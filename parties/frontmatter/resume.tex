	\chapter*{Résumé}
	\titreEntete{Résumé}
\addcontentsline{toc}{chapter}{Résumé}
	\medskip
	Ce mémoire, réalisé pour l'obtention du diplôme de Master 2 \og Technologies numériques appliquées à l'histoire\fg{} de l'École nationale des Chartes, retrace l'évolution des pratiques documentaires sur les référentiels en institution patrimoniale à travers l'étude des référentiels de l'\ac{ina} et leurs alignements. Cette étude de l'évolution des formes et des structures des référentiels est liée à l'évolution de la place de ces référentiels au sein des systèmes documentaires, ainsi qu'aux besoins qui leur sont liés.\\
	
	\textbf{Mots-clés~:} institut national de l'audiovisuel; référentiel; thésaurus; vocabulaire contrôlé; vocabulaire hiérarchique; ontologie; web de données; Wikidata; liens; alignement.
	
	\textbf{Informations bibliographiques~:} Maxime Challon, \textit{Les référentiels en institutions patrimoniales : évolution des pratiques et repositionnement. L’exemple des référentiels de l’Institut National de l’Audiovisuel.}, mémoire de master \og{}Technologies numériques appliquées à l'histoire\fg{}, dir. Gautier Poupeau, École nationale des chartes, 2020.