\chaptertoc{Conclusion}
\titreEntete{Conclusion}

Ce mémoire s'est attaché à montrer l'évolution de la structure et de la place des référentiels dans les systèmes documentaires des institutions patrimoniales. L'exemple du \ldd de l'\ac*{ina} a permis d'illustrer les structures des référentiels, les problématiques associées, ainsi que les solutions adoptées afin d'obtenir une place nouvelle pour ces référentiels dans un modèle de données repensé.\\

La recherche du moyen de classement de documents le plus efficace, ainsi que de l'outil de description le plus adapté, est une recherche millénaire qui évolue encore actuellement. La nécessité de se dégager du langage naturel est apparue dès l'Antiquité avec la conscience des difficultés qu'il représentait. Les siècles qui ont suivi ont permis de longues réflexions sur la structure idéale des classements hiérarchiques, des arbres, afin de définir au mieux chaque Chose du monde. Cette réflexion constante a conduit à la constatation des limites de cette représentation hiérarchique du monde. Pourtant, Boèce l'avait déjà évoqué avec la création infinie d'arbres selon le contexte.

\noindent L'influence de l'arbre dans les systèmes documentaires et dans les référentiels est importante puisqu'elle est à l'origine de l'ensemble des vocabulaires contrôlés comme les \textit{thesauri}. Comme les arbres, les vocabulaires contrôlés portent difficilement du sens au-delà des termes qui le composent: la hiérarchie ne rentre pas dans la définition sémantique des termes.

\noindent La possibilité infinie d'arbres conduit au vertige des labyrinthes qui se réinventent sans cesse selon le lecteur et sa position dans le labyrinthe. Au \textsc{XX}\textsuperscript{ème}siècle, les labyrinthes trouvent leur application dans l'informatique et dans le Web, créant ainsi des référentiels à leur image, désordonné et en constante évolution selon les requêtes qui y sont effectuées. Ainsi, les liens qui unissent les termes des référentiels portent du sens et permettent une définition d'un terme accompagnée de sens obtenu par ces liens. Ce \og modèle-réseau\fg{}, comme le nomme \nP{Umberto}{Eco}, atteint son apogée avec la création de Wikidata. Cette plateforme est à la fois une encyclopédie basée sur des entités, des propriétés et des valeurs, conformément aux principes du Web de données, mais elle est également un lieu centralisé depuis lequel il est possible de se rediriger vers d'autres ressources par un simple lien.

\noindent Contrairement aux volontés universalistes de \nP{Thomas}{d'Aquin} ou des Lumières, il n'existe pas un référentiel unique qui puisse décrire et définir l'ensemble du monde, mais bien autant de référentiels et de jeux de données que nécessaire, suivant les spécificités de chacun et leur domaine d'activité. Le graphe de données présent sur le Web a permis de relier ces multiples référentiels et de s'affranchir de la création d'un jeu de données unique représentant le monde: l'ensemble des données reliées du monde constitue un référentiel qui décrit seul le monde. Alors, une institution patrimoniale qui souhaite apporter une description précise à ses documents peut utiliser ce graphe de données et ne conserver qu'un point d'entrée avec son identifiant.

\noindent À plus petite échelle, dans une institution ou une entreprise, la déconstruction des jeux de données et des référentiels en données reliées offre les mêmes avantages de navigation et de centralisation des connaissances du domaine d'activité.\\

En plus de cette évolution de la structure des référentiels, ce mémoire a montré comment les usages et les besoins ont évolué avec les référentiels. En effet, l'usage qui est fait d'un référentiel dirige sa structure jusqu'à la fin des années 2000. Ainsi, à l'\ac{ina}, nous avons montré pour un même type de référentiel, celui des personnes physiques, mais pour deux métiers différents, la \ac*{dj} et la \ac*{ddcol}, les différences qui existent et qui trouvent leur origine dans les usages qui en sont faits: les niveaux de précision des informations et les structures divergent pour la description d'un même objet, une personne.

\noindent La nécessité de centraliser les données des diverses bases de données, et de leur redonner leur cohérence, a permis le lancement du \ldd. La BnF ou bien le Centre Pompidou virtuel ont également mené à bien cette transformation du système documentaire dans la dernière décennie, afin de faciliter les usages faits des données. En effet, le processus de création du référentiel est retourné: le modèle de données est pensé selon la donnée qui est stocké, au lieu de partir des usages. Ce renversement permet de s'abstraire des usages qui seront faits des données pour proposer à l'avenir autant de services adaptés aux besoins des utilisateurs, à partir des mêmes données.\\

Ce changement dans la manière de concevoir un modèle de données provoque un déplacement du référentiel au sein des systèmes documentaires. Les \textit{thesauri} et autres vocabulaires contrôlés étaient considérés comme de simples aides à la description: peu de termes étaient associés aux notices ce qui rendait la description sommaire et peu efficace pour la valorisation des documents. Le référentiel n'est qu'un outil de contrôle, comme peuvent l'être les règles de catalogage.

\noindent L'apparition du Web de données permet un enrichissement des données des instituions à partir d'un identifiant. Cependant, si cet enrichissement permet d'avoir plus d'informations concernant une entrée d'un thésaurus, il ne permet pas d'enrichir la description d'un document. L'utilisation du Web de données est alors subordonnée à la description préalable qui a été réalisée avec les référentiels internes.

\noindent L'impact du \ac*{led} est important puisqu'il permet d'établir un modèle de données global qui peut accueillir tout type de données sous la forme de nœuds et de liens. Avec ce \ac{led}, l'intelligence artificielle peut désormais décrire automatiquement chaque document avec précision, selon plusieurs outils de reconnaissance de textes, d'images ou de sons. Les descriptions créées sont alors plus complètes, et les liens vers les concepts du \ac{led} ou les entités du \ac*{lod} se multiplient et accroissent la quantité d'informations qui sera disponible pour l'utilisateur final.\\

Les référentiels, bien qu'éclatés en données et en liens, seront toujours nécessaires aux institutions et aux entreprises pour la description contrôlée de leurs documents. L'évolution de leur forme, de leur structure et de leurs usages s'est produite sur plusieurs millénaires, et connaît de profonds bouleversements depuis les années 2000. Cette évolution est constante et nous pouvons imaginer que les futurs référentiels seront des référentiels partagés sur le Web certes, mais encore plus centralisés. Cette centralisation doit avoir lieu au-delà des limites longtemps perçues entre les institutions patrimoniales et le reste du Web, afin de décloisonner les référentiels et de les intégrer pleinement dans le Web sémantique.