\chapter{\label{I-A}Le référentiel comme clé}
\titreEntete{Le référentiel comme clé}

Considéré comme une simple aide ou outil au service du documentaliste ou de l'utilisateur, le référentiel trouve d'abord sa place comme fournisseur de clés. Son utilisation principale est d'offrir au document décrit des vedettes qui puissent permettre une classification ou une recherche aisée de ce document. Cependant, pour être efficaces, ces vedettes doivent partager un langage contrôlé, des règles de graphie, de syntaxe, \dots D'abord conservées sur des fichiers papier en institutions patrimoniales, ces vedettes ont été parmi les premiers éléments rétroconvertis, donnant naissance aux fichiers d'autorité numériques, et permettant une interopérabilité entre les référentiels par le biais des portails numériques.

\section{\label{I-A-1}Du langage libre au langage contrôlé: vers l'indexation}
\titreEntete{Vers l'indexation}

\begin{quote}
	\og La nature n'a pas juré de ne nous offrir que des objets exprimables par des formes simples de langage \footcite[p.18]{valery_variete_1936} \fg{}
\end{quote}

Le langage permet aux hommes de communiquer entre eux. Ce langage libre, naturel, comprend l'ensemble des langues, et donne aux hommes la possibilité de décrire le plus précisément possible le monde qui les entoure, sans jamais atteindre le description idéale. Seulement, ce langage conduit à des variations graphiques ou syntaxiques, selon la déclinaison des noms ou la conjugaison des verbes. La polysémie est également l'une des conséquences de ce langage naturel selon le contexte de chaque mot. Enfin, le langage libre conduit à la synonymie. Toutes ces caractéristiques du langage humain perturbent et complexifient la tâche de description documentaire, bien qu'elles soient essentielles à la communication entre les hommes.\\

Afin de régler ces confusions possibles entre les mots et de régir leur formation, des langages contrôlés ont très vite fait leur apparition. Ils permettent de décrire des concepts, des thèmes, des ouvrages, tout en permettant un classement potentiel. Ce recours aux langages contrôlés est une pratique très ancienne, née avant l'apparition des \textit{codices} lorsque déjà la recherche d'informations était nécessaire. Pratique millénaire, l'attribution de termes contrôlés à une information se perpétue encore actuellement, par exemple sous la forme de \og hashtags\fg{} sur les réseaux sociaux, qui permettent de décrire un texte et de le retrouver ensuite aux côtés d'autres similaires.\\

Dans l'Antiquité, les index n'existent pas encore. Cependant, des \index[ref]{typologie@Typologie!vocabulaires controles@Vocabulaires contrôlés}vocabulaires contrôlés sont utilisés pour le classement et pour la mémorisation des textes. Ces termes contrôlés se retrouvent dans des notes marginales, des tables de concordance ou bien dans les catalogues. Au \textsc{III}\textsuperscript{ème}siècle av. J.C., \nP{Callimaque}{de Silène} réalise le catalogue de la bibliothèque d'Alexandrie en utilisant le genre du texte pour lui déterminer une classe, puis les \textit{volumina} sont rangés dans des rayons selon un ordre alphabétique, ces rayons reflétant les classes attribuées selon le genre.\\

Au Moyen-Âge, les premiers \index[ref]{typologie@Typologie!index@Index}index apparaissent, s'ajoutant aux tables de concordance. \nP{Isidore}{de Séville} ne créé qu'un classement alphabétique dans son Livre X des \textit{Étymologies}, sans indexer son ouvrage. Cinq siècles plus tard, les vedettes commencent à être normalisées dans certaines œuvres, le nominatif ou l'ablatif étant considérés comme la forme retenue, et rassemblées dans un index alphabétique\footnote{\nP{Jean}{Berger}, dans son analyse du \textit{Liber de honoribus}, le plus vieil index alphabétique compilé au \textsc{XII}\textsuperscript{ème}siècle, étudie avec précision l'indexation des chartes du Cartulaire de Saint-Julien de Brioude: les lieux et les personnes sont ainsi indexés. Voir \cite[pp.97 et suivantes]{berger_indexation_2006}}.\\

Avec la Renaissance puis l'Ancien Régime, l'indexation devient plus fine et les index de fin de volumes de plus en plus imposants. Ils permettent au lecteur un accès direct aux passages du texte contenant l'entrée d'index. Plus encore, ces index lient une classification générale suivie d'alphabétique, tout en normalisant leurs entrées\footnote{\nP{Jean-Daniel}{Schoepflin} dans son \textit{Alsatia illustrata} de 1751 créé ainsi deux index distincts: l'un pour les personnes (\textit{Index auctorum}), l'autre pour les termes évoqués dans son œuvre(\textit{Index rerum}). L'ensemble des noms est indexé au nominatif puis ils sont parfois subdivisés en thèmes ou événements. L'index devient ainsi indépendant de la graphie et de la grammaire de la langue utilisée. Voir \cite{schoepflin_alsatia_1751}. Voir \reference{annexe_index_schoepflin}.}\footnote{\nP{Robert}{Estienne} pousse plus loin encore l'indexation, un siècle et demi avant \nP{Jean-Daniel}{Schoepflin}, en créant de multiples \index[ref]{typologie@Typologie!index@Index}index: celui des populations, des villes, \dots. Ces index sont eux-mêmes subdivisés, normalisés et classés alphabétiquement, les rendant œuvre à part entière. Voir \cite{estienne_thesaurus_1573}.}.

\section{\label{I-A-2}Une clé entre les données: une terminologie maîtrisée, objectif des vocabulaires contrôlés}

\section{\label{I-A-3}Une clé entre les jeux de données: l'interopérabilité par les fichiers d'autorité et les portails}
\titreEntete{Une clé entre les jeux de données}

Comme nous l'avons évoqué précédemment (voir \reference{I-A-2}), les vocabulaires contrôlés sont de nouveaux langages, spécifiques et uniformisés, se substituant au langage naturel humain pour un domaine précis. Le vocabulaire est par conséquent un référentiel propre à l'institution qui l'a créé et a pour seul utilisateur cette institution. Seulement, deux institutions aux activités proches créent deux vocabulaires similaires, se distinguant par la complétude de certaines vedettes ou par des variantes de graphies.\\

Le domaine bibliothéconomique a été le premier à informatiser en masse ses vocabulaires et ses \index[ref]{autorites@Autorités!fichiers autorite@Fichiers d'autorité}fichiers d'autorité, permettant ainsi une amélioration de l'expérience utilisateur et du catalogage, tout comme un partage possible avec des institutions proches.

\subsection{\label{I-A-3-a}La naissance des autorités par rétroconversion}
\titreEntete{Les fichiers d'autorité}

\begin{citationLongue}
	Les fichiers d'autorité appartiennent bien à un	ensemble : fonctionnant comme un tout, avec des règles d’interdépendance et d’interopérabilité de ses constituants, ils permettent le contrôle de la cohérence des métadonnées bibliographiques.\footcite[p.6]{aymonin_arabesques_2017}
\end{citationLongue}

Avant la naissance du Web, chaque ouvrage était décrit dans un catalogue et classé par ordre alphabétique des noms d'auteurs. Des \index[ref]{autorites@Autorités!fichiers autorite@Fichiers d'autorité}catalogues thématiques ont été créés, de même que des fichiers physiques en bibliothèque, permettant la recherche de documents selon un sujet précis. Cependant, l'indexation des documents est réduite au titre, à l'auteur, et à quelques sujets. En effet, la structure même d'un fichier papier en bibliothèque nécessite de dupliquer la notice d'un exemplaire en plusieurs notices qui vont être placées par la suite dans le fichier correspondant au sujet.\\

Ces fichiers physiques des bibliothèques, bien qu'utiles aux lecteurs par leur classement thématique, présentent plusieurs difficultés: d'abord, l'indexation se trouve limitée à quelques mots; ensuite, la création d'un fichier thématique est complexe à réaliser par le choix des vedettes et produit alors un immense silence; enfin, la consultation d'une fiche par un lecteur empêche un second de la consulter dans le même temps.\\

Dès les années 1970, les bibliothèques se sont engagées dans une vaste opération de rétroconversion de leurs notices documentaires. Les fichiers physiques et les notices cartonnées sont alors informatisés et \og reproduits presque à l’identique [\dots] sous forme de bases de données\fg{}\footcite{bermes_du_2013}. L'informatisation des notices et des fichiers permet, par conséquent, d'améliorer l'indexation des documents. L'utilisateur va donc pouvoir trouver plus rapidement plus de documents correspondant à sa recherche. Ainsi, les autorités \index[ref]{lod@Linked Open Data (LOD)!lcsh@LCSH}\index[ref]{autorites@Autorités!lcsh@LCSH}\ac{lcsh}, créées en 1914 sous format papier, ont été informatisées; les autorités \index[ref]{lod@Linked Open Data (LOD)!rameau@RAMEAU}\index[ref]{autorites@Autorités!rameau@RAMEAU}\ac{rameau} créées dans les années 1980 reprennent celles de \ac{lcsh} en les complétant.\\

Cependant, ces \index[ref]{autorites@Autorités!fichiers autorite@Fichiers d'autorité}fichiers d'autorité comportent, comme nous l'avons évoqué plus haut (\reference{I-A-2-c}), des formes retenues et des formes rejetées des termes, ce qui crée de multiples renvois à l'intérieur des fichiers physique ou informatique. L'arrivée des moteurs de recherche, dans les années 2000, permet de supprimer ces différences de termes, en indexant à la fois les formes retenues et les formes rejetées et en permettant de trouver directement la vedette recherchée.

\subsection{\label{I-A-3-b}Partager des vocabulaires: à la recherche de la meilleure interopérabilité}
\titreEntete{Partager des vocabulaires}

La problématique du partage des référentiels entre institutions se pose avant l'informatisation des catalogues et des fichiers des bibliothèques. En effet, le format \index[ref]{echanges@Échanges!formats@Formats!marc@MARC}\ac{marc}, né en 1968 à la Bibliothèque du Congrès, permet l'échange de données entre les institutions et la \og duplication les notices d’un catalogue à un autre\fg{}\footcite{bermes_convergence_2013}. Malgré de multiples variantes nationales, l'\ac{unimarc} reste aujourd'hui le format d'échange privilégié entre les bibliothèques.\\

Pour partager les \index[ref]{autorites@Autorités!fichiers autorite@Fichiers d'autorité}fichiers d'autorité et aboutir à une interopérabilité totale des données entre deux institutions par le biais des machines, différents protocoles d'échange ont été utilisés --- ou délaissés en fonction des difficultés imposées par chacun. Dès les années 1980 est développé le \index[ref]{echanges@Échanges!protocoles@Protocoles!z3950@Z39-50}protocole Z39-50. Ce protocole permet d'interroger une base de données de manière synchrone, selon la requête du client, et de récupérer des données en format \ac{marc}\footcite{bibliotheque_nationale_de_france_protocole_nodate}.\\

Ce protocole Z39-50 est destiné aux catalogueurs qui peuvent ainsi \og repérer puis télécharger une notice dans un catalogue distant plutôt que d’avoir à la saisir \textit{ex nihilo}\fg{}\footcite{bermes_convergence_2013}. Le partage, \og par conversion et copie\fg{}\footnote{\cite{bermes_convergence_2013}. Voir \reference{annexe_types_interop} (\reference{conversion}).}, n'est alors qu'une simple copie de données, dont la mise à jour est difficile. L'existence de ce protocole, bien que destiné aux professionnels de la documentation, a suscité la création de \index[ref]{relier@Relier!portails@Portails}portails de consultation de notices documentaires ou de \index[ref]{autorites@Autorités!fichiers autorite@Fichiers d'autorité}fichiers d'autorité, interrogeant de manière synchrone les bases de données: cette utilisation orientée utilisateur du \index[ref]{echanges@Échanges!protocoles@Protocoles!z3950@Z39-50}protocole Z39-50 permet à la Bibliothèque nationale de France d'offrir différents services(intégration des notices dans \index[ref]{lod@Linked Open Data (LOD)!oclc@OCLC}\ac{oclc}, recherche dans le Catalogue Collectif de France (CCFR), \dots\footcite{bibliotheque_nationale_de_france_protocole_nodate}). Cependant, face aux temps de réponses importants et aux résultats appauvris retournés par la requête, les portails se sont révélés décevants et peu efficaces. De plus, l'utilisation d'un portail nécessite, de la part de l'utilisateur, qu'il connaisse précisément ce qu'il cherche, de manière à se connecter au \index[ref]{relier@Relier!portails@Portails}portail correspondant (qui lui-même doit être connu de cet utilisateur)\footcite{dalbin_approches_2011}.\\

La multiplication des formats d'échanges --- \index[ref]{echanges@Échanges!formats@Formats!marc@MARC}\ac{marc} et \ac{unimarc} pour les bibliothèques, \index[ref]{echanges@Échanges!formats@Formats!ead@EAD}\ac{ead} pour les archives ---, ainsi que la volonté d'offrir au public lien entre les différentes bases de données patrimoniales, ont conduit à la création d'un nouveau protocole, \index[ref]{echanges@Échanges!protocoles@Protocoles!oaipmh@OAI-PMH}\ac{oaipmh}. Ce protocole asynchrone repose sur deux acteurs: le fournisseur qui met à disposition ses données dans un \og entrepôt\fg{}, et le moissonneur qui collecte ces données pour les intégrer à son système\footcite{bibliotheque_nationale_de_france_protocole_nodate-1}.\\

Cependant, si les performances du protocole sont améliorées avec \ac{oaipmh}, les documents et les \index[ref]{autorites@Autorités!fichiers autorite@Fichiers d'autorité}fichiers d'autorité ne peuvent pas être sélectionnés et filtrés: un format d'échange simple, minimal, est nécessaire. Ce format est le \index[ref]{echanges@Échanges!formats@Formats!dublin core@Dublin Core}Dublin Core\footcite{noauthor_dublin_nodate} comprenant quinze champs d'informations.
Ce partage de données et de métadonnées entre les institutions permet une \og interopérabilité par le plus petit dénominateur commun\fg{}\footnote{\cite{bermes_convergence_2013}. Voir \reference{annexe_types_interop} (\reference{denom})}, qui est le Dublin Core. Celui-ci peut cependant présenter un appauvrissement des données puisque les champs sont très réduits, ou au contraire permettre de grandes différences au sein d'un même champ.

%conclusion