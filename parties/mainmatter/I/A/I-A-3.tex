\section{\label{I-A-3}Une clé entre les jeux de données: l'interopérabilité par les fichiers d'autorité et les portails}
\titreEntete{Une clé entre les jeux de données}

Comme nous l'avons évoqué précédemment (voir \reference{I-A-2}), les vocabulaires contrôlés sont de nouveaux langages, spécifiques et uniformisés, se substituant au langage naturel humain pour un domaine précis. Le vocabulaire est par conséquent un référentiel propre à l'institution qui l'a créée et a pour seul utilisateur cette institution. Seulement, deux institutions aux activités proches créent deux vocabulaires similaires, se distinguant par la complétude de certaines vedettes ou par des variantes de graphies.\\

Le domaine bibliothéconomique a été le premier à informatiser ses vocabulaires et ses fichiers d'autorités en masse, permettant ainsi une amélioration de l'expérience utilisateur et du catalogage, et un partage possible avec des institutions proches.

\subsection{\label{I-A-3-a}La naissance des autorités par rétroconversion}
\titreEntete{Les fichiers d'autorité}

\begin{citationLongue}
	Les fichiers d'autorité appartiennent bien à un	ensemble : fonctionnant comme un tout, avec des règles d’interdépendance et d’interopérabilité de ses constituants, ils permettent le contrôle de	la cohérence des métadonnées bibliographiques.\footcite[p.6]{aymonin_arabesques_2017}
\end{citationLongue}

Avant la naissance du web, chaque ouvrage était décrit dans un catalogue et classé par ordre alphabétique des noms d'auteur. Des catalogues thématiques ont été créés, de même que des fichiers physiques en bibliothèque, permettant la recherche de documents selon un sujet précis. Cependant, l'indexation des documents est réduite au titre, à l'auteur, et à quelques sujets. En effet, la structure même d'un fichier papier en bibliothèque nécessite de dupliquer la notice d'un exemplaire en plusieurs notices qui vont être placées par la suite dans le fichier correspondant au sujet.\\

Ces fichiers physiques des bibliothèques, bien qu'utiles aux lecteurs par leur classement thématique, présentent plusieurs difficultés: d'abord, l'indexation se trouve limitée à quelques mots; ensuite, la création d'un fichier thématique est complexe à réaliser par le choix des vedettes et produit alors un immense silence; enfin, la consultation d'une fiche par un lecteur empêche un second de la consulter dans le même temps.\\

Dès les années 1970, les bibliothèques se sont engagées dans une vaste opération de rétroconversion de leurs notices documentaires. Les fichiers physiques et les notice cartonnées sont alors informatisés et \og reproduits presque à l’identique [\dots] sous forme de bases de données\fg{}\footcite{bermes_1_2013}. L'informatisation des notices et des fichiers permet par conséquent d'améliorer l'indexation des documents, et à l'utilisateur de pouvoir trouver plus de documents correspondant à sa recherche plus rapidement. Ainsi, les autorités \ac{lcsh}, créées en 1914 sous format papier, ont été informatisées; les autorités \ac{rameau} créées dans les années 1980 reprennent celles \ac{lcsh} en les complétant.\\

Cependant, ces fichiers d'autorité comportent, comme nous l'avons évoqué plus haut (\reference{I-A-2-c}), des formes retenues et des formes rejetées des termes, ce qui créé de multiples renvois à l'intérieur du fichier physique ou informatique. L'arrivée des moteurs de recherche dans les années 2000 permet de supprimer ces différences de termes en indexant à la fois les formes retenues et les formes rejetées, permettant de trouver directement la vedette recherchée.

\subsection{\label{I-A-3-b}Partager des vocabulaires: à la recherche de la meilleure interopérabilité}
\titreEntete{Partager des vocabulaires}

La problématique du partage des référentiels entre institutions se pose avant l'informatisation des catalogues et des fichiers des bibliothèques. En effet, le format \ac{marc}, né en 1968 à la Bibliothèque du Congrès, permet l'échange de données entre les institutions et la \og duplication les notices d’un catalogue à un autre\fg{}\footcite{bermes_2_2013}. Malgré de multiples variantes nationales, l'\ac{unimarc} reste aujourd'hui le format d'échange privilégié entre les bibliothèques.\\

Pour partager les fichiers d'autorité et aboutir à une interopérabilité totale des données entre deux institutions par le biais des machines, différents protocoles d'échange ont été utilisés --- ou délaissés en fonction des difficultés imposées par chacun---. Dès les années 1980 est développé le protocole Z39-50. Ce protocole permet d'interroger une base de données de manière synchrone, selon la requête du client, et de récupérer des données en format \ac{marc}\footcite{bibliotheque_nationale_de_france_protocole_nodate}.\\

Ce protocole Z39-50 est destiné aux catalogueurs qui peuvent ainsi \og repérer puis télécharger une notice dans un catalogue distant plutôt que d’avoir à la saisir \textit{ex nihilo}\fg{}\footcite{bermes_2_2013}. Le partage, \og par conversion et copie\fg{}\footnote{\cite{bermes_2_2013}. Voir \reference{annexe_types_interop}}, n'est alors qu'une simple copie de données, dont la mise à jour est difficile. L'existence de ce protocole, bien que destiné aux professionnels de la documentation, a suscité la création de portails de consultation de notices documentaires ou de fichiers d'autorité, interrogeant de manière synchrone les bases de données: cette utilisation orientée utilisateur du protocole Z39-50 permet à la Bibliothèque nationale de France d'offrir différents services(intégration des notices dans \ac{oclc}, recherche dans le Catalogue Collectif de France (CCFR), \dots\footcite{bibliotheque_nationale_de_france_protocole_nodate}). Cependant, face aux temps de réponses importants et aux résultats appauvris retournés par la requête, les portails se sont révélés décevants et peu efficaces. De plus, l'utilisation d'un portail nécessite de la part de l'utilisateur qu'il connaisse précisément ce qu'il cherche de manière à se connecter au portail correspondant (sui lui-même doit être connu de cet utilisateur)\footcite{dalbin_approches_2011}.\\

La multiplication des formats d'échanges --- \ac{marc} et \ac{unimarc} pour les bibliothèques, \ac{ead} pour les archives ---, ainsi que la volonté d'offrir au public lien entre les différentes bases de données patrimoniales, ont conduit à la création d'un nouveau protocole, \ac{oaipmh}. Ce protocole asynchrone repose sur deux acteurs: le fournisseur qui met à disposition ses données dans un \og entrepôt\fg{}, et le moissonneur qui collecte ces données pour les intégrer à son système\footcite{bibliotheque_nationale_de_france_protocole_nodate-1}.\\

Cependant, si les performances du protocole sont améliorées avec \ac{oaipmh}, les documents et les fichiers d'autorité ne peuvent pas être sélectionnés et filtrés: un format d'échange simple, minimal, est nécessaire. Ce format est le Dublin Core\footcite{noauthor_dublin_nodate} comprenant quinze champs d'informations.
Ce partage de données et de métadonnées entre les institutions permet une \og interopérabilité par le plus petit dénominateur commun\fg{}\footnote{\cite{bermes_2_2013}. Voir \reference{annexe_types_interop}}, où ce dénominateur est le Dublin Core. Ce dénominteur commun peut néanmoins présenter un appauvrissement des données puisque les champs sont très réduits, ou au contraire permettre de grandes différences au sein d'un même champ.