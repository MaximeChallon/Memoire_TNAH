\section{\label{I-A-1}Du langage libre au langage contrôlé: vers l'indexation}
\titreEntete{Vers l'indexation}

\begin{quote}
	\og La nature n'a pas juré de ne nous offrir que des objets exprimables par des formes simples de langage \footcite[p.18]{valery_variete_1936} \fg{}
\end{quote}

Le langage permet aux hommes de communiquer entre eux. Ce langage libre, naturel, comprend l'ensemble des langues, et donne aux hommes la possibilité de décrire le plus précisément possible le monde qui les entoure, sans jamais atteindre le description idéale. Seulement, ce langage conduit à des variations graphiques ou syntaxiques, selon la déclinaison des noms ou la conjugaison des verbes. La polysémie est également l'une des conséquences de ce langage naturel selon le contexte de chaque mot. Enfin, le langage libre conduit à la synonymie. Toutes ces caractéristiques du langage humain perturbent et complexifient la tâche de description documentaire, bien qu'elles soient essentielles à la communication entre les hommes.\\

Afin de régler ces confusions possibles entre les mots et de régir leur formation, des langages contrôlés ont très vite fait leur apparition. Ils permettent de décrire des concepts, des thèmes, des ouvrages, tout en permettant un classement potentiel. Ce recours aux langages contrôlés est une pratique très ancienne, née avant l'apparition des \textit{codices} lorsque déjà la recherche d'informations était nécessaire. Pratique millénaire, l'attribution de termes contrôlés à une information se perpétue encore actuellement, par exemple sous la forme de \og hashtags\fg{} sur les réseaux sociaux, qui permettent de décrire un texte et de le retrouver ensuite aux côtés d'autres similaires.\\

Dans l'Antiquité, les index n'existent pas encore. Cependant, des \index[ref]{typologie@Typologie!vocabulaires controles@Vocabulaires contrôlés}vocabulaires contrôlés sont utilisés pour le classement et pour la mémorisation des textes. Ces termes contrôlés se retrouvent dans des notes marginales, des tables de concordance ou bien dans les catalogues. Au \textsc{III}\textsuperscript{ème}siècle av. J.C., \nP{Callimaque}{de Silène} réalise le catalogue de la bibliothèque d'Alexandrie en utilisant le genre du texte pour lui déterminer une classe, puis les \textit{volumina} sont rangés dans des rayons selon un ordre alphabétique, ces rayons reflétant les classes attribuées selon le genre.\\

Au Moyen-Âge, les premiers \index[ref]{typologie@Typologie!index@Index}index apparaissent, s'ajoutant aux tables de concordance. \nP{Isidore}{de Séville} ne créé qu'un classement alphabétique dans son Livre X des \textit{Étymologies}, sans indexer son ouvrage. Cinq siècles plus tard, les vedettes commencent à être normalisées dans certaines œuvres, le nominatif ou l'ablatif étant considérés comme la forme retenue, et rassemblées dans un index alphabétique\footnote{\nP{Jean}{Berger}, dans son analyse du \textit{Liber de honoribus}, le plus vieil index alphabétique compilé au \textsc{XII}\textsuperscript{ème}siècle, étudie avec précision l'indexation des chartes du Cartulaire de Saint-Julien de Brioude: les lieux et les personnes sont ainsi indexés. Voir \cite[pp.97 et suivantes]{berger_indexation_2006}}.\\

Avec la Renaissance puis l'Ancien Régime, l'indexation devient plus fine et les index de fin de volumes de plus en plus imposants. Ils permettent au lecteur un accès direct aux passages du texte contenant l'entrée d'index. Plus encore, ces index lient une classification générale suivie d'alphabétique, tout en normalisant leurs entrées\footnote{\nP{Jean-Daniel}{Schoepflin} dans son \textit{Alsatia illustrata} de 1751 crée ainsi deux index distincts: l'un pour les personnes (\textit{Index auctorum}), l'autre pour les termes évoqués dans son œuvre(\textit{Index rerum}). L'ensemble des noms est indexé au nominatif puis ils sont parfois subdivisés en thèmes ou événements. L'index devient ainsi indépendant de la graphie et de la grammaire de la langue utilisée. Voir \cite{schoepflin_alsatia_1751}. Voir \reference{annexe_index_schoepflin}.}\footnote{\nP{Robert}{Estienne} pousse plus loin encore l'indexation, un siècle et demi avant \nP{Jean-Daniel}{Schoepflin}, en créant de multiples \index[ref]{typologie@Typologie!index@Index}index: celui des populations, des villes, \dots. Ces index sont eux-mêmes subdivisés, normalisés et classés alphabétiquement, les rendant œuvre à part entière. Voir \cite{estienne_thesaurus_1573}.}.