\section{\label{I-B-1}De multiples fonds à décrire}
\titreEntete{De multiples fonds à décrire}


\subsection{\label{I-B-1-a}Les archives professionnelles}
\titreEntete{Les archives professionnelles}

Après l'éclatement de l'\ac{ortf}, des fonds divers deviennent la propriété de l'\ac{ina} et participent à la diversité des fonds audiovisuels conservés à l'\ac{ina}. Ainsi, les \og Actualité françaises\fg{} --- la presse filmée diffusée dans les selles de cinéma --- ont été transférées à l'\ac{ina} en 1974. Les grands moments des débuts de la télévision --- le premier journal télévisé, les premières grandes émissions ou les magazines de reportage --- sont conservés et participent, avant l'instauration du dépôt légal, à la mémoire de l'audiovisuel français; de même pour les fonds radio qui retracent les grands moments historiques depuis les années 1940, et qui deviennent de plus en plus complets avec la généralisation de la bande magnétique à partir du milieu des années 1950.\\

Les archives professionnelles de l'\ac{ina} ne sont que des archives: elles ne couvrent pas l'ensemble de la production audiovisuelle depuis les années 1940 ou la création de l'Institut. Des tranches horaires de diffusion télé- ou radiodiffusées ne sont ainsi pas conservées et peu de traces demeurent alors de la production audiovisuelle, notamment avant l'arrivée du kinéscope. Pour combler ces manques, l'\ac{ina} dispose d'un fonds photographique créé à partir des services de l'\ac{ortf} ou de l'\ac{ina} et portant sur la réalisation des émissions et des tournages.\\

Enfin, des délégations régionales se chargent de la conservation et de la communication des archives télévisées et radiodiffusées des stations régionales --- apparues dans les années 1950: la vie et l'histoire des régions sont ainsi couvertes par l'\ac{ina}.\\

Les fonds d'archives professionnelles de l'\ac{ina} sont conséquents et divers, témoins de la vie et de la société française depuis l'Après-Guerre. Irremplaçables, leur description n'en est pas moins difficile par la diversité des sujets évoqués ou présents dans les documents.

\subsection{\label{I-B-1-b}Les fonds issus du dépôt légal}
\titreEntete{Les fonds issus du dépôt légal}

Depuis la loi sur le dépôt légal de l'audiovisuel de 1992, l'\ac{ina} en est le dépositaire. À partir de 1995, l'\ac{ina} enregistre la globalité de la programmation des stations de Radio-France --- France-Inter, France-Musique, France-Culture, France-Info et France-Bleue ---, enregistrement étendu en 2001 aux stations privées généralistes comme RTL ou NRJ\footnote{En 2020, l'ensemble des stations captées au titre du dépôt légal par l'INA est décrit dans l'\reference{annexe_dl_captation} (\reference{dl_radio})}.\\

Pour les programmes télévisés, le dépôt légal ne concerne d'abord --- entre 1995 et 2001 --- que les sept chaînes principales --- TF1, France 2, France 3, Canal +, M6, Arte, France 5 --- et leurs programmes en première diffusion. La captation directe et intégrale des chaînes n'apparaît qu'en 2002 et est élargie à douze autres chaînes. Enfin, depuis 2005, les chaînes de la Télévision numérique terrestre (TNT) sont toutes captées\footnote{Les chaînes de télévision captées en 2002 pour le dépôt légal sont décrites dans l'\reference{annexe_dl_captation} (\reference{dl_tv})}.

\bigskip
\bigskip

La diversité des fonds d'archives, la captation directe en intégralité des chaînes de télévision et de radio, ainsi que la captation de sites web, plateformes ou comptes de réseaux sociaux au titre du dépôt légal audiovisuel, représentent une masse très importante de documents à conserver et de données. En 2019\footcite[p.5]{institut_national_de_laudiovisuel_rapport_2019}, l'\ac{ina} conserve 20 873 143 heures de programmes de télévision et de radio, dont plus de 18 millions captés par le dépôt légal. 1,2 million de photos s'ajoutent à ces documents. La majorité de ces documents, issus du dépôt légal, sont destinés à une gestion patrimoniale et à une valorisation dans l'INAthèque, alors que les documents des archives professionnelles sont destinées à la valorisation commerciale au travers notamment le site \href{https://www.inamediapro.com}{INAMediaPro} destinés aux professionnels.