\section{\label{I-B-2}Un système documentaire pluriel répondant aux besoins}
\titreEntete{Un système documentaire pluriel}

La masse des documents, l'évolution de leur récupération auprès des producteurs et leurs usages divers conduisent à la création d'un système documentaire pluriel, créé à partir des besoins et non des données. Les deux usages commerciaux et patrimoniaux des documents\footnote{É. Alquier décrit ces deux usages dans un article de 2017 évoque ces usages et la plateforme qui les met en œuvre. En 2020, le service de vidéo à la demande Madelen vient s'ajouter à l'offre \og grand public\fg{} de l'\ac{ina}. Voir \cite{alquier_production_2017}.}, évoqués précédemment, dirigent le nombre des bases de données, leur structure et le partage de référentiels. Avant le projet du \ldd lancé en 2015, le système documentaire de l'\ac{ina} est pluriel, constitué de plusieurs bases de données distinctes ainsi que de plusieurs référentiels non communs.\\

Deux types de données sont présents dans les bases de l'\ac{ina}. D'abord, il y a du texte libre, décrivant les titres propres des documents, les titres de collections ou indiquant un identifiant, ou bien des notes diverses ou des chiffres. Ensuite, il y a les données contrôlées, issues de référentiels et de lexiques, permettant de décrire les contenus, les particularités de ces contenus et les événements associés à ces contenus (diffusion, archivage, exploitation)\footnote{Voir \reference{annexe_type_donnees_axel} (\reference{type_donnees_axel}).}.

\subsection{\label{I-B-2-a}Les bases de données du dépôt légal (DL)}
\titreEntete{Les bases de données du dépôt légal}

L'\ac{ina} capte en permanence et en direct plus de 170 chaînes de télévision et stations de radio\footcite[p.5]{institut_national_de_laudiovisuel_rapport_2019}. Ce flux ininterrompu est décrit lors du catalogage par des techniciens spécialisés dans la gestion de collections multimédia: pour chaque document y est notamment indiqué le titre, le générique, les dates et heures de diffusion, ainsi que des descripteurs pour indexer la chaîne de diffusion, les thématiques présentes, \dots\\

Quand le document est décrit, les données complétées par le technicien de gestion des collections multimédia dans son interface graphique sont dirigées vers les bases de données du dépôt légal, scindées en quatre pour correspondre à la provenance du document. Ainsi, bien que les documents proviennent de la même source --- la captation pour le dépôt légal ---, ils sont éclatés dans quatre bases de données différentes pour correspondre à leur provenance:
\begin{itemize}
	\item la base DLRADIO (Dépôt Légal de la Radio) comprend les documents diffusés en radio, sans autre distinction de provenance
	\item la base DLTV (Dépôt Légal de la Télévision (Nationale)) ne comprend pas l'ensemble des documents diffusés à la télévision, mais seulement les chaînes nationales
	\item la base DLREG (Dépôt Légal de la Télévision Régionale) comprend les documents télévisuels diffusés sur une chaîne de télévision régionale comme France 3
	\item enfin, la base DLSAT (Dépôt Légal de la Télévision Satellite) comprend les documents diffusés sur les chaînes de télévision satellite
\end{itemize}
\bigskip

Cependant, malgré cette scission des données dans plusieurs bases de données, ces quatre bases partagent un même schéma pour les \index[ref]{typologie@Typologie!vocabulaires controles@Vocabulaires contrôlés}référentiels. Ce schéma permet de trouver des tables comprenant la signification d'identifiants de provenance de chaînes (le lien entre le code \og FR5\fg{} présent dans les données peut ainsi être établi avec son terme développé), de provenance de données, \dots Ce schéma est un fournisseur de mots-clés destinés à permettre la description, l'indexation et la recherche des documents.

\subsection{\label{I-B-2-b}Les bases de données des archives professionnelles (DA)}
\titreEntete{Les bases de données des archives professionnelles}

Le dépôt légal se concentre sur la diffusion des documents et conserve alors à chaque instant l'ensemble de ce qui est diffusé à la télévision ou à la radio --- les émissions, les films, les publicités, les journaux télévisés, \dots~ Cette conservation des premières diffusions et des rediffusions permet, ainsi que l'exige le dépôt légal, d'avoir un panorama complet du paysage audiovisuel français, comme c'est le cas à la Bibliothèque nationale de France pour les imprimés ou les périodiques.\\

Les archives professionnelles ne sont pas soumises à cette exhaustivité: lorsqu'un producteur de contenu audiovisuel mandate l'\ac{ina} pour la conservation et/ou la commercialisation de ses contenus, les données de ces contenus sont récupérées dans les bases du dépôt légal puis copiées dans celles des archives professionnelles. Ainsi, la même donnée est dupliquée au \ac{dl} et au \ac{da}. Cependant, le \ac{da} s'intéressant non pas à la diffusion elle-même du document mais au document lui-même, ces données vont être transformées et complétées de manière à être plus précises et à avoir une meilleure description. Cette description plus fine permet la vente des extraits.\\

Comme pour le \ac{dl}, le \ac{da} possède plusieurs bases de données partageant les mêmes référentiels:
\begin{itemize}
	\item la base DAV (Archives Professionnelles de la Télévision Nationale)
	\item la base DAVREG (Archives Professionnelles de la Télévision Régionale)
	\item la base DAVRAD (Archives Professionnelles de la Radio)
\end{itemize}
\bigskip

Ces trois bases de données sont appuyées par plusieurs lexiques et \index[ref]{typologie@Typologie!thesaurus@Thésaurus}\textit{thesauri}, notamment celui des noms communs\footnote{L'importance --- et la complexité --- de ce thésaurus au \ac{da} nécessite une interface graphique, \og Totem\fg{}, pour le visualiser et cataloguer les documents. Un exemple de visualisation de ce thésaurus est possible en \reference{annexe_thesaurus} (\reference{thesaurus_cadreur}).} et des personnes physiques et morales.

\subsection{\label{I-B-2-c}La base de donnés juridique (DJ)}
\titreEntete{La base de donnés juridique}

La base de données \og Adaje\fg{} de la \ac{dj} comprend l'ensemble des données permettant d'identifier et de rémunérer les ouvrants-droit\footnote{Personnes auxquelles les droits ont été ouverts, le producteur lui-même ou ses ayants-droit.} et les ayants-droit\footnote{\og Un ayant droit est une personne ayant acquis un droit d'une autre personne\fg{} in \url{https://droit-finances.commentcamarche.com/faq/4010-ayant-droit-definition}.} des documents et extraits vendus. Cette base juridique contient par conséquent des tables de Personnes, de Contributions, d'Informations personnelles, \dots\\

Les bases \ac{dl} et \ac{da}, et celle de la \ac{dj} n'ont aucun lien entre elles, mais leurs données semblent redondantes notamment pour les personnes physiques et morales. Le projet du \index[ref]{led@Linked Enterprise Data (LED)!ldd@Lac de données (INA)}\index[ref]{modelisation@Modélisation!ldd@Lac de données (INA)}\ldd\footnote{Ce projet est évoqué au \reference{III-B}} devrait permettre l'alignement de ces bases entre elles en évitant les doublons: la base de la \ac{dj} enrichira notamment les concepts de personnes physiques et morales déjà créés à partir des données de la \ac{ddcol}.

\bigskip
\bigskip

Plusieurs référentiels, parfois similaires, sont présents dans les bases de la \ac{ddcol}\footnote{Voir \reference{annexe_bdd_ina} (\reference{bdd_ddcol_ina}).} et la \ac{dj} présentées ici. Leur structure\footnote{Ces structures sont détaillées dans les chapitres consacrés aux alignements des données de l'\ac{ina}.} est différente selon les usages qui ont conduit à leur création, et aux besoins qui en résultent: des notes qualités décrivant la fonction précise des personnes sont présentes dans le lexique des personnes de la \ac{ddcol} alors que seul un domaine d'activité général est conservé à la \ac{dj}. Les systèmes documentaire et juridique de l'\ac{ina} ne sont pas interopérables et n'ont pas été conçus pour l'être: d'un côté, soit l'événement de diffusion est prioritaire, soit l'extrait documentaire l'est; de l'autre, l'information juridique joue ce rôle. Les usages sont tous différents et dirigent le stockage des données dans l'Institut.