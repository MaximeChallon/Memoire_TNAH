\chapter{\label{I-B}Les référentiels à l’INA}
\titreEntete{Les référentiels à l’INA}

%intro: historique INA
\begin{citationLongue}
	Conserver, c’est d’abord faire en sorte que chaque minute audiovisuelle qui passe soit une archive. Si pour le téléspectateur la diffusion d’une émission renvoie au moment présent, pour l’\ac{ina}, il s’agit déjà d’une parcelle de mémoire à conserver\footcite[p.45]{hoog_lina_2006}.
\end{citationLongue}
Historiquement créé avec la scission de l'\ac{ortf} en 1974 pour la collecte des archives audiovisuelles, la recherche et la formation professionnelle, l'\ac{ina} a subi un bouleversement dans les années 1990 avec la loi du 20 juin 1992 sur le dépôt légal audiovisuel instaurant le dépôt légal des radios publiques --- en 1994 ---, des télévisions nationales hertziennes --- en 1995 ---, et de la télévision par câble et numérique et de radios privées --- en 2002, captées directement, permettant alors de s'affranchir du dépôt physique par les producteurs et d'augmenter en masse les données de l'Institut.\\

Défini comme un \ac{epic} dans la loi de création de 1974, l'\ac{ina} a, en plus des missions patrimoniales décrites ci-dessus, des missions commerciales comme la commercialisation des archives ou bien la vente de services auprès de producteurs audiovisuels --- l'\ac{ina} devient alors un tiers-archiveur par le biais de mandats signés avec ces producteurs pour la conservation et/ou la commercialisation de leurs archives. Des compétences juridiques sont ainsi indispensables à l'\ac{ina} pour cette commercialisation et le reversement des droits aux ouvrants- et ayants-droits.\\

Chacune de ces missions a des besoins différents et dirige la structure des données  ainsi que leur usages. La création et l'usage de référentiels sont alors différents selon la mission du département gestionnaire des données: la \ac{ddcol} est en charge de la mission patrimoniale, tandis que la \ac{dj} est en charge des aspects commerciaux et juridiques.

\section{\label{I-B-1}De multiples fonds à décrire}
\titreEntete{De multiples fonds à décrire}


\subsection{\label{I-B-1-a}Les archives professionnelles}
\titreEntete{Les archives professionnelles}

Après l'éclatement de l'\ac{ortf}, des fonds divers deviennent la propriété de l'\ac{ina} et participent à la diversité des fonds audiovisuels conservés à l'\ac{ina}. Ainsi, les \og Actualités françaises\fg{} --- la presse filmée diffusée dans les salles de cinéma --- ont été transférées à l'\ac{ina} en 1974. Les grands moments des débuts de la télévision --- le premier journal télévisé, les premières grandes émissions ou les magazines de reportage --- sont conservés et participent, avant l'instauration du dépôt légal, à la mémoire de l'audiovisuel français; de même pour les fonds radio qui retracent les grands moments historiques depuis les années 1940, et qui deviennent de plus en plus complets avec la généralisation de la bande magnétique à partir du milieu des années 1950.\\

Les archives professionnelles de l'\ac{ina} ne sont que des archives: elles ne couvrent pas l'ensemble de la production audiovisuelle depuis les années 1940 ou la création de l'Institut. Des tranches horaires de diffusion télé- ou radiodiffusées ne sont ainsi pas conservées; peu de traces demeurent alors de la production audiovisuelle, notamment avant l'arrivée du kinéscope. Pour combler ces manques, l'\ac{ina} dispose d'un fonds photographique créé à partir des services de l'\ac{ortf} ou de l'\ac{ina} et portant sur la réalisation des émissions et des tournages.\\

Enfin, des délégations régionales se chargent de la conservation et de la communication des archives télévisées et radiodiffusées des stations régionales --- apparues dans les années 1950: la vie et l'histoire des régions sont ainsi couvertes par l'\ac{ina}.\\

Les fonds d'archives professionnelles de l'\ac{ina} sont conséquents et divers, témoins de la vie et de la société française depuis l'Après-Guerre. Irremplaçables, leur description n'en est pas moins difficile par la diversité des sujets évoqués ou présents dans les documents.

\subsection{\label{I-B-1-b}Les fonds issus du dépôt légal}
\titreEntete{Les fonds issus du dépôt légal}

Depuis la loi sur le dépôt légal de l'audiovisuel de 1992, l'\ac{ina} en est le dépositaire. À partir de 1995, l'\ac{ina} enregistre la globalité de la programmation des stations de Radio-France --- France-Inter, France-Musique, France-Culture, France-Info et France-Bleue ---, enregistrement étendu en 2001 aux stations privées généralistes comme RTL ou NRJ\footnote{En 2020, l'ensemble des stations captées au titre du dépôt légal par l'INA est décrit dans l'\reference{annexe_dl_captation} (\reference{dl_radio})}.\\

Pour les programmes télévisés, le dépôt légal ne concerne d'abord --- entre 1995 et 2001 --- que les sept chaînes principales --- TF1, France 2, France 3, Canal +, M6, Arte, France 5 --- et leurs programmes en première diffusion. La captation directe et intégrale des chaînes n'apparaît qu'en 2002 et est élargie à douze autres chaînes. Enfin, depuis 2005, les chaînes de la Télévision numérique terrestre (TNT) sont toutes captées\footnote{Les chaînes de télévision captées en 2002 pour le dépôt légal sont décrites dans l'\reference{annexe_dl_captation} (\reference{dl_tv})}.

\bigskip
\bigskip

La diversité des fonds d'archives, la captation directe en intégralité des chaînes de télévision et de radio, ainsi que la captation de sites web, plateformes ou comptes de réseaux sociaux au titre du dépôt légal audiovisuel, représentent une masse très importante de données et de documents à conserver. En 2019\footcite[p.5]{institut_national_de_laudiovisuel_rapport_2019}, l'\ac{ina} conserve 20 873 143 heures de programmes de télévision et de radio, dont plus de 18 millions captés par le dépôt légal. 1,2 million de photos s'ajoutent à ces documents. La majorité de ces documents, issus du dépôt légal, sont destinés à une gestion patrimoniale et à une valorisation dans l'INAthèque, alors que les documents des archives professionnelles sont destinées à la valorisation commerciale au travers notamment du site \href{https://www.inamediapro.com}{INAMediaPro} destiné aux professionnels.
\section{\label{I-B-2}Un système documentaire pluriel répondant aux besoins}
\titreEntete{Un système documentaire pluriel}
\section{\label{I-B-3}Multiplication des sources de données et des référentiels}
\titreEntete{Multiplication des sources de données et des référentiels}

De manière à améliorer et enrichir ses données, à faciliter le travail de catalogage, de description et d'indexation, l'\ac{ina} récupère des métadonnées et des données à l'extérieur, auprès de plusieurs fournisseurs. Certains fournisseurs deviennent alors eux-mêmes des référentiels, dont l'identifiant qu'ils fournissent est présent dans les bases de données de l'\ac{ina} aux côtés des données fournies.\\

Ainsi, l'\ac{ina} reçoit des informations concernant les chaînes de provenance, les noms du générique avec les titres, les audiences et le public cible du document\footnote{Voir \reference{annexe_fournisseurs_exterieurs} (\reference{sheldon_mediametrie}).}, ou encore les grilles de diffusion prévisionnelles et réelles. L'ensemble de ces informations permet d'accompagner la tâche de catalogage en fournissant des champs préremplis. Les fournisseurs de ces données sont multiples\footnote{Voir \reference{annexe_fournisseurs_exterieurs} (\reference{enrichissement_dl}).} et fournissent des données tant sur les programmes que sur les producteurs eux-mêmes: 
\begin{itemize}
	\item Les données prévisionnelles de diffusion de la télévision sont achetées auprès de la société Plurimédia\footnote{Voir \url{http://www.plurimedia.fr/}.}. Les fictions, les documentaires, les dessins animés, les émissions de toutes natures, les magazines, \dots sont ainsi décrits au préalable par cette société.
	\item Les données réelles de la diffusion télévisuelle et radio --- date, horaires, parts d'audience, public --- sont fournies par Médiamétrie\footnote{Voir \url{https://www.mediametrie.fr/}.}, en complément des données ---programmation, diffusion, description des contenus --- reçues de la part des diffuseurs eux-mêmes.
	\item Des informations complémentaires sur les programmes sont acquises auprès d'agences de presse comme Kantarmédia\footnote{Voir \url{https://www.kantarmedia.com/fr}.}; pour les producteurs, les informations sont obtenues depuis la société Karl More Productions France.
\end{itemize}

%conclu
\bigskip
\bigskip
\bigskip

La masse des données conservées à l'\ac{ina}, ainsi que l'évolution de la législation concernant les documents à conserver, a créé un système documentaire pluriel, aux bases de données éclatées et aux référentiels internes divers. L'interopérabilité n'est pas recherchée et chaque base de données ou référentiel répond à un besoin spécifique --- conservation, commercialisation, juridique --- d'un département ainsi qu'à un usage particulier --- apporter des informations biographiques sur une personne, retrouver cette personne pour la rémunérer, \dots