\section{\label{I-B-3}Multiplication des sources de données et des référentiels}
\titreEntete{Multiplication des sources de données et des référentiels}

De manière à améliorer et enrichir ses données, à faciliter le travail de catalogage, de description et d'indexation, l'\ac{ina} récupère des métadonnées et des données à l'extérieur, auprès de plusieurs fournisseurs. Certains fournisseurs deviennent alors eux-mêmes des référentiels, dont l'identifiant qu'ils fournissent est présent dans les bases de données de l'\ac{ina} aux côtés des données fournies.\\

Ainsi, l'\ac{ina} reçoit des informations concernant les chaînes de provenance, les noms du générique avec les titres, les audiences et le public cible du document\footnote{Voir \reference{annexe_fournisseurs_exterieurs} (\reference{sheldon_mediametrie}).}, ou encore les grilles de diffusion prévisionnelles et réelles. L'ensemble de ces informations permet d'accompagner la tâche de catalogage en fournissant des champs préremplis. Les fournisseurs de ces données sont multiples\footnote{Voir \reference{annexe_fournisseurs_exterieurs} (\reference{enrichissement_dl}).} et fournissent des données tant sur les programmes que sur les producteurs eux-mêmes: 
\begin{itemize}
	\item Les données prévisionnelles de diffusion de la télévision sont achetées auprès de la société Plurimédia\footnote{Voir \url{http://www.plurimedia.fr/}.}. Les fictions, les documentaires, les dessins animés, les émissions de toutes natures, les magazines, \dots sont ainsi décrits au préalable par cette société.
	\item Les données réelles de la diffusion télévisuelle et radio --- date, horaires, parts d'audience, public --- sont fournies par Médiamétrie\footnote{Voir \url{https://www.mediametrie.fr/}.}, en complément des données ---programmation, diffusion, description des contenus --- reçues de la part des diffuseurs eux-mêmes.
	\item Des informations complémentaires sur les programmes sont acquises auprès d'agences de presse comme Kantarmédia\footnote{Voir \url{https://www.kantarmedia.com/fr}.}; pour les producteurs, les informations sont obtenues depuis la société Karl More Productions France.
\end{itemize}