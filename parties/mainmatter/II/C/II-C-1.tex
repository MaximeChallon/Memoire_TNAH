\section{\label{II-C-1}Effectuer des requêtes sur Wikidata}
\titreEntete{Effectuer des requêtes sur Wikidata}

%intro
Wikidata a son propre modèle de données, mais les données sont représentables et exportables en \ac{rdf}. Ainsi, Wikidata est totalement intégré dans le Web sémantique. Le langage de requête de \ac{rdf}, \ac{sparql}, doit permettre un accès rapide aux données par des requêtes parfois complexes. Cependant, d'autres méthodes existent pour accéder aux données, comme une \ac{api} fournie par \href{https://www.mediawiki.org/wiki/Wikibase/API}{Wikibase} via l'URL \url{https://wikidata.org/w/api.php}.\\

L'alignement des personnes physiques de l'\ac{ina} a, pour des raisons expliquées plus bas dans le propos\footnote{Voir \reference{II-C-4}.}, nécessité l'utilisation de ces deux méthodes afin d'améliorer les délais de récupération des donnée, et par conséquent d'alignement.

\subsection{\label{II-C-1-a}La structure des données de Wikidata}
\titreEntete{La structure des données de Wikidata}

La compréhension de la structure des données de Wikidata, particulière, est nécessaire pour effectuer des requêtes précises et efficaces. Bien que proche de \ac{rdf}, cette structure comporte quelques particularités\footnote{Voir \reference{annexe_nvx_modeles} (\reference{wikidata}).}. L'affirmation, la structure de base de Wikidata, se compose d'un élément suivi d'une paire propriété--valeur nommée déclaration. Cette affirmation est très proche, sinon identique, au triplet \ac{rdf} sujet-prédicat-objet. Un élément peut contenir autant de déclarations que nécessaire et est identifié par une URI unique --- comme l'est une instance \ac{rdf} --- désignant une page Wikidata commençant par la lettre Q (la lettre P désigne les propriétés).\\

Wikidata peut, en plus de ces données liées semblables à \ac{rdf}, contenir des informations plus complexes, complexifiant par conséquent la représentation des informations en \ac{rdf} et les requêtes associées. Ainsi, les déclarations ne sont pas seulement composées de la paire propriété--valeur. Des qualificatifs et des références peuvent y être ajoutés de manière à préciser la paire, eux-mêmes étant sous la forme d'une paire propriété-valeur et constituant un triplet \ac{rdf} valeur d'une propriété--propriété--valeur.\\

De plus, de même que les ontologies, des axiomes peuvent être introduits pour les propriétés: une propriété peut n'accepter qu'une seule valeur, ou bien une multiplicité de valeurs; une propriété peut accepter uniquement des éléments identifiés par une URI et non des littéraux, \dots~\\

Enfin, Wikidata propose au début de chaque page le label préférentiel de l'élément, une description sommaire, ainsi que des labels alternatifs, pour chaque langue. De manière à mettre ces données dans son modèle de données, les ontologies telles que \ac{rdfs}, \ac{skos} ou \href{http://wikiba.se/ontology#}{wikibase}\footnote{Cette ontologie est construite à partir de \ac{owl} et \ac{rdfs}.} sont utilisées et permettent un accès facilité aux données avec \ac{rdf}.

\subsection{\label{II-C-1-b}Le \ac{sparql}-EndPoint}
\titreEntete{Le SPARQL-EndPoint}

Un premier accès aux données de Wikidata peut se faire par le \ac{sparql}-EndPoint. \ac{sparql} est le langage de requête de données \ac{rdf} et permet d'effectuer des requêtes complexes en parcourant les triplets. Deux entrées dans ce \ac{sparql}-EndPoint sont possibles:
\begin{itemize}
	\item Une interface\footnote{Disponible à l'URL \url{https://query.wikidata.org/}} est disponible pour y écrire les requêtes puis les exécuter. Une prévisualisation des résultats apparaît au bas de l'interface, et il est possible d'exporter les données en plusieurs formats (Comma Separated Values (CSV), \ac{json}, \ac{rdf}, \dots).
	\item Un accès direct aux données est possible en construisant des URLs de requête avec des paramètres: la requête \ac{sparql} est à insérer après \textit{https://query.wikidata.org/sparql?query=}; ensuite, le paramètre \og format\fg{} permet la spécification du format de sortie (\textit{\&format=json} par exemple)
\end{itemize}
\medskip

L'utilisation du \ac{sparql}-EndPoint est adaptée aux requêtes générées humainement. En effet, plus la requête est complexe, plus le temps de réponse est élevé. Il est très fréquent d'obtenir un \textit{timeout} --- il peut être du à la complexité de la requête, mais également à la charge des ressources du service de Wikidata ou du réseau Internet ---, ce qui empêche l'utilisation de \ac{sparql} dans un processus d'alignement automatique.\\

S'ajoutent à ces limites d'autres limites imposées par Wikidata\footnote{Consultables dans la documentation: \url{https://www.mediawiki.org/wiki/Wikidata_Query_Service/User_Manual/fr\#Limites_des_requ\%C3\%AAtes}}:
\begin{itemize}
	\item limitation à cinq requêtes simultanées par adresse IP;
	\item soixante secondes de traitement toutes les soixante secondes;
	\item cette limitation de temps induit des erreurs qui sont limitées à trente par minute; passée cette limitation à trente erreurs par minute --- qui est atteinte rapidement avec un traitement automatique ---, l'adresse IP est bloquée
\end{itemize}

\subsection{\label{II-C-1-c}L'\ac{api} Wikibase}
\titreEntete{L'API Wikibase}

Face à ces nombreuses limites empêchant une utilisation du \ac{sparql}-EndPoint dans l'alignement de données avec Wikidata, l'\ac{api} permet de meilleurs temps de réponse et l'absence d'erreurs --- exceptées celles générées pour les requêtes ne renvoyant aucun résultat\footnote{L'outil Open Refine utilise cette \ac{api} et non le \ac{sparql}-EndPoint. Voir le code source du logiciel: \url{https://github.com/OpenRefine/OpenRefine}.}. Cette \ac{api} de Wikibase est uniquement disponible par la méthode GET d'URLs. Elle est représentée par un ensemble de modules qui permettent, sans avoir à écrire de requête \ac{sparql}, de filtrer les résultats de la recherche.\\

L'accès se fait par l'URL \url{https://wikidata.org/w/api.php}. À cette URL peuvent être ajoutés des paramètres\footnote{Les paramètres et les modules présentés ci-dessous sont uniquement ceux utilisés dans le traitement avec Talend pour l'alignement des personnes physiques avec Wikidata.}:
\begin{itemize}
	\item \textit{action}: permet d'indiquer le module de l'\ac{api} Wikibase utilisé;
	\item \textit{format}: permet de choisir le format de sortie des résultats (JSON ou XML)
\end{itemize}
\medskip

Trois modules ont été utilisés pour réaliser cet alignement:
\begin{itemize}
	\item \textit{wbsearchentities} permet de rechercher la présence d'une chaîne de caractères dans les libellés et les libellés alternatifs des entités de Wikidata; plusieurs paramètres sont alors disponibles:
	\begin{table}
		\centering
		\begin{tabularx}{15cm}{|X|X|X|}
			\hline
			\begin{center}Paramètre\end{center}&\begin{center}Fonction\end{center}&\begin{center}Particularités\end{center}  \tabularnewline \hline
			\textit{search}&Contient la chaîne de caractères à rechercher&Obligatoire\tabularnewline \hline
			\textit{language}&Spécifie la langue du texte à rechercher&Obligatoire\tabularnewline \hline
			\textit{limit}&Nombre de résultats&\tabularnewline \hline
		\end{tabularx}
	\caption[Paramètres principaux du module \textit{wbsearchentities} de l'\ac{api} Wikibase]{Paramètres principaux du module \textit{wbsearchentities} de l'\ac{api} Wikibase [Source: \url{https://www.wikidata.org/w/api.php?action=help\&modules=wbsearchentities]}}
	\label{wbsearchentities}
	\end{table}
	\item \textit{wbgetentities} permet d'obtenir tout ou partie des entités de Wikidata
	\begin{table}
		\centering
		\begin{tabularx}{15cm}{|X|X|X|}
			\hline
			\begin{center}Paramètre\end{center}&\begin{center}Fonction\end{center}&\begin{center}Particularités\end{center}  \tabularnewline \hline
			\textit{ids}&Liste des entités à récupérer&\tabularnewline \hline
			\textit{props}&Type de données à récupérer&Sont utilisés les libellés, les déclarations et les libellés alternatifs\tabularnewline \hline
			\textit{languages}&Spécifie la langue du texte à rechercher&Obligatoire\tabularnewline \hline
		\end{tabularx}
		\caption[Paramètres principaux du module \textit{wbgetentities} de l'\ac{api} Wikibase]{Paramètres principaux du module \textit{wbgetentities} de l'\ac{api} Wikibase [Source: \url{https://www.wikidata.org/w/api.php?action=help\&modules=wbgetentities]}}
		\label{wbgetentities}
	\end{table} 
	\item \textit{wbgetclaims} permet d'obtenir l'ensemble des déclarations de l'entité demandée
	\begin{table}
		\centering
		\begin{tabularx}{15cm}{|X|X|X|}
			\hline
			\begin{center}Paramètre\end{center}&\begin{center}Fonction\end{center}&\begin{center}Particularités\end{center}  \tabularnewline \hline
			\textit{entity}&Identifiant de l'entité&Obligatoire \tabularnewline \hline
		\end{tabularx}
		\caption[Paramètres principaux du module \textit{wbgetclaims} de l'\ac{api} Wikibase]{Paramètres principaux du module \textit{wbgetclaims} de l'\ac{api} Wikibase [Source: \url{https://www.wikidata.org/w/api.php?action=help\&modules=wbgetclaims]}}
		\label{wbgetclaims}
	\end{table} 
\end{itemize}
\medskip

La possibilité de lancer plusieurs requêtes \ac{http} simultanément --- en raison de l'absence de vérification de l'adresse IP --- et d'y insérer plusieurs entités --- jusqu'à cinquante ---, quand le paramètre \textit{ids} est possible, rend l'\ac{api} Wikibase efficace et plus adaptée à un alignement automatique d'un jeu de données avec Wikidata.

%conclu
\bigskip
\bigskip
Plusieurs points d'accès vers Wikidata sont possibles: l'un peut être utilisé pour des requêtes uniques et complexes, l'autre est utilisé en cas d'automatisation d'un processus. Ainsi, nous le verrons par la suite, un alignement de données avec Wikidata peut nécessiter plusieurs appels à l'\ac{api} avant de pouvoir choisir l'entité correspondante, en étant plus rapide que les requêtes effectuées avec le \ac{sparql}-EndPoint. 