\section{\label{II-C-4}Comprendre les limites}
\titreEntete{Comprendre les limites}

%intro
Le lien offert par l'identifiant \index[ref]{lod@Linked Open Data (LOD)!wikidata@Wikidata}Wikidata est précieux grâce aux nombreuses informations et données qu'il peut fournir. Seulement,  pour obtenir ce lien automatiquement, un alignement doit être réalisé grâce à des comparaisons de données qui se complexifient en fonction de la nature des données, de leur structure et surtout de leur similarité de part et d'autre de l'alignement.\\

Ces limites, associées à celles du Web sémantique et de Wikidata, réduisent  l'efficacité de cet alignement: l'\ac{ina} possède environ 330000 personnes physiques qui n'ont pas d'identifiant Wikidata associé; l'enjeu de cet alignement est donc important. Cependant, seuls 31000 identifiants de Wikidata ont pu être récupérés: 90\% des personnes physiques n'ont, par conséquent, pas pu être alignées. 

\subsection{\label{II-C-4-a}Les raisons de l'absence d'alignement}
\titreEntete{Les raisons de l'absence d'alignement}

Les raisons de l'absence d'alignement des personnes physiques de l'\ac{ina} avec \index[ref]{lod@Linked Open Data (LOD)!wikidata@Wikidata}Wikidata sont multiples et diverses, et principalement liées à la donnée elle-même, dans sa forme, sa graphie ou sa structure.\\

D'abord, de nombreuses personnes physiques de l'\ac{ina} ont des données éparses, sans dates de naissance ou de décès, sans genre. Quand la fonction indiquée n'est pas réellement une fonction, mais une notion d'appartenance à un événement ou à une famille, cette personne se retrouve sans point de comparaison possible. Ainsi, \nP{Abdesslam}{Guerouaz} ne dispose ni de dates ni de fonction utilisables dans l'alignement --- la note qualité est \og Attentat, Maroc 1994.Maroc\fg{}. Cette appartenance à un événement n'est pas une valeur correspondant à la propriété \textit{Occupation} (P106) servant aux comparaisons sur les fonctions des notes qualité.\\

Ensuite, la divergence des états civils est une autre source d'empêchement des alignements. Si les noms et prénoms français --- qui ne sont, par conséquent, traduits ni à l'\ac{ina} ni sur Wikidata --- ne sont pas massivement concernés, ceux d'origines étrangères le sont car les traductions varient selon le traducteur et les normes de catalogage et de remplissage. C'est pourquoi \nP{Tomasz}{Popakul} du jeu de données de l'\ac{ina}, \textit{réalisateur} selon la note qualité, ne peut pas être aligné avec son entité Wikidata Q19269951. En effet, le libellé de Wikidata est \nP{Tomek}{Popakul}. Malgré la concordance de la fonction de la note qualité avec la valeur de la propriété P106, cette personne n'a pas pu être alignée en raison de différences de traduction de l'état civil.\\

L'alignement des fonctions est également l'une des raisons de l'échec des alignements de personnes. En effet, comme constaté au \reference{I-C}, les premiers termes des notes qualité peuvent être génériques et bloquer les traitements. Ainsi, \nP{Cameron}{Alborzian}, \og Ancien mannequin, maître de yoga\fg dans la note qualité, ne peut pas être aligné avec son entité Wikidata Q5026166 en raison de l'échec de l'alignement de la fonction. La propriété P106 de Q5026166 comporte la valeur \textit{Mannequin} (Q4610556) alors que la fonction de l'\ac{ina} est \textit{Ancien mannequin}.\\

Cependant, la majorité des personnes physiques non alignées avec Wikidata concerne l'absence de ces personnes dans \index[ref]{lod@Linked Open Data (LOD)!wikidata@Wikidata}Wikidata: l'\ac{ina} documente chaque personne de générique, du scripte au cadreur, réalisateur, \dots~ Ces personnes, peu ou pas connues, ne font, par conséquent, pas l'objet d'une entité sur Wikidata.

\subsection{\label{II-C-4-b}Les limites du Web sémantique}
\titreEntete{Les limites du Web sémantique}

Les limites imposées par le Web sémantique compliquent également le processus d'alignement entre deux jeux de données. Les performances de \index[ref]{echanges@Échanges!protocoles@Protocoles!sparql@SPARQL}\ac{sparql}, et plus généralement de \index[ref]{echanges@Échanges!formats@Formats!rdf@RDF}\ac{rdf}, sont difficilement compatibles avec des requêtes de masse et complexes. Les temps de réponse augmentent considérablement dès lors qu'un appel au service de langage de \index[ref]{relier@Relier!wikibase@Wikibase}Wikibase est effectué, que trop de clauses optionnelles ou obligatoires sont introduites, ou bien que le nombre de résultats est trop important. Ainsi, pour éviter ces temps de réponses trop longs, \index[ref]{lod@Linked Open Data (LOD)!wikidata@Wikidata}Wikidata a introduit une limite à soixante secondes. La présence de cette limite contraint à ne pas utiliser le \ac{sparql}-EndPoint dans le cadre d'un processus automatique d'alignement de données, et à se tourner vers une autre solution, l'\ac{api} Wikibase qui offre des performances supérieures, sans retour d'erreurs liées aux temps de réponses.\\

La rapidité d'un service sur le Web est essentielle et fait partie, comme le rappelle \nP{Gautier}{Poupeau} dans son billet de blog\footcite{poupeau_au-a_2018}, des axes indispensables du Big Data\footnote{\og Parmi les trois axes qui définissent traditionnellement le Big Data, vitesse, volume et variété (les \og 3V\fg{}), les deux premières caractéristiques ne sont pas encore atteintes par ces technologies\fg{} in \cite{poupeau_au-a_2018}}. Ainsi, la cinquième étoile de la classification de \nP{Tim}{Berners-Lee}, composée de \index[ref]{echanges@Échanges!formats@Formats!rdf@RDF}\ac{rdf}, paraît difficilement atteignable et utilisable universellement: l'\ac{api} Wikibase renvoie les résultats sous forme de \index[ref]{echanges@Échanges!formats@Formats!json@JSON}\ac{json} ou de \index[ref]{echanges@Échanges!formats@Formats!xml@XML}XML, et non en \ac{rdf}. Une démonstration des difficultés de \ac{rdf} pour s'imposer avec son langage de requête \ac{sparql} est son utilisation et sa maîtrise par peu de développeurs\footnote{\og il est à craindre que les technologies du Web sémantique restent des technologies de niche maîtrisées par peu de développeurs\fg{} in \cite{poupeau_au-a_2018}}; cependant, le modèle en\index[ref]{typologie@Typologie!graphe@Graphe de nœuds et de liens} graphe développé à partir de la réflexion de \nP{Tim}{Berners-Lee} n'est pas remis en cause et est de plus en plus utilisé.\\

Enfin, le Web sémantique et \index[ref]{lod@Linked Open Data (LOD)!wikidata@Wikidata}Wikidata se placent dans une vision d'interopérabilité à partir des données dont nous avons développé les avantages avec la disparition de la notion de référentiel. Cette intéropérabilité, théoriquement, doit permettre de s'affranchir des règles du langage naturel et de proposer des données partageables et réutilisables par tous. Cependant, ces données devront toujours subir des traitements et des transformations pour pouvoir être intégrées dans des systèmes documentaires ou des jeux de données. La normalisation des libellés et des dates des entités de Wikidata en est un exemple, de manière à pouvoir être mis en concordance avec d'autres données, celles de l'\ac{ina}.

%conclu
\bigskip
\bigskip
Face aux possibilités d'obtenir un enrichissement de ses propres données à partir de jeux de données externes et publics comme Wikidata, les institutions patrimoniales sont de plus en plus nombreuses à utiliser le Web de données. Mais, au-delà de ces avantages, les limites et les difficultés sont réelles: faut-il alors bâtir un modèle de données, ou un référentiel, internes sur un modèle du Web de données? Est-il possible d'utiliser un jeu de données externe comme référentiel interne, adapté aux besoins et aux usages spécifiques à une institution?\\

Le traitement de la donnée sera toujours nécessaire pour pouvoir faire correspondre cette donnée aux usages de chacun selon des besoins spécifiques. Bien que Wikidata ou d'autres jeux de données du Linked Open Data soient complets et se veulent universels dans leur domaine, les besoins et les usages sont tous différents; ainsi, les normes et les conventions de rédaction des données varient selon ces derniers dans chaque institution, rendant impossible l'écriture du dictionnaire universel dans lequel seront recherchées les données nécessaires.