\chapter{\label{II-C}Relier ses données à Wikidata: l'exemple de l'alignement des personnes physiques de l'\ac{ina}}
\titreEntete{Relier ses données à Wikidata}

%intro
\lettrine{S}i certaines institutions patrimoniales ont fait le choix d'exposer l'ensemble de leur données sur le Web avec le Web de données, certaines ne font pas ce choix. C'est le cas de l'\ac{ina}. En effet, l'Institut se concentre sur la mise en cohérence et la centralisation de ses données, tout en les enrichissant le plus possible. Dans ce but d'enrichissement des données, l'\ac{ina}, nous l'avons évoqué au \reference{I-B}, récupère des données complémentaires auprès de sociétés et créé ainsi du lien avec ces dernières par la conservation dans les bases de données des identifiants propres à chaque document et à chaque société. Cependant, les données fournies sont spécifiques à un domaine d'activité et restreintes. Par conséquent, l'utilisation du Web de données, avec des données ouvertes et accessibles à tous, permet la liaison avec plusieurs jeux de données et la récupération des informations manquantes.\\

\href{https://www.wikidata.org/}{Wikidata} est une base de connaissances collaborative \ac{rdf} concentrant le plus de données et de connaissances sur le monde. L'\ac{ina} utilise ainsi les identifiants de cette base comme liens vers les données qui y sont stockées\footnote{Le \reference{III-A} évoque en détail les possibilités offertes par Wikidata dans le parcours des liens et du graphe.}. Ces identifiants peuvent être ajoutés dès l'opération de catalogage des documents audiovisuels; ou bien ajoutés \textit{a posteriori} par un alignement.\\

Pour réaliser cet alignement entre des données institutionnelles et un jeu de données externes du Web sémantique, plusieurs outils existent:
\begin{itemize}
	\item Des logiciels comme \href{https://openrefine.org/}{Open Refine}\footnote{Ce logiciel est notamment utilisé par les archives Nationales pour aligner les personnes de la \href{http://www2.culture.gouv.fr/documentation/leonore/}{base Léonore de la Légion d'Honneur} avec Wikidata.} permettent d'aligner des données avec Wikidata en un clic sur l'interface; seulement, ce type de logiciel est peu personnalisable dans la constitution des requêtes effectuées.
	\item Des logiciels ETL permettant de créer des chaînes de récupération et de traitement des données, peu importe le lieu et le type de leur stockage. \href{https://www.talend.com/fr/}{Talend} est ainsi utilisé par l'\ac{ina}.
\end{itemize}
\medskip

De même que l'alignement entre les personnes physiques et le thésaurus de noms communs de l'\ac{ina}\footnote{Voir \reference{I-B}.}, celui-ci présente de nombreuses difficultés liées au langage naturel de chaque jeu de données. Plusieurs points de comparaison doivent alors être utilisés, ce qui multiplie les alignements avec Wikidata. De plus, ce type d'alignement révèle de multiples limites, tant techniques que linguistiques.

%input
\section{\label{II-C-1}Effectuer des requêtes sur Wikidata}
\titreEntete{Effectuer des requêtes sur Wikidata}

%intro
Wikidata a son propre modèle de données, mais les données sont représentables et exportables en \ac{rdf}. Ainsi, Wikidata est totalement intégré dans le Web sémantique. Le langage de requête de \ac{rdf}, \ac{sparql}, doit permettre un accès rapide aux données par des requêtes parfois complexes. Cependant, d'autres méthodes existent pour accéder aux données, comme une \ac{api} fournie par \href{https://www.mediawiki.org/wiki/Wikibase/API}{Wikibase} via l'URL \url{https://wikidata.org/w/api.php}.\\

L'alignement des personnes physiques de l'\ac{ina} a nécessité, pour des raisons expliquées plus bas dans le propos\footnote{Voir \reference{II-C-4}.}, l'utilisation de ces deux méthodes afin d'améliorer les délais de récupération des données, et par conséquent d'alignement.

\subsection{\label{II-C-1-a}La structure des données de Wikidata}
\titreEntete{La structure des données de Wikidata}

La compréhension de la structure des données de \index[ref]{lod@Linked Open Data (LOD)!wikidata@Wikidata}Wikidata, particulière, est nécessaire pour effectuer des requêtes précises et efficaces. Bien que proche de \index[ref]{echanges@Échanges!formats@Formats!rdf@RDF}\ac{rdf}, cette structure comporte quelques particularités\footnote{Voir \reference{annexe_nvx_modeles} (\reference{wikidata}).}. L'affirmation, la structure de base de Wikidata, se compose d'un élément suivi d'une paire propriété--valeur nommée déclaration. Cette affirmation est très proche, sinon identique, au triplet \ac{rdf} sujet-prédicat-objet. Un élément peut contenir autant de déclarations que nécessaire et est identifié par une URI unique --- comme l'est une instance \ac{rdf} --- désignant une page Wikidata commençant par la lettre \textit{Q} (la lettre \textit{P} désigne les propriétés).\\

Wikidata peut, en plus de ces données liées semblables à \ac{rdf}, contenir des informations plus complexes, compliquant par conséquent la représentation des informations en \ac{rdf} et les requêtes associées. Ainsi, les déclarations ne sont pas seulement composées de la paire propriété--valeur. Des qualificatifs et des références peuvent y être ajoutés de manière à préciser la paire, eux-mêmes étant sous la forme d'une paire propriété--valeur et constituant un triplet \ac{rdf} valeur d'une propriété.\\

De plus, comme pour les ontologies, des axiomes peuvent être introduits pour les propriétés: une propriété peut n'accepter qu'une seule valeur, ou bien une multiplicité de valeurs; une propriété peut accepter que des éléments identifiés par une URI et non des littéraux, \dots~\\

Enfin, \index[ref]{lod@Linked Open Data (LOD)!wikidata@Wikidata}Wikidata propose au début de chaque page le label préférentiel de l'élément, une description sommaire, ainsi que pour chaque langue des labels alternatifs. De manière à mettre ces données dans son modèle de données, des ontologies telles que \index[ref]{relier@Relier!rdfs@RDFS}\ac{rdfs}, \index[ref]{relier@Relier!skos@SKOS}\ac{skos} ou \index[ref]{relier@Relier!wikibase@Wikibase}\href{http://wikiba.se/ontology#}{wikibase}\footnote{Cette ontologie est construite à partir de \ac{owl} et \ac{rdfs}.} sont utilisées et permettent un accès facilité aux données avec \index[ref]{echanges@Échanges!formats@Formats!rdf@RDF}\ac{rdf}.

\subsection{\label{II-C-1-b}Le \ac{sparql}-EndPoint}
\titreEntete{Le SPARQL-EndPoint}

Un premier accès aux données de \index[ref]{lod@Linked Open Data (LOD)!wikidata@Wikidata}Wikidata peut se faire par le \index[ref]{echanges@Échanges!protocoles@Protocoles!sparql@SPARQL}\ac{sparql}-EndPoint. \ac{sparql} est le langage de requête de données \ac{rdf} et permet d'effectuer des requêtes complexes en parcourant les triplets. Deux entrées dans ce \ac{sparql}-EndPoint sont possibles:
\begin{itemize}
	\item Une interface\footnote{Disponible à l'URL \url{https://query.wikidata.org/}} est disponible pour y écrire les requêtes puis les exécuter. Une prévisualisation des résultats apparaît au bas de l'interface, et il est possible d'exporter les données en plusieurs formats (\index[ref]{echanges@Échanges!formats@Formats!csv@CSV}Comma Separated Values (CSV), \index[ref]{echanges@Échanges!formats@Formats!json@JSON}\ac{json}, \index[ref]{echanges@Échanges!formats@Formats!rdf@RDF}\ac{rdf}, \dots).
	\item Un accès direct aux données est possible en construisant des \index[ref]{echanges@Échanges!protocoles@Protocoles!http@HTTP}URLs de requête avec des paramètres: la requête \ac{sparql} est à insérer après \url{https://query.wikidata.org/sparql?query=}; ensuite, le paramètre \og format\fg{} permet la spécification du format de sortie (\textit{\&format=json} par exemple)
\end{itemize}
\medskip

L'utilisation du \ac{sparql}-EndPoint est adaptée aux requêtes générées humainement. En effet, plus la requête est complexe, plus le temps de réponse est élevé. Il est très fréquent d'obtenir un \textit{timeout} --- il peut être dû à la complexité de la requête, mais également à la charge des ressources du service de Wikidata ou du réseau Internet ---, ce qui empêche l'utilisation de \ac{sparql} dans un processus d'alignement automatique.\\

S'ajoutent à ces limites d'autres limites imposées par \index[ref]{echanges@Échanges!protocoles@Protocoles!sparql@SPARQL}\index[ref]{lod@Linked Open Data (LOD)!wikidata@Wikidata}Wikidata\footnote{Consultables dans la documentation: \url{https://www.mediawiki.org/wiki/Wikidata_Query_Service/User_Manual/fr\#Limites_des_requ\%C3\%AAtes}}:
\begin{itemize}
	\item limitation à cinq requêtes simultanées par adresse IP;
	\item soixante secondes de traitement toutes les soixante secondes;
	\item cette limitation de temps induit des erreurs qui sont fixées à trente par minute; si ce plafond de trente erreurs par minute est atteint et dépassé --- ce qui est atteint rapidement avec un traitement automatique ---, l'adresse IP est bloquée
\end{itemize}

\subsection{\label{II-C-1-c}L'\ac{api} Wikibase}
\titreEntete{L'API Wikibase}

Face à ces nombreuses limites empêchant une utilisation du \ac{sparql}-EndPoint dans l'alignement de données avec \index[ref]{lod@Linked Open Data (LOD)!wikidata@Wikidata}Wikidata, l'\ac{api} permet de meilleurs temps de réponse et l'absence d'erreurs --- exceptées celles générées pour les requêtes ne renvoyant aucun résultat\footnote{L'outil Open Refine utilise cette \ac{api} et non le \ac{sparql}-EndPoint. Voir le code source du logiciel: \url{https://github.com/OpenRefine/OpenRefine}.}. Cette \ac{api} de Wikibase est uniquement disponible par la méthode \index[ref]{echanges@Échanges!protocoles@Protocoles!http@HTTP}GET d'URLs. Elle est représentée par un ensemble de modules qui permettent, sans avoir à écrire de requête \index[ref]{echanges@Échanges!protocoles@Protocoles!sparql@SPARQL}\ac{sparql}, de filtrer les résultats de la recherche.\\

L'accès se fait par \index[ref]{echanges@Échanges!protocoles@Protocoles!http@HTTP}l'URL \url{https://wikidata.org/w/api.php}. À cette URL peuvent être ajoutés des paramètres\footnote{Les paramètres et les modules présentés ci-dessous sont uniquement ceux utilisés dans le traitement avec Talend pour l'alignement des personnes physiques avec Wikidata.}:
\begin{itemize}
	\item \textit{action}: permet d'indiquer le module de l'\ac{api} Wikibase utilisé;
	\item \textit{format}: permet de choisir le format de sortie des résultats (\index[ref]{echanges@Échanges!formats@Formats!json@JSON}\ac{json} ou \index[ref]{echanges@Échanges!formats@Formats!xml@XML}XML)
\end{itemize}
\medskip

Trois modules ont été utilisés pour réaliser cet alignement:
\begin{itemize}
	\item \textit{wbsearchentities} permet de rechercher la présence d'une chaîne de caractères dans les libellés et les libellés alternatifs des entités de \index[ref]{lod@Linked Open Data (LOD)!wikidata@Wikidata}Wikidata; plusieurs paramètres sont alors disponibles(\reference{wbsearchentities}).
	\item \textit{wbgetentities} permet d'obtenir tout ou partie des entités de Wikidata (\reference{wbgetentities}).
	\item \textit{wbgetclaims} permet d'obtenir l'ensemble des déclarations de l'entité demandée (\reference{wbgetclaims}).
\end{itemize}

\begin{table}[h!]
	\centering
	\begin{tabularx}{15cm}{|X|X|X|}
		\hline
		\begin{center}Paramètre\end{center}&\begin{center}Fonction\end{center}&\begin{center}Particularités\end{center}  \tabularnewline \hline
		\textit{search}&Contient la chaîne de caractères à rechercher&Obligatoire\tabularnewline \hline
		\textit{language}&Spécifie la langue du texte à rechercher&Obligatoire\tabularnewline \hline
		\textit{limit}&Nombre de résultats&\tabularnewline \hline
	\end{tabularx}
	\caption[Paramètres principaux du module \textit{wbsearchentities} de l'\ac{api} Wikibase]{Paramètres principaux du module \textit{wbsearchentities} de l'\ac{api} Wikibase [Source: \url{https://www.wikidata.org/w/api.php?action=help\&modules=wbsearchentities]}}
	\label{wbsearchentities}
\end{table}


\begin{table}[h!]
	\centering
	\begin{tabularx}{15cm}{|X|X|X|}
		\hline
		\begin{center}Paramètre\end{center}&\begin{center}Fonction\end{center}&\begin{center}Particularités\end{center}  \tabularnewline \hline
		\textit{ids}&Liste des entités à récupérer&\tabularnewline \hline
		\textit{props}&Type de données à récupérer&Sont utilisés les libellés, les déclarations et les libellés alternatifs\tabularnewline \hline
		\textit{languages}&Spécifie la langue du texte à rechercher&Obligatoire\tabularnewline \hline
		\textit{titles}&Titre de la page correspondante&\tabularnewline \hline
		\textit{sites}&Identifiant du site dans lequel rechercher&\tabularnewline \hline
	\end{tabularx}
	\caption[Paramètres principaux du module \textit{wbgetentities} de l'\ac{api} Wikibase]{Paramètres principaux du module \textit{wbgetentities} de l'\ac{api} Wikibase [Source: \url{https://www.wikidata.org/w/api.php?action=help\&modules=wbgetentities]}}
	\label{wbgetentities}
\end{table} 


\begin{table}[h!]
	\centering
	\begin{tabularx}{15cm}{|X|X|X|}
		\hline
		\begin{center}Paramètre\end{center}&\begin{center}Fonction\end{center}&\begin{center}Particularités\end{center}  \tabularnewline \hline
		\textit{entity}&Identifiant de l'entité&Obligatoire \tabularnewline \hline
	\end{tabularx}
	\caption[Paramètres principaux du module \textit{wbgetclaims} de l'\ac{api} Wikibase]{Paramètres principaux du module \textit{wbgetclaims} de l'\ac{api} Wikibase [Source: \url{https://www.wikidata.org/w/api.php?action=help\&modules=wbgetclaims]}}
	\label{wbgetclaims}
\end{table} 
\medskip

La possibilité de lancer plusieurs requêtes \index[ref]{echanges@Échanges!protocoles@Protocoles!http@HTTP}\ac{http} simultanément --- en raison de l'absence de vérification de l'adresse IP --- et d'y insérer plusieurs entités --- jusqu'à cinquante --- quand le paramètre \textit{ids} est disponible, rend l'\ac{api} Wikibase efficace et plus adaptée à un alignement automatique d'un jeu de données avec \index[ref]{lod@Linked Open Data (LOD)!wikidata@Wikidata}Wikidata.

%conclu
\bigskip
\bigskip
Plusieurs points d'accès vers Wikidata sont possibles: l'un peut être utilisé pour des requêtes uniques et complexes, l'autre est utilisé en cas d'automatisation d'un processus. Ainsi, nous le verrons par la suite, un alignement de données avec Wikidata peut nécessiter plusieurs appels à l'\ac{api} avant de pouvoir choisir l'entité correspondante, en étant plus rapide que les requêtes effectuées avec le \ac{sparql}-EndPoint. 
\section{\label{II-C-2}Aligner des personnes depuis des données contrôlées}
\titreEntete{Aligner des personnes depuis des données contrôlées}

%intro
L'\ac{ina} conserve ses données sous deux formes: des données contrôlées, et des données en texte libre\footnote{Voir \reference{annexe_type_donnees_axel} (\reference{type_donnees_axel}).}. Les premières permettent une meilleure exploitation et des traitements plus faciles, notamment lors d'un alignement avec un référentiel ou un autre jeu de données\footnote{L'alignement des notes qualité de l'\ac{ina} avec un thésaurus interne a déjà démontré la difficulté de l'utilisation du texte libre.}. Cependant, le contrôle du langage naturel se fait différemment selon le contexte de création et d'utilisation des données. Ainsi, deux jeux de données contrôlés peuvent avoir pour un même concept deux graphies et deux normes différentes, tout en respectant de chaque côté un contrôle du langage qui leur est propre.\\

Bien que les personnes physiques décrites à l'\ac{ina} respectent des règles de structure, et le soient dans plusieurs attributs d'une table de base de données, elles ne correspondent pas exactement aux entités de Wikidata qui sont créées selon d'autres règles de graphie. Ainsi, quel que soit le jeu de données, il est toujours nécessaire de lui apporter un premier traitement afin de trouver le lien vers un autre jeu de données.

\subsection{\label{II-C-2-a}Choix des déclarations des entités de Wikidata}
\titreEntete{Choix des déclarations des entités de Wikidata}

À partir des mêmes personnes physiques de la \reference{I-C-3}, il est nécessaire de créer du lien avec \index[ref]{lod@Linked Open Data (LOD)!wikidata@Wikidata}Wikidata de manière à les enrichir. Plusieurs informations sont ainsi disponibles pour chaque personne: les dates de naissance et de mort, et la note qualité\footnote{Voir \reference{table_roberts_1}.}.
\begin{table}[h!]
	\centering
	\csvautotabular[separator=semicolon]{images/ex_II_C_2.csv}
	\caption{Informations disponibles pour Howard Roberts à l'\ac{ina}}
	\label{table_roberts_1}
\end{table}
\medskip

%propriété Wiki correspondantes
L'entité\footnote{Howard Roberts: \url{https://www.wikidata.org/wiki/Q1631895}} Howard Roberts comporte plusieurs dizaines de déclarations --- paire propriété--valeur. Cependant, la présence de ces déclarations n'est pas uniforme selon les entités de personnes. De manière à aligner les entités Wikidata avec un jeu de données, il est nécessaire de chercher les propriétés les plus communes et le plus souvent présentes, pouvant correspondre à chacune des informations disponibles dans le jeu de données à aligner.\\

Ce choix, pour les dates de naissance ou de mort, le sexe ou le pays de citoyenneté, est aisé et correspond à des propriétés utilisées presque systématiquement dans les entités de type personne\index[ref]{lod@Linked Open Data (LOD)!wikidata@Wikidata}:
\begin{itemize}
	\item le sexe de la personne est défini par la propriété P21\footnote{Sexe (P21): \url{https://www.wikidata.org/wiki/Property:P21}} de Wikidata; une entité correspond à la valeur
	\item la date de naissance est la propriété P569\footnote{Date de naissance (P569): \url{https://www.wikidata.org/wiki/Property:P569}}; la valeur de cette déclaration est un littéral
	\item la date de décès est la propriété P570\footnote{Date de décès (P570): \url{https://www.wikidata.org/wiki/Property:P570}}; la valeur de cette déclaration est un littéral
	\item enfin, le pays de citoyenneté est la propriété P27\footnote{Pays de citoyenneté (P27): \url{https://www.wikidata.org/wiki/Property:P27}}; la valeur est une entité de pays
\end{itemize} 

\subsection{\label{II-C-2-b} Adapter les données contrôlées pour les valeurs des déclarations}
\titreEntete{Adapter les données contrôlées}

%table comparaison ina wiki avec propriétés
Bien que contrôlés, les jeux de données diffèrent dans la graphie de leurs valeurs\footnote{Voir \reference{table_roberts_2}.}. Il est par conséquent nécessaire d'effectuer un premier traitement sur les données de l'\ac{ina} de manière à faciliter leur alignement avec\index[ref]{lod@Linked Open Data (LOD)!wikidata@Wikidata} Wikidata.
\begin{table}[h!]
	\centering
	\csvautotabular[separator=semicolon]{images/ex_II_C_2_2.csv}
	\csvautotabular[separator=semicolon]{images/ex_II_C_2_2a.csv}
	\caption{Comparaison des informations disponibles pour Howard Roberts à l'\ac{ina} et sur Wikidata}
	\label{table_roberts_2}
\end{table}

%traiteemnt préalbale
D'abord, une inversion des noms et prénoms des personnes physiques de l'\ac{ina} doit être effectuée afin de correspondre au libellé de Wikidata. Il existe des propriétés \textit{nom de famille} (P734)\footnote{Nom de famille (P734): \url{https://www.wikidata.org/wiki/Property:P734}} et \textit{prénom} (P735)\footnote{Prénom (P735): \url{https://www.wikidata.org/wiki/Property:P735}}: ce formalisme trop important par rapport aux données de l'\ac{ina} n'aurait pas permis de comparer également les autres libellés qui ne sont pas divisés ainsi. L'inversion du nom et du prénom a, par conséquent, été réalisée pour pouvoir correspondre au libellé préférentiel de l'entité, ainsi qu'à ses libellés alternatifs.\\

Ensuite, le pays doit être extrait de la note qualité selon la méthode expliquée à la \reference{I-C-3}. Dans le cas d'une note avec plusieurs pays, la ligne et ses informations sont répétées autant de fois que nécessaire, avec un pays par ligne: cela permet de traiter la double nationalité ou les différents pays d'exercice des fonctions de chaque personne. Puisque la propriété P27 n'indique que le pays de citoyenneté, conserver les multiples pays de la note qualité permet d'augmenter les chances d'alignement avec Wikidata.\\

%années
Enfin, afin d'éviter l'écueil des différentes notations des dates --- en plein texte, avec des tirets ou des barres obliques, avec les horaires, \dots ---, les dates de naissance et de décès sont réduites à l'année autant pour les données de l'\ac{ina} que pour celles de \index[ref]{lod@Linked Open Data (LOD)!wikidata@Wikidata}Wikidata. Il est en effet possible de considérer qu'une personne et une entité partageant le même libellé et ayant les mêmes années de naissance et de décès soient équivalentes.
\begin{table}[h!]
	\centering
	\csvautotabular[separator=semicolon]{images/ex_II_C_2_3a.csv}
	\csvautotabular[separator=semicolon]{images/ex_II_C_2_3b.csv}
	\caption{Comparaison des informations disponibles pour Howard Roberts à l'\ac{ina} et sur Wikidata après un premier traitement}
	\label{table_roberts_3}
\end{table}
\medskip

%scinder jeu selon critères
Quand les deux jeux de données présentent la même forme et la même graphie, pour des propriétés équivalentes, l'alignement peut débuter\footnote{Voir \reference{table_roberts_3}.}. Cependant, les personnes physiques de l'\ac{ina} peuvent ne pas avoir de date de naissance connue, un pays absent, \dots~ Afin de ne pas comparer ces données manquantes avec celles certainement présentes sur \index[ref]{lod@Linked Open Data (LOD)!wikidata@Wikidata}Wikidata, il convient de regrouper les personnes physiques selon les caractéristiques remplies, pour adapter les requêtes et les comparaisons.

\subsection{\label{II-C-2-c}Effectuer l'alignement par de multiples requêtes}
\titreEntete{Effectuer l'alignement par de multiples requêtes}

%genre
Le premier traitement effectué ci-dessus ne suffit cependant pas à rendre les données alignables entre-elles. En effet, \textit{Q6581097} et \textit{Homme} ne sont pas similaires, tout comme le pays avec l'identifiant \index[ref]{lod@Linked Open Data (LOD)!wikidata@Wikidata}Wikidata. Il est, par conséquent, nécessaire d'attribuer ces identifiants Wikidata dans le jeu de données de l'\ac{ina}. Ainsi, les mentions de genre, \og Homme\fg{} et \og Femme\fg{} sont, quand elles sont présentes, remplacées par l'identifiant des entités \og masculin\fg{}\footnote{masculin (Q6581097): \url{https://www.wikidata.org/wiki/Q6581097}} et \og féminin\fg{}\footnote{féminin (Q1775415): \url{https://www.wikidata.org/wiki/Q1775415}} de Wikidata.\\

%pays
L'ajout de l'identifiant du pays est également nécessaire. Les pays se comptant par centaines, l'alignement doit être réalisé automatiquement à partir d'une requête dans le \index[ref]{echanges@Échanges!protocoles@Protocoles!sparql@SPARQL}\ac{sparql}-EndPoint. En effet, une seule requête, non répétée ensuite, est nécessaire pour récupérer l'ensemble des pays de Wikidata, leur identifiant, leur libellé préférentiel et leurs libellés alternatifs\footnote{Voir \reference{sparql_pays}.}. Les pays récupérés, ainsi que les pays de l'\ac{ina}, voient leur ponctuation et leur accentuation supprimées, et les majuscules transformées en minuscules pour arriver à un rapprochement le plus grand possible entre les graphies d'un même pays. Les Ètats-Unis indiqués dans les données de l'\ac{ina} sont alors alignés avec le même identifiant que le pays de citoyenneté de l'entité \nP{Howard}{Roberts} de \index[ref]{lod@Linked Open Data (LOD)!wikidata@Wikidata}Wikidata.

\begin{figure}[!h]
	\centering
\begin{minted}{sparql}
SELECT DISTINCT ?country ?countryLabel ?altLabel
WHERE
{
?country wdt:P31 wd:Q3624078 .
#en excluant les États historiques n'existant plus
FILTER NOT EXISTS {?country wdt:P31 wd:Q3024240}
#ainsi que les anciennes civilisations
FILTER NOT EXISTS {?country wdt:P31 wd:Q28171280}
OPTIONAL{?country skos:altLabel ?altLabel . FILTER (lang(?altLabel) = "fr")} .
		
SERVICE wikibase:label { bd:serviceParam wikibase:language "fr" }
}
ORDER BY ?countryLabel
\end{minted}
\caption{Requête \ac{sparql} de récupération des pays}
\label{sparql_pays}
\end{figure}

%personnes
À la suite de ces deux précédents alignements, l'alignement de la personne elle-même devient possible: les quatre points de comparaison sont désormais potentiellement utilisables. La rédaction d'une requête individuelle \index[ref]{echanges@Échanges!protocoles@Protocoles!sparql@SPARQL}\ac{sparql}, avec des filtres selon les données disponibles à la comparaison, demande beaucoup de temps de traitement et de recherche. Pour cette raison, l'\ac{api} Wikibase est réalisée en deux étapes\footnote{Il a été constaté qu'avec le même logiciel --- Talend --- et la même machine, l'exécution de deux requêtes avec l'\ac{api} prend quarante fois moins de temps qu'une unique requête dans le \ac{sparql}-EndPoint.}:
\begin{itemize}
	\item l'utilisation du module \textit{wbsearchentities} permet la recherche par le nom de la personne\footnote{Pour \nP{Howard}{Roberts}, cette recherche est la suivante: \url{https://www.wikidata.org/w/api.php?action=wbsearchentities\&language=fr\&search=howard\%20roberts\&format=json}.}; l'ensemble des identifiants retournés d'entités sont alors stockés pour l'étape suivante
	\item il est ensuite possible, avec le module \textit{wbgetclaims}\footnote{Voir \url{https://www.wikidata.org/w/api.php?action=wbgetclaims\&entity=Q1631895\&format=json}.}, d'obtenir les déclarations de chaque entité, et, par conséquent, les valeurs des propriétés recherchées --- P21, P27, P569 et P570
\end{itemize}
La récupération des valeurs de ces propriétés permet l'obtention de la \reference{table_roberts_3} puis l'alignement de la personne physique de l'\ac{ina} avec la bonne entité de \index[ref]{lod@Linked Open Data (LOD)!wikidata@Wikidata}Wikidata, si cette dernière existe.
Cependant, la seule utilisation de ces quatre points de comparaison ne permet qu'un petit nombre d'alignements. En effet, les données de l'\ac{ina} comportent peu ou pas de dates de naissance et de décès: la réalisation d'un alignement sur les seuls genre et pays de citoyenneté ne peut pas être satisfaisante et ne permet pas une bonne gestion des homonymes.

%conclu
\bigskip
\bigskip
La présence de données contrôlées dans une institution ne permet pas à elle seule d'obtenir un alignement avec des référentiels ou des jeux de données externes: bien que structurées et contrôlées, les données diffèrent par leur graphie et leur norme de rédaction d'une institution à une autre, d'un jeu de données à un autre, \dots~Il sera, par conséquent, toujours nécessaire d'effectuer des traitements préalables de données que l'on souhaite mettre en relation avec d'autres.
\section{\label{II-C-3}Aligner des personnes depuis du texte libre}
\titreEntete{Aligner des personnes depuis du texte libre}

%intro

%conclu
\section{\label{II-C-4}Comprendre les limites}
\titreEntete{Comprendre les limites}

%intro

%conclu