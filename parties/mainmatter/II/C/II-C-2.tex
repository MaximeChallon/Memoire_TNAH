\section{\label{II-C-2}Aligner des personnes depuis des données contrôlées}
\titreEntete{Aligner des personnes depuis des données contrôlées}

%intro
L'\ac{ina} conserve ses données sous deux formes: des données contrôlées, et des données en texte libre\footnote{Voir \reference{annexe_type_donnees_axel} (\reference{type_donnees_axel}).}. Les premières permettent une meilleure exploitation et des traitements plus faciles, notamment lors d'un alignement avec un référentiel ou un autre jeu de données\footnote{L'alignement des notes qualité de l'\ac{ina} avec un thésaurus interne a déjà démontré la difficulté de l'utilisation du texte libre.}. Cependant, le contrôle du langage naturel se fait différemment selon le contexte de création et d'utilisation des données. Ainsi, deux jeux de données contrôlés peuvent avoir pour un même concept deux graphies et deux normes différentes, tout en respectant de chaque côté un contrôle du langage qui leur est propre.\\

Bien que les personnes physiques décrites à l'\ac{ina} respectent des règles de structure, et le soient dans plusieurs attributs d'une table de base de données, elles ne correspondent pas exactement aux entités de Wikidata qui sont créées selon d'autres règles de graphie. Ainsi, quel que soit le jeu de données, il est toujours nécessaire de lui apporter un premier traitement afin de trouver le lien vers un autre jeu de données.

\subsection{\label{II-C-2-a}Choix des déclarations des entités de Wikidata}
\titreEntete{Choix des déclarations des entités de Wikidata}

À partir des mêmes personnes physiques de la \reference{I-C-3}, il est nécessaire de créer du lien avec \index[ref]{lod@Linked Open Data (LOD)!wikidata@Wikidata}Wikidata de manière à les enrichir. Plusieurs informations sont ainsi disponibles pour chaque personne: les dates de naissance et de mort, et la note qualité\footnote{Voir \reference{table_roberts_1}.}.
\begin{table}[h!]
	\centering
	\csvautotabular[separator=semicolon]{images/ex_II_C_2.csv}
	\caption{Informations disponibles pour Howard Roberts à l'\ac{ina}}
	\label{table_roberts_1}
\end{table}
\medskip

%propriété Wiki correspondantes
L'entité\footnote{Howard Roberts: \url{https://www.wikidata.org/wiki/Q1631895}} Howard Roberts comporte plusieurs dizaines de déclarations --- paire propriété--valeur. Cependant, la présence de ces déclarations n'est pas uniforme selon les entités de personnes. De manière à aligner les entités Wikidata avec un jeu de données, il est nécessaire de chercher les propriétés les plus communes et le plus souvent présentes, pouvant correspondre à chacune des informations disponibles dans le jeu de données à aligner.\\

Ce choix, pour les dates de naissance ou de mort, le sexe ou le pays de citoyenneté, est aisé et correspond à des propriétés utilisées presque systématiquement dans les entités de type personne\index[ref]{lod@Linked Open Data (LOD)!wikidata@Wikidata}:
\begin{itemize}
	\item le sexe de la personne est défini par la propriété P21\footnote{Sexe (P21): \url{https://www.wikidata.org/wiki/Property:P21}} de Wikidata; une entité correspond à la valeur
	\item la date de naissance est la propriété P569\footnote{Date de naissance (P569): \url{https://www.wikidata.org/wiki/Property:P569}}; la valeur de cette déclaration est un littéral
	\item la date de décès est la propriété P570\footnote{Date de décès (P570): \url{https://www.wikidata.org/wiki/Property:P570}}; la valeur de cette déclaration est un littéral
	\item enfin, le pays de citoyenneté est la propriété P27\footnote{Pays de citoyenneté (P27): \url{https://www.wikidata.org/wiki/Property:P27}}; la valeur est une entité de pays
\end{itemize} 

\subsection{\label{II-C-2-b} Adapter les données contrôlées pour les valeurs des déclarations}
\titreEntete{Adapter les données contrôlées}

%table comparaison ina wiki avec propriétés
Bien que contrôlés, les jeux de données diffèrent dans la graphie de leurs valeurs\footnote{Voir \reference{table_roberts_2}.}. Il est par conséquent nécessaire d'effectuer un premier traitement sur les données de l'\ac{ina} de manière à faciliter leur alignement avec\index[ref]{lod@Linked Open Data (LOD)!wikidata@Wikidata} Wikidata.
\begin{table}[h!]
	\centering
	\csvautotabular[separator=semicolon]{images/ex_II_C_2_2.csv}
	\csvautotabular[separator=semicolon]{images/ex_II_C_2_2a.csv}
	\caption{Comparaison des informations disponibles pour Howard Roberts à l'\ac{ina} et sur Wikidata}
	\label{table_roberts_2}
\end{table}

%traiteemnt préalbale
D'abord, une inversion des noms et prénoms des personnes physiques de l'\ac{ina} doit être effectuée afin de correspondre au libellé de Wikidata. Il existe des propriétés \textit{nom de famille} (P734)\footnote{Nom de famille (P734): \url{https://www.wikidata.org/wiki/Property:P734}} et \textit{prénom} (P735)\footnote{Prénom (P735): \url{https://www.wikidata.org/wiki/Property:P735}}: ce formalisme trop important par rapport aux données de l'\ac{ina} n'aurait pas permis de comparer également les autres libellés qui ne sont pas divisés ainsi. L'inversion du nom et du prénom a, par conséquent, été réalisée pour pouvoir correspondre au libellé préférentiel de l'entité, ainsi qu'à ses libellés alternatifs.\\

Ensuite, le pays doit être extrait de la note qualité selon la méthode expliquée à la \reference{I-C-3}. Dans le cas d'une note avec plusieurs pays, la ligne et ses informations sont répétées autant de fois que nécessaire, avec un pays par ligne: cela permet de traiter la double nationalité ou les différents pays d'exercice des fonctions de chaque personne. Puisque la propriété P27 n'indique que le pays de citoyenneté, conserver les multiples pays de la note qualité permet d'augmenter les chances d'alignement avec Wikidata.\\

%années
Enfin, afin d'éviter l'écueil des différentes notations des dates --- en plein texte, avec des tirets ou des barres obliques, avec les horaires, \dots ---, les dates de naissance et de décès sont réduites à l'année autant pour les données de l'\ac{ina} que pour celles de \index[ref]{lod@Linked Open Data (LOD)!wikidata@Wikidata}Wikidata. Il est en effet possible de considérer qu'une personne et une entité partageant le même libellé et ayant les mêmes années de naissance et de décès soient équivalentes.
\begin{table}[h!]
	\centering
	\csvautotabular[separator=semicolon]{images/ex_II_C_2_3a.csv}
	\csvautotabular[separator=semicolon]{images/ex_II_C_2_3b.csv}
	\caption{Comparaison des informations disponibles pour Howard Roberts à l'\ac{ina} et sur Wikidata après un premier traitement}
	\label{table_roberts_3}
\end{table}
\medskip

%scinder jeu selon critères
Quand les deux jeux de données présentent la même forme et la même graphie, pour des propriétés équivalentes, l'alignement peut débuter\footnote{Voir \reference{table_roberts_3}.}. Cependant, les personnes physiques de l'\ac{ina} peuvent ne pas avoir de date de naissance connue, un pays absent, \dots~ Afin de ne pas comparer ces données manquantes avec celles certainement présentes sur \index[ref]{lod@Linked Open Data (LOD)!wikidata@Wikidata}Wikidata, il convient de regrouper les personnes physiques selon les caractéristiques remplies, pour adapter les requêtes et les comparaisons.

\subsection{\label{II-C-2-c}Effectuer l'alignement par de multiples requêtes}
\titreEntete{Effectuer l'alignement par de multiples requêtes}

%genre
Le premier traitement effectué ci-dessus ne suffit cependant pas à rendre les données alignables entre-elles. En effet, \textit{Q6581097} et \textit{Homme} ne sont pas similaires, tout comme le pays avec l'identifiant \index[ref]{lod@Linked Open Data (LOD)!wikidata@Wikidata}Wikidata. Il est, par conséquent, nécessaire d'attribuer ces identifiants Wikidata dans le jeu de données de l'\ac{ina}. Ainsi, les mentions de genre, \og Homme\fg{} et \og Femme\fg{} sont, quand elles sont présentes, remplacées par l'identifiant des entités \og masculin\fg{}\footnote{masculin (Q6581097): \url{https://www.wikidata.org/wiki/Q6581097}} et \og féminin\fg{}\footnote{féminin (Q1775415): \url{https://www.wikidata.org/wiki/Q1775415}} de Wikidata.\\

%pays
L'ajout de l'identifiant du pays est également nécessaire. Les pays se comptant par centaines, l'alignement doit être réalisé automatiquement à partir d'une requête dans le \index[ref]{echanges@Échanges!protocoles@Protocoles!sparql@SPARQL}\ac{sparql}-EndPoint. En effet, une seule requête, non répétée ensuite, est nécessaire pour récupérer l'ensemble des pays de Wikidata, leur identifiant, leur libellé préférentiel et leurs libellés alternatifs\footnote{Voir \reference{sparql_pays}.}. Les pays récupérés, ainsi que les pays de l'\ac{ina}, voient leur ponctuation et leur accentuation supprimées, et les majuscules transformées en minuscules pour arriver à un rapprochement le plus grand possible entre les graphies d'un même pays. Les Ètats-Unis indiqués dans les données de l'\ac{ina} sont alors alignés avec le même identifiant que le pays de citoyenneté de l'entité \nP{Howard}{Roberts} de \index[ref]{lod@Linked Open Data (LOD)!wikidata@Wikidata}Wikidata.

\begin{figure}[!h]
	\centering
\begin{minted}{sparql}
SELECT DISTINCT ?country ?countryLabel ?altLabel
WHERE
{
?country wdt:P31 wd:Q3624078 .
#en excluant les États historiques n'existant plus
FILTER NOT EXISTS {?country wdt:P31 wd:Q3024240}
#ainsi que les anciennes civilisations
FILTER NOT EXISTS {?country wdt:P31 wd:Q28171280}
OPTIONAL{?country skos:altLabel ?altLabel . FILTER (lang(?altLabel) = "fr")} .
		
SERVICE wikibase:label { bd:serviceParam wikibase:language "fr" }
}
ORDER BY ?countryLabel
\end{minted}
\caption{Requête \ac{sparql} de récupération des pays}
\label{sparql_pays}
\end{figure}

%personnes
À la suite de ces deux précédents alignements, l'alignement de la personne elle-même devient possible: les quatre points de comparaison sont désormais potentiellement utilisables. La rédaction d'une requête individuelle \index[ref]{echanges@Échanges!protocoles@Protocoles!sparql@SPARQL}\ac{sparql}, avec des filtres selon les données disponibles à la comparaison, demande beaucoup de temps de traitement et de recherche. Pour cette raison, l'\ac{api} Wikibase est réalisée en deux étapes\footnote{Il a été constaté qu'avec le même logiciel --- Talend --- et la même machine, l'exécution de deux requêtes avec l'\ac{api} prend quarante fois moins de temps qu'une unique requête dans le \ac{sparql}-EndPoint.}:
\begin{itemize}
	\item l'utilisation du module \textit{wbsearchentities} permet la recherche par le nom de la personne\footnote{Pour \nP{Howard}{Roberts}, cette recherche est la suivante: \url{https://www.wikidata.org/w/api.php?action=wbsearchentities\&language=fr\&search=howard\%20roberts\&format=json}.}; l'ensemble des identifiants retournés d'entités sont alors stockés pour l'étape suivante
	\item il est ensuite possible, avec le module \textit{wbgetclaims}\footnote{Voir \url{https://www.wikidata.org/w/api.php?action=wbgetclaims\&entity=Q1631895\&format=json}.}, d'obtenir les déclarations de chaque entité, et, par conséquent, les valeurs des propriétés recherchées --- P21, P27, P569 et P570
\end{itemize}
La récupération des valeurs de ces propriétés permet l'obtention de la \reference{table_roberts_3} puis l'alignement de la personne physique de l'\ac{ina} avec la bonne entité de \index[ref]{lod@Linked Open Data (LOD)!wikidata@Wikidata}Wikidata, si cette dernière existe.
Cependant, la seule utilisation de ces quatre points de comparaison ne permet qu'un petit nombre d'alignements. En effet, les données de l'\ac{ina} comportent peu ou pas de dates de naissance et de décès: la réalisation d'un alignement sur les seuls genre et pays de citoyenneté ne peut pas être satisfaisante et ne permet pas une bonne gestion des homonymes.

%conclu
\bigskip
\bigskip
La présence de données contrôlées dans une institution ne permet pas à elle seule d'obtenir un alignement avec des référentiels ou des jeux de données externes: bien que structurées et contrôlées, les données diffèrent par leur graphie et leur norme de rédaction d'une institution à une autre, d'un jeu de données à un autre, \dots~Il sera, par conséquent, toujours nécessaire d'effectuer des traitements préalables de données que l'on souhaite mettre en relation avec d'autres.