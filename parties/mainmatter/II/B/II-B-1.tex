\section{\label{II-B-1}L'ontologie, un vocabulaire structurant}
\titreEntete{L'ontologie, un vocabulaire structurant}

%intro
Dans le chapitre précédent\footnote{Voir \reference{II-A}.}, nous avons évoqué un premier type de référentiel --- les vocabulaires de valeurs --- présent dans le Web de données. De manière à pouvoir décrire ces ressources\footnote{\og L'une des fonctions des ontologies est de permettre de définir la nature des ressources\fg{} in \cite[§49]{bermes_convergence_2013}.}, d'autres référentiels sont nécessaires, les ontologies, en fournissant les classes et les propriétés utiles aux descriptions. Au-delà de l'apport de ces éléments, l'ontologie permet également une description formelle par des axiomes et des règles de raisonnement, visibles dans le Web de données avec \ac{rdfs} et \ac{owl}.\\

L'ontologie informatique est un concept récent, né à la fin du \textsc{XX}\textsuperscript{ème}siècle comme le Web. Plusieurs types d'ontologies existent, reflétant leur caractère universel ou non, leur domaine de description; leur structuration et leur formation doivent cependant répondre à des critères précis de manière à structurer le plus efficacement possible de référentiel.

\subsection{\label{II-B-1-a}Origines de l'ontologie informatique}
\titreEntete{Origines de l'ontologie informatique}

\begin{citationLongue}
	[Les ontologies sont] des vocabulaires de termes --- classes, relations, fonctions, constantes d'objet --- avec des définitions communes, sous la forme d'un texte compréhensible par les humains et applicable à la machine, de contraintes déclaratives dans leur forme la mieux formée.\footnote{\og vocabulaires of representational terms --- classes, relations, functions, object constants ---  with agreed-upon definitions, in the form of human-readable text and machine-enforceable, declarative constraints on their well formed use\fg{} in \cite[p.2]{gruber_role_1991}}
\end{citationLongue}

L'ontologie est d'abord une science philosophique, née avec les \textit{Catégories} d'Aristote, étudiant la réalité des entités, les relations qu'elles entretiennent --- hiérarchie, similarité --- pour trouver les similarités et les différences présentes dans le monde. Au \textsc{XIX}\textsuperscript{ème}siècle, le siècle de la taxonomie, cette science philosophique devient l'étude de l'ensemble des connaissances existantes dans le monde\footcite{welty_supporting_2011}.\\

Ce n'est qu'en 1991\footcite{gruber_role_1991} puis 1993\footcite{gruber_toward_1993} que \nP{Thomas R.}{Gruber}, souhaitant améliorer l'intelligence artificielle et l'indexation structurée, évoque l'ontologie informatique, pensée comme un ensemble d'entités déclaratif destinée au partage des connaissances entre les machines: \og Une ontologie est une spécification explicite d’une conceptualisation\fg{}\footnote{\og An ontology is an explicit specification of a conceptualization. \fg{} in \cite[p.1]{gruber_toward_1993}.}. Cette définition donnée très tôt par \nP{Thomas R.}{Gruber} permet d'observer deux principes de l'ontologie: premièrement, elle est une conceptualisation d'un domaine, par conséquent elle est un choix de description sur un domaine précis; deuxièmement, cette conceptualisation est spécifiée, c'est à dire qu'elle a une description formelle.\\

\nP{Rudi}{Studer} apporte des précisions en 1998\footcite{studer_knowledge_1998} en proposant une nouvelle définition plus spécifique de l'ontologie: \og Une ontologie est
une spécification formelle et explicite d’une conceptualisation partagée\fg{}\footnote{\og An ontologyis a formal, explicit specification of a shared conceptualization\fg{} in \cite{studer_knowledge_1998}}. Une ontologies est formelle de manière à pouvoir être comprise par une machine; elle est une spécification explication par la déclarativité de ses concepts, de ses propriétés, \dots~; elle est partagée car elle prend l'ensemble des connaissances d'une communauté, d'un domaine; enfin, la conceptualisation renvoie au domaine décrit par cette ontologie.\\

L'ontologie est un référentiel de classes et de propriétés, ne s'appliquant qu'à un seul domaine particulier de la connaissance, mais permettant de le structurer. Son fort développement a permis une application dans le Web de données et dans le milieu bibliothéconomique, qui considère les ontologies comme \og des éléments de description de métadonnées\fg{}\footcite{baker_rapport_2012}.

\subsection{\label{II-B-1-b}Des ontologies différentes}
\titreEntete{Des ontologies différentes}

Une grande diversité d'ontologies existe. Certaines sont plus importantes que d'autres du fait du nombre d'utilisations qu'elles entraînent et de leur généralité; d'autres, plus spécifiques, paraissent moins importantes par le faible nombre de liens qu'elles suscitent. Les ontologies peuvent également être classées selon les usages qui en sont faits, selon si leur finalité est une publication sur le Web sémantique ou simplement une utilisation interne à une institution.\\

Au plus haut niveau se trouvent des ontologies \og noyaux\fg{}\footcite[p.4]{isaac_les_2012} qui modélisent les connaissances communs, partageables et réutilisables d'un domaine à un autre: le modèle \ac{cidoccrm} des musées propose ainsi une ontologie réutilisée dans \ac{rdfs} notamment\footnote{Voir \reference{annexe_onto} (\reference{onto_crm}).}. Son vocabulaire correspond aux événements, aux objets, aux moments, \dots ~ ce qui en fait un vocabulaire générique pour d'autres ontologies de plus bas niveau.\\

Au niveau inférieur se trouvent les ontologies de domaine, propres à un domaine en particulier: elles modélisent les connaissances de ce domaine uniquement; elles offrent des concepts et des relations permettant de décrire les activités et les vocabulaires du domaine en question. Les concepts de ces ontologies de domaine sont souvent des spécialisations\footnote{Ainsi que le faisait remarquer \nP{Rudi}{Studer} en 1998 in \cite{studer_knowledge_1998}.} d'ontologies de plus haut niveau. L'ontologie \ac{frbr}\footnote{Décrite dans la \reference{II-B-3}.} peut être considérée comme une de ces ontologies de domaine car elle utilise \ac{rdfs}, les \textit{Dublin Core Terms} (DC Terms), et d'autres ontologies de haut niveau\footnote{Voir \reference{annexe_onto} (\reference{onto_frbr})}.\\

Plus bas encore, il existe des ontologies utilisées dans l'unique cadre d'utilisations applicatives, par un petit nombre d'utilisateurs. Elles ne modélisent par conséquent que les termes et les relations nécessaires à l'application, en utilisant principalement des ontologies de plus haut niveau, notamment celles de domaine\footnote{L'ontologie \textit{Citation Counting and Context Characterization Ontology}(C4O) est l'une de ces ontologies applicatives, ayant peu d'ontologies l'utilisant, et utilisant un grand nombre d'ontologies de plus haut niveau. Voir \reference{annexe_onto} (\reference{onto_c4o}).}.\\

De multiples ontologies existent et diffèrent de cette hiérarchisation\footcite[p.2]{isaac_les_2012} et participent à la diversité des ontologies. Cette diversité est essentielle pour décrire tout type d'objets ou d'événements --- que ce soit dans le milieu culturel ou non ---, ou pour relier à ces ontologies des vocabulaires propres: elle participe à l'interopérabilité des référentiels entre eux. Cette interopérabilité, grâce aux ontologies, peut être entre un système documentaire interne et le Web de données, ou bien entre deux jeux de données du même Web de données.

\subsection{\label{II-B-1-c}Les principes de l'ontologie}
\titreEntete{Les principes de l'ontologie}

Permettre l'interopérabilité entre tout type de données et de jeux de données nécessite une structure et des principes. Dans la publication de 1993\footcite{gruber_toward_1993}, \nP{Thomas R.}{Gruber} édicte déjà cinq critères sans lesquels une ontologie ne peut pas être formée correctement. Ces critères, généraux, contraignent la graphie tout en laissant le champ ouvert aux modifications futures d'une ontologie. Il préconise ainsi:
\begin{itemize}
	\item la clarté des termes décrits: leur description doit être objective, complète, et réalisée dans le langage naturel;
	\item la cohérence des axiomes retenus et l'interdiction de la discordance entre les termes et les axiomes: elle permet la spécialisation de la conceptualisation qui est la définition de l'ontologie;
	\item la possibilité d'étendre l'ontologie même après sa création: cela permet d'accepter les changements d'usages ou de besoins liés à cette ontologie, et par conséquent de la faire évoluer facilement;
	\item le biais d'encodage doit être minimal de manière à permettre la plus grande interopérabilité;
	\item l'engagement dans l'ontologie doit être minimal, les termes utilisés souvent être ceux les plus souvent utilisés: cela permet la réutilisation de l'ontologie
\end{itemize}
\bigskip

Ces cinq critères ontologiques dirigent la structure et la nature des éléments essentiels aux ontologies et les constituant. Les concepts sont le premier de ces éléments: ils permettent de définir des idées, des objets ou des notions. De même que pour les vocabulaires contrôlés\footnote{Voir \reference{I-A-2}.}, plusieurs propriétés peuvent s'appliquer à ces concepts. \nP{Nicola}{Guarino} décrit ces propriétés\footcite{guarino_formal_1998} de généricité --- absence d'extension pour le concept ---, d'identité, de rigidité --- si une instance du concept reste en permanence son instance ---, d'anti-rigidité --- si une instance est principalement définie par son appartenance à un autre concept --- et d'unité des concepts. Ainsi, deux concepts peuvent être disjoints, équivalents ou dépendants.\\

Les relations sont une autre partie essentielle des ontologies, sans lesquelles les concepts n'ont pas de sens entre eux et ne peuvent pas être formalisés. Elles peuvent être inclusives --- hiérarchiques ---, ou ensemblistes avec des unions, des intersections ou des exclusions. À ces relations peuvent d'ajouter des axiomes, des règles, qui viennent régir les contraintes, les relations ou les concepts eux-mêmes.\\

Les différents principes des ontologies les rendent strictes sur leur formation et leur structure, de nombreuses propriétés s'imposant. Complexes, ils permettent néanmoins la création d'un vocabulaire à la fois contrôlé, hiérarchique et utilisable par tous.

%conclu
\bigskip
\bigskip
L'ontologie peut alors trouver son intérêt sur le Web avec une indexation réalisée par les moteurs de recherche; sur le Web et en institution en permettant la description de jeux de données par des ontologies publiques; en structurant la connaissance du monde. L'une des applications principale est bibliothéconomique avec la possibilité de valoriser et de publier les collections sous forme de métadonnées. Enfin, l'ontologie permet de relier deux jeux de données, deux référentiels, pourtant éloignés, selon un vocabulaire commun partagé publiquement.