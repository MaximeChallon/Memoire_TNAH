\chapter{\label{II-A}Le Web de données: une exposition commune des référentiels}
\titreEntete{Le Web de données: une exposition commune des référentiels}

\begin{citationLongue}
	Le web de données, en proposant une forme d'interopérabilité basée sur des standards du Web et sur des liens entre les ressources, semble à même de faciliter l'accès à des données structurées, stockées dans des bases telles que les catalogues de bibliothèques, les inventaires d'archives ou les bases culturelles des musées.\footcite[p.45]{dalbin_approches_2011}
\end{citationLongue}
\medskip
\lettrine{L}e domaine bibliothéconomique, et plus généralement celui de la culture, est l'un des premiers à s'être intégré dans le Web de données. Les avantages apportés par le Web, tels que le partage et la mise en commun de référentiels et de données, ont permis une large adoption des standards et formats du Web de données dans les institutions patrimoniales. Cependant, les pratiques individualistes\footnote{En 1991, \nP{Thomas R.}{Gruber} fait remarquer que ces \og blocs monolithiques\fg{} propres à chaque institutions sont un frein au développement de liens avec d'autres institutions. \og Que peut-il être fait pour permettre l'accumulation, le partage et la réutilisation des bases de connaissances? \fg{}. Voir \cite[p.1]{gruber_role_1991}.} qui étaient celles des institutions auparavant se retrouvent dans le développement de ce Web de données et ont conduit à une efficacité limitée lors de ses débuts.\\

Malgré ces difficultés des premiers temps, les institutions patrimoniales se sont désormais emparées de ce Web de données, devenu un lieu de partage de liens et un fournisseur d'identifiants que les institutions peuvent stocker en vue d'enrichir leurs propres données\footnote{Un premier exemple a été étudié précédemment avec l'achat de ressources extérieures à l'\ac{ina}. Voir \reference{I-B-3}.}.\\

Plus encore que le partage de liens, le Web de données est également un apport considérable dans l'expérience de l'utilisateur qui recherche des informations spécifiques sans savoir vers quelle institution se tourner. Il peut, avec les technologies du Web, naviguer de lien en lien, d'institution en institution, rebondir de document en document, sans se rendre compte des frontières techniques ou institutionnelles\footnote{\og Sur le Web, un utilisateur a la possibilité de naviguer d'un site à un autre sans avoir connaissance des moyens techniques utilisés pour publier les données, ni même avoir conscience des ruptures ou des frontières entre chacun des sites. \fg{} in \cite[p.45]{dalbin_approches_2011}.}. Le Web permet un affranchissement des frontières, à la fois pour l'utilisateur final que pour les machines.\\

Enfin, les technologies du Web, utilisées dans le Web de données, proposent de nouveaux formats et de nouvelles modélisations de données, conduisant à la disparition progressive de la notion de référentiel. De plus, la massification des données du Web de données posent la problématique de leur accès rapide pour l'utilisateur.

\section{\label{II-A-1}Le Web de données: naissance et principes}
\titreEntete{Le Web de données: naissance et principes}

La recherche d'un protocole et d'un format d'échange de données entre les institutions est constante. Nous avons évoqué précédemment\footnote{Voir \reference{I-C-2}.} les difficultés rencontrées avec les protocoles Z039-50 et \ac{oaipmh}. Ces derniers sont insuffisants pour permettre un partage massif de données et de référentiels, mais ils ont permis l'évolution de la réflexion sur l'interopérabilité. Le grand bouleversement est survenu au milieu des années 2000 avec le Web de données qui ne crée pas de protocole nouveau, mais s'appuie sur un autre largement répandu et utilisé, \ac{http}. Les règles édictées ont permis sa bonne utilisation pour créer le web de données et la naissance d'un format d'échange, \ac{rdf}.

\subsection{\label{II-A-1-a}Créer un modèle de données nativement compatible avec le Web: le Web de données}
\titreEntete{Un modèle de données nativement compatible avec le Web}

\subsubsection{\label{II-A-1-a-i}Naissance du Web de données}
\titreEntete{Naissance du Web de données}

Dès 1989, \nP{Tim}{Berners-Lee} propose un \index[ref]{typologie@Typologie!graphe@Graphe de nœuds et de liens}\og espace d'information commun\fg{}\footnote{\og pool of information\fg{} in \cite{berners-lee_information_1989}.} où les textes seraient liés par des liens\footnote{\og a web of notes with links\fg{} in \cite{berners-lee_information_1989}}. Il propose ainsi un modèle de nœuds et de liens qui permet d'entrer n'importe quel type d'informations: grâce aux liens, une ressource peut être trouvée sans avoir eu à la chercher. Cependant, plusieurs difficultés demeurent encore: il est nécessaire, comme dans un arbre, d'avoir des nœuds uniques; la modélisation du monde réel est impossible, et le modèle de nœuds et de liens se heurte aux mêmes réflexions que Porphyre et les encyclopédistes quant à la possibilité de représenter le monde en un seul arbre. Pour y faire face, \nP{Tim}{Berners-Lee} propose comme solution le lien hypertexte.\\

En 1994, \nP{Tim}{Berners-Lee} continue la réflexion sur le Web et les liens hypertexte\footcite{berners-lee_plenary_1994}. En 1989, seuls les documents étaient mis sur le Web et \nP{Tim}{Berners-Lee} décrivait la manière de les relier entre eux. En 1994, il propose d'intégrer au Web des données du monde réel qui seraient reliées par des liens hypertexte, comme les documents, toujours sous la forme de nœuds et de liens. Cette proposition est le déclencheur de la réflexion sur le Web de données. Cependant, la réalité n'est pas compréhensible par une machine, et les liens ne peuvent se comprendre que par leur contexte: l'ajout de valeurs aux relations permet de donner du sens au Web --- c'est le \index[ref]{typologie@Typologie!graphe@Graphe de nœuds et de liens}Web sémantique. \nP{Tim}{Berners-Lee} se fait ainsi le promoteur de la donnée structurée à la fois pour la machine et pour l'humain.\\

Une feuille de route est par conséquent écrite par \nP{Tim}{Berners-Lee} en 1998\footcite{berners-lee_semantic_1998}. Il y évoque pour la première fois le terme \og Web de données\fg{}\footnote{\og web of data\fg{} in \cite{berners-lee_semantic_1998}}, mais la feuille de route n'est pas appliquée et il faut attendre 2006 et la publication \textit{Linked data}\footcite{berners-lee_linked_2006} pour que les recommandations du Web sémantique soient expliquées et adoptées. Fondamentale, cette publication évoque les principes du Web de données actuel en décrivant les bonnes pratiques à adopter. Le Web est ainsi perçu comme une \og base de données globale\fg{}\footcite[§29]{bermes_convergence_2013} où les données sont reliées de la même manière que les documents HTML avec des liens hypertexte.\\

Cette interopérabilité basée sur les liens, théorisée notamment par \nP{Tim}{Berners-Lee}, est à l'origine du \index[ref]{typologie@Typologie!graphe@Graphe de nœuds et de liens}Web de données et de l'actuel partage de données et de référentiels entre les institutions patrimoniales. Le lien apparaît comme essentiel et permet de décloisonner chaque institution pour les faire communiquer ensemble de manière à améliorer la recherche de données et de documents par l'utilisateur final\footnote{L'utilisateur final n'est pas seulement le grand public, il peut être chercheur, professionnel dans une institution, consommateur commercial, \dots}.

\subsubsection{\label{II-A-1-a-ii}Principes généraux}
\titreEntete{Principes généraux}

La publication de 2006 de \nP{Tim}{Berners-Lee} décrit précisément les principes du Web de données qu'il est nécessaire de développer pour comprendre l'évolution des pratiques documentaires des institutions depuis le milieu des années 2000. Ces principes s'appuient sur l'architecture du Web existant et ne visent pas la création d'un Web: l'interopérabilité des données doit passer par une interopérabilité des protocoles et des formats utilisés avec ceux du Web, qui connaît une utilisation croissante en 2006.\\

Le premier principe évoqué est celui de l'utilisation des Uniform Resource Identifier (URI) comme clé unique d'une ressource: en cas de non utilisation du standard des URIs, le \index[ref]{typologie@Typologie!graphe@Graphe de nœuds et de liens}Web sémantique n'est plus possible\footnote{\og If it doesn't use the universal URI set of symbols, we don't call it Semantic Web\fg{} in \cite{berners-lee_linked_2006}.}. En effet, une URI possède une syntaxe précise qu'il convient de respecter et d'adopter: \textit{scheme:autorité/chaîne\_de\_caractères}\footcite[§40]{bermes_convergence_2013}.\\

Le second principe est celui de l'utilisation du protocole du WorldWideWeb, \index[ref]{echanges@Échanges!protocoles@Protocoles!http@HTTP}\ac{http}.\\

Le troisième principe impose le renvoi d'informations et de données dans des formats standards du Web, en \index[ref]{echanges@Échanges!formats@Formats!rdf@RDF}\ac{rdf}/XML, ou en N3 ou Turtle\footcite{berners-lee_linked_2006}. Tous ces formats acceptent le langage de requête \index[ref]{echanges@Échanges!protocoles@Protocoles!sparql@SPARQL}SPARQL.\\

Enfin, le quatrième principe est celui de la création de liens entre les ressources --- donc les URIs --- sans lesquels les efforts réalisés avec les trois premiers principes sont vains. Ainsi, un Web fiable, sans frontières, est créé\footnote{\og serious, unbounded web in which one can find al kinds of things, just as on the hypertext web we have managed to build\fg{} in \cite{berners-lee_linked_2006}}; l'utilisateur peut y naviguer facilement grâce aux liens hypertextes. Le \index[ref]{typologie@Typologie!graphe@Graphe de nœuds et de liens}Web de données est par conséquent moins une base de données qu'un lieu où les liens donnent de la valeur aux ressources liées: plus une ressource possède de liens, plus celle-ci a une description précise et fiable, plus elle devient visible à l'utilisateur.\\

Nous l'aurons remarqué, depuis le début de cette description du Web de données, la notion de référentiel semble s'estomper au profit de ressources et de données liées. En effet, un référentiel n'est qu'une mise en forme spécifique d'un jeu de données selon une structure propre à son producteur. Cette spécificité de chaque jeu de données n'est pas valable dans le Web de données: un retour à la donnée est nécessaire, les liens qui lui seront affectée permettront alors de représenter son ancienne structure dans le référentiel. \nP{Tim}{Berners-Lee} décrit la nécessité de se dégager de ses propres formats sur le Web pour évoluer vers des formats compréhensibles par une machine: il crée l'échelle des cinq étoiles, le \index[ref]{echanges@Échanges!formats@Formats!rdf@RDF}\ac{rdf} étant la meilleure des solutions d'exposition des données.

\subsection{\label{II-A-1-b}Inventer un format d'échange compatible avec ce modèle de données: RDF}
\titreEntete{Inventer un format d'échange compatible avec ce modèle de données}

\index[ref]{echanges@Échanges!protocoles@Protocoles!http@HTTP}\ac{http} est le protocole utilisé pour le Web de données; le format d'échange est \index[ref]{echanges@Échanges!formats@Formats!rdf@RDF}\ac{rdf}. C'est un standard, développé pour le Web, capable d'assurer l'interopérabilité des données. Seules des URIs peuvent constituer des ressources. Ces ressources sont ensuite reliées par un lien typé dû au formalisme offert par \ac{rdf}. \ac{rdf} ne permet pas, comme cela est le cas avec les encodages XML archivistiques ou codicologiques, un schéma prédéfini, mais un modèle logique de description des ressources.\\

Avec \ac{rdf}, deux ressources ne peuvent être reliées directement, seule leur relation peut être typée afin que la machine puisse interpréter la nature de leur lien, peu importe la localisation des deux ressources. Ainsi, la forme d'un triplet \index[ref]{echanges@Échanges!formats@Formats!rdf@RDF}\ac{rdf} reflète cette distinction: le \og sujet\fg est nécessairement une ressource --- par conséquent une URI ---, il est suivi d'un \og prédicat\fg{} qui défini la nature de la relation avec le troisième élément du triplet, l'\og objet\fg{}, qui peut être une ressource ou un littéral. Le triplet est donc une simple phrase sujet-verbe-complément compréhensible par une machine. Une ressource pouvant être à la fois sujet dans un triplet, prédicat dans un autre, ou objet dans d'autres; un graphe se construit alors. L'information est donc totalement déconstruite pour un humain, mais elle devient compréhensible par une machine, qui permet ensuite la reconstruction de l'information par des \index[ref]{echanges@Échanges!protocoles@Protocoles!sparql@SPARQL}requêtes efficaces sur ces triplets --- cette reconstruction pouvant être personnalisée selon la requête effectuée.

\bigskip
\bigskip
Avec l'apparition du Web de données, un changement d'échelle des référentiels a lieu: ils cessent d'être utilisés par leur seul créateur dès lors qu'ils sont transformés puis envoyés dans le Web de données, ils peuvent désormais être partagés et réutilisés grâce aux URIs. L'utilisation d'un protocole existant, ainsi que d'un nouveau format d'échange, a permis de s'éloigner des modèles d'interopérabilité par conversion et copie, ou par le plus petit dénominateur commun: les référentiels sont des nœuds autour desquels les jeux de données sont rattachés\footnote{C'est l'intéropérabilité de la \og roue et de l'essieu\fg{}, ou\og hub and spoke\fg{}, décrite dans \cite{bermes_convergence_2013}. Voir \reference{annexe_types_interop} (\reference{hub_spoke})}. 
\section{\label{II-A-2}La mise en commun de référentiels au service des institutions}
\titreEntete{La mise en commun de référentiels au service des institutions}

\begin{citationLongue}
	Le travail d’alignement, c’est-à-dire de mise en relation, des référentiels entre eux dans l’objectif de créer du lien (et donc de l’interopérabilité) entre les bases et au-delà entre les institutions, initié dans le cadre de la réflexion autour du Web de données, va se poursuivre pour faciliter le maintien du référentiel et son enrichissement.\footcite{poupeau_reflexions_2018}
\end{citationLongue}
\medskip
La mise en commun de référentiels est essentielle pour créer de l'interopérabilité. Une mise en commun de référentiels au sein même d'une institution est possible de manière à créer du lien entre ses données, mais l'utilisation de plusieurs référentiels externes permet souvent d'exprimer plus de relations et de propriétés que la simple utilisation de référentiels internes.\\

C'est pourquoi les institutions patrimoniales et culturelles se sont engagées dans le Web de données très tôt et ont permis l'émergence de référentiels internationaux qui font aujourd'hui autorité. Pour accroître encore la puissance de ces référentiels, des passerelles sont créées entre les référentiels, de manière à créer plus de liens.

\subsection{\label{II-A-2-a}L'adoption du Web de données en institutions patrimoniales}
\titreEntete{L'adoption du Web de données en institutions patrimoniales}

\nP{Tim}{Berners-Lee}, dans sa feuille de route pour le Web sémantique en 1998, plaide pour le Web de données. Seulement, sa publication de 2006 rappelle les avantages de ce Web de données ainsi que les pratiques qui y sont liées. En effet, ces technologies étant nouvelles au début des années 2000, elles sont utilisées principalement pour de la recherche: les pratiques du Web sémantique sont par conséquent individuelles et peut conformes aux recommandations de \nP{Tim}{Berners-Lee} pour le Web sémantique. La finalité n'étant pas la publication sur le Web, ces données sont peu exploitables et parfois non accessibles\footnote{\nP{Tim}{Berners-Lee} dans la publication de 2006 se plaint de cette production conséquente de triplets non accessibles dans le Web sémantique: \og Many research and evaluation projects in the few years of the Semantic Web technologies produced ontologies, and significant data stores, but the data, if available at all, is buried in a zip archive somewhere, rather than being accessible on the web as linked data.\fg{}. Voir \cite{berners-lee_linked_2006}.}.\\

À la suite des travaux d'un groupe de travail du W3C, le Semantic Web Education and Outreach Interest Group (SWEO)\footnote{\url{https://www.w3.org/blog/SWEO/}}, destiné à promouvoir les technologies du Web sémantique, l'initiative \og Linking Open Data\fg{}\footnote{Cette initiative est aujourd'hui omniprésente dans le Web de données: en juillet 2020, 1260 jeux de données sont présents dans le \textit{Linked Open Data Cloud}. Leur représentation graphique, guidée par les liens entre ces jeux de données, devient au fil des années un exercice de plus en plus complexe face à l'augmentation constante des jeux de données présents. Voir \url{https://www.lod-cloud.net/}. Voir \reference{annexe_lod}} naît pour encourager à la publication de données dépourvues de droits. La publication de DBpédia \footnote{\url{fr.dbpedia.org}} a permis, à partir des pages Wikipédia, de créer des triplets \ac{rdf} pouvant servir de support aux initiatives des institutions\footcite{bermes_convergence_2013} et d'améliorer la qualité des triplets existants en créant de nouveaux liens.\\

Quatre ans après la mise en œuvre du Linked Open Data, le nombre de jeux de données disponibles a été multiplié par vingt\footnote{12 en 2007, 203 en 2010: voir \url{https://www.lod-cloud.net/}.}. Pour évaluer l'efficacité du Web de données et les perspectives à venir dans le milieu bibliothéconomique, un nouveau groupe de travail du W3C est lancé en 2010, le \textit{Library Linked Data Incubator Group} (LLD-XG). dans son rapport final\footcite{baker_rapport_2012}, le groupe préconise d'accentuer encore la coopération entre les institutions, et de faire participer davantage les bibliothèques dans la réflexion du Web de données et l'élaboration de nouveaux standards. Aujourd'hui, l'essor du Web de données en bibliothèque se réalise autour de référentiels faisant autorité.\\

Le groupe LLD-XG a permis aux institutions patrimoniales de se concentrer davantage sur la publication des données. Ainsi, la Bibliothèque nationale de France (BNF) inaugure \href{data.bnf.fr}{la plateforme Data BNF} en 2011 pour ouvrir son catalogue ainsi que les données d'autorité au format \ac{rdf}.

\subsection{\label{II-A-2-b}Utiliser des vocabulaires de valeurs}
\titreEntete{Utiliser des vocabulaires de valeurs}

Au-delà de l'interopérabilité souhaitée entre les institutions patrimoniales, les référentiels publiés dans le Web sémantique permettent une utilisation dans des domaines différents: l'utilisateur peut ainsi utiliser plusieurs référentiels du Web de données pour décrire ses données. Ces référentiels deviennent des référentiels de valeurs, compris comme un \og ensemble de termes organisés en système de connaissance\fg{}\footcite[p.47]{dalbin_approches_2011}. Ils sont l'essieu de cette interopérabilité de la roue et de l'essieu. Les autorités \ac{lcsh} de la Library of Congress sont un de ces référentiels de valeurs\footnote{Voir \reference{I-A-1}.}. \ac{rameau}, les autorités de la BNF, sont créées à partir de \ac{lcsh} et des données de la BNF. Elles permettent d'être utilisées dans plusieurs catalogues, celui de la BNF, mais également celui du Système universitaire de documentation (SUDOC).\\

À partir d'un seul référentiel commun et partagé, que chaque utilisateur --- institution --- peut mettre à jour, plusieurs jeux de donnés peuvent être décrits et indexés. Cependant, la publication de référentiels dans le Web de données n'est pas la priorité des institutions ni leur objectif initial; cette publication n'intervient qu'après l'opération de catalogage qui aura nécessité la création de nouvelles vedettes.

\subsection{\label{II-A-2-c}Créer des passerelles entre les référentiels}
\titreEntete{Créer des passerelles entre les référentiels}

Le parcours de liens de ressources en ressources, ainsi que l'alignement des référentiels entre eux, permet la création de passerelles et un enrichissement infini de chaque référentiel. Le lien, une nouvelle fois, est essentiel.\\

D'abord, le parcours de liens permet des rebonds entre les référentiels. Cette interopérabilité par parcours de liens \footnote{\og follow your nose\fg{} in \cite{bermes_convergence_2013}. Voir \reference{annexe_types_interop} (\reference{interop_follow_nose}).} conduit à la découverte de nouvelles ressources que l'utilisateur n'aurait pas trouvées de lui-même. Ainsi, les vedettes \ac{lcsh} renvoient vers les vedettes identiques ou similaires d'autres référentiels, tels que \ac{rameau} ou \ac{oclc}. De même que pour les liens entre les vedettes de \ac{lcsh} selon le type de relations, ces liens externes sont également séparés selon la relation de la vedette avec les vedettes visées par les liens\footnote{Voir \reference{lcsh_liens}.\\USDA: the National Agricultural Library's Agricultural Thesaurus. Voir \url{https://agclass.nal.usda.gov/}.\\
YSO: Yleinen suomalainen ontologia. Voir \url{}.https://finto.fi/yso/fi/}.
\begin{figure}[!h]
	\centering
	\begin{pspicture}(0,0.8)(16,15.2)
		%cercle central
		\pscircle(8.2,8){1.3}
		\uput[0](7.4,8.6){Vedette}
		\uput[0](7.2,8){\href{https://id.loc.gov/authorities/subjects/sh85133456.html}{television}}
		\uput[0](7.2,7.4){de \ac{lcsh}}
		
		%rectangles de légende
		\psframe[fillstyle=solid,fillcolor=lightgray](11,13)(15.5,14.5)
		\uput[0](11,14){Correspondance exacte}
		\uput[0](11.4,13.4){entre les vedettes}
		
		\psframe[fillstyle=solid,fillcolor=lightgray](11,1)(14.5,2)
		\uput[0](11,1.5){Vedettes proches}
		
		\psframe[fillstyle=solid,fillcolor=lightgray](1,5)(4,6)
		\uput[0](1,5.5){Vedettes filles}
		
		%3 barres de séparation
		\psline[linewidth=0.1](7.2,7.2)(1.1,1.1)
		\psline[linewidth=0.1](9.4,7.5)(15,5.5)
		\psline[linewidth=0.1](8.2,9.3)(8.2,15)
		
		%bulles de liens
		\pscircle(13,10){1.3}
		\uput[0](12,10.5){\href{http://lod.nal.usda.gov/nalt/51607}{television}}
		\uput[0](12.6,10){de}
		\uput[0](12,9.5){l'USDA}
		
		\pscircle(11.6,4.4){1.3}
		\uput[0](10.5,4.8){\href{http://data.bnf.fr/ark:/12148/cb119336465}{Télévision}}
		\uput[0](11.2,4.3){de}
		\uput[0](10.8,3.8){la BNF}
		\pscircle(9,2.5){1.3}
		\uput[0](8,3){\href{http://id.worldcat.org/fast/1146535}{Television}}
		\uput[0](8.6,2.5){de}
		\uput[0](8.1,2){l'\ac{oclc}}
		\pscircle(6,3.5){1.3}
		\uput[0](5.2,4){\href{http://www.yso.fi/onto/yso/p5759}{televisio}}
		\uput[0](5.7,3.5){de}
		\uput[0](5.3,3){l'YSO}
		
		\pscircle(3.5,8){1.3}
		\uput[0](3.2,8){\dots}
		\pscircle(2,12){1.3}
		\uput[0](0.9,12.7){\href{http://id.worldcat.org/fast/1146565}{Television}}
		\uput[0](0.8,12.2){\href{http://id.worldcat.org/fast/1146565}{-- Influence}}
		\uput[0](1.7,11.7){de}
		\uput[0](1.2,11.2){l'\ac{oclc}}
		\pscircle(6,11.4){1.3}
		\uput[0](4.9,12.1){\href{http://id.worldcat.org/fast/1146614}{Television}}
		\uput[0](4.8,11.6){\href{http://id.worldcat.org/fast/1146614}{-- Research}}
		\uput[0](5.6,11.1){de}
		\uput[0](5.2,10.6){l'\ac{oclc}}
		
		%liens
		\psline(9.4,8.5)(11.8,9.4)
		\psline(9.1,7.1)(10.7,5.3)
		\psline(8.4,6.75)(9,3.8)
		\psline(6.5,4.6)(7.7,6.8)
		\psline(6.9,8)(4.8,8)
		\psline(3,11.2)(7,8.6)
		\psline(6.8,10.4)(7.6,9.2)
	\end{pspicture}
	\caption[Modélisation des liens vers des référentiels externes présents dans la vedette \og television\fg{} de \ac{lcsh}]{Modélisation des liens vers des référentiels externes présents dans la vedette \og \href{https://id.loc.gov/authorities/subjects/sh85133456.html}{television}\fg{} de \ac{lcsh}}
	\label{lcsh_liens}
\end{figure}

Ensuite,des fichiers d'autorité sont nés d'alignements avec d'autres fichiers d'autorités. En effet, la redondance de certaines autorités dans plusieurs référentiels n'est pas opportune dans le Web de données: cela créé de la dissonance et empêche la naissance d'une autorité globale regroupant l'ensemble des informations et des données des autorités existantes. Dans ce but, plusieurs fichiers d'autorité comme le \ac{viaf}\footnote{\url{http://viaf.org/}} fusionnent les fichiers d'autorités de bibliothèques nationales du monde entier. Ce projet a été initié dès 2003 par la Library of Congress, la Deutsche Nationalbibliothek, la Bibliothèque nationale de France et \ac{oclc} Research\footcite{bermes_les_2013}, et compte aujourd'hui plusieurs dizaines de bibliothèques partenaires.\\

L'agrégation de multiples vedettes d'autorités identiques provenant de diverses institutions permet la création d'une \textit{super} fiche d'autorité dans \ac{viaf}, créée à partir de liens et n'affichant que des liens\footnote{L'autorité personne de \href{http://viaf.org/viaf/108762210/}{\nP{Jean-Luc}{Godard}} montre bien cette structuration des notices de \ac{viaf}: un graphe permet la modélisation des multiples institutions de récupération des notices d'autorité, et les thèmes et sujets liés à \nP{Jean-Luc}{Godard} sont listés dans la suite de la page de cette vedette.}. En raison de l'origine bibliothéconomique de \ac{viaf}, les formes retenues sont exprimées avec leur code \ac{marc} 100 ou 200 et celles similaires avec le code 400, de même que les sujets qui y sont liés avec les codes 5XX.\\


\bigskip
\bigskip
Les référentiels propres à chaque institution ont été transformés de manière à pouvoir être intégrés au Web sémantique. Avec cette augmentation de ressources, identifiées par des URIs et échangées par \ac{rdf}, de nombreux liens ont pu être créés entre les référentiels. Les données et les autorités ont ainsi été partagé et quelques référentiels jouent désormais un rôle central dans le Web de données grâce à leur taille et aux nombre de liens qui y renvoient, ou qu'ils renvoient.
\section{\label{II-A-3}Vers la fin de la notion de référentiels?}
\titreEntete{Vers la fin de la notion de référentiels?}

L'éclatement du document en données sur le Web a permis de grandes avancées pour les institutions patrimoniales qui partagent non plus des notices bibliographiques ou d'autorités, mais des données liées au travers d'URIs. Elles ont trouvé avec le Web de données un protocole ainsi qu'un format d'échange standardisés et utilisés par tous les utilisateurs. Cependant, ce règne de la donnée sur le Web conduit à de nouvelles réflexions quant à la définition des référentiels: personnes, lieux et sujets sont considérés comme des données de référence; pourquoi alors ne pas considérer une œuvre comme une donnée de référence elle aussi? Cette conceptualisation de la réalité conduit ainsi à repenser les modèles de données dans les institutions ainsi que les formats de description des documents afin de partager des formats et des standards sur le Web pour profiter au plus grand nombre.

\subsection{\label{II-A-3-a}Quand tout devient un potentiel référentiel}
\titreEntete{Quand tout devient un potentiel référentiel}

D'un référentiel qui était une liste de mots contrôlés et hiérarchisés avec les \textit{thesauri}, le Web de données le fait passer vers une nouvelle définition et de nouveaux usages: un référentiel est désormais un ensemble d'informations qui sont susceptibles d'être partagées puis réutilisées dans divers systèmes documentaires afin de créer du lien\footcite[§49]{bermes_les_2013}. Ces informations ne sont plus spécifiquement des termes choisis et contrôlés par des documentalistes: tout peut devenir information et par conséquent référentiel s'il y a des relations établies.\\

Dès la fin du \textsc{XX}\textsuperscript{ème} siècle, deux modèles voient le jour pour repenser la structure de la donnée et la place des référentiels. En 1996, la réflexion autour de la description des collections muséales aboutit à la création du modèle du \ac{cidoccrm}\footcite{noauthor_cidoc-crm_nodate} qui est un modèle orienté document -- objet --- permettant de pouvoir décrire les interactions de chaque objet avec d'autres entités.\\

En parallèle de ce modèle destiné aux descriptions de collections de musées, les bibliothèques mènent également une réflexion similaire entre 1992 et 1997, ce qui conduit à l'élaboration des modèles \ac{frbr}\footnote{Plusieurs modèles \ac{frbr} spécifiques ont vu le jour: les \ac{frad} (pour les données d'autorité; voir \cite{federation_internationale_des_associations_de_bibliothecaires_et_de_bibliotheques_fonctionnalites_2010}) et les \ac{frsad} (pour les sujets; voir \cite{federation_internationale_des_associations_de_bibliothecaires_et_de_bibliotheques_fonctionnalites_2010-1}), destinés à modéliser les relations et les entités des points d'accès.}. 
Il est nécessaire de s'attarder sur ces \ac{frbr} afin de mieux comprendre la structure du \textit{Lac de données} de l'\ac{ina} expliquée par la suite\footnote{Voir \reference{III-B}.}, bien que celle-ci ne soit pas exactement similaire aux \ac{frbr}.
Trois groupes sont distingués dans les \ac{frbr}\footnote{Le rapport détaillé des \ac{frbr} indique l'ensemble des groupes et des relations possibles, ce que nous ne développerons pas ici. Voir \cite{federation_internationale_des_associations_de_bibliothecaires_et_de_bibliotheques_fonctionnalites_2012}}: l'un correspond à la notice bibliographique elle-même, les deux autres aux points d'accès.\\

La notice bibliographique est, avec les \ac{frbr}, divisé en quatre sections, partant des caractéristiques propres de l'exemplaire décrit --- l'item ---, suivies par les caractéristiques de la publication auquel le document appartient --- la manifestation --- et par celles de son contenu --- l'expression ---, pour terminer avec celles de la création abstraite auquel le document appartient --- l'œuvre.\\

Le premier point d'accès est le groupe des personnes et des collectivités qui permettent de décrire les responsabilités de chacun --- auteur, producteur, ayant-droit, \dots ~ ---, de l'œuvre à l'item. Le second point d'accès permet la description du contenu, le sujet de l'activité intellectuelle ou artistique à travers des concepts, des objets, des événements ou des lieux.\\

La création de liens entre les différents groupes et les différentes entités permet une description fine des contenus, des responsabilités et des œuvres; la liaison entre des entités conformément au Web de données; la création infinie de référentiels avec une œuvre pouvant être sujet d'une autre œuvre par exemple\footnote{Voir \reference{annexe_nvx_modeles} (\reference{frbr}).}.\\

Avec ces nouveaux modèles --- \ac{cidoccrm} et \ac{frbr} --- , toutes les notions peuvent devenir des référentiels: les œuvres, les expressions, les sujets, les familles, \dots~. Leur avantage est leur partage désormais possible avec d'autres métiers, et plus largement sur le Web de données puisqu'une exposition en \ac{rdf} est possible et réalisée. Plusieurs institutions peuvent alors utiliser les données modélisées selon les \ac{frbr} afin de créer plus de liens que si elles ne s'étaient appuyées uniquement sur les référentiels communs, tels que \ac{lcsh} ou \ac{rameau}, pour établir la description de leur objet.

\subsection{\label{II-A-3-b}Vers une uniformisation internationale de la donnée sur le Web et l'adoption de \ac{rdf} comme format de production}
\titreEntete{Uniformisation de la donnée sur le Web et adoption de RDF}

L'apparition des nouveaux modèles centrés sur les entités, dans lesquels toutes les entités sont potentiellement partageables et utilisables par d'autres pour servir de référentiel, favorise de nouvelles réflexions sur le catalogage des documents: la finalité devenant de plus en plus souvent la publication des données sur le Web sémantique, est-il toujours nécessaire de cataloguer dans un format pour ensuite convertir les données en \ac{rdf}?

La nécessité d'améliorer les règles de catalogage dans les pays anglo-saxons dans les années 2000 a conduit à l'utilisation des nouveaux modèles \ac{frbr} dans le code \ac{rda} publié en 2010. Ce changement majeur est testé à partir de 2011 à la Library of Congress, puis adopté en 2013 dans les pays anglo-saxons; la France et ses agences bibliographies ABES et BNF tente d'intégrer ces nouvelles règles.\\

En effet, \ac{rda} est nativement pensé autour du Web de données et de l'utilisation en ligne des données de manière à pouvoir ensuite exprimer les entités et leurs relations sous la forme de triplets \ac{rdf} grâce à l'attribution d'un identifiant à chaque élément ou valeur\footnote{Les vocabulaires \ac{rda} ont fait l'objet d'un groupe de travail après la création du code de catalogage \ac{rda}.}. Ce code de catalogage a vocation à être internationalement utilisé, de manière à uniformiser les données produites et favoriser ainsi les échanges.\\

En adoptant \ac{rda}, le format \ac{marc} est délaissé et semble ne plus pouvoir répondre aux changements impliqués par le Web. Cependant, \ac{rda}, publié et utilisé, subit déjà des évolutions et un nouveau modèle de données, nativement centré sur \ac{rdf}, est créé en 2012: Bibframe\footcite{library_of_congress_overview_2016}. Le format \ac{marc} est abandonné au profit d'un modèle pensé autour de \ac{rdf}, et par conséquent des usages numériques des utilisateurs. Le modèle de données de Bibframe est semblable à celui des \ac{frbr} avec les \textit{work}, les \textit{instance} et les \textit{item}. Seulement, \ac{rdf} n'apparaît plus comme un format de sortie des données après leur catalogage; il est désormais le format natif de catalogage. Ainsi, chaque donnée, chaque entité, devient un référentiel en ce qu'elle est nativement liée à d'autres ressources.

%conclu
\bigskip
\bigskip
L'impact du Web de données sur la notion de référentiel s'étend au-delà de la réflexion sur la définition et la structure d'un référentiel: toute la modélisation des données des institutions est remise en question. Ces dernières doivent s'adapter et tenter de trouver de nouveaux modèles de données qui puissent répondre à la fois à leurs besoins internes de description et de signalement des collections, et à la fois aux besoins croissants des utilisateurs sur le Web de pouvoir trouver des ressources en pouvant s'affranchir des barrières technologiques et institutionnelles. Ainsi, il n'y a plus de référentiels dans lesquels l'utilisateur peut aller, les catalogues des institutions ouverts en \ac{rdf} sont eux-mêmes ces référentiels.

%conclu
\bigskip
\bigskip
\bigskip
L'apparition du Web et son adoption par le public l'a rendu indispensable pour les institutions et l'obtention d'informations. Les frontières auparavant visibles avec les portails ont été effacées avec le Web de données qui permet la liaison des jeux de données entre eux, peu importe leur provenance et leur lieu de stockage. La réflexion entamée dès le début des années 2000 a permis la réalisation des premiers jeux de données du Web sémantique au milieu de cette décennie. Les possibilités offertes par le Web de données ont conduit l'ensemble des institutions à adopter cette exposition des données, de manière à enrichir leurs propres données avec l'ajout de liens multiples.\\

L'effacement de la distinction entre document et référentiel conduit à la disparition progressive de la notion de référentiel au profit de nouveaux modèles de données reliées entre elles à travers un modèle partagé, \ac{rdf}. Cependant, la création de triplets \ac{rdf} nécessite toujours un certain type de référentiel, l'ontologie. Cette dernière permet d'établir les liens entre les ressources et de les typer.