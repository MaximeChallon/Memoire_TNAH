\section{\label{II-A-3}Vers la fin de la notion de référentiels?}
\titreEntete{Vers la fin de la notion de référentiels?}

L'éclatement du document en données sur le Web a permis de grandes avancées pour les institutions patrimoniales qui partagent non plus des notices bibliographiques ou d'autorités, mais des données liées au travers d'URIs. Elles ont trouvé avec le Web de données un protocole ainsi qu'un format d'échange standardisés et utilisés par tous les utilisateurs. Cependant, ce règne de la donnée sur le Web conduit à de nouvelles réflexions quant à la définition des référentiels: personnes, lieux et sujets sont considérés comme des données de référence; pourquoi alors ne pas considérer une œuvre comme une donnée de référence elle aussi? Cette conceptualisation de la réalité conduit ainsi à repenser les modèles de données dans les institutions ainsi que les formats de description des documents, afin de partager, sur le Web, des formats et des standards pour profiter au plus grand nombre.

\subsection{\label{II-A-3-a}Quand tout devient un potentiel référentiel}
\titreEntete{Quand tout devient un potentiel référentiel}

Ce référentiel, qui était une liste de mots contrôlés et hiérarchisés avec les \textit{thesauri}, se tourne, grâce au Web de données, vers une nouvelle définition et de nouveaux usages: un référentiel est désormais un ensemble d'informations susceptibles d'être partagées puis réutilisées dans divers systèmes documentaires pour créer du lien\footcite[§49]{bermes_les_2013}. Ces informations ne sont plus spécifiquement des termes choisis et contrôlés par des documentalistes: tout peut devenir information et par conséquent référentiel si des relations sont établies.\\

Dès la fin du \textsc{XX}\textsuperscript{ème} siècle, deux modèles voient le jour pour repenser la structure de la donnée et la place des référentiels. En 1996, la réflexion autour de la description des collections muséales permet la création du modèle du \index[ref]{modelisation@Modélisation!cidoc@CIDOC-CRM}\ac{cidoccrm}\footcite{noauthor_cidoc-crm_nodate}, modèle orienté document -- objet --- permettant de pouvoir décrire les interactions de chaque objet avec d'autres entités.\\

Parallèlement à ce modèle destiné aux descriptions de collections de musées, les bibliothèques mènent également une réflexion similaire entre 1992 et 1997, permettant ainsi l'élaboration des modèles \index[ref]{modelisation@Modélisation!frbr@FRBR}\ac{frbr}\footnote{Plusieurs modèles \ac{frbr} spécifiques ont vu le jour: les \index[ref]{modelisation@Modélisation!frad@FRAD}\ac{frad} (pour les données d'autorité; voir \cite{federation_internationale_des_associations_de_bibliothecaires_et_de_bibliotheques_fonctionnalites_2010}) et les \index[ref]{modelisation@Modélisation!frsad@FRSAD}\ac{frsad} (pour les sujets; voir \cite{federation_internationale_des_associations_de_bibliothecaires_et_de_bibliotheques_fonctionnalites_2010-1}), destinés à modéliser les relations et les entités des points d'accès.}. 
Il est nécessaire de s'attarder sur ces \index[ref]{modelisation@Modélisation!frbr@FRBR}\ac{frbr} afin de mieux comprendre la structure du \index[ref]{led@Linked Enterprise Data (LED)!ldd@Lac de données (INA)}\index[ref]{modelisation@Modélisation!ldd@Lac de données (INA)}\ldd de l'\ac{ina} expliquée par la suite\footnote{Voir \reference{III-B}.}, bien que celle-ci ne soit pas exactement similaire aux \ac{frbr}.
Trois groupes sont distingués dans les \ac{frbr}\footnote{Le rapport détaillé des \ac{frbr} indique l'ensemble des groupes et des relations possibles, ce que nous ne développerons pas ici. Voir \cite{federation_internationale_des_associations_de_bibliothecaires_et_de_bibliotheques_fonctionnalites_2012}}: l'un correspond à la notice bibliographique elle-même, les deux autres aux points d'accès.\\

La notice bibliographique est, avec les \ac{frbr}, divisée en quatre sections, partant des caractéristiques propres de l'exemplaire décrit --- l'item ---, puis par les caractéristiques de la publication auquel le document appartient --- la manifestation --- et par celles de son contenu --- l'expression ---, pour terminer avec celles de la création abstraite auquel le document appartient --- l'œuvre.\\

Le premier point d'accès est le groupe des personnes et des collectivités qui permettent de décrire les responsabilités de chacun --- auteur, producteur, ayant-droit, \dots ~ ---, de l'œuvre à l'item. Le second point d'accès permet la description du contenu, le sujet de l'activité intellectuelle ou artistique à travers des concepts, des objets, des événements ou des lieux.\\

La création de liens entre les différents groupes et les différentes entités permet une description fine des contenus, des responsabilités et des œuvres. Elle permet également la liaison entre des entités conformément aux principes du Web de données. Enfin, la création infinie de référentiels, avec une œuvre pouvant être sujet d'une autre œuvre par exemple, est possible\footnote{Voir \reference{annexe_nvx_modeles} (\reference{frbr}).}.\\

Avec ces nouveaux modèles --- \index[ref]{modelisation@Modélisation!cidoc@CIDOC-CRM}\ac{cidoccrm} et \index[ref]{modelisation@Modélisation!frbr@FRBR}\ac{frbr} --- , toutes les notions peuvent devenir des référentiels: les œuvres, les expressions, les sujets, les familles, \dots~. Leur avantage est leur partage désormais possible avec d'autres métiers, et plus largement sur le Web de données puisqu'une exposition en \index[ref]{echanges@Échanges!formats@Formats!rdf@RDF}\ac{rdf} est possible et réalisée. Plusieurs institutions peuvent alors utiliser les données modélisées selon les \ac{frbr} afin de créer plus de liens que si elles ne s'étaient appuyées, pour établir la description de leur objet, que sur les référentiels communs, tels que \index[ref]{lod@Linked Open Data (LOD)!lcsh@LCSH}\index[ref]{autorites@Autorités!lcsh@LCSH}\ac{lcsh} ou \index[ref]{lod@Linked Open Data (LOD)!rameau@RAMEAU}\index[ref]{autorites@Autorités!rameau@RAMEAU}\ac{rameau}.

\subsection{\label{II-A-3-b}Vers une uniformisation internationale de la donnée sur le Web et l'adoption de \ac{rdf} comme format de production}
\titreEntete{Uniformisation de la donnée sur le Web et adoption de RDF}

L'apparition des nouveaux modèles centrés sur les entités, dans lesquels toutes les entités sont potentiellement partageables et utilisables par d'autres pour servir de référentiel, favorise de nouvelles réflexions sur le catalogage des documents: la finalité devenant de plus en plus souvent la publication des données sur le \index[ref]{typologie@Typologie!graphe@Graphe de nœuds et de liens}Web sémantique, est-il toujours nécessaire de cataloguer dans un format pour ensuite convertir les données en \ac{rdf}?\\

La nécessité d'améliorer les règles de catalogage dans les pays anglo-saxons dans les années 2000 a conduit à l'utilisation des nouveaux modèles \index[ref]{modelisation@Modélisation!frbr@FRBR}\ac{frbr} dans le code \index[ref]{echanges@Échanges!formats@Formats!rda@RDA}\ac{rda} publié en 2010. Ce changement majeur est testé à partir de 2011 à la Library of Congress, puis adopté en 2013 dans les pays anglo-saxons; la France et ses agences bibliographies ABES et BNF tentent d'intégrer ces nouvelles règles.\\

En effet, \ac{rda} est nativement pensé autour du Web de données et de l'utilisation en ligne des données, de manière à pouvoir ensuite exprimer les entités et leurs relations sous la forme de triplets \index[ref]{echanges@Échanges!formats@Formats!rdf@RDF}\ac{rdf} grâce à l'attribution d'un identifiant à chaque élément ou valeur\footnote{Les vocabulaires \ac{rda} ont fait l'objet d'un groupe de travail après la création du code de catalogage \ac{rda}.}. Ce code de catalogage est destiné à être internationalement utilisé, de manière à uniformiser les données produites et favoriser ainsi les échanges.\\

En adoptant \index[ref]{echanges@Échanges!formats@Formats!rda@RDA}\ac{rda}, le format \ac{marc} est délaissé et semble ne plus pouvoir répondre aux changements impliqués par le Web. Cependant, \ac{rda}, publié et utilisé, subit déjà des évolutions et un nouveau modèle de données, nativement centré sur \ac{rdf}, est créé en 2012: \index[ref]{echanges@Échanges!formats@Formats!bibframe@Bibframe}Bibframe\footcite{library_of_congress_overview_2016}. Le format \index[ref]{echanges@Échanges!formats@Formats!marc@MARC}\ac{marc} est abandonné au profit d'un modèle pensé autour de \ac{rdf}, et, par conséquent, des usages numériques des utilisateurs. Le modèle de données de Bibframe est semblable à celui des \index[ref]{modelisation@Modélisation!frbr@FRBR}\ac{frbr} avec les \textit{work}, les \textit{instance} et les \textit{item}. Seulement, \index[ref]{echanges@Échanges!formats@Formats!rdf@RDF}\ac{rdf} n'apparaît plus comme un format de sortie des données après leur catalogage; il est désormais le format natif de catalogage. Ainsi, chaque donnée, chaque entité, devient un référentiel en ce qu'elle est nativement liée à d'autres ressources.

%conclu
\bigskip
\bigskip
L'impact du Web de données sur la notion de référentiel s'étend au-delà de la réflexion sur la définition et la structure d'un référentiel: toute la modélisation des données des institutions est remise en question. Ces dernières doivent s'adapter et tenter de trouver de nouveaux modèles de données qui puissent répondre d'une part à leurs besoins internes de description et de signalement des collections, d'autre part aux besoins croissants des utilisateurs sur le Web à la recherche des ressources libérées des barrières technologiques et institutionnelles. Ainsi, il n'y a plus de référentiels dans lesquels l'utilisateur peut aller, les catalogues des institutions ouverts en \ac{rdf} sont eux-mêmes ces référentiels.