\section{\label{II-A-1}Le web de données: naissance et principes}
\titreEntete{Le web de données: naissance et principes}

La recherche d'un protocole et d'un format d'échange de données entre les institutions est constante. Nous avons évoqué précédemment\footnote{Voir \reference{I-C-2}.} les difficultés rencontrées avec les protocoles Z039-50 et \ac{oaipmh}. Ces derniers sont insuffisants pour permettre un partage massif de données et de référentiels, mais ils ont permis l'évolution de la réflexion sur l'interopérabilité. Le grand bouleversement est survenu au milieu des années 2000 avec le Web de données qui ne créé pas de protocole nouveau, mais en utilise un largement répandu et utilisé, \ac{http}. Les règles édictées ont permis sa bonne utilisation pour créer le web de données et la seule naissance d'un format d'échange, \ac{rdf}.

\subsection{\label{II-A-1-a}Créer un modèle de données nativement compatible avec le web: le web de données}
\titreEntete{Un modèle de données nativement compatible avec le web}

\subsubsection{\label{II-A-1-a-i}Naissance du Web de données}
\titreEntete{Naissance du Web de données}

Dès 1989, \nP{Tim}{Berners-Lee} propose un \og espace d'information commun\fg{}\footnote{\og poll of information\fg{} in \cite{berners-lee_information_1989}.} où les textes seraient liés par des liens\footnote{\og a web of notes with links\fg{} in \cite{berners-lee_information_1989}}. Il propose ainsi un modèle de nœuds et de liens qui permet d'entrer n'importe quel type d'informations, de manière, grâce aux liens, à trouver une ressource sans avoir eu à la chercher. Cependant, plusieurs difficultés s'offrent encore: il est nécessaire, comme dans un arbre, d'avoir des nœuds uniques; la modélisation du monde réel est impossible, et le modèle de nœuds et de liens se heurte aux mêmes réflexions que Porphyre et les encyclopédistes quant à la possibilité de représenter le monde en un seul arbre. Pour y faire face, \nP{Tim}{Berners-Lee} propose le lien hypertexte comme solution.\\

En 1994, \nP{Tim}{Berners-Lee} continue la réflexion sur le Web et les liens hypertexte\footcite{berners-lee_plenary_1994}. En 1989, seuls les documents étaient mis sur le Web et \nP{Tim}{Berners-Lee} décrivait la manière de les relier entre eux. En 1994, il propose d'intégrer au Web des données du monde réel qui seraient reliées par des liens hypertexte, comme les documents, toujours sous la forme de nœuds et de liens. Cette proposition débute la réflexion sur le Web de données. Cependant, la réalité n'est pas compréhensible d'une machine, et les liens ne peuvent se comprendre que par leur contexte: l'ajout de valeurs aux relations permet de donner du sens au Web --- c'est le Web sémantique. \nP{Tim}{Berners-Lee} se fait ainsi le promoteur de la donnée structurée à la fois pour la machine et pour l'humain.\\

Une feuille de route est par conséquent écrite par \nP{Tim}{Berners-Lee} en 1998\footcite{berners-lee_semantic_1998}. Il y évoque pour la première fois le terme \og web de données\fg{}\footnote{\og web of data\fg{} in \cite{berners-lee_semantic_1998}}, mais la feuille de route n'est pas appliquée et il faut attendre 2006 et la publication \textit{Linked data}\footcite{berners-lee_linked_2006} pour que les recommandations du Web sémantique soient expliquées et adoptées. Fondamentale, cette publication évoque les principes du Web de données actuel en décrivant les bonnes pratiques à adopter. Le Web est ainsi perçu comme une \og base de données globale\fg{}\footcite[§29]{bermes_2_2013} où les données sont reliées de la même manière que les documents HTML avec des liens hypertexte.\\

Cette interopérabilité basée sur les liens, théorisée notamment par \nP{Tim}{Berners-Lee}, est à l'origine du web de données et de l'actuel partage de données et de référentiels entre les institutions patrimoniales. Le lien apparaît comme essentiel et permet de décloisonner chaque institution pour les faire communiquer ensemble de manière à améliorer la recherche de données et de documents par l'utilisateur final\footnote{L'utilisateur final n'est pas seulement le grand public, il peut être chercheur, professionnel dans une institution, consommateur commercial, \dots}.

\subsubsection{\label{II-A-1-a-ii}Principes généraux}
\titreEntete{Principes généraux}

La publication de 2006 de \nP{Tim}{Berners-Lee} décrit précisément les principes du web de données qu'il est nécessaire de développer pour comprendre l'évolution des pratiques documentaires des institutions depuis le milieu des années 2000. Ces principes s'appuient sur l'architecture du Web existant et ne visent pas la création d'un Web: l'interopérabilité des données doit passer par une interopérabilité des protocoles et des formats utilisés avec ceux du Web qui connaît une utilisation croissante en 2006.\\

Le premier principe évoqué est celui de l'utilisation des Uniform Resource Identifier (URI) comme clé unique d'une ressource: en cas de non utilisation du standard des URIs, le Web sémantique n'est plus possible\footnote{\og If it doesn't use the universal URI set of symbols, we don't call it Semantic Web\fg{} in \cite{berners-lee_linked_2006}.}. En effet, une URI possède une syntaxe précise qu'il convient de respecter et d'adopter: scheme:autorité/chaîne\_de\_caractères\footcite[§40]{bermes_2_2013}.\\

Le second principe est celui de l'utilisation du protocole du WorldWideWeb, \ac{http}.\\

Le troisième principe impose le renvoi d'informations et de données dans des formats standards du Web, en \ac{rdf}/XML, ou en N3 ou Turtle\footcite{berners-lee_linked_2006}. Tous ces formats admettent le langage de requête SPARQL.\\

Enfin, le quatrième principe est celui de la création de liens entre les ressources --- donc les URIs --- sans lesquels les efforts réalisés avec les trois premiers principes sont vains. Ainsi, un web fiable, sans frontières est créé\footnote{\og serious, unbounded web in which one can find al kinds of things, just as on the hypertext web we have managed to build\fg{} in \cite{berners-lee_linked_2006}}; l'utilisateur peut y naviguer facilement grâce aux liens hypertextes. Le web de données est par conséquent moins une base de données qu'un lieu où les liens donnent de la valeur aux ressources liées: plus une ressource possède de liens, plus celle-ci a une description précise et fiable, plus elle devient visible à l'utilisateur.\\

Nous l'aurons remarqué, depuis le début de cette description du Web de données, la notion de référentiel semble d'estomper au profit de ressources et de données liées. En effet, un référentiel n'est qu'une mise en forme spécifique d'un jeu de données selon une structure propre à son producteur. Cette spécificité de chaque jeu de données n'est pas valable dans le Web de données: un retour à la donnée est nécessaire, les liens qui lui seront affectés permettront alors de représenter son ancienne structure dans le référentiel. \nP{Tim}{Berners-Lee} décrit cette nécessité de se dégager de ses propres formats sur le Web pour évoluer vers des formats compréhensibles par une machine: il créé l'échelle des cinq étoiles dans ce but, le \ac{rdf} étant la meilleure des solutions d'exposition des données.

\subsection{\label{II-A-1-b}Inventer un format d'échange compatible avec ce modèle de données: RDF}
\titreEntete{Inventer un format d'échange compatible avec ce modèle de données}

\ac{http} est le protocole utilisé pour le Web de données; le format d'échange est \ac{rdf}. C'est un standard développé pour le Web capable d'assurer l'interopérabilité des données. Seules des URIs peuvent constituer des ressources. Ces ressources sont ensuite reliées grâce à un lien typé par le formalisme offert par \ac{rdf}. \ac{rdf} n'offre pas, comme cela est le cas avec les encodages XML archivistiques ou codicologiques, un schéma prédéfini, mais un modèle logique de description des ressources.\\

\ac{rdf} ne permet pas de relier directement deux ressources, seulement de typer leur relation afin que la machine puisse interpréter la nature de leur lien, peut importe la localisation des deux ressources. Ainsi, la forme d'un triplet \ac{rdf} reflète cette distinction: le \og sujet\fg est nécessairement une ressource --- par conséquent une URI ---, il est suivi d'un \og prédicat\fg{} qui défini la nature de la relation avec le troisième élément du triplet, l'\og objet\fg{} qui peut être une ressource ou un littéral. Le triplet est donc une simple phrase sujet-verbe-complément compréhensible par une machine. Une ressource pouvant être à la fois sujet dans un triplet, prédicat dans un autre, ou objet dans d'autres, un graphe se construit alors. L'information est donc totalement déconstruite pour un humain, mais elle devient compréhensible par une machine, qui permet ensuite la reconstruction de l'information par des requêtes efficaces sur ces triplets --- cette reconstruction pouvant être personnalisée selon la requête effectuée.

\bigskip
\bigskip
Avec l'apparition du Web de données, un changement d'échelle des référentiels a lieu: ils cessent d'être utilisés par leur seul créateur dès lors qu'ils sont transformés puis envoyés dans le Web de données, ils peuvent désormais être partagés et réutilisés grâce aux URIs. L'utilisation d'un protocole existant, ainsi que d'un nouveau format d'échange, a permis de s'éloigner des modèles d'interopérabilité par conversion et copie, ou par le plus petit dénominateur commun: les référentiels sont des nœuds autour desquels les jeux de données sont rattachés\footnote{C'est l'intéropérabilité de la \og roue et de l'essieu\fg{}, ou\og hub and spoke\fg{}, décrite dans \cite{bermes_2_2013}. Voir \reference{annexe_types_interop} (\reference{hub_spoke})}. 