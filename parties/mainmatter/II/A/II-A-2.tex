\section{\label{II-A-2}La mise en commun de référentiels au service des institutions}
\titreEntete{La mise en commun de référentiels au service des institutions}

\begin{citationLongue}
	Le travail d’alignement, c’est-à-dire de mise en relation, des référentiels entre eux dans l’objectif de créer du lien (et donc de l’interopérabilité) entre les bases et au-delà entre les institutions, initié dans le cadre de la réflexion autour du Web de données, va se poursuivre pour faciliter le maintien du référentiel et son enrichissement.\footcite{poupeau_reflexions_2018}
\end{citationLongue}
\medskip
La mise en commun de référentiels est essentielle pour créer de l'interopérabilité. Une mise en commun de référentiels au sein même d'une institution est possible de manière à créer du lien entre ses données, mais l'utilisation de plusieurs référentiels externes permet souvent d'exprimer plus de relations et de propriétés que la simple utilisation de référentiels internes.\\

C'est pourquoi les institutions patrimoniales et culturelles se sont engagées dans le Web de données très tôt et ont permis l'émergence de référentiels internationaux qui font aujourd'hui autorité. Pour accroître encore la puissance de ces référentiels, des passerelles sont créées entre les référentiels, de manière à créer plus de liens.

\subsection{\label{II-A-2-a}L'adoption du Web de données en institutions patrimoniales}
\titreEntete{L'adoption du Web de données en institutions patrimoniales}

\nP{Tim}{Berners-Lee}, dans sa feuille de route pour le Web sémantique en 1998, plaide pour le Web de données. Seulement, sa publication de 2006 rappelle les avantages de ce Web de données ainsi que les pratiques qui y sont liées. En effet, ces technologies étant nouvelles au début des années 2000, elles sont utilisées principalement pour de la recherche: les pratiques du Web sémantique sont par conséquent individuelles et peut conformes aux recommandations de \nP{Tim}{Berners-Lee} pour le Web sémantique. La finalité n'étant pas la publication sur le Web, ces données sont peu exploitables et parfois non accessibles\footnote{\nP{Tim}{Berners-Lee} dans la publication de 2006 se plaint de cette production conséquente de triplets non accessibles dans le Web sémantique: \og Many research and evaluation projects in the few years of the Semantic Web technologies produced ontologies, and significant data stores, but the data, if available at all, is buried in a zip archive somewhere, rather than being accessible on the web as linked data.\fg{}. Voir \cite{berners-lee_linked_2006}.}.\\

À la suite des travaux d'un groupe de travail du W3C, le Semantic Web Education and Outreach Interest Group (SWEO)\footnote{\url{https://www.w3.org/blog/SWEO/}}, destiné à promouvoir les technologies du Web sémantique, l'initiative \og Linking Open Data\fg{}\footnote{Cette initiative est aujourd'hui omniprésente dans le Web de données: en juillet 2020, 1260 jeux de données sont présents dans le \textit{Linked Open Data Cloud}. Leur représentation graphique, guidée par les liens entre ces jeux de données, devient au fil des années un exercice de plus en plus complexe face à l'augmentation constante des jeux de données présents. Voir \url{https://www.lod-cloud.net/}. Voir \reference{annexe_lod}} naît pour encourager à la publication de données dépourvues de droits. La publication de DBpédia \footnote{\url{fr.dbpedia.org}} a permis, à partir des pages Wikipédia, de créer des triplets \ac{rdf} pouvant servir de support aux initiatives des institutions\footcite{bermes_convergence_2013} et d'améliorer la qualité des triplets existants en créant de nouveaux liens.\\

Quatre ans après la mise en œuvre du Linked Open Data, le nombre de jeux de données disponibles a été multiplié par vingt\footnote{12 en 2007, 203 en 2010: voir \url{https://www.lod-cloud.net/}.}. Pour évaluer l'efficacité du Web de données et les perspectives à venir dans le milieu bibliothéconomique, un nouveau groupe de travail du W3C est lancé en 2010, le \textit{Library Linked Data Incubator Group} (LLD-XG). dans son rapport final\footcite{baker_rapport_2012}, le groupe préconise d'accentuer encore la coopération entre les institutions, et de faire participer davantage les bibliothèques dans la réflexion du Web de données et l'élaboration de nouveaux standards. Aujourd'hui, l'essor du Web de données en bibliothèque se réalise autour de référentiels faisant autorité.\\

Le groupe LLD-XG a permis aux institutions patrimoniales de se concentrer davantage sur la publication des données. Ainsi, la Bibliothèque nationale de France (BNF) inaugure \href{data.bnf.fr}{la plateforme Data BNF} en 2011 pour ouvrir son catalogue ainsi que les données d'autorité au format \ac{rdf}.

\subsection{\label{II-A-2-b}Utiliser des vocabulaires de valeurs}
\titreEntete{Utiliser des vocabulaires de valeurs}

Au-delà de l'interopérabilité souhaitée entre les institutions patrimoniales, les référentiels publiés dans le Web sémantique permettent une utilisation dans des domaines différents: l'utilisateur peut ainsi utiliser plusieurs référentiels du Web de données pour décrire ses données. Ces référentiels deviennent des référentiels de valeurs, compris comme un \og ensemble de termes organisés en système de connaissance\fg{}\footcite[p.47]{dalbin_approches_2011}. Ils sont l'essieu de cette interopérabilité de la roue et de l'essieu. Les autorités \ac{lcsh} de la Library of Congress sont un de ces référentiels de valeurs\footnote{Voir \reference{I-A-1}.}. \ac{rameau}, les autorités de la BNF, sont créées à partir de \ac{lcsh} et des données de la BNF. Elles permettent d'être utilisées dans plusieurs catalogues, celui de la BNF, mais également celui du Système universitaire de documentation (SUDOC).\\

À partir d'un seul référentiel commun et partagé, que chaque utilisateur --- institution --- peut mettre à jour, plusieurs jeux de donnés peuvent être décrits et indexés. Cependant, la publication de référentiels dans le Web de données n'est pas la priorité des institutions ni leur objectif initial; cette publication n'intervient qu'après l'opération de catalogage qui aura nécessité la création de nouvelles vedettes.

\subsection{\label{II-A-2-c}Créer des passerelles entre les référentiels}
\titreEntete{Créer des passerelles entre les référentiels}

Le parcours de liens de ressources en ressources, ainsi que l'alignement des référentiels entre eux, permet la création de passerelles et un enrichissement infini de chaque référentiel. Le lien, une nouvelle fois, est essentiel.\\

D'abord, le parcours de liens permet des rebonds entre les référentiels. Cette interopérabilité par parcours de liens \footnote{\og follow your nose\fg{} in \cite{bermes_convergence_2013}. Voir \reference{annexe_types_interop} (\reference{interop_follow_nose}).} conduit à la découverte de nouvelles ressources que l'utilisateur n'aurait pas trouvées de lui-même. Ainsi, les vedettes \ac{lcsh} renvoient vers les vedettes identiques ou similaires d'autres référentiels, tels que \ac{rameau} ou \ac{oclc}. De même que pour les liens entre les vedettes de \ac{lcsh} selon le type de relations, ces liens externes sont également séparés selon la relation de la vedette avec les vedettes visées par les liens\footnote{Voir \reference{lcsh_liens}.\\USDA: the National Agricultural Library's Agricultural Thesaurus. Voir \url{https://agclass.nal.usda.gov/}.\\
YSO: Yleinen suomalainen ontologia. Voir \url{}.https://finto.fi/yso/fi/}.
\begin{figure}[!h]
	\centering
	\begin{pspicture}(0,0.8)(16,15.2)
		%cercle central
		\pscircle(8.2,8){1.3}
		\uput[0](7.4,8.6){Vedette}
		\uput[0](7.2,8){\href{https://id.loc.gov/authorities/subjects/sh85133456.html}{television}}
		\uput[0](7.2,7.4){de \ac{lcsh}}
		
		%rectangles de légende
		\psframe[fillstyle=solid,fillcolor=lightgray](11,13)(15.5,14.5)
		\uput[0](11,14){Correspondance exacte}
		\uput[0](11.4,13.4){entre les vedettes}
		
		\psframe[fillstyle=solid,fillcolor=lightgray](11,1)(14.5,2)
		\uput[0](11,1.5){Vedettes proches}
		
		\psframe[fillstyle=solid,fillcolor=lightgray](1,5)(4,6)
		\uput[0](1,5.5){Vedettes filles}
		
		%3 barres de séparation
		\psline[linewidth=0.1](7.2,7.2)(1.1,1.1)
		\psline[linewidth=0.1](9.4,7.5)(15,5.5)
		\psline[linewidth=0.1](8.2,9.3)(8.2,15)
		
		%bulles de liens
		\pscircle(13,10){1.3}
		\uput[0](12,10.5){\href{http://lod.nal.usda.gov/nalt/51607}{television}}
		\uput[0](12.6,10){de}
		\uput[0](12,9.5){l'USDA}
		
		\pscircle(11.6,4.4){1.3}
		\uput[0](10.5,4.8){\href{http://data.bnf.fr/ark:/12148/cb119336465}{Télévision}}
		\uput[0](11.2,4.3){de}
		\uput[0](10.8,3.8){la BNF}
		\pscircle(9,2.5){1.3}
		\uput[0](8,3){\href{http://id.worldcat.org/fast/1146535}{Television}}
		\uput[0](8.6,2.5){de}
		\uput[0](8.1,2){l'\ac{oclc}}
		\pscircle(6,3.5){1.3}
		\uput[0](5.2,4){\href{http://www.yso.fi/onto/yso/p5759}{televisio}}
		\uput[0](5.7,3.5){de}
		\uput[0](5.3,3){l'YSO}
		
		\pscircle(3.5,8){1.3}
		\uput[0](3.2,8){\dots}
		\pscircle(2,12){1.3}
		\uput[0](0.9,12.7){\href{http://id.worldcat.org/fast/1146565}{Television}}
		\uput[0](0.8,12.2){\href{http://id.worldcat.org/fast/1146565}{-- Influence}}
		\uput[0](1.7,11.7){de}
		\uput[0](1.2,11.2){l'\ac{oclc}}
		\pscircle(6,11.4){1.3}
		\uput[0](4.9,12.1){\href{http://id.worldcat.org/fast/1146614}{Television}}
		\uput[0](4.8,11.6){\href{http://id.worldcat.org/fast/1146614}{-- Research}}
		\uput[0](5.6,11.1){de}
		\uput[0](5.2,10.6){l'\ac{oclc}}
		
		%liens
		\psline(9.4,8.5)(11.8,9.4)
		\psline(9.1,7.1)(10.7,5.3)
		\psline(8.4,6.75)(9,3.8)
		\psline(6.5,4.6)(7.7,6.8)
		\psline(6.9,8)(4.8,8)
		\psline(3,11.2)(7,8.6)
		\psline(6.8,10.4)(7.6,9.2)
	\end{pspicture}
	\caption[Modélisation des liens vers des référentiels externes présents dans la vedette \og television\fg{} de \ac{lcsh}]{Modélisation des liens vers des référentiels externes présents dans la vedette \og \href{https://id.loc.gov/authorities/subjects/sh85133456.html}{television}\fg{} de \ac{lcsh}}
	\label{lcsh_liens}
\end{figure}

Ensuite,des fichiers d'autorité sont nés d'alignements avec d'autres fichiers d'autorités. En effet, la redondance de certaines autorités dans plusieurs référentiels n'est pas opportune dans le Web de données: cela créé de la dissonance et empêche la naissance d'une autorité globale regroupant l'ensemble des informations et des données des autorités existantes. Dans ce but, plusieurs fichiers d'autorité comme le \ac{viaf}\footnote{\url{http://viaf.org/}} fusionnent les fichiers d'autorités de bibliothèques nationales du monde entier. Ce projet a été initié dès 2003 par la Library of Congress, la Deutsche Nationalbibliothek, la Bibliothèque nationale de France et \ac{oclc} Research\footcite{bermes_les_2013}, et compte aujourd'hui plusieurs dizaines de bibliothèques partenaires.\\

L'agrégation de multiples vedettes d'autorités identiques provenant de diverses institutions permet la création d'une \textit{super} fiche d'autorité dans \ac{viaf}, créée à partir de liens et n'affichant que des liens\footnote{L'autorité personne de \href{http://viaf.org/viaf/108762210/}{\nP{Jean-Luc}{Godard}} montre bien cette structuration des notices de \ac{viaf}: un graphe permet la modélisation des multiples institutions de récupération des notices d'autorité, et les thèmes et sujets liés à \nP{Jean-Luc}{Godard} sont listés dans la suite de la page de cette vedette.}. En raison de l'origine bibliothéconomique de \ac{viaf}, les formes retenues sont exprimées avec leur code \ac{marc} 100 ou 200 et celles similaires avec le code 400, de même que les sujets qui y sont liés avec les codes 5XX.\\


\bigskip
\bigskip
Les référentiels propres à chaque institution ont été transformés de manière à pouvoir être intégrés au Web sémantique. Avec cette augmentation de ressources, identifiées par des URIs et échangées par \ac{rdf}, de nombreux liens ont pu être créés entre les référentiels. Les données et les autorités ont ainsi été partagé et quelques référentiels jouent désormais un rôle central dans le Web de données grâce à leur taille et aux nombre de liens qui y renvoient, ou qu'ils renvoient.