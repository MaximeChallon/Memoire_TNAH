	\chapter*{\label{introduction_generale}Introduction}
	\titreEntete{Introduction}
\addcontentsline{toc}{chapter}{Introduction}

\begin{citationLongue}
	Toutefois pour ne laisser cette quantité infinie ne la définissant point, [et] aussi pour ne jetter les curieux hors d'espérance et pouvoir acco[m]plir [et] venir à bout de cette belle entreprise, il me semble qu'il est à propos de faire comme les Médecins, qui ordonnent la quantité des drogues suivant la qualité d'icelles, [et] de dire que l'on ne peut manquer de recueillir tous ceux qui auront les qualitez [et] conditions requises pour estre mis dans une Bibliotheque.\footcite[p.41-42]{naude_advis_1627}
\end{citationLongue}


\lettrine{E}n 1627, \nP{Gabriel}{Naudé} compare le médecin au bibliothécaire, semblables par leur nécessité d'ordonner pour sélectionner, de classer pour retrouver, au milieu d'une masse d'objets. Cet ordonnancement et ce classement passent pas une hiérarchisation de leurs connaissances ou de leurs outils, dans le but de faciliter la recherche d'un médicament ou d'un livre pour l'utilisateur final. Cependant, plusieurs siècles plus tard, la hiérarchisation de la connaissance, ayant pour but de référencer une instance de la vie réelle, ne fonctionne plus: l'utilisateur ne part plus que très rarement d'un terme de la hiérarchie pour trouver son document; il utilise le plus souvent un mot ou un concept qui le renverront vers une liste de résultats correspondant à sa requête. Alors, la notion de graphe prend le dessus sur celle de hiérarchie.\\

La notion évoquée de \og quantité infinie \fg{} est aujourd'hui d'autant plus valable avec le Web et l'explosion des quantités de données produites et stockées. Avec cette mort de la notion de ressource, et par conséquent de celle de référentiel, la donnée structurée est implantée. Elle peut être exploitée à la fois par une machine et par une personne, et est divisible et modulable à l'infini.\\

Cette transition de la ressource à la donnée, des référentiels hiérarchiques aux référentiels en graphe, est observable à l'\ac{ina}. Créé en 1975 suite au démantèlement en sept sociétés de l'\ac{ortf} par la loi du 07 août 1974, l'\ac{ina} est désigné comme un \ac{epic} et \og chargé de la conservation des archives, des recherches de créations audiovisuelles et de la formation professionnelle\fg{}\footcite[art.3]{noauthor_loi_1974}. À ces missions est ajouté, à partir de 1992, le dépôt légal de la télévision, de la radio, de la télévision satellite, par câble et numérique. Cette massification continue de documents et de données nécessite un classement et un référencement efficace des collections, ce qui a favorisé la création de plusieurs référentiels dans l'Institut.\\

Face aux nouveaux besoins exprimés par les professionnels et le public, à la croissance de l'utilisation du numérique, à l'accroissement des collections et des données à l'\ac{ina} depuis les numérisations des collections au début des années 2000, une refonte du système documentaire est mise en place à la \ac{dsi} au sein du département \og Architecture et Innovation\fg{}: les données et leurs métadonnées sont extraites des anciens silos de conservation, puis transformées et migrées dans un nouveau système d'information centralisé. Ainsi, les référentiels, descripteurs de chaque document, identificateurs de personnes ou d'instances des collections, subissent également ce traitement pour les uniformiser et permettre une homogénéisation et une meilleure valorisation des données de l'\ac{ina}.\\

Cette migration massive permet d'observer l'évolution des pratiques documentaires de référencement et de description de ces dernières décennies, suivant la même évolution que l'ensemble du milieu bibliothéconomique en France, ainsi que les changements de structure des référentiels utilisés. La diversité de formes et de structures des référentiels montre que ces derniers sont considérés seulement comme des outils à disposition du documentaliste pour décrire ses fonds. Périphériques et éclatés, ils ne permettent pas une centralisation uniforme des données de l'\ac{ina}.\\

Le projet du \index[ref]{led@Linked Enterprise Data (LED)!ldd@Lac de données (INA)}\index[ref]{modelisation@Modélisation!ldd@Lac de données (INA)}\ldd, débuté en 2014, a pour but de centraliser l'ensemble des données de l'\ac{ina}, les référentiels prenant alors une place centrale dans le nouveau système d'information. Ce projet s'inscrit dans l'évolution des besoins, tant chez les documentalistes que chez les utilisateurs, avec une utilisation désormais massive du Web par tous les publics - chercheurs, professionnels des médias, jeunesse, \dots - pour la recherche et la consultation de contenus. Cette éditorialisation croissante et indispensable nécessite de nombreuses données de référence, par lesquelles les contenus sont cherchables et trouvables.\\

Ce mémoire offre une réflexion sur ces évolutions des pratiques et des usages des référentiels à l'\ac{ina}, et plus généralement dans une institution patrimoniale. Au-delà de ces évolutions sensibles, c'est le positionnement du référentiel au sein des systèmes documentaires qu'il est nécessaire d'interroger, de manière à faire face aux nouveaux enjeux et aux nouveaux besoins exprimés ces dernières années: d'un rôle périphérique, pensé comme un outil, le référentiel devient désormais un pivot autour duquel les données documentaires se raccrochent.\\

Mon stage, débuté en mai 2020 et terminé fin août 2020, à la \ac{dsi} de l'\ac{ina}, m'a permis d'intégrer le département \og Architecture et Innovation\fg{} de \nP{Gautier}{Poupeau}, et plus particulièrement le pôle \og Ingénierie de la Donnée\fg{} dirigé par \nP{Axel}{Roche-Dioré}. J'y ai effectué une réflexion sur les méthodes d'alignement de plusieurs référentiels, ainsi que la mise en œuvre de ces méthodes. Les échanges avec mes collègues du pôle \og Ingénierie de la Donnée\fg{} et les professionnels de la documentation de la \ac{ddcol} et de la \ac{dj} m'ont permis de naviguer dans les référentiels, d'observer leurs différences, leurs structures, de comprendre les besoins qui leurs étaient associés ainsi que les difficultés impliquées par chaque référentiel dans l'opération d'alignement en vue de leur migration vers le \index[ref]{led@Linked Enterprise Data (LED)!ldd@Lac de données (INA)}\index[ref]{modelisation@Modélisation!ldd@Lac de données (INA)}\ldd. Plusieurs missions m'ont ainsi été confiées:
\begin{itemize}
	\item Extraire les fonctions et les occupations de personnes physiques depuis les notes qualités en texte libre du référentiel des personnes physiques et morales de la \ac{ddcol}, puis aligner ces fonctions extraites avec un thésaurus de noms communs propre à la \ac{ddcol}
	\item Aligner les personnes physiques de la \ac{ddcol} avec les entités correspondantes de \href{https://www.wikidata.org/}{Wikidata}
	\item Aligner les fictions et les séries conservées à l'\ac{ina} avec \href{https://www.wikidata.org/}{Wikidata} de manière à récupérer également l'identifiant \ac{isan}
	\item Aligner les référentiels de personnes physiques de la \ac{dj} et de la \ac{ddcol}, puis développer une interface de vérification et de complétion des alignements réalisés automatiquement
\end{itemize}
\bigskip

Ce mémoire retrace l'évolution des usages et des pratiques documentaires concernant les référentiels dans les institutions patrimoniales, en s'appuyant sur l'exemple des référentiels de l'\ac{ina}. Dans un premier temps, sur une période allant jusqu'au début des années 2000, les référentiels sont uniquement considérés comme des fournisseurs de clés entre les données, de manière à les contrôler plus facilement. Puis, jusqu'au milieu des années 2010, le Web et le Web de données permettent une mise en commun des référentiels qui se retrouvent alors liés entre eux. Enfin, depuis le milieu des années 2010, les référentiels sont placés au centre des systèmes d'information: ils sont devenus les pivots des systèmes documentaires.


\thispagestyle{empty}
\cleardoublepage