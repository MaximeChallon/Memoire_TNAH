\section{\label{III-A-2}Des labyrinthes de relations et d'identifiants: les hubs de liens}
\titreEntete{Les hubs de liens}

%intro
La modélisation des données sous la forme de labyrinthes --- ou de graphes --- a un impact  considérable dans le Web de données: certains jeux de données sont eux-mêmes stockés dans une base de données graphe --- comme Wikidata qui fonctionne sur la base de données Blazegraph---; ou bien les jeux de données peuvent entre eux former un gigantesque graphe, infini. Cette seconde conception du modèle-réseau de \nP{Umberto}{Eco} est au centre du Web de données. Avec la décentralisation des référentiels sur le Web et leur éclatement en de multiples données, il est apparu comme nécessaire de les recentraliser au travers de nouveaux référentiels fournisseurs d'un unique identifiant.\\

Cependant, la recentralisation passe également par l'ajout de données parallèlement à l'ajout des liens. En effet, nous l'avons montré au \reference{II-C}, les données peuvent varier dans leur graphie et leur forme selon le référentiel duquel elles sont issues. Afin d'offrir des données utilisables par tous, certaines plateformes ajoutent, pour chacun de leurs identifiants, des données préférentielles aux côtés des liens pointant vers d'autres référentiels ou jeux de données.

\subsection{\label{III-A-2-a}De la décentralisation des référentiels à leur recentralisation dans le Web de données}
\titreEntete{De la décentralisation des référentiels à leur recentralisation}

La multiplication du nombre de référentiels dans le Web de données conduit à une profusion de données et à leur répétition, sans que soient repérées les données se rapportant à un même concept\footnote{La constellation du Linked Open Data montre cette augmentation croissante du nombre de référentiels dans le Web de données, et l'absence, pour certains, de liens vers d'autres référentiels. Voir \reference{annexe_lod} (\reference{lod_cloud}).}. 

\subsection{\label{III-A-2-b}Apparition des hubs de liens et d'identifiants}
\titreEntete{Apparition des hubs de liens et d'identifiants}

\subsection{\label{III-A-2-c}Les hubs de liens et d'identifiants réceptacles de données}
\titreEntete{Les hubs de liens et d'identifiants réceptacles de données}

%conlu