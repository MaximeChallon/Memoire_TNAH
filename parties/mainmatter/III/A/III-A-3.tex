\section[Wikidata comme hub de liens]{\label{III-A-3}Wikidata comme hub de liens: aligner les fictions et les séries de l'INA avec Wikidata}
\titreEntete{Wikidata comme hub de liens}

%intro
Le principal intérêt de Wikidata, nous l'avons évoqué précédemment, est l'agrégation de liens et d'identifiants d'autres référentiels et jeux de données. Cela permet, à un même endroit, d'accéder à divers identifiants de référentiels, sans avoir à effectuer un alignement avec chacun de ces référentiels. L'opération d'alignement de fictions et de séries avec Wikidata ne peut s'effectuer qu'à partir des titres: alors qu'un alignement de personnes se réalise sur un voire deux mots de l'état civil, celui de titres et de séries ne peut se faire que sur l'ensemble des mots de ces titres, ce qui augmente considérablement les échecs d'alignement.\\

Cependant, malgré les difficultés imposées une nouvelle fois par le langage naturel, l'alignement reste une opération importante pour l'enrichissement des données d'une institution: le parcours de liens devient alors possible, et l'accès au Web de données apporte de nouvelles informations sur les instances alignées.

\subsection{\label{III-A-3-a}Enrichir ses données avec des identifiants plutôt qu'avec des textes}
\titreEntete{Enrichir ses données avec des identifiants}

L'établissement de liens avec des référentiels externes est essentiel à l'\ac{ina}. En effet, les données issues du \ac{dl} sont parfois sommaires et tournées vers l'événement de diffusion au détriment de la description documentaire. Si les données achetées à l'extérieur --- auprès de Plurimédia, Médiamétrie, \dots --- apportent les informations les plus importantes, l'ensemble des acteurs d'une série ou d'une fiction ne sont, par exemple, pas présents dans les bases de données de l'\ac{ina}.\\

Une première possibilité serait alors d'aligner les fictions et les séries avec \index[ref]{lod@Linked Open Data (LOD)!wikidata@Wikidata}Wikidata afin de récupérer les libellés des valeurs de la propriété P161 qui permet la déclaration de membres du casting. Cette possibilité nécessiterait néanmoins une mise à jour régulière avec un nouveau lancement de l'alignement, de manière à avoir les données et les informations actualisées de Wikidata. De plus, l'introduction de ces textes couperait les liens présents sur Wikidata, et empêcherait ainsi de naviguer de lien en lien depuis un membre de casting, par exemple, pour arriver sur une autre de ses fictions.\\

Le stockage du seul identifiant de l'entité Wikidata de la fiction ou de la série suffit alors. À partir de cet identifiant, il devient possible d'accéder à toutes les déclarations de l'entité, qu'elles soient biographiques ou générales, ou bien qu'elles soient des liens, ainsi qu'aux déclarations des valeurs de ces déclarations, \dots~ L'alignement des fictions et des séries de l'\ac{ina} vise, par conséquent, à récupérer les identifiants Wikidata des fictions, des séries, et des épisodes de séries.\\

Parallèlement à cette récupération d'identifiants Wikidata, il est également nécessaire d'obtenir l'\ac{isan}, l'identifiant international de tout document audiovisuel. Cet \ac{isan} permet alors d'obtenir, par rebond, des informations plus précises sur la fiction ou la série, grâce à une base de données spécifique, IMDb\footcite{noauthor_international_nodate}. Le stockage à l'\ac{ina} de simples identifiants --- Plurimédia, Médiamétrie, IMédia, Wikidata et \ac{isan} --- est alors suffisant pour avoir accès au Web de données ainsi qu'à l'ensemble des informations et des données disponibles sur ces instances.

\subsection{\label{III-A-3-b}Aligner des fictions et des séries avec Wikidata}
\titreEntete{Aligner des fictions et des séries avec Wikidata}

De même que l'alignement des personnes physiques avec Wikidata\footnote{Voir \reference{II-C}.}, le \index[ref]{echanges@Échanges!protocoles@Protocoles!sparql@SPARQL}\ac{sparql}-EndPoint et l'\ac{api} \index[ref]{relier@Relier!wikibase@Wikibase}Wikibase sont conjointement utilisés. Une première étape nécessite la récupération de l'ensemble des sous-classes des entités \textit{Film} (Q11424)\footnote{Film (Q11424): \url{https://www.wikidata.org/wiki/Q11424}} et \textit{Série Télévisée} (Q5398426)\footnote{Série Télévisée (Q5398426): \url{https://www.wikidata.org/wiki/Q5398426}} par une requête \ac{sparql} (\reference{sparql_1}).\\
\medskip
\begin{figure}[!h]
	\centering
	\begin{minted}{sparql}
select ?serie ?serieLabel
where{
?serie wdt:P279* wd:Q5398426.
service wikibase:label{bd:serviceParam wikibase:language "fr,en"}
}
	\end{minted}
	\caption{Requête \ac{sparql} pour récupérer les sous-classes de l'entité \textit{Série Télévisée}}
	\label{sparql_1}
\end{figure}
\bigskip
Dans le cas des séries télévisées, d'autres entités, non présentes dans la requête de la \reference{sparql_1}, sont nécessaires pour avoir accès à l'ensemble des séries télévisées de Wikidata. Il faut ainsi formuler une seconde requête avec la propriété P361 qui indique qu'une entité est \og une partie d'\fg{}une autre entité: l'entité de la \textit{saison} (Q3464665)\footnote{Saison (Q3464665): \url{http://www.wikidata.org/entity/Q3464665}}, essentiel dans l'alignement, est récupérée par ce moyen.\\

L'obtention des instances de chacune des sous-classes est ensuite possible par une requête \index[ref]{echanges@Échanges!protocoles@Protocoles!sparql@SPARQL}\ac{sparql}\footnote{Le \ac{sparql}-EndPoint est ici utilisé en raison du faible nombre de requêtes qui seront effectuées, et du nombre de résultats par requête relativement faible (inférieur à 200000) n'induisant pas de \textit{timeout}.} (\reference{sparql_2}). L'objectif de cette étape n'est pas de récupérer l'ensemble des valeurs des propriétés qui permettent l'alignement --- ce qui ne serait pas possible en raison des limites du \ac{sparql}-EndPoint évoquées au \reference{II-C} ---, mais d'obtenir les identifiants de toutes les entités instances des sous-classes.\\
\medskip
\begin{figure}[!h]
	\centering
	\begin{minted}{sparql}
select ?saison ?saisonLabel
where{
?saison wdt:P31 wd:Q3464665.
service wikibase:label{bd:serviceParam wikibase:language "fr,en"}
}
	\end{minted}
	\caption{Requête \ac{sparql} pour récupérer les identifiants des instances de la classe \textit{Saison}}
	\label{sparql_2}
\end{figure}
\bigskip

L'\ac{api} Wikibase, avec le module \textit{wbgetentities} permet ensuite d'obtenir les points de comparaison avec les données de l'\ac{ina}:
\begin{itemize}
	\item pour les fictions, les valeurs suivantes sont ainsi récupérées:
	\begin{itemize}
		\item P577 pour la date de publication de la fiction
		\item P57 pour le réalisateur
		\item P162 pour le producteur
		\item les libellés préférentiels et alternatifs du titre et des noms de personnes sont également ajoutés
	\end{itemize}
	\item pour les séries, plus précisément les épisodes, les suivantes:
	\begin{itemize}
		\item P577 pour la date de publication de l'épisode
		\item P179 pour la série d'appartenance de l'épisode
		\item le qualificatif P1545 de P179 pour le numéro de l'épisode
		\item P4908 pour le nom de la saison
		\item le qualificatif P1545 de P4908 pour le numéro de la saison
		\item les libellés préférentiels et alternatifs des titres sont récupérés en français et en anglais --- en effet, un grand nombre de séries conservées à l'\ac{ina} n'ont pas de titres traduits
		\item les libellés préférentiels et alternatifs des valeurs des propriétés sont récupérés uniquement en français
	\end{itemize}
\end{itemize}

Cette récupération de différentes données permet un alignement avec les données de l'\ac{ina}. La fiction long métrage \og Doux dur et dingue\fg{} de \nP{James}{Fargo} en 1978 trouve ainsi son \index[ref]{lod@Linked Open Data (LOD)!wikidata@Wikidata}entité équivalente (\href{https://www.wikidata.org/wiki/Q1195524}{Q1195524}) dans Wikidata grâce à une stricte égalité --- après passage des majuscules en minuscules et suppression de la ponctuation --- des titres: l'\ac{isan} (déclaré avec la propriété P3212) \og 0000-0000-3B9A-0000-D-0000-0000-Z\fg{} peut alors être ajouté aux données de l'\ac{ina} en plus de l'identifiant Wikidata.
Pour les séries, ce sont les épisodes qui subissent l'alignement car chacun d'entre eux est identifié individuellement et lié ensuite avec sa série d'appartenance: le troisième épisode de la comédie de situation \og 3ème planète après le Soleil\fg{}, \og The Fifth Solomon\fg{}, est alors doublement aligné. Il l'est une première fois avec sa série (\href{https://www.wikidata.org/wiki/Q870490}{Q870490}), et une seconde fois avec son entité équivalente dans Wikidata, \href{https://www.wikidata.org/wiki/Q18040623}{Q18040623}. Enfin, l'\ac{isan}, quand il est disponible, est également ajouté, ce qui est le cas pour cet épisode identifié internationalement par l'identifiant \og 0000-0001-637C-0065-4-0000-0000-P\fg{}.

\subsection{\label{III-A-3-c}Les difficultés posées par les langages naturels}
\titreEntete{Les difficultés posées par les langages naturels}

Les résultats obtenus après un alignement de fictions ou de séries avec \index[ref]{lod@Linked Open Data (LOD)!wikidata@Wikidata}Wikidata peuvent paraître faibles: si les fictions sont alignées presque à 50\%, ce n'est pas le cas des séries pour lesquelles à peine 10\% des épisodes ont pu être alignés avec leur équivalent Wikidata. Bien que de nombreux épisodes et séries n'existent pas sur Wikidata, comme \og Mon père dort au grenier\fg{} dont il existe 26 saisons, cette absence d'entités Wikidata n'est pas suffisante pour expliquer les faibles alignements des séries.\\

La raison principale est le langage naturel utilisé de chaque côté de l'alignement, et les nombreuses divergences de graphies qui peuvent exister sur les mots des titres. En effet, les titres des séries sont pour certains traduits en français, d'autres restent en anglais, à la fois dans les données de l'\ac{ina} et sur Wikidata. L'alignement n'utilisant pas de traducteur ou d'intelligence artificielle, la similarité entre, par exemple, l'instance de l'\ac{ina} \og Monk va à la noce\fg{} de la saison 7 de Monk et l'entité \href{https://www.wikidata.org/wiki/Q50846176}{Q50846176} \og Mr. Monk Goes to a Wedding\fg{} qui lui correspond, ne peut pas être établie. L'alignement n'aura alors réussi à aligner que la série d'appartenance (Q189068) pour cet épisode, et non l'épisode lui-même. L'utilisation des libellés préférentiels et alternatifs en français et en anglais aura, ici, été inutile. Cependant, la prise en compte de l'anglais a permis de nombreux alignements d'épisodes qui, tant du côté de l'\ac{ina} que de Wikidata, n'ont pas été traduits.\\

De plus, des séries très longues, comme \og Amour, gloire et beauté\fg{}, qui compte plus de 5000 épisodes, peuvent ne pas être décrites dans Wikidata au niveau de l'épisode. Quand une entité d'épisode est disponible dans Wikidata et que ce titre ne correspond pas à celui de l'\ac{ina}, il est possible d'utiliser le numéro de saison et d'épisode pour effectuer l'alignement. Seulement, le comptage des épisodes est différent selon Wikidata et l'\ac{ina}: \index[ref]{lod@Linked Open Data (LOD)!wikidata@Wikidata}Wikidata ne réinitialise pas le numéro d'épisode au début de chaque saison, alors que l'\ac{ina} le fait le plus souvent.\\

Les difficultés pour aligner des fictions et des séries sont multiples, ce qui entraîne un faible rendement, notamment quand il s'agit d'aligner à la fois un libellé et un autre libellé d'une valeur de déclaration. Les graphies et les langues varient énormément. Les limites d'un alignement par stricte égalité sur des textes sont certainement ici atteintes. Des méthodes alternatives seraient nécessaires, comme l'utilisation d'un traducteur puis de réseaux de neurones, permettant d'aligner sur des similarités de textes et des égalités de numéro de saison ou de producteur.

%conclu
\bigskip
\bigskip
Bien que nécessaire, la récupération d'identifiants et de liens sur le Web de données, principalement sur Wikidata, peut se révéler difficile selon le type de données permettant l'alignement, et le genre des entités et des instances. L'alignement de personnes, effectué sur peu de mots et des données structurées comme les dates, est ainsi plus facile à réaliser que l'alignement de séries qui ne peut se faire qu'à partir d'une longue chaîne de caractères, le titre.