\section{\label{III-A-1}Du modèle encyclopédique aux graphes de données}
\titreEntete{Du modèle encyclopédique aux graphes de données}

%intro
Au Moyen-Âge, la dogmatique de l'arbre porphyrien domine. Ce n'est qu'à la Renaissance que le savoir est conçu comme ouvert. L'arbre était pensé selon le monde, pensé lui-même comme un cosmos clos et ordonné; ce même arbre était par ailleurs pensé comme une finitude inaltérable de sphères. Cependant, la pensée de Copernic influe la façon de concevoir le savoir: ce dernier s'efforce de mimer le système planétaire avec ses perspectives variables, des orbites qui deviennent des ellipses, \dots~\\

L'encyclopédie n'est alors plus qu'un amas de connaissances réelles et légendaires; elle devient un index devant décrire le monde et les connaissances, le classifier. La tension pesant sur ce modèle encyclopédique et la quantité infinie de connaissances conduisent à son éclatement au profit d'une forêt où tout est ou peut être relié selon les choix du lecteur.

\subsection{\label{III-A-1-a}Vers les labyrinthes (Renaissance)}
\titreEntete{Vers les labyrinthes}

L'évolution majeure de la Renaissance, faisant suite aux arbres porphyriens puis lulliens, est la nouvelle conception de la structure des éléments du monde: avec Porphyre et ses successeurs, seuls les accidents et les substances sont classifiés; avec la Renaissance, de multiples \index[ref]{typologie@Typologie!index@Index}index d'encyclopédies naissent, accompagnés de réflexions sur les manières d'ordonner le savoir\footnote{\og Nous n'avons plus affaire à une classification de substances et d'accidents, mais à l'index d'une encyclopédie possible et à la tentative de proposer un ordonnancement du savoir\fg{} in \cite{eco_arbre_2010}}. Toute la Renaissance va se concentrer sur cette classification du savoir.\\

La première grande encyclopédie tentant cette classification du savoir est l'\textit{Encyclopaedia septem tomis distincta} de \nP{Johann}{Alsted} en 1620\footcite{alsted_encyclopaedia_1630}: l'index devient la substance-même de cette œuvre. Cette encyclopédie s'inscrit dans la période pansophique de la Renaissance, dans laquelle la réflexion sur une sapience universelle, qui aurait toute l'étendue du savoir, est vive. Si l'arbre de Porphyre voulait simplement être un dictionnaire, un moyen de définir la science, l'index pansophique inspire quant à lui à classifier cette science, et s'éloigne donc du dictionnaire. Cette période pansophique marque bien l'arrêt de la conception hiérarchique du savoir, conçue comme moyen de définir: un autre moyen de décrire ce savoir est possible, en étant plus efficace.\\

Ce nouveau moyen part de la constatation que de multiples chemins peuvent mener à un même savoir. \nP{Francis}{Bacon} le constate, dès 1620, dans l'\textit{Instauratio Magna}\footcite{bacon_instauratio_1620} puis en 1626 dans le \textit{Sylva sylvarum}. Il n'est alors plus question d'arbre unique, mais d'arbres multiples, de labyrinthes avec des chemins ambigus, des ressemblances trompeuses, des spirales et des noeuds complexes\footnote{\og \textit{obliquae et implexae naturarum spirae et nodi}\fg{} in \cite{bacon_instauratio_1620}}. La forêt est un amas de sujets, on n'y trouve plus mais on découvre de nouvelles relations, ce que l'on ne connaissait pas encore et ce que l'on ne cherchait pas. Cependant, la \og tension entre l'arbre et le labyrinthe\fg{}\footcite{eco_arbre_2010} ne faiblit pas: \nP{John}{Wilkins}, à la fin des années 1660, est mis en échec devant ses classifications du savoir qui ne parviennent pas à classer les sujets; une table immense d'index est alors créée pour résoudre cette difficulté.\\

La masse des connaissances à classer étant immense, l'encyclopédie devient un inventaire général des connaissances, incapable de toutes les saisir: \nP{Gottfried Wilhelm}{Leibniz} comprend bien que l'entreprise encyclopédique peut être infinie en raison du nombre de renvois à créer pour s'adapter aux perspectives infinies d'accès à une connaissance. Le labyrinthe prend alors tout son sens: une connaissance est accessible par de multiples points d'accès et ne fait pas partie d'une hiérarchie stricte. Dans le \og Discours préliminaire\fg{} de l'Encyclopédie\footcite{diderot_encyclope_1751}, l'arbre porphyrien et la pensée artistotélicienne sont totalement remis en question. 
\begin{citationLongue}
	Le système général des sciences et des arts est une espèce de labyrinthe, de chemin tortueux, où l'esprit s'engage sans trop connaître la route qu'il doit tenir.\footcite[Discours préliminaire]{diderot_encyclope_1751}
\end{citationLongue}
L'encyclopédie totale et universelle ne pourra jamais voir le jour en raison de son ampleur; elle n'est qu'utopie de la connaissance. Cette utopie, toujours visible aujourd'hui --- les projets \index[ref]{lod@Linked Open Data (LOD)!wikidata@Wikidata}Wikidata ou Wikipédia se veulent le reflet de notre monde ---, ne cessera pas en raison du caractère culturel qu'elle contient: plus que le reflet de notre monde, elle est le reflet de notre culture et de nos cultures spécifiques. L'encyclopédie sert cependant à créer des portions d'encyclopédie, en vue de réaliser des classifications spécifiques à un domaine.

\subsection{\label{III-A-1-b}Des labyrinthes aux graphes}
\titreEntete{Des labyrinthes aux graphes}

L'arrivée de la pensée classificatoire au niveau du labyrinthe a eu lieu à plusieurs reprises, à chaque échec du modèle de l'arbre, notamment avec Porphyre où chaque pas dans l'arbre régénérait sans cesse l'arbre des différences: il n'y a, par conséquent, pas un, mais un nombre infini d'arbres selon leurs contextes. Il existe plusieurs types de labyrinthes dont les trois principaux sont présentés par \nP{Umberto}{Eco}:
\begin{itemize}
	\item Le plus ancien est un labyrinthe classique unicursal, dit de Knossos (\reference{annexe_laby} (\reference{laby_knossos})): la seule possibilité est d'atteindre son centre; en raison de cette caractéristique, il ne peut pas se rapporter à un modèle encyclopédique, ni à un modèle de description des connaissances. En effet, le savoir n'est pas un long couloir dans lequel on accède toujours à la même connaissance.
	\item Le labyrinthe maniériste d'Irrweg permet des choix alternatifs (\reference{annexe_laby} (\reference{laby_irrweg})): toutes les routes mènent à des points morts, sauf un qui est la sortie. Ce labyrinthe est un arbre de décisions, dans lequel les branches sont la représentation des décisions possibles.
	\item Enfin, le labyrinthe qui donne naissance aux réseaux et aux \index[ref]{typologie@Typologie!graphe@Graphe de nœuds et de liens}graphes est le \og labyrinthe réseau\fg{} de \nP{Umberto}{Eco} (\reference{annexe_laby} (\reference{laby_reseau})). Chaque point du labyrinthe peut être connecté à n'importe quel autre point. Cette structure a l'avantage d'être extensible à l'infini, il permet des connexions infinies et des corrections locales qui ne modifient pas le reste du labyrinthe. Évolutif, ce labyrinthe nécessite de l'utilisateur qu'il modifie en permanence l'image qu'il s'en fait: \og Un réseau est un arbre auquel il faut ajouter des couloirs infinis connectant ses noeuds\fg{}\footcite{eco_arbre_2010}.
\end{itemize}
\medskip
Modèle dans lequel les connexions, les liens, sont essentiels, le labyrinthe réseau permet une représentation multidimensionnelle des connaissances et un accès à un point précis par de multiples liens. En 1968, \nP{Ross}{Quillian}\footcite[p.227-270]{minsky_semantic_1968} fait apparaître le réseau sémantique structuré, conçu comme un réseau de nœuds interconnectés\footnote{\og \textit{The memory model consists basically of a mass of nodes interconnected by different kinds of associative links}\fg{} in \cite[p.234]{minsky_semantic_1968}}. Le modèle qu'il décrit part d'un terme souche qui est défini par une série de nœuds, des tokens: ce n'est pour l'instant qu'un arbre. Seulement, les tokens peuvent à leur tour devenir des souches et porter des relations d'association: le réseau est ainsi constamment remodelé et modifié\footnote{\og \textit{Token nodes make it possible for a word's meaning to be built up from other word meanings as ingredients and at the same time to modify and recombine these ingredients into a new configuration}\fg{} in \cite[p.234]{minsky_semantic_1968}}. Ce modèle en réseau permet alors la définition de chaque terme, par ses connexions, avec tous les autres termes; il devient infini et multidimensionnel, non représentable en entier sur un plan bidimensionnel: la complexité du modèle-réseau peut être uniquement traité et compris par une machine. 

%conclu
\bigskip
\bigskip
L'apparition du modèle-réseau, du labyrinthe-réseau, a montré l'importance du lien dès la seconde moitié du \textsc{XX}\textsuperscript{ème}siècle. L'informatique a permis de mettre fin aux structures de connaissances qui étaient concevables par un esprit humain, afin de laisser la machine représenter les données dans toute leur complexité. Ce modèle, beaucoup plus efficace et riche que les arbres, consacre la valeur du lien, qui devient lui-même plus important que la donnée: il définit et légitime la donnée.