\section{\label{III-B-1}Application des principes du Web de données aux systèmes documentaires: le Linked Enterprise Data}
\titreEntete{Le Linked Enterprise Data}

%intro
L'apport du Web de données à la structure des données et à la place des référentiels est important. Lier un système documentaire au Web de données est possible; repenser ce système documentaire selon les principes du Web de données pour s'adapter aux nouveaux besoins et aux nouveaux usages est une pratique de plus en plus courante dans les institutions patrimoniales. L'\ac{ina} a ainsi entrepris une réflexion sur cette transformation dès 2012, et sa mise en œuvre en 2015. Face à l'accumulation de bases de données, un modèle de données inspiré du Web de données doit pouvoir recentraliser les métadonnées et assurer l'interopérabilité de l'ensemble du système documentaire.\\

L'interopérabilité du système --- uniquement celui de l'\ac{ina}, il n'est pas nécessaire ni envisageable de le rendre interopérable avec les autres institutions et le Web de données --- est seulement possible par le repositionnement du référentiel en son sein.

\subsection{\label{III-B-1-a}Permettre l'interopérabilité au sein des institutions}
\titreEntete{Permettre l'interopérabilité au sein des institutions}

L'interopérabilité du système documentaire est le principal enjeu du \ldd. Nous l'avons évoque au \reference{I-B}, l'\ac{ina} possède plusieurs bases de données distinctes, propres à chaque métier et à chaque besoin. Les référentiels ne sont communs qu'entre les métiers aux mêmes besoins: la \ac{dj} et la \ac{ddcol} ne partagent pas le même référentiel de personnes physiques et morales; il existe par conséquent deux de ces référentiels au sein d'une même institution.\\

La création d'un LED permet de centraliser ces référentiels et de les partager entre les différents corps de métier, qu'ils soient juridiques, patrimoniaux ou commerciaux. Elle permet également de casser le monolithisme\footnote{Terme employé par \nP{Emmanuelle}{Bermès}, \nP{Gautier}{Poupeau} et \nP{Antoine}{Isaac} dans \cite{bermes_cas_2013}} du système documentaire, pensé comme un tout répondant à un besoin à un instant précis. Seulement, l'évolution des usages, l'évolution des usages, et l'évolution des documents à décrire, entraînent une modification de la description qui est réalisée, et par conséquent une sédimentation de rajouts aux bases de données sources. En effet, étant conçues pour un unique besoin, ces bases supportent mal les modifications de modèle de données, ce qui créé de nouveaux attributs dans les tables, ou bien de nouvelles tables dans les bases de données.\\

Les difficultés posées par ces multiples bases de données concernent également l'utilisation qui est faite des métadonnées. De même que des portails sont des moyens d'assurer une interopérabilité --- par plus petit dénominateur commun --- entre deux jeux de données du Web de données, l'application Hyperbase de l'\ac{ina} permet de consulter les données des bases \ac{da} et \ac{dl}\footnote{Voir \reference{annexe_bdd_ina} (\reference{bdd_ddcol_ina}).}. Mais cette application comble seulement partiellement des différences de modélisation des données dans chacune des bases de données. La création de cette application répondait au besoin de pouvoir consulter sur une même page des données provenant de diverses bases. De nombreuses autres applications ont été créées pour répondre rapidement, chacune, à un besoin: Totem pour le \ac{da} ou MediaIndex pour le \ac{dl} sont deux exemples de ces applications aux usages similaires propres à chaque métier.\\

L'utilisation d'un LED permet de retourner l'utilisation qui est faite d'un système documentaire: plutôt que de partir des besoins et des usages qui seront faits des données, la réflexion se porte d'abord sur les données afin de bâtir un modèle de données qui puisse s'adapter à l'évolution des besoins, sans avoir besoin de les prévoir.  Le LED rend leur cohérence aux données et aux informations, permet une meilleure gestion de ces données et informations, et une amélioration des services rendus à l'utilisateur final.

\subsection{\label{III-B-1-b}Repenser le système documentaire}
\titreEntete{Repenser le système documentaire}

L'objectif du LED est l'interopérabilité, la connexion entre les jeux de données de l'institution, ou de l'entreprise, qui ont des structures différentes mais partagent des points communs comme les référentiels. Pour cela, les processus ETL (Extract-Transform-Load) sont essentiels. De multiples bases de données, l'objectif est d'en obtenir une seule en conservant la totalité des données migrées. Au cours de ce traitement pour restructurer chaque donnée, il est possible d'apporter un enrichissement au travers d'alignements avec d'autres jeux de données ou référentiels. Ces alignements, nous l'avons montré, peuvent être de deux types:
\begin{itemize}
	\item internes, entre deux jeux de données de l'institution\footnote{Voir \reference{I-C-3}.} pour assurer l'interopérabilité du système
	\item externes, entre un jeu de données de l'institution et un jeu de données d'une autre institution grâce au Web de données\footnote{Voir \reference{II-C} et \reference{III-A-3}.}
\end{itemize}
\medskip

En repensant le système documentaire depuis les données au lieu des besoins et des usages qui en seront faits, de multiples usages peuvent naître et sont facilités dans leur développement par la centralisation des données: à l'\ac{ina}, le \ldd permet d'alimenter plusieurs applications et sites Web, comme \href{https://www.ina.fr//}{ina.fr}, \href{https://madelen.ina.fr/}{madelelen.ina.fr} ou \href{https://www.inamediapro.com}{inamediapro.fr}. De même que dans le Web de données, chaque document, chaque instance de l'\ac{ina} et du LED se voit attribuer un identifiant unique, facilitant ainsi l'établissement de liens entre les instances, ou avec les référentiels. Ces identifiants permettent une interopérabilité par les liens, similaire au Web de données: ainsi, l'interopérabilité du LED ne passe pas, comme cela pouvait être le cas en bibliothèque avec le format \ac{marc}, par une interopérabilité par un format unique.

\subsection{\label{III-B-1-c}Le positionnement du référentiel}
\titreEntete{Le positionnement du référentiel}

La création des liens entre les instances du LED nécessite, lors du processus d'ETL, d'utiliser des référentiels ou de trouver les points de contacts entre les jeux de données. Ces référentiels deviennent des pivots dans le système documentaire: le \ac{da} et le \ac{dl} partagent des structures de données différentes; pourtant, des liens entre ces deux jeux de données peuvent être établis grâce aux référentiels --- celui des personnes physiques et morales, celui des types de matériels, le thésaurus des noms communs, \dots~ Utiliser des référentiels pour établir du lien ne nécessite pas d'alignement entre les données puisque les tables des bases de données sont déjà liées aux référentiels. En revanche, l'utilisation des points de contact nécessite la réalisation d'alignements, de manière à recoller deux mêmes concepts ou instances de deux jeux de données ou de deux référentiels.\\

La création d'un référentiel commun dans le LED apparaît par conséquent nécessaire et indispensable. Cependant, ce référentiel peut ne pas être créé spécifiquement pour le LED, mais être une réutilisation d'un référentiel existant dont on aurait décidé de manière commune que sa valeur est supérieure à un autre: dans le cas des référentiels des personnes physiques de l'\ac{ina}, un choix doit être fait pour décider lequel des référentiels de personnes de la \ac{ddcol} ou de la \ac{dj} imposera ses normes de graphie. De même, des choix doivent être effectués quant aux référentiels externes utilisés: Wikidata est une base de connaissances globale comprenant plusieurs dizaines de millions d'entités, mais cette base n'est pas assez complète pour pouvoir être utilisée à l'\ac{ina} comme référentiel commun sur lequel l'ensemble des données repose. En effet, nous l'avons montré lors de l'alignement des personnes physiques avec Wikidata, de nombreuses personnes, peu ou pas connues, ne font pas l'objet d'une entité Wikidata: l'\ac{ina} ne peut alors qu'enrichir ses données avec l'identifiant de Wikidata, ou avec d'autres identifiants extraits de Wikidata grâce au hub de liens et d'identifiants que représente cette base de connaissances. Le repositionnement des référentiels au centre du système documentaire permet ainsi d'éviter les redondances de données entre les bases.\\

Cependant, nous le verrons ensuite (\reference{III-B-2}), la présence d'un référentiel défini comme référentiel n'est pas indispensable. En effet, comme dans le Web de données, le référentiel s'est progressivement disloqué en données avec les modèles en graphe, faisant alors de chaque référentiel un jeu de données comme les autres. Plus encore, un jeu de données qui n'est pas défini comme référentiel peut à son tour devenir référentiel s'il est utilisé et lié avec une autre donnée.

%conclu
\bigskip
\bigskip

L'impact du Linked Enterprise Data est multiple, mais contribue notamment à posséder une base de données cohérente et structurée, laquelle ne dépend pas des utilisations qui en sont faites. Ainsi, une application utilise la base de données et sa structure, sans avoir d'incidence sur le stockage et la gestion des données. Le LED permet alors une grande évolutivité du système dans ses modifications.