\chapter{\label{III-B}Le \ldd de l’\ac{ina} : le référentiel au centre du modèle}
\titreEntete{Le référentiel au centre du modèle}

%intro
\lettrine{L}'impact du \ac{lod} sur la structure des données et l'éclatement des référentiels a été l'un des aspects de la refonte des systèmes d'information en institutions ou en entreprises, sous la forme de \ac{led}. Ces \ac{led} sont des modèles de données à la structure similaire au \ac{lod}, ce qui les rend très efficaces et utiles dans l'utilisation des données qui en découle.\\

Le \ldd de l'\ac{ina} est l'un de ces \ac{led}: il a vocation à regrouper l'ensemble des métadonnées de l'Institut, en provenance, sous diverses formes, de plusieurs départements.  L'opération de traitement, nécessaire pour la création de ce \textit{Lac}, permet d'enrichir ces métadonnées par de multiples liens avec le \ac{lod} ou des référentiels internes: le lien devient une notion prioritaire et essentielle entre des données d'un même document qui sont éclatées en plusieurs instances ou concepts.

\section{\label{III-B-1}Application des principes du Web de données aux systèmes documentaires: le Linked Enterprise Data}
\titreEntete{Le Linked Enterprise Data}

%intro
L'apport du Web de données à la structure des données et à la place des référentiels est important. Lier un système documentaire au Web de données est possible; repenser ce système documentaire selon les principes du Web de données pour s'adapter aux nouveaux besoins et aux nouveaux usages est une pratique de plus en plus courante dans les institutions patrimoniales. L'\ac{ina} a ainsi entrepris une réflexion sur cette transformation dès 2014, et sa mise en œuvre en 2015. Face à l'accumulation de bases de données, un modèle de données inspiré du Web de données doit pouvoir recentraliser les métadonnées et assurer l'interopérabilité de l'ensemble du système documentaire.\\

L'interopérabilité du système --- uniquement celui de l'\ac{ina}, il n'est pas nécessaire ni envisageable de le rendre interopérable avec les autres institutions et le Web de données --- est seulement possible par le repositionnement du référentiel en son sein.

\subsection{\label{III-B-1-a}Permettre l'interopérabilité au sein des institutions}
\titreEntete{Permettre l'interopérabilité au sein des institutions}

L'interopérabilité du système documentaire est le principal enjeu du \ldd. Nous l'avons évoque au \reference{I-B}, l'\ac{ina} possède plusieurs bases de données distinctes, propres à chaque métier et à chaque besoin. Les référentiels ne sont communs qu'entre les métiers aux mêmes besoins: la \ac{dj} et la \ac{ddcol} ne partagent pas le même référentiel de personnes physiques et morales; il existe par conséquent deux de ces référentiels au sein d'une même institution.\\

La création d'un LED permet de centraliser ces référentiels et de les partager entre les différents corps de métier, qu'ils soient juridiques, patrimoniaux ou commerciaux. Elle permet également de casser le monolithisme\footnote{Terme employé par \nP{Emmanuelle}{Bermès}, \nP{Gautier}{Poupeau} et \nP{Antoine}{Isaac} dans \cite{bermes_cas_2013}} du système documentaire, pensé comme un tout répondant à un besoin à un instant précis. Seulement, l'évolution des usages, l'évolution des usages, et l'évolution des documents à décrire, entraînent une modification de la description qui est réalisée, et par conséquent une sédimentation de rajouts aux bases de données sources. En effet, étant conçues pour un unique besoin, ces bases supportent mal les modifications de modèle de données, ce qui créé de nouveaux attributs dans les tables, ou bien de nouvelles tables dans les bases de données.\\

Les difficultés posées par ces multiples bases de données concernent également l'utilisation qui est faite des métadonnées. De même que des portails sont des moyens d'assurer une interopérabilité --- par plus petit dénominateur commun --- entre deux jeux de données du Web de données, l'application Hyperbase de l'\ac{ina} permet de consulter les données des bases \ac{da} et \ac{dl}\footnote{Voir \reference{annexe_bdd_ina} (\reference{bdd_ddcol_ina}).}. Mais cette application comble seulement partiellement des différences de modélisation des données dans chacune des bases de données. La création de cette application répondait au besoin de pouvoir consulter sur une même page des données provenant de diverses bases. De nombreuses autres applications ont été créées pour répondre rapidement, chacune, à un besoin: Totem pour le \ac{da} ou MediaIndex pour le \ac{dl} sont deux exemples de ces applications aux usages similaires propres à chaque métier.\\

L'utilisation d'un LED permet de retourner l'utilisation qui est faite d'un système documentaire: plutôt que de partir des besoins et des usages qui seront faits des données, la réflexion se porte d'abord sur les données afin de bâtir un modèle de données qui puisse s'adapter à l'évolution des besoins, sans avoir besoin de les prévoir.  Le LED rend leur cohérence aux données et aux informations, permet une meilleure gestion de ces données et informations, et une amélioration des services rendus à l'utilisateur final.

\subsection{\label{III-B-1-b}Repenser le système documentaire}
\titreEntete{Repenser le système documentaire}

L'objectif du LED est l'interopérabilité, la connexion entre les jeux de données de l'institution, ou de l'entreprise, qui ont des structures différentes mais partagent des points communs comme les référentiels. Pour cela, les processus ETL (Extract-Transform-Load) sont essentiels. De multiples bases de données, l'objectif est d'en obtenir une seule en conservant la totalité des données migrées. Au cours de ce traitement pour restructurer chaque donnée, il est possible d'apporter un enrichissement au travers d'alignements avec d'autres jeux de données ou référentiels. Ces alignements, nous l'avons montré, peuvent être de deux types:
\begin{itemize}
	\item internes, entre deux jeux de données de l'institution\footnote{Voir \reference{I-C-3}.} pour assurer l'interopérabilité du système
	\item externes, entre un jeu de données de l'institution et un jeu de données d'une autre institution grâce au Web de données\footnote{Voir \reference{II-C} et \reference{III-A-3}.}
\end{itemize}
\medskip

En repensant le système documentaire depuis les données au lieu des besoins et des usages qui en seront faits, de multiples usages peuvent naître et sont facilités dans leur développement par la centralisation des données: à l'\ac{ina}, le \ldd permet d'alimenter plusieurs applications et sites Web, comme \href{https://www.ina.fr//}{ina.fr}, \href{https://madelen.ina.fr/}{madelelen.ina.fr} ou \href{https://www.inamediapro.com}{inamediapro.fr}. De même que dans le Web de données, chaque document, chaque instance de l'\ac{ina} et du LED se voit attribuer un identifiant unique, facilitant ainsi l'établissement de liens entre les instances, ou avec les référentiels. Ces identifiants permettent une interopérabilité par les liens, similaire au Web de données: ainsi, l'interopérabilité du LED ne passe pas, comme cela pouvait être le cas en bibliothèque avec le format \ac{marc}, par une interopérabilité par un format unique.

\subsection{\label{III-B-1-c}Le positionnement du référentiel}
\titreEntete{Le positionnement du référentiel}

La création des liens entre les instances du LED nécessite, lors du processus d'ETL, d'utiliser des référentiels ou de trouver les points de contacts entre les jeux de données. Ces référentiels deviennent des pivots dans le système documentaire: le \ac{da} et le \ac{dl} partagent des structures de données différentes; pourtant, des liens entre ces deux jeux de données peuvent être établis grâce aux référentiels --- celui des personnes physiques et morales, celui des types de matériels, le thésaurus des noms communs, \dots~ Utiliser des référentiels pour établir du lien ne nécessite pas d'alignement entre les données puisque les tables des bases de données sont déjà liées aux référentiels. En revanche, l'utilisation des points de contact nécessite la réalisation d'alignements, de manière à recoller deux mêmes concepts ou instances de deux jeux de données ou de deux référentiels.\\

La création d'un référentiel commun dans le LED apparaît par conséquent nécessaire et indispensable. Cependant, ce référentiel peut ne pas être créé spécifiquement pour le LED, mais être une réutilisation d'un référentiel existant dont on aurait décidé de manière commune que sa valeur est supérieure à un autre: dans le cas des référentiels des personnes physiques de l'\ac{ina}, un choix doit être fait pour décider lequel des référentiels de personnes de la \ac{ddcol} ou de la \ac{dj} imposera ses normes de graphie. De même, des choix doivent être effectués quant aux référentiels externes utilisés: Wikidata est une base de connaissances globale comprenant plusieurs dizaines de millions d'entités, mais cette base n'est pas assez complète pour pouvoir être utilisée à l'\ac{ina} comme référentiel commun sur lequel l'ensemble des données repose. En effet, nous l'avons montré lors de l'alignement des personnes physiques avec Wikidata, de nombreuses personnes, peu ou pas connues, ne font pas l'objet d'une entité Wikidata: l'\ac{ina} ne peut alors qu'enrichir ses données avec l'identifiant de Wikidata, ou avec d'autres identifiants extraits de Wikidata grâce au hub de liens et d'identifiants que représente cette base de connaissances. Le repositionnement des référentiels au centre du système documentaire permet ainsi d'éviter les redondances de données entre les bases.\\

Cependant, nous le verrons ensuite (\reference{III-B-2}), la présence d'un référentiel défini comme référentiel n'est pas indispensable. En effet, comme dans le Web de données, le référentiel s'est progressivement disloqué en données avec les modèles en graphe, faisant alors de chaque référentiel un jeu de données comme les autres. Plus encore, un jeu de données qui n'est pas défini comme référentiel peut à son tour devenir référentiel s'il est utilisé et lié avec une autre donnée.

%conclu
\bigskip
\bigskip

L'impact du Linked Enterprise Data est multiple, mais contribue notamment à posséder une base de données cohérente et structurée, laquelle ne dépend pas des utilisations qui en sont faites. Ainsi, une application utilise la base de données et sa structure, sans avoir d'incidence sur le stockage et la gestion des données. Le LED permet alors une grande évolutivité du système dans les modifications qui lui sont apportées.
\section{\label{III-B-2}Le \textit{Lac de données} de l'\ac{ina}: repositionnement du référentiel au centre du modèle de données}
\titreEntete{Le Lac de données de l'INA}

%intro

%conclu
\section{\label{III-B-3}Perspectives d'utilisation}
\titreEntete{Perspectives d'utilisation}

%intro
\begin{citationLongue}
	[L]e rôle central du référentiel va se poursuivre au-delà de [la] réflexion sur l’interopérabilité. En effet, ils sont la pierre angulaire des nouveaux bouleversements autour du \textit{machine learning} et du \textit{deep learning}.\footcite{poupeau_reflexions_2018}
\end{citationLongue}

L'interopérabilité des données --- comprises au sens large, avec les métadonnées et les référentiels --- est une avancée majeure dans la conception des modèles de données. En effet, en plus de résoudre les nombreuses difficultés qui résultaient de la multiplicité des bases de données à l'\ac{ina} et de leur conception par les besoins et les usages qui en étaient faits, la centralisation de ces données et leur interopérabilité ouvre des possibilités quant aux nouveaux usages qu'il est possible d'envisager ou de satisfaire.\\

L'intégration de l'\ac{ina} dans le \textit{big data} est l'une de ces possibilités: l'utilisation de l'intelligence artificielle permet à la fois une amélioration des descriptions déjà réalisées au catalogage, et à la fois la création de nouvelles descriptions. Plus encore que ces actions sur les métadonnées et la création de descriptions de contenu, l'amélioration de la valorisation des documents de l'\ac{ina} est possible avec le \ldd et l'uniformisation du modèle de données.

\subsection{\label{III-B-3-a}Permettre l’intégration des données issues de la description et de la segmentation de vidéos dans le \ldd : réutilisation des concepts et enrichissement des métadonnées}
\titreEntete{La description automatique de vidéos}
	
La reconnaissance d'entités nommées est un enjeu essentiel dans la description de documents. Cette dernière est facilitée, depuis quelques années, par des outils nés de programme de recherche sur l'extraction d'entités nommées dans les textes. La classification d'images est également une pratique facilitée par des algorithmes développés par des entreprises comme Google ou Amazon. Cependant, l'extraction d'entités nommées dans des vidéos reste peu pratiquée. L'\ac{ina} ne dispose alors pas d'outils suffisants et existants pour effectuer une recherche de personne, de logo ou de tableau dans une vidéo. Cette recherche et cette extraction d'entités visuelles dans des vidéos représente l'un des projets de l'\ac{ina}, DigInPix\footcite{institut_national_de_laudiovisuel_diginpix_nodate-1}.\\

Dans DigInPix, le rôle du référentiel est essentiel, le référentiel est indispensable au fonctionnement de l'algorithme: le dictionnaire d'entités nommées sur lequel repose l'algorithme permet de reconnaître des logos, des peintures, des personnes physiques ou morales\footnote{Bien que différent par la nature des données stockées, des images, ce dictionnaire est similaire à tout autre dictionnaire comme décrit plus tôt dans notre propos, afin de décrire la diversité d'une entité: \og Nous appelons “dictionnaire” une liste d'entités nommées, regroupées pour leur appartenance à certains concepts de niveau hiérarchique supérieur (par exemple, personnes morales, personnes physiques, peintures, bâtiments, etc.).\fg{} in \cite{institut_national_de_laudiovisuel_diginpix_nodate}}. Les bases de données initiales de l'\ac{ina} n'étant pas suffisamment complètes, les entités ont été enrichies de représentations visuelles trouvées sur le Web, afin d'établir un imposant corpus de comparaison face aux vidéos qui seront à traiter. Une image est tirée de chaque vidéo à intervalle régulier, afin de la comparer à l'ensemble du dictionnaire: plusieurs entités nommées peuvent ainsi être reconnues dans une même image d'une vidéo. De plus, un taux de fiabilité est attribué à chaque rapprochement (\reference{diginpix_result}).
\begin{figure}[!h]
	\centering
	\includegraphics[width=7cm]{images/diginpix_resultat.jpg}
	\caption[Extraction des entités nommées d'un programme de France 2 avec DigInPix]{Extraction des entités nommées d'un programme de France 2 avec DigInPix [Source: \cite{institut_national_de_laudiovisuel_diginpix_nodate}]}
	\label{diginpix_result}
\end{figure}

Avec le projet du \ldd, la création de descriptions de contenus peut aller plus loin encore. En effet, le modèle de données unifié, comprenant l'ensemble des anciens référentiels de la \ac{ddcol}, permet de mettre en relation les données issues automatiquement d'un processus de traitement des vidéos par l'intelligence artificielle avec les concepts du \ldd. La segmentation automatique de vidéos\footnote{Le projet est SAAJ, Segmentation et Analyse Automatique des Journaux télévisés.} montre la diversité des utilisations possibles d'un référentiel quand celui-ci est uniforme et centralisé. Ce nouveau projet, au sein de celui du \ldd, débute en 2018 et conduit à la création de multiples outils, tous permettant une description automatique du contenu d'une vidéo --- les journaux télévisés et les chaînes d'information. Ainsi, la segmentation automatique par l'intelligence artificielle permet:
\begin{itemize}
	\item l'établissement d'une grille de programmation à partir des métadonnées fournies par les diffuseurs, les producteurs, \dots afin de déterminer les horaires prévus et habituels de chaque programme pour les chaînes d'information
	\item la classification automatique du programme ou des segments de programme selon une typologie précise --- plateau, présentateur, reportage, \dots
	\item la transcription des voix et de la parole, produisant ainsi un texte non formaté à partir duquel une description du contenu est effectuée avec des entités nommées qui en sont extraites et un alignement avec les entités de Wikidata
	\item l'océrisation des textes présents dans l'image, ce qui produit également un texte qui permet une description intellectuelle du contenu et un alignement des entités nommées avec Wikidata; cette océrisation concerne notamment les bandeaux des journaux télévisés dans lesquels le nom et la fonction de chaque personne sont indiqués
	\item la description automatique d'une image par un tagging d'entités nommées
	\item la reconnaissance de visages afin d'identifier le présentateur du journal télévisé, ou les protagonistes des vidéos
	\item la reconnaissance d'images et de logos, afin d'enrichir la description déjà précise de la vidéo ou du segment
\end{itemize} 

Chacun de ces outils fonctionne avec ou en relation avec un ou plusieurs référentiels: ils peuvent être internes, c'est à dire propres à l'\ac{ina}, ou bien externes comme Wikidata qui permet un enrichissement et une ouverture des métadonnées vers l'extérieur. Ces données générées automatiquement sont créées sous le modèle du \ldd et sont par conséquent en relation avec ses concepts. 

\subsection{\label{III-B-3-b}Faciliter et améliorer le catalogage des documents de l’\ac{ina} par l’extraction automatique de données}
\titreEntete{Faciliter et améliorer le catalogage des documents}

L'apport du projet SAAJ est une description fine et précise de l'ensemble d'une vidéo. Au-delà de la génération automatique de métadonnées dans le \ldd, il devient une aide pour le technicien de gestion des contenus multimédia de l'\ac{ina}. En effet, il n'a plus à créer les métadonnées associées au document, mais à superviser leur qualité et leur véracité: \og Le documentaliste « humain » est-il destiné à
passer du statut de producteur de données à celui de contrôleur de la qualité des fruits de l’automatisation ?\fg{}\footcite[p.134]{alquier_production_2017}. Cet aspect de contrôle qualité est un usage indirect des données générées automatiquement: il est positif pour la création précise, fine et complète de métadonnées sur un programme; mais il contraint à un changement de pratiques de catalogage, où l'humain n'a plus le rôle principal, qui est intellectuel, dans lequel il décrit le document et son contenu. \\

Cette amélioration et cette facilitation du travail de catalogage a lieu depuis des données nouvelles générées par l'intelligence artificielle. Cependant, le contrôle de la qualité des métadonnées, et leur enrichissement, peut également passer par un traitement \textit{a posteriori}. En effet, l'apparition du Web sémantique et son adoption par un grand nombre d'institutions a poussé l'\ac{ina}, associé à d'autres institutions, à mener le projet Qualinca entre 2012 et 2015 afin \og d'améliorer la richesse, la cohérence et l’interopérabilité des métadonnées du système documentaire de l’Ina à travers la mise en œuvre d’une activité de recherche dans le domaine des techniques de liage de données\fg{}\footcite[p.129]{alquier_production_2017}. Qualinca repose sur de nombreux enjeux, comme la possibilité de partager des identifiants communs entre les différents métiers, l'amélioration des descriptions de contenus grâce aux données extérieures du \ac{lod}, mais également d'effacer les ambiguïtés des termes des lexiques de l'\ac{ina}.\\

Se basant sur deux algorithmes, ProbFr et Agreg, Qualinca s'est surtout tourné vers les alignements de corpus de musique, et d'homonymes de personnes physiques et d'émissions. Dans cet alignement des homonymes avec le \ac{lod}, la base DBpedia --- Wikidata n'est né qu'en 2014 ---, les résultats sont peu exploitables et se heurtent, comme nous avons pu le constater lors de l'alignement des personnes physiques avec Wikidata, au langage naturel des fonctions que les algorithmes sont incapables de dépasser: sur 5000 \nP{Jacques}{Martin}, 667 différents ont été identifiés par les algorithmes\footcite[p.133]{alquier_production_2017}.\\

L'extraction automatique d'entités nommées a permis la création de nouvelles métadonnées associées non pas au matériel ou aux données de diffusion, mais au contenu intellectuel des vidéos, ainsi que l'apport d'une aide au catalogage par la qualité des entités fournies et leur précision.

\subsection{\label{III-B-3-c}Améliorer la valorisation des documents et offrir une meilleure expérience utilisateur}
\titreEntete{Améliorer la valorisation des documents}

La centralisation des données de l'\ac{ina} au sein du \ldd ouvre des possibilités pour la valorisation des documents auprès de tous les publics\footnote{Pour l'ensemble de l'offre disponible, voir la \reference{I-B}.}. D'abord, cette centralisation permet la création de multiples applications pour l'utilisateur, sans que cela ne modifie la structure des données\footnote{Voir \reference{annexe_lac} (\reference{lac_infra}).}; ainsi, la présence de l'\ac{ina} en est modifiée par l'apparition d'un \textit{hub} regroupant l'ensemble de l'offre numérique de l'Institut\footnote{\og L’autre grand défi posé à l’Institut, c’est celui de l’accessibilité de ses propositions. Les rassembler au sein d’un grand portail numérique, un hub qui offrira en quelques clics un accès renouvelé, simplifié et cohérent à l’ensemble des activités, contenus et services de l’Ina, est ainsi l’objectif qui mobilise aujourd’hui toute l’entreprise à l’horizon de 2019.\fg{} in \cite{vallet_ina_nodate}}.\\

Cette centralisation de l'offre est aussi présente pour les usages internes avec la création d'une nouvelle interface de consultation des métadonnées et des documents en eux-mêmes, Notilus, afin d'éviter la consultation croisée de multiples interfaces selon la provenance du document comme cela était le cas avant le \ldd. Par cette interface, le \ac{dl}, le \ac{da}, puis la \ac{dj} et l'ensemble des professionnels de l'\ac{ina}, ont accès aux mêmes données et aux mêmes documents, en un point unique, une interface de consultation qui reconstitue les instances du \ldd.\\

L'amélioration de l'expérience utilisateur est également une priorité dans une période où la modification des pratiques est radicale: ces pratiques sont quasiment toutes numériques et contraignent l'\ac{ina} à s'adapter. Si le site \url{https://www.ina.fr} est né dès 2009, le \ldd va pouvoir lui apporter d'importantes améliorations. En effet, la présence des référentiels y est limitée et limite les possibilités de rebonds de la part de l'utilisateur\footnote{Voir \reference{inafr_ref}.}. Ainsi, les liens des personnes au générique ne renvoient pas à une vedette personne de l'\ac{ina}, mais à des résultats de recherche sur le nom de cette personne. 
\begin{figure}[!h]
	\centering
	\includegraphics[width=10cm]{images/inafr_ecole_fans.png}
	\caption[Métadonnées associées à un document sur ina.fr]{Métadonnées associées à un document sur ina.fr [Source: \url{https://www.ina.fr/video/I11297765/mimi-mathy-et-yves-lecoq-jouant-a-l-ecole-des-fans-video.html}]}
	\label{inafr_result}
\end{figure}\\
Les possibilités offertes par le \ldd sont multiples et non pouvons en imaginer certaines, basées sur le seul usage des concepts et de leurs relations, qui faciliteraient la recherche de l'utilisateur. Ainsi, de même que la BnF, la création de vedettes de personnes est envisageable afin de regrouper en une même page les informations biographiques, ainsi que les documents liés (les instances) ou bien les thématiques principales (les concepts liés). Les termes d'indexation et de description des vidéos peuvent également être concernés par ce regroupement d'informations et de liens. Cependant, plus encore que ces regroupement de métadonnées, d'instances et de concepts relatifs à un concept, il est désormais possible, avec les liens établis avec le \ac{lod}, d'obtenir des informations manquantes et d'enrichir les données proposées à l'utilisateur: un lien peut être inséré, comme c'est le cas dans \ac{viaf}, ou bien les champs peuvent être directement remplis sur la page HTML.\\

Ces possibilités ne sont possibles que grâce à la déconstruction de l'information dans le \ldd, permettant alors une grande modularité des données dans les usages qui en sont faits. Ces usages ne sont pas tous nés et le \ac{ldd} doit pouvoir permettre à l'\ac{ina} de les remplir sans avoir recours à un modification du modèle de données ou à la création d'une nouvelle de données. Ainsi, s'il est nécessaire de publier les données\footnote{Cet aspect semble difficile pour l'\ac{ina} en raison des données personnelles qui y sont conservées. Cependant, la loi de 2016 pour une République numérique (\cite{noauthor_loi_2016}) encourage à la publication des données de référence qui peuvent être réutilisées par d'autres services ou d'autres institutions.} sur le Web et plus particulièrement sur le Web de données, une représentation \ac{rdf} est possible; avec la présence forte de l\ac{ina} sur les réseaux sociaux, il peut être envisager de créer des publications automatiquement à partir de tags issus de concepts; \dots

%conclu
\bigskip
\bigskip
Le référentiel a atteint une place centrale dans le \ldd: l'ensemble des applications et des sites de l'Institut fonctionnent ou vont fonctionner depuis ce silo de métadonnées qui a été pensé selon la donnée et non plus les besoins. Ces besoins, évolutifs et dépendants de la période, ne peuvent pas tous être prédits, ce qui a conduit à la constitution d'un modèle de données souples et d'un processus intermédiaire de traitement de ces données de manière à offrir à chaque application les données qui lui sont nécessaires.
