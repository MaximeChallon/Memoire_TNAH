%%%%%%%%%%%%%%%%%%%%%
%%%%% PREAMBULE %%%%%
%%%%%%%%%%%%%%%%%%%%%
\documentclass[a4paper,12pt,twoside]{book}
\usepackage{fontspec}

%%%%% Index %%%%%
%\usepackage{index}
%\usepackage{imakeidx}%pour les index, à charger avant hyperref
%\makeindex
%\makeindex[name=lieux, title=Index des noms de lieux]%créer un index des lieux

\usepackage[pdfusetitle, pdfsubject ={Mémoire TNAH}, pdfkeywords={les mots-clés}]{hyperref}
\usepackage[french]{babel}
\usepackage{morewrites}
\usepackage{tocbibind} %paquet pour mettre index et bib dans la toc

% configurer le document selon les normes de l'école
\usepackage[margin=2.5cm]{geometry}
\usepackage{setspace}
\onehalfspacing
\setlength\parindent{1cm}

\usepackage{lettrine}

%%%%% Dessin %%%%%
%\usepackage{qtree}
% avec un code comme ça pour un arbre de classification
%\Tree[.IP [.NP [.Det \textit{the} ]
%[.N\1 [.N \textit{package} ]]]
%[.I\1 [.I \textsc{3sg.Pres} ]
%[.VP [.V\1 [.V \textit{is} ]
%[.AP [.Deg \textit{really} ]
%[.A\1 [.A \textit{simple} ]
%\qroof{\textit{to use}}.CP ]]]]]]
\usepackage{tikz} %paquet pour dessiner ; à placer avant graphicx
\usepackage{graphicx} %paquet image


\usepackage{caption} %pour que la mention de figure n'apparaisse pas dans les légendes de l'image


%%%%% Bibliographie %%%%%
%\usepackage[backend=biber, sorting=nyt, style=enc, maxbibnames=3]{biblatex}
%\addbibresource{biblio.bib}
%\nocite{*}
%\defbibnote{intro}{Cette bibliographie contient toutes les références utilisées pour ce cours} %pour des notes introductives dans le début de la biblio

%%%%% Abréviations %%%%%
\usepackage{acro}
\DeclareAcronym{ina}{short = INA, long = Institut National de l'Audiovisuel}
% apple par \ac{ina}


%%%%% Nouvelles commandes %%%%%
% \newcommand{\reference}[1]{\autoref{#1}: \nameref{#1}}

%%%%% Nouveaux environnements %%%%%

\author{Maxime Challon - M2 TNAH}
\title{Les référentiels en institutions patrimoniales : évolution des pratiques et repositionnement. L’exemple des référentiels de l’Institut National de l’Audiovisuel.}

%%%%%%%%%%%%%%%%%%%%
%%%%% DOCUMENT %%%%%
%%%%%%%%%%%%%%%%%%%%
\begin{document}	
	\frontmatter
	\begin{titlepage}
		\begin{center}
			
			\bigskip
			
			\begin{large}
				\'ECOLE NATIONALE DES CHARTES
			\end{large}
			\begin{center}\rule{2cm}{0.02cm}\end{center}
			
			\bigskip
			\bigskip
			\bigskip
			\begin{Large}
				\textbf{Maxime Challon}\\
			\end{Large}
			\begin{normalsize} \textit{licencié ès histoire}
			\end{normalsize}
			
			\bigskip
			\bigskip
			\bigskip
			
			\begin{Huge}
				\textbf{Les référentiels en institutions patrimoniales : évolution des pratiques et repositionnement}\\
			\end{Huge}
			\bigskip
			\bigskip
			\begin{LARGE}
				\textbf{L’exemple des référentiels de l’Institut national de l’Audiovisuel}\\
			\end{LARGE}
			
			\bigskip
			\bigskip
			\bigskip
			\begin{large}
			\end{large}
			\vfill
			
			\begin{large}
				Mémoire 
				pour le diplôme de master \\
				\og{} Technologies numériques appliquées à l'histoire \fg{} \\
				\bigskip
				2020
			\end{large}
			
		\end{center}
	\end{titlepage}

\thispagestyle{empty}	
\cleardoublepage
	
		\chapter*{Résumé}
	\titreEntete{Résumé}
\addcontentsline{toc}{chapter}{Résumé}
	\medskip
	Ce mémoire, réalisé pour l'obtention du diplôme de Master 2 \og Technologies numériques appliquées à l'histoire\fg{} de l'École nationale des Chartes, retrace l'évolution des pratiques documentaires sur les référentiels en institution patrimoniale à travers l'étude des référentiels de l'\ac{ina} et leurs alignements. Cette étude de l'évolution des formes et des structures des référentiels est liée à l'évolution de la place de ces référentiels au sein des systèmes documentaires, ainsi qu'aux besoins qui leur sont liés.\\
	
	\textbf{Mots-clés~:} institut national de l'audiovisuel; référentiel; thésaurus; vocabulaire contrôlé; vocabulaire hiérarchique; ontologie; web de données; Wikidata; liens; alignement.
	
	\textbf{Informations bibliographiques~:} Maxime Challon, \textit{Les référentiels en institutions patrimoniales : évolution des pratiques et repositionnement. L’exemple des référentiels de l’Institut National de l’Audiovisuel.}, mémoire de master \og{}Technologies numériques appliquées à l'histoire\fg{}, dir. Gautier Poupeau, École nationale des chartes, 2020.
	
		\chapter*{Remerciements}
	\titreEntete{Remerciements}
	\addcontentsline{toc}{chapter}{Remerciements}
	
	\lettrine{M}es remerciements vont tout d'abord à Gautier \textsc{Poupeau}, mon maître de stage, qui m'a accueilli, guidé, conseillé et intégré à son équipe malgré le travail à distance imposé par le contexte actuel. Je souhaite également remercier Axel \textsc{Roche-Dioré} pour ses explications et son soutien dans la réalisation technique de mon stage.\\
	
	J'adresse aussi mes remerciements aux membres du pôle \og Ingénierie de la Donnée\fg{} pour le temps qu'ils m'ont accordé, Lauryne \textsc{Lemosquet}, Otmane \textsc{Elabboubi} et Akli \textsc{Abdi}, ainsi qu'à Florence \textsc{Bréant}, cheffe de projet pour le \textit{Lac de données}. \\
	
	Que soit également remercié l'ensemble du département \og Architecture et Innovation\fg{} de l'\ac{ina} pour l'accompagnement fourni tout au long de mon stage, notamment Stanislas \textsc{de Maigret} et Matthieu \textsc{Boricaud} pour le déploiement de l'application, et Olivio \textsc{Ségura} pour la présentation des archives de l'\ac{ina}.
	
	\chapter*{Liste des abréviations}
	\addcontentsline{toc}{chapter}{Liste des abréviations}
	\printacronyms[heading=none]
	
		\chapter*{Introduction}
\addcontentsline{toc}{chapter}{Introduction}
% texte de l'intro ici



\thispagestyle{empty}
\cleardoublepage
	
	\mainmatter
	
	\part{\label{controler}CONTRÔLER. A la recherche de clés (années 1960 – fin des années 1990)}
	
	\chapter{\label{I-A}Le référentiel comme clé}
	\chapter{\label{I-B}L'arbre, un vocabulaire contrôlé hiérarchique}
	\chapter{\label{I-C}Les référentiels à l’INA}
	
	\part{\label{relier}RELIER. Vers le partage de référentiels communs (début des années 2000 – milieu des années 2010)}
	
	\chapter{\label{II-A}Le web de données: une exposition commune des référentiels}
	\chapter{\label{II-B}Partager des structurations similaires de jeux de données par les classes et les propriétés : les ontologies, grammaires communes mais spécifiques}
	\chapter{\label{II-C}Relier ses données à Wikidata}
	
	\part{\label{centraliser}CENTRALISER. Le référentiel, clé de voûte et pivot (depuis le milieu des années 2010)}	
	
	\chapter{\label{III-A}Les labyrinthes comme réseaux de données et de liens}
	\chapter{\label{III-B}Le Lac de données de l’INA : le référentiel au centre du modèle}
	\chapter{\label{III-C}Centraliser les référentiels de l’INA dans le Lac de données: l'exemple de l'alignement de deux référentiels de personnes physiques}
	
	\chaptertoc{Conclusion}
\titreEntete{Conclusion}

%historique de struct ref: arbre au laby et modele reseau

%changement des usages et des besoins...

%... qui produit changement place ref
	
	\appendix
	\part*{Annexes}	
	\addcontentsline{toc}{part}{Annexes}
	\setcounter{chapter}{0}

\chapter{\label{annexe_index_schoepflin}Les index de la Renaissance, termes contrôlés et classification alphabétique (les index de l'\textit{Alsatia Illustrata} de \nP{Jean-Daniel}{Schoepflin})}
\titreEntete{Annexe \thechapter}

\begin{figure}[!h]
	\centering
	\begin{minipage}[c]{.46\linewidth}
		\includegraphics[width=6cm]{images/index_auctorum_alsatia.jpg}
		\caption{Index auctorum}
	\end{minipage} \hfill
	\begin{minipage}[c]{.46\linewidth}
		\includegraphics[width=6cm]{images/index_rerum_alsatia.jpg}
		\caption{Index rerum}
	\end{minipage} 
	\medskip
	Extraits des deux index de l'œuvre de \nP{Jean-Daniel}{Schoepflin} [Source: \url{http://bibliotheque-numerique.inha.fr/idurl/1/12532}, p.804 et 813]
\end{figure}

\chapter{\label{annexe_types_interop}Les différents types d'interopérabilité}
\titreEntete{Annexe \thechapter}

\begin{figure}[!h]
	\centering
	\includegraphics[width=12cm]{images/interop_conversion_copie.jpeg}
	\medskip
	\caption[L'interopérabilité par conversion et copie]{L'interopérabilité par conversion et copie [Source: \cite{bermes_2_2013}]}
\end{figure}

\begin{figure}[!h]
	\centering
	\includegraphics[width=12cm]{images/interop_denom_commun.jpeg}
	\medskip
	\caption[L'interopérabilité par le plus petit dénominateur commun]{L'interopérabilité par le plus petit dénominateur commun [Source: \cite{bermes_2_2013}]}
\end{figure}

\chapter{\label{annexe_thesaurus}Le thésaurus de noms communs de l'\ac{ina}}
\titreEntete{Annexe \thechapter}

\begin{figure}[!h]
	\centering
	\includegraphics[width=7cm]{images/cadreur_hierarchie.png}
	\medskip
	\caption[Extrait du thésaurus de noms communs de l'\ac{ina}]{Extrait du thésaurus de noms communs de l'\ac{ina} autour du terme \og Cadreur\fg{}}
\end{figure}
	
	\backmatter
	
	
%bibliographie ici dans les normes de l'école
%\printbibliography[title= Bibliographie sélective, prenote=intro]%postnote est aussi possible
%\printbibliography[heading=subbibliography, keyword={semantique}, title={Sémantique}]%biblio sélective pour un mot clé donné

% index à mettre ici si index	
%	\printindex
%\printindex[lieux]

% si figures
%	\listoffigures

	\tableofcontents
	
\end{document}