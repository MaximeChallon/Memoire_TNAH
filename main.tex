%%%%%%%%%%%%%%%%%%%%%
%%%%% PREAMBULE %%%%%
%%%%%%%%%%%%%%%%%%%%%
\documentclass[a4paper,12pt,twoside]{book}
\usepackage{fontspec}

%%%%% Index %%%%%
\usepackage{index}
\usepackage{imakeidx}%pour les index, à charger avant hyperref
\makeindex
\makeindex[name=ref, title=Index des noms et des types de référentiels]

\usepackage[pdfusetitle, pdfsubject ={Mémoire TNAH}, pdfkeywords={institut national de l'audiovisuel; référentiel; thésaurus; vocabulaire contrôlé; vocabulaire hiérarchique; ontologie; web de données; Wikidata; liens; alignement}]{hyperref}
\usepackage[english, french]{babel}
\usepackage{morewrites}
\usepackage{tocbibind} %paquet pour mettre index et bib dans la toc

% configurer le document selon les normes de l'école
\usepackage[margin=2.5cm]{geometry}
\usepackage{setspace}
\onehalfspacing
\setlength\parindent{1cm}

\usepackage{lettrine}
\usepackage{minted}

%%%%% Dessin %%%%%
\usepackage{qtree}
\usepackage{pstricks}
\usepackage{tikz} %paquet pour dessiner ; à placer avant graphicx
\usepackage{graphicx} %paquet image
\usepackage{wrapfig}
\usepackage{caption} %pour que la mention de figure n'apparaisse pas dans les légendes de l'image

%%%%% Tableaux %%%%%
\usepackage{csvsimple}
\usepackage{tabularx}
\newcolumntype{C}{>{\centering}X}
\usepackage{longtable}


%%%%% Bibliographie %%%%%
\usepackage[backend=biber, sorting=nyt, style=enc, maxbibnames=3]{biblatex}
\addbibresource{bibliographie/bib.bib}
\nocite{*}
%\defbibnote{intro}{Cette bibliographie contient toutes les références utilisées pour ce cours} %pour des notes introductives dans le début de la biblio

%%%%% Abréviations %%%%%
\usepackage{acro}
\DeclareAcronym{api}{short = API, long = Application Programming Interface}
\DeclareAcronym{bibo}{short = bibo, long = the Bibliographic Ontology}
\DeclareAcronym{cidoccrm}{short = CIDOC-CRM, long = Comité International pour la Documentation - Conceptual Reference Model}
\DeclareAcronym{da}{short = DA, long = Département des Archives Professionnelles}
\DeclareAcronym{ddcol}{short = DDCOL, long = Direction déléguée aux collections}
\DeclareAcronym{dj}{short = DJ, long = Direction juridique}
\DeclareAcronym{dl}{short = DL, long = Dépôt Légal}
\DeclareAcronym{dsi}{short = DSI, long = Direction des systèmes d'information}
\DeclareAcronym{ead}{short = EAD, long = Encoded Archival Description}
\DeclareAcronym{epic}{short = ÉPIC, long = Établissement public à caractère industriel et commercial}
\DeclareAcronym{foaf}{short = FOAF, long = Friend of a Friend}
\DeclareAcronym{frad}{short = FRAD, long = Functional Requirements for Authority Data}
\DeclareAcronym{frbr}{short = FRBR, long = Functional Requirements for Bibliographic Records}
\DeclareAcronym{frsad}{short = FRSAD, long = Functional Requirements for Subject Authority Data}
\DeclareAcronym{gemet}{short = GEMET, long = General Multilingual Environmental Thesaurus}
\DeclareAcronym{http}{short = HTTP, long = HyperText Transfer Protocol}
\DeclareAcronym{idref}{short = IdRef, long = Identifiants et Référentiels pour l'enseignement supérieur et la recherche}
\DeclareAcronym{ina}{short = INA, long = Institut national de l'Audiovisuel}
\DeclareAcronym{isan}{short = ISAN, long = \textit{International Standard Audiovisual Number}}
\DeclareAcronym{isni}{short = ISNI, long = \textit{International Standard Name Identifier}}
\DeclareAcronym{json}{short = JSON, long = JavaScript Object Notation}
\DeclareAcronym{kos}{short = KOS, long = Knowledge Organization Systems}
\DeclareAcronym{lcsh}{short = LCSH, long = Library of Congress Subject Headings}
\DeclareAcronym{led}{short = LED, long = Linked Enterprise Data}
\DeclareAcronym{lod}{short = LOD, long = Linked Open Data}
\DeclareAcronym{marc}{short = MARC, long = MAchine-Readable Cataloging}
\DeclareAcronym{oaipmh}{short = OAI-PMH, long = Open Archive Intiative Protocol for Metadata Harvesting}
\DeclareAcronym{oclc}{short = OCLC, long = Online Computer Library Center}
\DeclareAcronym{ortf}{short = ORTF, long = Office de la radio-télévision française}
\DeclareAcronym{rameau}{short = RAMEAU, long =Répertoire d'autorité-matière encyclopédique et alphabétique unifié}
\DeclareAcronym{owl}{short = OWL, long = Web Ontology Language}
\DeclareAcronym{rda}{short = RDA, long = Resource Description and Access}
\DeclareAcronym{rdf}{short = RDF, long = Resource Description Framework}
\DeclareAcronym{rdfs}{short = RDFS, long = RDF Schema}
\DeclareAcronym{skos}{short = SKOS, long = Simple Knowledge Organization System}
\DeclareAcronym{sparql}{short = SPARQL, long = SPARQL Protocol and RDF Query Language}
\DeclareAcronym{unimarc}{short = UNIMARC, long =UNIversal MAchine-Readable Cataloging}
\DeclareAcronym{viaf}{short = VIAF, long = Virtual International Authority File}


%%%%% Nouvelles commandes %%%%%
\newcommand{\reference}[1]{\autoref{#1}: \nameref{#1}}

\newcommand{\ldd}{\textit{Lac de données~}}

\newcommand{\chaptertoc}[1]{\chapter*{#1}
	\addcontentsline{toc}{chapter}{#1}
	\markboth{\slshape\MakeUppercase{#1}}{\slshape\MakeUppercase{#1}}}

\newcommand{\titreEntete}[1]{\markboth{\slshape\MakeUppercase{#1}}{}}

\newcommand{\nP}[2]{#1 \textsc{#2}}

%%%%% Nouveaux environnements %%%%%
\newenvironment{citationLongue}{\begin{quotation}\og}{\fg{}\end{quotation}}

\author{Maxime Challon - M2 TNAH}
\title{Les référentiels en institutions patrimoniales : évolution des pratiques et repositionnement. L’exemple des référentiels de l’Institut national de l’Audiovisuel.}

%%%%%%%%%%%%%%%%%%%%
%%%%% DOCUMENT %%%%%
%%%%%%%%%%%%%%%%%%%%
\begin{document}
	\renewcommand*\appendixautorefname{Annexe}
	\renewcommand*\chapterautorefname{Chapitre}
	\renewcommand*\partautorefname{Partie}
	\renewcommand*\subsectionautorefname{Sous-section}
		
	\frontmatter
	\begin{titlepage}
		\begin{center}
			
			\bigskip
			
			\begin{large}
				\'ECOLE NATIONALE DES CHARTES
			\end{large}
			\begin{center}\rule{2cm}{0.02cm}\end{center}
			
			\bigskip
			\bigskip
			\bigskip
			\begin{Large}
				\textbf{Maxime Challon}\\
			\end{Large}
			\begin{normalsize} \textit{licencié ès histoire}
			\end{normalsize}
			
			\bigskip
			\bigskip
			\bigskip
			
			\begin{Huge}
				\textbf{Les référentiels en institutions patrimoniales : évolution des pratiques et repositionnement}\\
			\end{Huge}
			\bigskip
			\bigskip
			\begin{LARGE}
				\textbf{L’exemple des référentiels de l’Institut national de l’Audiovisuel}\\
			\end{LARGE}
			
			\bigskip
			\bigskip
			\bigskip
			\begin{large}
			\end{large}
			\vfill
			
			\begin{large}
				Mémoire 
				pour le diplôme de master \\
				\og{} Technologies numériques appliquées à l'histoire \fg{} \\
				\bigskip
				2020
			\end{large}
			
		\end{center}
	\end{titlepage}

\thispagestyle{empty}	
\cleardoublepage
	
		\chapter*{Résumé}
	\titreEntete{Résumé}
\addcontentsline{toc}{chapter}{Résumé}
	\medskip
	Ce mémoire, réalisé pour l'obtention du diplôme de Master 2 \og Technologies numériques appliquées à l'histoire\fg{} de l'École nationale des Chartes, retrace l'évolution des pratiques documentaires sur les référentiels en institution patrimoniale à travers l'étude des référentiels de l'\ac{ina} et leurs alignements. Cette étude de l'évolution des formes et des structures des référentiels est liée à l'évolution de la place de ces référentiels au sein des systèmes documentaires, ainsi qu'aux besoins qui leur sont liés.\\
	
	\textbf{Mots-clés~:} institut national de l'audiovisuel; référentiel; thésaurus; vocabulaire contrôlé; vocabulaire hiérarchique; ontologie; web de données; Wikidata; liens; alignement.
	
	\textbf{Informations bibliographiques~:} Maxime Challon, \textit{Les référentiels en institutions patrimoniales : évolution des pratiques et repositionnement. L’exemple des référentiels de l’Institut National de l’Audiovisuel.}, mémoire de master \og{}Technologies numériques appliquées à l'histoire\fg{}, dir. Gautier Poupeau, École nationale des chartes, 2020.
	
		\chapter*{Remerciements}
	\titreEntete{Remerciements}
	\addcontentsline{toc}{chapter}{Remerciements}
	
	\lettrine{M}es remerciements vont tout d'abord à Gautier \textsc{Poupeau}, mon maître de stage, qui m'a accueilli, guidé, conseillé et intégré à son équipe malgré le travail à distance imposé par le contexte actuel. Je souhaite également remercier Axel \textsc{Roche-Dioré} pour ses explications et son soutien dans la réalisation technique de mon stage.\\
	
	J'adresse aussi mes remerciements aux membres du pôle \og Ingénierie de la Donnée\fg{} pour le temps qu'ils m'ont accordé, Lauryne \textsc{Lemosquet}, Otmane \textsc{Elabboubi} et Akli \textsc{Abdi}, ainsi qu'à Florence \textsc{Bréant}, cheffe de projet pour le \textit{Lac de données}. \\
	
	Que soit également remercié l'ensemble du département \og Architecture et Innovation\fg{} de l'\ac{ina} pour l'accompagnement fourni tout au long de mon stage, notamment Stanislas \textsc{de Maigret} et Matthieu \textsc{Boricaud} pour le déploiement de l'application, et Olivio \textsc{Ségura} pour la présentation des archives de l'\ac{ina}.
	
	\chaptertoc{Liste des abréviations}
	\printacronyms[heading=none]
	
		\chapter*{Introduction}
\addcontentsline{toc}{chapter}{Introduction}
% texte de l'intro ici



\thispagestyle{empty}
\cleardoublepage
	
	\mainmatter
	
	\part{\label{controler}CONTRÔLER. A la recherche de clés (années 1960 – fin des années 1990)}
	
	Simple liste de mots ou thésaurus, un référentiel peut être conçu sous différentes formes. Dans un premier temps, son utilisation répond à un unique besoin: contrôler les formes que peuvent prendre les termes de manière à pouvoir les associer plusieurs fois à des documents, pour éviter une redondance de la même information à différents endroits. Le référentiel n'a alors qu'une fonction de clé. Cette opération, simple au premier abord, peut se complexifier avec l'intégration de synonymes ou de termes similaires associés au terme principal. L'arbre de classification, modèle du thésaurus, permet de contrôler un ensemble de termes tout en leur apportant du sens.\\
	
	Les référentiels de l'\ac{ina} reflètent cette évolution et les pratiques qui lui sont liées. L'\ac{ina} est pourvu de listes de noms associés à des synonymes ou des variantes pour les noms de personnes, mais les noms communs --- plus difficiles à exprimer de manière contrôlée --- se trouvent eux dans un thésaurus permettant d'établir plus de relations sémantiques entre les termes similaires.
	
	\chapter{\label{I-A}Le référentiel comme clé}
\titreEntete{Le référentiel comme clé}

Considéré comme une simple aide ou outil au service du documentaliste ou de l'utilisateur, le référentiel trouve d'abord sa place comme fournisseur de clés. Son utilisation principale est d'offrir au document décrit des vedettes qui puissent permettre une classification ou une recherche aisée de ce document. Cependant, pour être efficaces, ces vedettes doivent partager un langage contrôlé, des règles de graphie, de syntaxe, \dots D'abord conservées sur des fichiers papier en institutions patrimoniales, ces vedettes ont été parmi les premiers éléments rétroconvertis, donnant naissance aux fichiers d'autorité numériques, et permettant une interopérabilité entre les référentiels par le biais des portails numériques.

\section{\label{I-A-1}Du langage libre au langage contrôlé: vers l'indexation}
\titreEntete{Vers l'indexation}

\begin{quote}
	\og La nature n'a pas juré de ne nous offrir que des objets exprimables par des formes simples de langage \footcite[p.18]{valery_variete_1936} \fg{}
\end{quote}

Le langage permet aux hommes de communiquer entre eux. Ce langage libre, naturel, comprend l'ensemble des langues, et donne aux hommes la possibilité de décrire le plus précisément possible le monde qui les entoure, sans jamais atteindre le description idéale. Seulement, ce langage conduit à des variations graphiques ou syntaxiques, selon la déclinaison des noms ou la conjugaison des verbes. La polysémie est également l'une des conséquences de ce langage naturel selon le contexte de chaque mot. Enfin, le langage libre conduit à la synonymie. Toutes ces caractéristiques du langage humain perturbent et complexifient la tâche de description documentaire, bien qu'elles soient essentielles à la communication entre les hommes.\\

Afin de régler ces confusions possibles entre les mots et de régir leur formation, des langages contrôlés ont très vite fait leur apparition. Ils permettent de décrire des concepts, des thèmes, des ouvrages, tout en permettant un classement potentiel. Ce recours aux langages contrôlés est une pratique très ancienne, née avant l'apparition des \textit{codices} lorsque déjà la recherche d'informations était nécessaire. Pratique millénaire, l'attribution de termes contrôlés à une information se perpétue encore actuellement, par exemple sous la forme de \og hashtags\fg{} sur les réseaux sociaux, qui permettent de décrire un texte et de le retrouver ensuite aux côtés d'autres similaires.\\

Dans l'Antiquité, les index n'existent pas encore. Cependant, des \index[ref]{typologie@Typologie!vocabulaires controles@Vocabulaires contrôlés}vocabulaires contrôlés sont utilisés pour le classement et pour la mémorisation des textes. Ces termes contrôlés se retrouvent dans des notes marginales, des tables de concordance ou bien dans les catalogues. Au \textsc{III}\textsuperscript{ème}siècle av. J.C., \nP{Callimaque}{de Silène} réalise le catalogue de la bibliothèque d'Alexandrie en utilisant le genre du texte pour lui déterminer une classe, puis les \textit{volumina} sont rangés dans des rayons selon un ordre alphabétique, ces rayons reflétant les classes attribuées selon le genre.\\

Au Moyen-Âge, les premiers \index[ref]{typologie@Typologie!index@Index}index apparaissent, s'ajoutant aux tables de concordance. \nP{Isidore}{de Séville} ne créé qu'un classement alphabétique dans son Livre X des \textit{Étymologies}, sans indexer son ouvrage. Cinq siècles plus tard, les vedettes commencent à être normalisées dans certaines œuvres, le nominatif ou l'ablatif étant considérés comme la forme retenue, et rassemblées dans un index alphabétique\footnote{\nP{Jean}{Berger}, dans son analyse du \textit{Liber de honoribus}, le plus vieil index alphabétique compilé au \textsc{XII}\textsuperscript{ème}siècle, étudie avec précision l'indexation des chartes du Cartulaire de Saint-Julien de Brioude: les lieux et les personnes sont ainsi indexés. Voir \cite[pp.97 et suivantes]{berger_indexation_2006}}.\\

Avec la Renaissance puis l'Ancien Régime, l'indexation devient plus fine et les index de fin de volumes de plus en plus imposants. Ils permettent au lecteur un accès direct aux passages du texte contenant l'entrée d'index. Plus encore, ces index lient une classification générale suivie d'alphabétique, tout en normalisant leurs entrées\footnote{\nP{Jean-Daniel}{Schoepflin} dans son \textit{Alsatia illustrata} de 1751 créé ainsi deux index distincts: l'un pour les personnes (\textit{Index auctorum}), l'autre pour les termes évoqués dans son œuvre(\textit{Index rerum}). L'ensemble des noms est indexé au nominatif puis ils sont parfois subdivisés en thèmes ou événements. L'index devient ainsi indépendant de la graphie et de la grammaire de la langue utilisée. Voir \cite{schoepflin_alsatia_1751}. Voir \reference{annexe_index_schoepflin}.}\footnote{\nP{Robert}{Estienne} pousse plus loin encore l'indexation, un siècle et demi avant \nP{Jean-Daniel}{Schoepflin}, en créant de multiples \index[ref]{typologie@Typologie!index@Index}index: celui des populations, des villes, \dots. Ces index sont eux-mêmes subdivisés, normalisés et classés alphabétiquement, les rendant œuvre à part entière. Voir \cite{estienne_thesaurus_1573}.}.

\section{\label{I-A-2}Une clé entre les données: une terminologie maîtrisée, objectif des vocabulaires contrôlés}

\section{\label{I-A-3}Une clé entre les jeux de données: l'interopérabilité par les fichiers d'autorité et les portails}
\titreEntete{Une clé entre les jeux de données}

Comme nous l'avons évoqué précédemment (voir \reference{I-A-2}), les vocabulaires contrôlés sont de nouveaux langages, spécifiques et uniformisés, se substituant au langage naturel humain pour un domaine précis. Le vocabulaire est par conséquent un référentiel propre à l'institution qui l'a créé et a pour seul utilisateur cette institution. Seulement, deux institutions aux activités proches créent deux vocabulaires similaires, se distinguant par la complétude de certaines vedettes ou par des variantes de graphies.\\

Le domaine bibliothéconomique a été le premier à informatiser en masse ses vocabulaires et ses \index[ref]{autorites@Autorités!fichiers autorite@Fichiers d'autorité}fichiers d'autorité, permettant ainsi une amélioration de l'expérience utilisateur et du catalogage, tout comme un partage possible avec des institutions proches.

\subsection{\label{I-A-3-a}La naissance des autorités par rétroconversion}
\titreEntete{Les fichiers d'autorité}

\begin{citationLongue}
	Les fichiers d'autorité appartiennent bien à un	ensemble : fonctionnant comme un tout, avec des règles d’interdépendance et d’interopérabilité de ses constituants, ils permettent le contrôle de la cohérence des métadonnées bibliographiques.\footcite[p.6]{aymonin_arabesques_2017}
\end{citationLongue}

Avant la naissance du Web, chaque ouvrage était décrit dans un catalogue et classé par ordre alphabétique des noms d'auteurs. Des \index[ref]{autorites@Autorités!fichiers autorite@Fichiers d'autorité}catalogues thématiques ont été créés, de même que des fichiers physiques en bibliothèque, permettant la recherche de documents selon un sujet précis. Cependant, l'indexation des documents est réduite au titre, à l'auteur, et à quelques sujets. En effet, la structure même d'un fichier papier en bibliothèque nécessite de dupliquer la notice d'un exemplaire en plusieurs notices qui vont être placées par la suite dans le fichier correspondant au sujet.\\

Ces fichiers physiques des bibliothèques, bien qu'utiles aux lecteurs par leur classement thématique, présentent plusieurs difficultés: d'abord, l'indexation se trouve limitée à quelques mots; ensuite, la création d'un fichier thématique est complexe à réaliser par le choix des vedettes et produit alors un immense silence; enfin, la consultation d'une fiche par un lecteur empêche un second de la consulter dans le même temps.\\

Dès les années 1970, les bibliothèques se sont engagées dans une vaste opération de rétroconversion de leurs notices documentaires. Les fichiers physiques et les notices cartonnées sont alors informatisés et \og reproduits presque à l’identique [\dots] sous forme de bases de données\fg{}\footcite{bermes_du_2013}. L'informatisation des notices et des fichiers permet, par conséquent, d'améliorer l'indexation des documents. L'utilisateur va donc pouvoir trouver plus rapidement plus de documents correspondant à sa recherche. Ainsi, les autorités \index[ref]{lod@Linked Open Data (LOD)!lcsh@LCSH}\index[ref]{autorites@Autorités!lcsh@LCSH}\ac{lcsh}, créées en 1914 sous format papier, ont été informatisées; les autorités \index[ref]{lod@Linked Open Data (LOD)!rameau@RAMEAU}\index[ref]{autorites@Autorités!rameau@RAMEAU}\ac{rameau} créées dans les années 1980 reprennent celles de \ac{lcsh} en les complétant.\\

Cependant, ces \index[ref]{autorites@Autorités!fichiers autorite@Fichiers d'autorité}fichiers d'autorité comportent, comme nous l'avons évoqué plus haut (\reference{I-A-2-c}), des formes retenues et des formes rejetées des termes, ce qui crée de multiples renvois à l'intérieur des fichiers physique ou informatique. L'arrivée des moteurs de recherche, dans les années 2000, permet de supprimer ces différences de termes, en indexant à la fois les formes retenues et les formes rejetées et en permettant de trouver directement la vedette recherchée.

\subsection{\label{I-A-3-b}Partager des vocabulaires: à la recherche de la meilleure interopérabilité}
\titreEntete{Partager des vocabulaires}

La problématique du partage des référentiels entre institutions se pose avant l'informatisation des catalogues et des fichiers des bibliothèques. En effet, le format \index[ref]{echanges@Échanges!formats@Formats!marc@MARC}\ac{marc}, né en 1968 à la Bibliothèque du Congrès, permet l'échange de données entre les institutions et la \og duplication les notices d’un catalogue à un autre\fg{}\footcite{bermes_convergence_2013}. Malgré de multiples variantes nationales, l'\ac{unimarc} reste aujourd'hui le format d'échange privilégié entre les bibliothèques.\\

Pour partager les \index[ref]{autorites@Autorités!fichiers autorite@Fichiers d'autorité}fichiers d'autorité et aboutir à une interopérabilité totale des données entre deux institutions par le biais des machines, différents protocoles d'échange ont été utilisés --- ou délaissés en fonction des difficultés imposées par chacun. Dès les années 1980 est développé le \index[ref]{echanges@Échanges!protocoles@Protocoles!z3950@Z39-50}protocole Z39-50. Ce protocole permet d'interroger une base de données de manière synchrone, selon la requête du client, et de récupérer des données en format \ac{marc}\footcite{bibliotheque_nationale_de_france_protocole_nodate}.\\

Ce protocole Z39-50 est destiné aux catalogueurs qui peuvent ainsi \og repérer puis télécharger une notice dans un catalogue distant plutôt que d’avoir à la saisir \textit{ex nihilo}\fg{}\footcite{bermes_convergence_2013}. Le partage, \og par conversion et copie\fg{}\footnote{\cite{bermes_convergence_2013}. Voir \reference{annexe_types_interop} (\reference{conversion}).}, n'est alors qu'une simple copie de données, dont la mise à jour est difficile. L'existence de ce protocole, bien que destiné aux professionnels de la documentation, a suscité la création de \index[ref]{relier@Relier!portails@Portails}portails de consultation de notices documentaires ou de \index[ref]{autorites@Autorités!fichiers autorite@Fichiers d'autorité}fichiers d'autorité, interrogeant de manière synchrone les bases de données: cette utilisation orientée utilisateur du \index[ref]{echanges@Échanges!protocoles@Protocoles!z3950@Z39-50}protocole Z39-50 permet à la Bibliothèque nationale de France d'offrir différents services(intégration des notices dans \index[ref]{lod@Linked Open Data (LOD)!oclc@OCLC}\ac{oclc}, recherche dans le Catalogue Collectif de France (CCFR), \dots\footcite{bibliotheque_nationale_de_france_protocole_nodate}). Cependant, face aux temps de réponses importants et aux résultats appauvris retournés par la requête, les portails se sont révélés décevants et peu efficaces. De plus, l'utilisation d'un portail nécessite, de la part de l'utilisateur, qu'il connaisse précisément ce qu'il cherche, de manière à se connecter au \index[ref]{relier@Relier!portails@Portails}portail correspondant (qui lui-même doit être connu de cet utilisateur)\footcite{dalbin_approches_2011}.\\

La multiplication des formats d'échanges --- \index[ref]{echanges@Échanges!formats@Formats!marc@MARC}\ac{marc} et \ac{unimarc} pour les bibliothèques, \index[ref]{echanges@Échanges!formats@Formats!ead@EAD}\ac{ead} pour les archives ---, ainsi que la volonté d'offrir au public lien entre les différentes bases de données patrimoniales, ont conduit à la création d'un nouveau protocole, \index[ref]{echanges@Échanges!protocoles@Protocoles!oaipmh@OAI-PMH}\ac{oaipmh}. Ce protocole asynchrone repose sur deux acteurs: le fournisseur qui met à disposition ses données dans un \og entrepôt\fg{}, et le moissonneur qui collecte ces données pour les intégrer à son système\footcite{bibliotheque_nationale_de_france_protocole_nodate-1}.\\

Cependant, si les performances du protocole sont améliorées avec \ac{oaipmh}, les documents et les \index[ref]{autorites@Autorités!fichiers autorite@Fichiers d'autorité}fichiers d'autorité ne peuvent pas être sélectionnés et filtrés: un format d'échange simple, minimal, est nécessaire. Ce format est le \index[ref]{echanges@Échanges!formats@Formats!dublin core@Dublin Core}Dublin Core\footcite{noauthor_dublin_nodate} comprenant quinze champs d'informations.
Ce partage de données et de métadonnées entre les institutions permet une \og interopérabilité par le plus petit dénominateur commun\fg{}\footnote{\cite{bermes_convergence_2013}. Voir \reference{annexe_types_interop} (\reference{denom})}, qui est le Dublin Core. Celui-ci peut cependant présenter un appauvrissement des données puisque les champs sont très réduits, ou au contraire permettre de grandes différences au sein d'un même champ.

\bigskip
\bigskip
\bigskip
Auteurs, catalogueurs et bibliothécaires ont très vite ressenti le besoin de se dégager du langage naturel de manière à renvoyer rapidement vers des passages de leur texte ou à décrire le plus précisément possible les documents, pour faciliter la lecture ou la recherche de l'utilisateur final. D'abord effectuées sur des supports papier, ces opérations de descriptions ont été informatisées et ont permis le partage de données et de métadonnées entre les institutions: les notices et les fichiers d'autorité disponibles sont la source constante d'interrogations quant au meilleur moyen de les mettre à disposition, tant pour le professionnel que pour le public. L'ouverture de ces vocabulaires a permis une amélioration des descriptions et une uniformisation des pratiques d'indexation.\\

Cependant, ces vocabulaires contrôlés restent peu précis et sont limités à leur terminologie pour en tirer le sens: il ne comprennent pas de terminologie sémantique, qui permettrait d'améliorer plus encore la description effectuée, en pouvant se référer aux termes parents, frères ou fils. L'anneau de synonymie évoqué (\reference{I-A-2-b}) permet la prise en compte des synonymes, mais ne donne pas de sens supplémentaire à la vedette.
	\chapter{\label{I-C}L'arbre, un vocabulaire contrôlé hiérarchique}
\titreEntete{L'arbre, un vocabulaire contrôlé hiérarchique}

Nous l'avons évoqué (\reference{I-C-2}), le contexte d'un terme de vocabulaire peut lui donner un sens complémentaire ou différent. La hiérarchisation des vocabulaires permet un ajout de contexte à chaque terme, mais également un accroissement de la précision de la définition donnée à ce terme. Le vocabulaire hiérarchique contrôlé le plus fréquent est le thésaurus: la diversité de ses relations et de ses caractéristiques lui permet une adaptation à chaque vocabulaire. Cependant, la hiérarchie n'offre plus assez d'autorités pour décrire précisément les données de l'\ac{ina}.

\section{\label{I-C-1}L'arbre de Porphyre: origines et influences}
\titreEntete{L'arbre de Porphyre: origines et influences}

%intro
La définition d'un terme est une réflexion millénaire, et la recherche d'un référentiel, d'un dictionnaire pur, n'est toujours pas aboutie. En effet, l'intelligence artificielle nécessitant des référentiels solides, la réflexion sur la pureté du dictionnaire utilisé est constante. \nP{Umberto}{Eco} considère que le dictionnaire \og ne devrait comporter, pour la définition d'un  terme, que les propriétés nécessaires et suffisantes pour distinguer ce concept d'un autre\fg\footcite[chap.1]{eco_arbre_2010}. Ces propriétés nécessaires à la définition du terme ne doivent pas être une connaissance du monde, mais bien des propriétés analytiques: \og Animal\fg{} est une propriété analytique de \og Chien\fg{} alors que l'aboiement est une connaissance.\\

La théorisation du dictionnaire remonte à l'Antiquité et a eu de nombreuses influences dans les systèmes classificatoires jusqu'à nos jours: les vocabulaires utilisés en institutions patrimoniales sont pour la plupart des hiérarchies de termes.

\subsection{\label{I-C-1-a}L'arbre de Porphyre}
\titreEntete{L'arbre de Porphyre}

La pensée aristotélicienne considère la définition d'un terme comme la forme substantielle, c'est à dire les attributs essentiels: l'\og homme\fg{} est un \og Animal rationnel mortel\fg{}\footcite[chap.1]{eco_arbre_2010}. L'assemblage de ces propriétés essentielles crée une définition, mais chacune de ces propriétés peut s'appliquer à d'autres entités. \\

Le commentateur des \textit{Catégories} d'Aristote au \textsc{III}\textsuperscript{ème}siècle, Porphyre, établit des arbres pour décrire le monde: celui des \og Substances\fg{} a le plus de postérité en étant \og un ensemble hiérarchisé et fini de genres et de substances\fg{}\footcite[chap.1]{eco_arbre_2010}. Il part du \textit{Summus genus}, la Substance, pour atteindre une espèce indivisible, définie uniquement par ses attributs analytiques appelés genres\footnote{Voir \reference{arbre_porphyre_analytique}}. Un arbre de Porphyre est, par conséquent, une succession de genres divisés en espèces qui deviennent elles-mêmes des genres.\\

\begin{figure}[!h]
	\centering
	\Tree[.\textsc{Substance} 
		Incorporelle
		[.Corporelle 
			{Non vivante}
			[.Vivante 
				{Non animale}
				[.Animale 
					[.\textbf{Homme/Cheval} ]]]]]
	
	\caption[Arbre porphyrien de l'homme avec les seuls attributs analytiques]{Arbre porphyrien de l'homme avec les seuls attributs analytiques [d'après \cite{eco_arbre_2010}]}
	\label{arbre_porphyre_analytique}
\end{figure}

\begin{figure}[!h]
	\centering
	\includegraphics[width=12cm]{images/arbre_porprhyre_differences.png}
	\caption[Arbre porphyrien prenant en compte les différences]{Arbre porphyrien prenant en compte les différences [Source: \cite[chap.1]{eco_arbre_2010}]}
	\label{arbre_porphyre_differences}
\end{figure}

L'impossibilité de la distinction entre l'homme et le cheval impose de tenir compte des différences qui ne sont pas des attributs analytiques: \og La rationalité est la différence de l'homme\fg{}\footcite[chap.1]{eco_arbre_2010}. Ainsi, ces différences vont s'ajouter aux genres des espèces. Ces différences deviennent elles-mêmes divisibles et constitutives: elles deviennent genres. Ces différences sont essentielles pour distinguer une espèce d'une autre (voir \reference{arbre_porphyre_differences}).\\

Cependant, si la prise en compte des différences permet de différencier l'homme du cheval, elles ne permettent pas de distinguer le cheval de l'âne par exemple. Un même genre doit donc être utilisé plusieurs fois dans l'arbre, ce qui le rend infini, et l'établissement d'un dictionnaire impossible à réaliser (voir \reference{arbre_porphyre_boucle}).\\

\begin{figure}[!h]
	\centering
	\includegraphics[width=12cm]{images/arbre_porphyre_boucle.png}
	\caption[Infinitude de l'arbre de Porphyre]{Infinitude de l'arbre de Porphyre [Source: \cite[chap.1]{eco_arbre_2010}]}
	\label{arbre_porphyre_boucle}
\end{figure}

Face à cette impossibilité de décrire le monde avec des divisions uniques dans un seul arbre, c'est à dire d'établir un dictionnaire universel, absolu et global, la seule solution paraît être la création d'un nombre infini d'arbres, composés de propriétés s'articulant selon le contexte et le domaine d'utilisation de l'arbre: d'un seul arbre insaisissable, une forêt réorganisable à l'envi et à l'infini est apparue, laissant le choix à l'utilisateur de l'arbre utilisé selon le sujet.

\subsection{\label{I-C-1-b}L'encyclopédisme (Antiquité - Moyen-Âge): la recherche d'un arbre global mimant le monde réel}
\titreEntete{L'encyclopédisme}

L'utopie de saisie totale du monde se retrouve dans l'encyclopédisme, dès l'\textit{Historia naturalis} de \nP{Pline}{l'Ancien}. Sur le même principe que l'arbre porphyrien, la hiérarchie de \index[ref]{typologie@Typologie!index@Index}l'index de cette encyclopédie de 37 volumes part de l'original vers le dérivé, du naturel à l'artifice: \og Une encyclopédie, pour s’organiser, tente de suivre le modèle de l’arbre --- qui est toujours plus ou moins consciemment celui de la subdivision binaire d’un arbre porphyrien\fg{}\footnote{\cite[chap.1]{eco_arbre_2010}. Voir \reference{index_pline}}. Cependant, l'index d'une encyclopédie se distingue des termes d'un arbre porphyrien en ce qu'il est défini dans un autre développement --- un article d'encyclopédie ---, alors que les termes de l'arbre de Porphyre ne peuvent pas être définis par la suite.

\begin{figure}[!h]
	\centering
	\includegraphics[width= 13cm]{images/index_pline.png}
	\caption{Extrait de l'arborescence de l'index de \nP{Pline}{l'Ancien}}
	\label{index_pline}
\end{figure}

Avec le passage au christianisme, l'encyclopédisme doit décrire les textes sacrés et non plus le monde. Ainsi, des éléments moralisateurs et allégoriques se retrouvent dans les \index[ref]{typologie@Typologie!index@Index}index, devant les éléments matériels du monde\footnote{La tradition moralisatrice encyclopédique naît avec le \textit{Physiologos} d'un auteur grec et s'inspire de l'œuvre de \nP{Pline}{l'Ancien}, puis se poursuit tout au long du Moyen-Âge avec les \textit{Étymologies} d'\nP{Isidore}{de Séville} notamment.}. À partir du \textsc{XIII}\textsuperscript{ème}siècle, les encyclopédies montrent l'ordre qui les dirige: cela conduit à \textit{L'arbre de science} de \nP{Raymond}{Lulle} qui crée seize arbres représentant l'Être, chacun représentant un savoir différent en se divisant en sept parties (racines, tronc, branches, rameaux, feuilles, fleurs, fruits)\footcite[chap.10]{eco_arbre_2010}. Contrairement à l'arbre de Porphyre qui est un arbre vide que l'on peut remplir selon le contexte, les arbres que propose \nP{Raymond}{Lulle} sont pleins et ont pour vocation de décrire et de classer le monde, la Grande Chaîne de l'Être.

\subsection{\label{I-C-1-c}Influences: une diversité de référentiels hiérarchiques}
\titreEntete{Influences: une diversité de référentiels hiérarchiques}

La pensée aristotélicienne puis le commentaire porphyrien ont produit une tradition de hiérarchisation du monde qui s'est poursuivie pendant plus d'un millénaire, sans cesse confrontée à l'impossibilité d'une description totale de ce monde. La multiplicité des arbres est, chez \nP{Umberto}{Eco} puis dans celle de \nP{Raymond}{Lulle}, la conclusion de leur réflexion. L'influence de cette tradition de description est sensible jusqu'à aujourd'hui, notamment dans le domaine de l'indexation et de la bibliothéconomie.\\

En effet, une diversité de référentiels est apparue, chacun étant dérivé d'un arbre. Des schémas de classification sont définissables à l'infini, emboîtant les genres, les espèces et les différences\footnote{\og Un simple artifice classificatoire consiste à emboîter des genres, des espèces et des différences sans en expliquer le \textit{definiendum}\fg{} in \cite[chap.1]{eco_arbre_2010}}. La \index[ref]{typologie@Typologie!taxonomie@Taxonomie}taxonomie naît de ce modèle d'arbre. Elle n'a pas pour but de dire comment repérer le concept décrit, elle permet seulement de classer en renvoyant, pour chaque nœud, vers un autre chapitre où l'on décrit ces propriétés. La taxonomie, bien qu'historiquement appliquée aux sciences de la terre, a été reprise par \nP{Melvil}{Dewey} dans sa classification décimale Dewey\index[ref]{autorites@Autorités!dewey@Dewey} en 1876.\\

Définie comme un \og classement hiérarchique de termes préférentiels\fg{} par \nP{Louis}{Rosenfeld} et \nP{Peter}{Morville}\footcite{rosenfeld_information_2015}, la taxonomie ne veut pas définir, mais simplement permettre l'utilisation correcte et logique du terme, par l'attribution de catégories et l'utilisation exclusive de relations hiérarchiques.\\

Les \index[ref]{typologie@Typologie!thesaurus@Thésaurus}\textit{thesauri}\footnote{Ils sont décrits comme une \og liste organisée de termes contrôlées et normalisés (descripteurs et non-descripteurs) servant à l’indexation des documents et des questions dans un système documentaire\fg{} dans \cite{degez_thesauroglossaire_2001}. Peu formels, ils sont néanmoins le vocabulaire le plus utilisé pour l'indexation. L'un des \textit{thesauri} les plus utilisés est le \ac{gemet}(\cite{noauthor_general_nodate}). Le \ac{gemet}\index[ref]{autorites@Autorités!gemet@GEMET} est disponible en plus de trente langues et diffusé par l'Agence européenne de l'Énergie. Voir \reference{I-C-2}.} utilisent plus de relations et de types de termes, de manière à indexer des contenus avec des mots-clés et faciliter ainsi la recherche. Ce vocabulaire contrôlé et hiérarchique reste proche du langage naturel en y intégrant les variantes, les synonymes, les descriptions, les traductions et les équivalences.\\

Pour avoir une plus grande formalisation du thésaurus, il faut utiliser une \index[ref]{typologie@Typologie!ontologie@Ontologie}ontologie. Cette ontologie est la spécification formelle d'un espace de noms, d'un domaine particulier de la connaissance\footnote{L'une des ontologies les plus utilisées, notamment dans le web sémantique, est \ac{foaf}. \ac{foaf}\index[ref]{relier@Relier!foaf@FOAF} permet la description précise des personnes. Voir \cite{noauthor_foaf_nodate}.}. Elle identifie alors les objets à décrire, leurs relations au sein de ce domaine ainsi que leurs propriétés. L'ontologie n'est pas utilisée directement dans l'indexation ou la recherche, elle est d'abord utilisée pour instancier et raisonner, en s'éloignant du langage naturel avec l'utilisation d'identifiants techniques.\\

Les \index[ref]{typologie@Typologie!taxonomie@Taxonomie}taxonomies, les \textit{thesauri} ainsi que les \index[ref]{typologie@Typologie!ontologie@Ontologie}ontologies héritent tous du modèle de l'arbre, la description ou la classification par la hiérarchie étant la plus efficace. Ces \index[ref]{typologie@Typologie!vocabulaires controles@Vocabulaires contrôlés}vocabulaires sont les plus complexes par les relations qui les composent. \nP{Louis}{Rosenfeld} et \nP{Peter}{Morville}\footnote{\cite{rosenfeld_information_2015}. Voir \reference{frise_voca}} considèrent l'anneau de synonymie comme le plus simple des vocabulaires, avec des relations d'équivalence, alors que les \index[ref]{autorites@Autorités!fichiers autorite@Fichiers d'autorité}fichiers d'autorité et les taxonomies, fonctionnant sur la hiérarchie, sont plus complexes. Les \index[ref]{typologie@Typologie!thesaurus@Thésaurus}\textit{thesauri} et les ontologies sont encore plus complexes puisqu'ils sont constitués de relations hiérarchiques et associatives.

\begin{figure}[!h]
	\centering
	
	\begin{pspicture}(0,2)(10,8)
		\psline[linewidth=1.5pt]{->}(0,5)(10,5)
		\uput[0](-2.5,5){\textsc{Simple}}
		\uput[0](10.5,5){\textsc{Complexe}}
		\uput[0](3,3){\textbf{\textsc{Type de relations}}}
		\uput[0](2.5,7){\textbf{\textsc{Type de vocabulaire}}}
		\uput[0](0,4){Équivalence}
		\uput[0](3.8,4){Hiérarchique}
		\uput[0](7.6,4){Associative}
		\uput[0](-1,6.3){Anneau de synonymie}
		\uput[0](2.5,5.5){Fichiers d'autorité}
		\uput[0](5,6.1){Taxonomies}
		\uput[0](7,5.5){Thésaurus}
		\uput[0](9,6.3){Ontologies}
	\end{pspicture}
	
	\caption[Classification des vocabulaires selon leur complexité]{Classification des vocabulaires selon leur complexité [d'après \cite{rosenfeld_information_2015}]}
	\label{frise_voca}
\end{figure}

\section{\label{I-C-2}Les \textit{thesauri}, vocabulaire contrôlé hiérarchique le plus fréquent}
\titreEntete{Les thesauri}

\section{\label{I-C-3}Passer du texte libre à un vocabulaire contrôlé: aligner des notes qualité et un thésaurus de noms communs}
\titreEntete{Passer du texte libre à un vocabulaire contrôlé}



%conclu
\bigskip
\bigskip
\bigskip
L'utilisation d'un thésaurus permet d'aligner des termes ensemble et de relier du texte libre avec un vocabulaire contrôlé de manière à disposer d'un vocabulaire commun de description. Plus encore, la hiérarchie d'un thésaurus permet la classification d'un ensemble de concepts --- ici les fonctions --- selon quelques catégories globales. Le thésaurus a par conséquent la double fonction d'offrir un enrichissement du terme préférentiel par ses relations d'association, et de proposer une classification par ses relations hiérarchiques.\\

Aligner du texte libre avec un thésaurus nécessite plusieurs étapes et la prise en compte des différences de langage --- l'un étant un contrôle minimal du langage humain naturel, l'autre un vocabulaire contrôlé natif --- :
\begin{itemize}
	\item une normalisation est d'abord nécessaire de chaque côté de l'alignement selon les mêmes règles;
	\item puis un alignement selon l'exactitude avec le terme préférentiel peut être réalisé
	\item suivi par un autre alignement selon l'exactitude avec une variante du terme préférentiel;
	\item ensuite, aligner selon l'exactitude du commencement de la fonction avec le terme préférentiel est possible, 
	\item ainsi qu'aligner selon l'exactitude du commencement de la fonction avec une variante du terme préférentiel
	\item pour enfin aligner selon l'exactitude du deuxième terme non polysémique de la fonction avec le terme préférentiel
\end{itemize}


	\chapter{\label{I-B}Les référentiels à l’INA}
\titreEntete{Les référentiels à l’INA}

%intro: historique INA

\section{\label{I-B-1}De multiples fonds à décrire}
\titreEntete{De multiples fonds à décrire}


\subsection{\label{I-B-1-a}Les archives professionnelles}
\titreEntete{Les archives professionnelles}

Après l'éclatement de l'\ac{ortf}, des fonds divers deviennent la propriété de l'\ac{ina} et participent à la diversité des fonds audiovisuels conservés à l'\ac{ina}. Ainsi, les \og Actualités françaises\fg{} --- la presse filmée diffusée dans les salles de cinéma --- ont été transférées à l'\ac{ina} en 1974. Les grands moments des débuts de la télévision --- le premier journal télévisé, les premières grandes émissions ou les magazines de reportage --- sont conservés et participent, avant l'instauration du dépôt légal, à la mémoire de l'audiovisuel français; de même pour les fonds radio qui retracent les grands moments historiques depuis les années 1940, et qui deviennent de plus en plus complets avec la généralisation de la bande magnétique à partir du milieu des années 1950.\\

Les archives professionnelles de l'\ac{ina} ne sont que des archives: elles ne couvrent pas l'ensemble de la production audiovisuelle depuis les années 1940 ou la création de l'Institut. Des tranches horaires de diffusion télé- ou radiodiffusées ne sont ainsi pas conservées; peu de traces demeurent alors de la production audiovisuelle, notamment avant l'arrivée du kinéscope. Pour combler ces manques, l'\ac{ina} dispose d'un fonds photographique créé à partir des services de l'\ac{ortf} ou de l'\ac{ina} et portant sur la réalisation des émissions et des tournages.\\

Enfin, des délégations régionales se chargent de la conservation et de la communication des archives télévisées et radiodiffusées des stations régionales --- apparues dans les années 1950: la vie et l'histoire des régions sont ainsi couvertes par l'\ac{ina}.\\

Les fonds d'archives professionnelles de l'\ac{ina} sont conséquents et divers, témoins de la vie et de la société française depuis l'Après-Guerre. Irremplaçables, leur description n'en est pas moins difficile par la diversité des sujets évoqués ou présents dans les documents.

\subsection{\label{I-B-1-b}Les fonds issus du dépôt légal}
\titreEntete{Les fonds issus du dépôt légal}

Depuis la loi sur le dépôt légal de l'audiovisuel de 1992, l'\ac{ina} en est le dépositaire. À partir de 1995, l'\ac{ina} enregistre la globalité de la programmation des stations de Radio-France --- France-Inter, France-Musique, France-Culture, France-Info et France-Bleue ---, enregistrement étendu en 2001 aux stations privées généralistes comme RTL ou NRJ\footnote{En 2020, l'ensemble des stations captées au titre du dépôt légal par l'INA est décrit dans l'\reference{annexe_dl_captation} (\reference{dl_radio})}.\\

Pour les programmes télévisés, le dépôt légal ne concerne d'abord --- entre 1995 et 2001 --- que les sept chaînes principales --- TF1, France 2, France 3, Canal +, M6, Arte, France 5 --- et leurs programmes en première diffusion. La captation directe et intégrale des chaînes n'apparaît qu'en 2002 et est élargie à douze autres chaînes. Enfin, depuis 2005, les chaînes de la Télévision numérique terrestre (TNT) sont toutes captées\footnote{Les chaînes de télévision captées en 2002 pour le dépôt légal sont décrites dans l'\reference{annexe_dl_captation} (\reference{dl_tv})}.

\bigskip
\bigskip

La diversité des fonds d'archives, la captation directe en intégralité des chaînes de télévision et de radio, ainsi que la captation de sites web, plateformes ou comptes de réseaux sociaux au titre du dépôt légal audiovisuel, représentent une masse très importante de données et de documents à conserver. En 2019\footcite[p.5]{institut_national_de_laudiovisuel_rapport_2019}, l'\ac{ina} conserve 20 873 143 heures de programmes de télévision et de radio, dont plus de 18 millions captés par le dépôt légal. 1,2 million de photos s'ajoutent à ces documents. La majorité de ces documents, issus du dépôt légal, sont destinés à une gestion patrimoniale et à une valorisation dans l'INAthèque, alors que les documents des archives professionnelles sont destinées à la valorisation commerciale au travers notamment du site \href{https://www.inamediapro.com}{INAMediaPro} destiné aux professionnels.
\section{\label{I-B-2}Un système documentaire pluriel répondant aux besoins}
\titreEntete{Un système documentaire pluriel}

La masse des documents, l'évolution de leur récupération auprès des producteurs et leurs usages divers conduisent à la création d'un système documentaire pluriel, créé à partir des besoins et non des données. Les deux usages commerciaux et patrimoniaux des documents\footnote{É. Alquier décrit ces deux usages dans un article de 2017 évoque ces usages et la plateforme qui les met en œuvre. En 2020, le service de vidéo à la demande Madelen vient s'ajouter à l'offre \og grand public\fg{} de l'\ac{ina}. Voir \cite{alquier_production_2017}.}, évoqués précédemment, dirigent le nombre des bases de données, leur structure et le partage de référentiels. Avant le projet du \ldd lancé en 2015, le système documentaire de l'\ac{ina} est pluriel, constitué de plusieurs bases de données distinctes ainsi que de plusieurs référentiels non communs.\\

Deux types de données sont présents dans les bases de l'\ac{ina}. D'abord, il y a du texte libre, décrivant les titres propres des documents, les titres de collections ou indiquant un identifiant, ou bien des notes diverses ou des chiffres. Ensuite, il y a les données contrôlées, issues de référentiels et de lexiques, permettant de décrire les contenus, les particularités de ces contenus et les événements associés à ces contenus (diffusion, archivage, exploitation)\footnote{Voir \reference{annexe_type_donnees_axel} (\reference{type_donnees_axel}).}.

\subsection{\label{I-B-2-a}Les bases de données du dépôt légal (DL)}
\titreEntete{Les bases de données du dépôt légal}

L'\ac{ina} capte en permanence et en direct plus de 170 chaînes de télévision et stations de radio\footcite[p.5]{institut_national_de_laudiovisuel_rapport_2019}. Ce flux ininterrompu est décrit lors du catalogage par des techniciens spécialisés dans la gestion de collections multimédia: pour chaque document y est notamment indiqué le titre, le générique, les dates et heures de diffusion, ainsi que des descripteurs pour indexer la chaîne de diffusion, les thématiques présentes, \dots\\

Quand le document est décrit, les données complétées par le technicien de gestion des collections multimédia dans son interface graphique sont dirigées vers les bases de données du dépôt légal, scindées en quatre pour correspondre à la provenance du document. Ainsi, bien que les documents proviennent de la même source --- la captation pour le dépôt légal ---, ils sont éclatés dans quatre bases de données différentes pour correspondre à leur provenance:
\begin{itemize}
	\item la base DLRADIO (Dépôt Légal de la Radio) comprend les documents diffusés en radio, sans autre distinction de provenance
	\item la base DLTV (Dépôt Légal de la Télévision (Nationale)) ne comprend pas l'ensemble des documents diffusés à la télévision, mais seulement les chaînes nationales
	\item la base DLREG (Dépôt Légal de la Télévision Régionale) comprend les documents télévisuels diffusés sur une chaîne de télévision régionale comme France 3
	\item enfin, la base DLSAT (Dépôt Légal de la Télévision Satellite) comprend les documents diffusés sur les chaînes de télévision satellite
\end{itemize}
\bigskip

Cependant, malgré cette scission des données dans plusieurs bases de données, ces quatre bases partagent un même schéma pour les \index[ref]{typologie@Typologie!vocabulaires controles@Vocabulaires contrôlés}référentiels. Ce schéma permet de trouver des tables comprenant la signification d'identifiants de provenance de chaînes (le lien entre le code \og FR5\fg{} présent dans les données peut ainsi être établi avec son terme développé), de provenance de données, \dots Ce schéma est un fournisseur de mots-clés destinés à permettre la description, l'indexation et la recherche des documents.

\subsection{\label{I-B-2-b}Les bases de données des archives professionnelles (DA)}
\titreEntete{Les bases de données des archives professionnelles}

Le dépôt légal se concentre sur la diffusion des documents et conserve alors à chaque instant l'ensemble de ce qui est diffusé à la télévision ou à la radio --- les émissions, les films, les publicités, les journaux télévisés, \dots~ Cette conservation des premières diffusions et des rediffusions permet, ainsi que l'exige le dépôt légal, d'avoir un panorama complet du paysage audiovisuel français, comme c'est le cas à la Bibliothèque nationale de France pour les imprimés ou les périodiques.\\

Les archives professionnelles ne sont pas soumises à cette exhaustivité: lorsqu'un producteur de contenu audiovisuel mandate l'\ac{ina} pour la conservation et/ou la commercialisation de ses contenus, les données de ces contenus sont récupérées dans les bases du dépôt légal puis copiées dans celles des archives professionnelles. Ainsi, la même donnée est dupliquée au \ac{dl} et au \ac{da}. Cependant, le \ac{da} s'intéressant non pas à la diffusion elle-même du document mais au document lui-même, ces données vont être transformées et complétées de manière à être plus précises et à avoir une meilleure description. Cette description plus fine permet la vente des extraits.\\

Comme pour le \ac{dl}, le \ac{da} possède plusieurs bases de données partageant les mêmes référentiels:
\begin{itemize}
	\item la base DAV (Archives Professionnelles de la Télévision Nationale)
	\item la base DAVREG (Archives Professionnelles de la Télévision Régionale)
	\item la base DAVRAD (Archives Professionnelles de la Radio)
\end{itemize}
\bigskip

Ces trois bases de données sont appuyées par plusieurs lexiques et \index[ref]{typologie@Typologie!thesaurus@Thésaurus}\textit{thesauri}, notamment celui des noms communs\footnote{L'importance --- et la complexité --- de ce thésaurus au \ac{da} nécessite une interface graphique, \og Totem\fg{}, pour le visualiser et cataloguer les documents. Un exemple de visualisation de ce thésaurus est possible en \reference{annexe_thesaurus} (\reference{thesaurus_cadreur}).} et des personnes physiques et morales.

\subsection{\label{I-B-2-c}La base de donnés juridique (DJ)}
\titreEntete{La base de donnés juridique}

La base de données \og Adaje\fg{} de la \ac{dj} comprend l'ensemble des données permettant d'identifier et de rémunérer les ouvrants-droit\footnote{Personnes auxquelles les droits ont été ouverts, le producteur lui-même ou ses ayants-droit.} et les ayants-droit\footnote{\og Un ayant droit est une personne ayant acquis un droit d'une autre personne\fg{} in \url{https://droit-finances.commentcamarche.com/faq/4010-ayant-droit-definition}.} des documents et extraits vendus. Cette base juridique contient par conséquent des tables de Personnes, de Contributions, d'Informations personnelles, \dots\\

Les bases \ac{dl} et \ac{da}, et celle de la \ac{dj} n'ont aucun lien entre elles, mais leurs données semblent redondantes notamment pour les personnes physiques et morales. Le projet du \index[ref]{led@Linked Enterprise Data (LED)!ldd@Lac de données (INA)}\index[ref]{modelisation@Modélisation!ldd@Lac de données (INA)}\ldd\footnote{Ce projet est évoqué au \reference{III-B}} devrait permettre l'alignement de ces bases entre elles en évitant les doublons: la base de la \ac{dj} enrichira notamment les concepts de personnes physiques et morales déjà créés à partir des données de la \ac{ddcol}.

\bigskip
\bigskip

Plusieurs référentiels, parfois similaires, sont présents dans les bases de la \ac{ddcol}\footnote{Voir \reference{annexe_bdd_ina} (\reference{bdd_ddcol_ina}).} et la \ac{dj} présentées ici. Leur structure\footnote{Ces structures sont détaillées dans les chapitres consacrés aux alignements des données de l'\ac{ina}.} est différente selon les usages qui ont conduit à leur création, et aux besoins qui en résultent: des notes qualités décrivant la fonction précise des personnes sont présentes dans le lexique des personnes de la \ac{ddcol} alors que seul un domaine d'activité général est conservé à la \ac{dj}. Les systèmes documentaire et juridique de l'\ac{ina} ne sont pas interopérables et n'ont pas été conçus pour l'être: d'un côté, soit l'événement de diffusion est prioritaire, soit l'extrait documentaire l'est; de l'autre, l'information juridique joue ce rôle. Les usages sont tous différents et dirigent le stockage des données dans l'Institut.
\section{\label{I-B-3}Multiplication des sources de données et des référentiels}
\titreEntete{Multiplication des sources de données et des référentiels}

De manière à améliorer et enrichir ses données, à faciliter le travail de catalogage, de description et d'indexation, l'\ac{ina} récupère des métadonnées et des données à l'extérieur, auprès de plusieurs fournisseurs. Certains fournisseurs deviennent alors eux-mêmes des référentiels, dont l'identifiant qu'ils fournissent est présent dans les bases de données de l'\ac{ina} aux côtés des données fournies.\\

Ainsi, l'\ac{ina} reçoit des informations concernant les chaînes de provenance, les noms du générique avec les titres, les audiences et le public cible du document\footnote{Voir \reference{annexe_fournisseurs_exterieurs} (\reference{sheldon_mediametrie}).}, ou encore les grilles de diffusion prévisionnelles et réelles. L'ensemble de ces informations permet d'accompagner la tâche de catalogage en fournissant des champs préremplis. Les fournisseurs de ces données sont multiples\footnote{Voir \reference{annexe_fournisseurs_exterieurs} (\reference{enrichissement_dl}).} et fournissent des données tant sur les programmes que sur les producteurs eux-mêmes: 
\begin{itemize}
	\item Les données prévisionnelles de diffusion de la télévision sont achetées auprès de la société Plurimédia\footnote{Voir \url{http://www.plurimedia.fr/}.}. Les fictions, les documentaires, les dessins animés, les émissions de toutes natures, les magazines, \dots sont ainsi décrits au préalable par cette société.
	\item Les données réelles de la diffusion télévisuelle et radio --- date, horaires, parts d'audience, public --- sont fournies par Médiamétrie\footnote{Voir \url{https://www.mediametrie.fr/}.}, en complément des données ---programmation, diffusion, description des contenus --- reçues de la part des diffuseurs eux-mêmes.
	\item Des informations complémentaires sur les programmes sont acquises auprès d'agences de presse comme Kantarmédia\footnote{Voir \url{https://www.kantarmedia.com/fr}.}; pour les producteurs, les informations sont obtenues depuis la société Karl More Productions France.
\end{itemize}

%concli
	
	Le référentiel, tel que présenté dans les chapitres précédents, n'a pour destination que l'institution qui l'a créé, dans un unique but qui est de répondre à ses propres besoins selon ses activités. Cependant, nous l'avons évoqué, une même institution peut disposer de plusieurs référentiels, parfois similaires mais structurés différemment. Ces référentiels prennent la forme de listes de termes, ou de \textit{thesauri}, qui offrent des clés et des termes normalisés aux documents décrits. L'interopérabilité entre les bases de données n'est pas recherchée, comme celle avec des référentiels externes: le lien n'apparaît pas encore comme essentiel.\\
	
	Ainsi, le référentiel comme fournisseur de clés et de termes contrôlés n'a qu'un usage interne et spécifique; il n'est pas utilisable autre part comme le montre la \reference{schema_general_controler}.
	\begin{figure}[!h]
	\centering
	
	\begin{pspicture}(0,0)(14,7)		
		\uput[0](0.8,6.4){Entrepôt de documents A}
		\uput[0](7.8,6.4){Entrepôt de documents B}
		%cercles globaux
		\pscircle(3,3){3}
		\pscircle(10,3){3}
		%cercles du A
		\pscircle(4.8,3.4){0.6}
		\pscircle(3,4.2){0.6}
		\pscircle(1.5,3.2){0.6}
		\pscircle(2.2,1.6){0.6}
		\pscircle(3.8,1.4){0.6}
		%cercles du B
		\pscircle(9.2,3.9){0.6}
		\pscircle(11.4,2.8){0.6}
		\pscircle(9.2,1.6){0.6}
		%labels du A
		\uput[0](2.6,4.2){D1}
		\uput[0](3.4,1.4){D2}
		\uput[0](1,3.2){D3}
		\uput[0](4.4,3.4){R1}
		\uput[0](1.7,1.6){R2}
		%labels du B
		\uput[0](8.8,1.6){D1}
		\uput[0](10.9,2.8){D2}
		\uput[0](8.8,3.9){R1}
		%lignes du A
		\psline(4,2)(4.6,2.8)
		\psline(4.2,3.8)(3.6,4)
		\psline(3,3.6)(2.4,2.1)
		\psline(1.6,2.6)(1.85,2.25)
		%lignes du B
		\psline(9.2,2.2)(9.2,3.3)
		\psline(9.8,3.8)(10.95,3.3)
	\end{pspicture}
	
	\caption[Utilisation principale des référentiels conçus comme fournisseurs de clés]{Utilisation principale des référentiels conçus comme fournisseurs de clés (R: Référentiel; D: Document)}
	\label{schema_general_controler}
\end{figure}
	
	\part{\label{relier}RELIER. Vers le partage de référentiels communs (début des années 2000 – milieu des années 2010)}
	
	%intro
	Avec la naissance du Web à la fin du \textsc{XX}\textsuperscript{ème}siècle, le référentiel voit ses usages étendus et multipliés; sa place en est modifiée. Les différents types de référentiels ont dû s'adapter aux nouvelles technologies qui ont alors été offertes. Cependant, ces nouvelles technologies, accompagnées de nouveaux formats, de nouveaux standards et de nouveaux protocoles, ont conduit à la progressive disparition de la notion de référentiel: le référentiel a été divisé en données. Le lien entre ces données devient alors essentiel pour (re)donner du sens entre elles. Mais ces liens peuvent ne pas être seulement présents pour établir une liaison entre deux données: ils peuvent permettre de relier deux jeux de données, et notamment un jeu de données d'une institution avec celui d'une autre. Si le lien n'était présent que dans les \textit{thesauri} pour définir le type de relations, il est désormais l'enjeu principal de l'utilisation d'un autre référentiel ou d'un autre jeu de données: il permet l'enrichissement de ses propres données avec des données externes.\\
	
	Le Web de données a permis cet éclatement du référentiel et des jeux de données, et , par sa structure, a nécessité la création de ces liens. Cependant, un type de référentiel, l'ontologie, est toujours indispensable sur le Web car il offre un vocabulaire pour la description du monde réel. L'ontologie permet aussi l'établissement d'un grand nombre de liens entre les données, et permet à une institution de n'utiliser plus un seul mais autant de référentiels qu'elle le souhaite.\\
	
	C'est pourquoi le milieu bibliothéconomique s'est très tôt intéressé au Web de données et participe aux nombreuses réflexions qui l'animent. Les référentiels sont partagés sur ce Web de données et sont repris par d'autres institutions. Si le lien entre les institutions n'est pas l'enjeu principal de la réutilisation de ces référentiels, il est néanmoins créé, et permet la naissance de référentiels de rang supérieur, agrégateurs de données.
	
	\chapter{\label{II-A}Le web de données: une exposition commune des référentiels}
\titreEntete{Le web de données: une exposition commune des référentiels}

\begin{citationLongue}
	Le web de données, en proposant une forme d'interopérabilité basée sur des standards du Web et sur des liens entre les ressources, semble à même de faciliter l'accès à des données structurées, stockées dans des bases telles que les catalogues de bibliothèques, les inventaires d'archives ou les bases culturelles des musées.\footcite[p.45]{dalbin_approches_2011}
\end{citationLongue}
\medskip
Le domaine bibliothéconomique, et plus généralement celui culturel, est l'un des premiers à s'être intégré dans le Web de données. Les avantages apportés par le Web, tels que le partage et la mise en commun de référentiels et de données, ont permis une large adoption des standards et formats du Web de données dans les institutions patrimoniales. Cependant, les pratiques individualistes qui étaient celles des institutions auparavant se retrouvent dans le développement de ce web de données et ont conduit à une efficacité limitée lors de ses débuts.\\

Malgré ces difficultés des premiers temps, les institutions patrimoniales se sont désormais emparées de ce web de données, devenu un lieu de partage de liens et un fournisseur d'identifiants que les institutions peuvent stocker en vue d'enrichir leurs propres données\footnote{Un premier exemple a été étudié précédemment avec l'achat de ressources extérieures à l'\ac{ina}. Voir \reference{I-B-3}.}.\\

Plus encore que le partage de liens, le web de données est également un apport considérable dans l'expérience de l'utilisateur qui recherche des données spécifiques sans savoir vers quelle institution se tourner. Il peut, avec les technologies du Web, naviguer de lien en lien, d'institution en institution, rebondir de document en document, sans se rendre compte des frontières techniques ou institutionnelles\footnote{\og Sur le Web, un utilisateur a la possibilité de naviguer d'un site à un autre sans avoir connaissance des moyens techniques utilisés pour publier les données, ni même avoir conscience des ruptures ou des frontières entre chacun des sites. \fg{} in \cite[p.45]{dalbin_approches_2011}.}. Le Web permet un affranchissement des frontières, à la fois pour l'utilisateur final que pour les machines.\\

Enfin, les technologies du Web, utilisées dans le web de données, proposent de nouveaux formats et de nouvelles modélisations de données, conduisant à la disparition progressive de la notion de référentiel. De plus, la massification des données du web de données posent la problématique de leur accès rapide pour l'utilisateur.

\section{\label{II-A-1}Le Web de données: naissance et principes}
\titreEntete{Le Web de données: naissance et principes}

La recherche d'un protocole et d'un format d'échange de données entre les institutions est constante. Nous avons évoqué précédemment\footnote{Voir \reference{I-C-2}.} les difficultés rencontrées avec les protocoles Z039-50 et \ac{oaipmh}. Ces derniers sont insuffisants pour permettre un partage massif de données et de référentiels, mais ils ont permis l'évolution de la réflexion sur l'interopérabilité. Le grand bouleversement est survenu au milieu des années 2000 avec le Web de données qui ne crée pas de protocole nouveau, mais s'appuie sur un autre largement répandu et utilisé, \ac{http}. Les règles édictées ont permis sa bonne utilisation pour créer le web de données et la naissance d'un format d'échange, \ac{rdf}.

\subsection{\label{II-A-1-a}Créer un modèle de données nativement compatible avec le Web: le Web de données}
\titreEntete{Un modèle de données nativement compatible avec le Web}

\subsubsection{\label{II-A-1-a-i}Naissance du Web de données}
\titreEntete{Naissance du Web de données}

Dès 1989, \nP{Tim}{Berners-Lee} propose un \index[ref]{typologie@Typologie!graphe@Graphe de nœuds et de liens}\og espace d'information commun\fg{}\footnote{\og pool of information\fg{} in \cite{berners-lee_information_1989}.} où les textes seraient liés par des liens\footnote{\og a web of notes with links\fg{} in \cite{berners-lee_information_1989}}. Il propose ainsi un modèle de nœuds et de liens qui permet d'entrer n'importe quel type d'informations: grâce aux liens, une ressource peut être trouvée sans avoir eu à la chercher. Cependant, plusieurs difficultés demeurent encore: il est nécessaire, comme dans un arbre, d'avoir des nœuds uniques; la modélisation du monde réel est impossible, et le modèle de nœuds et de liens se heurte aux mêmes réflexions que Porphyre et les encyclopédistes quant à la possibilité de représenter le monde en un seul arbre. Pour y faire face, \nP{Tim}{Berners-Lee} propose comme solution le lien hypertexte.\\

En 1994, \nP{Tim}{Berners-Lee} continue la réflexion sur le Web et les liens hypertexte\footcite{berners-lee_plenary_1994}. En 1989, seuls les documents étaient mis sur le Web et \nP{Tim}{Berners-Lee} décrivait la manière de les relier entre eux. En 1994, il propose d'intégrer au Web des données du monde réel qui seraient reliées par des liens hypertexte, comme les documents, toujours sous la forme de nœuds et de liens. Cette proposition est le déclencheur de la réflexion sur le Web de données. Cependant, la réalité n'est pas compréhensible par une machine, et les liens ne peuvent se comprendre que par leur contexte: l'ajout de valeurs aux relations permet de donner du sens au Web --- c'est le \index[ref]{typologie@Typologie!graphe@Graphe de nœuds et de liens}Web sémantique. \nP{Tim}{Berners-Lee} se fait ainsi le promoteur de la donnée structurée à la fois pour la machine et pour l'humain.\\

Une feuille de route est par conséquent écrite par \nP{Tim}{Berners-Lee} en 1998\footcite{berners-lee_semantic_1998}. Il y évoque pour la première fois le terme \og Web de données\fg{}\footnote{\og web of data\fg{} in \cite{berners-lee_semantic_1998}}, mais la feuille de route n'est pas appliquée et il faut attendre 2006 et la publication \textit{Linked data}\footcite{berners-lee_linked_2006} pour que les recommandations du Web sémantique soient expliquées et adoptées. Fondamentale, cette publication évoque les principes du Web de données actuel en décrivant les bonnes pratiques à adopter. Le Web est ainsi perçu comme une \og base de données globale\fg{}\footcite[§29]{bermes_convergence_2013} où les données sont reliées de la même manière que les documents HTML avec des liens hypertexte.\\

Cette interopérabilité basée sur les liens, théorisée notamment par \nP{Tim}{Berners-Lee}, est à l'origine du \index[ref]{typologie@Typologie!graphe@Graphe de nœuds et de liens}Web de données et de l'actuel partage de données et de référentiels entre les institutions patrimoniales. Le lien apparaît comme essentiel et permet de décloisonner chaque institution pour les faire communiquer ensemble de manière à améliorer la recherche de données et de documents par l'utilisateur final\footnote{L'utilisateur final n'est pas seulement le grand public, il peut être chercheur, professionnel dans une institution, consommateur commercial, \dots}.

\subsubsection{\label{II-A-1-a-ii}Principes généraux}
\titreEntete{Principes généraux}

La publication de 2006 de \nP{Tim}{Berners-Lee} décrit précisément les principes du Web de données qu'il est nécessaire de développer pour comprendre l'évolution des pratiques documentaires des institutions depuis le milieu des années 2000. Ces principes s'appuient sur l'architecture du Web existant et ne visent pas la création d'un Web: l'interopérabilité des données doit passer par une interopérabilité des protocoles et des formats utilisés avec ceux du Web, qui connaît une utilisation croissante en 2006.\\

Le premier principe évoqué est celui de l'utilisation des Uniform Resource Identifier (URI) comme clé unique d'une ressource: en cas de non utilisation du standard des URIs, le \index[ref]{typologie@Typologie!graphe@Graphe de nœuds et de liens}Web sémantique n'est plus possible\footnote{\og If it doesn't use the universal URI set of symbols, we don't call it Semantic Web\fg{} in \cite{berners-lee_linked_2006}.}. En effet, une URI possède une syntaxe précise qu'il convient de respecter et d'adopter: \textit{scheme:autorité/chaîne\_de\_caractères}\footcite[§40]{bermes_convergence_2013}.\\

Le second principe est celui de l'utilisation du protocole du WorldWideWeb, \index[ref]{echanges@Échanges!protocoles@Protocoles!http@HTTP}\ac{http}.\\

Le troisième principe impose le renvoi d'informations et de données dans des formats standards du Web, en \index[ref]{echanges@Échanges!formats@Formats!rdf@RDF}\ac{rdf}/XML, ou en N3 ou Turtle\footcite{berners-lee_linked_2006}. Tous ces formats acceptent le langage de requête \index[ref]{echanges@Échanges!protocoles@Protocoles!sparql@SPARQL}SPARQL.\\

Enfin, le quatrième principe est celui de la création de liens entre les ressources --- donc les URIs --- sans lesquels les efforts réalisés avec les trois premiers principes sont vains. Ainsi, un Web fiable, sans frontières, est créé\footnote{\og serious, unbounded web in which one can find al kinds of things, just as on the hypertext web we have managed to build\fg{} in \cite{berners-lee_linked_2006}}; l'utilisateur peut y naviguer facilement grâce aux liens hypertextes. Le \index[ref]{typologie@Typologie!graphe@Graphe de nœuds et de liens}Web de données est par conséquent moins une base de données qu'un lieu où les liens donnent de la valeur aux ressources liées: plus une ressource possède de liens, plus celle-ci a une description précise et fiable, plus elle devient visible à l'utilisateur.\\

Nous l'aurons remarqué, depuis le début de cette description du Web de données, la notion de référentiel semble s'estomper au profit de ressources et de données liées. En effet, un référentiel n'est qu'une mise en forme spécifique d'un jeu de données selon une structure propre à son producteur. Cette spécificité de chaque jeu de données n'est pas valable dans le Web de données: un retour à la donnée est nécessaire, les liens qui lui seront affectée permettront alors de représenter son ancienne structure dans le référentiel. \nP{Tim}{Berners-Lee} décrit la nécessité de se dégager de ses propres formats sur le Web pour évoluer vers des formats compréhensibles par une machine: il crée l'échelle des cinq étoiles, le \index[ref]{echanges@Échanges!formats@Formats!rdf@RDF}\ac{rdf} étant la meilleure des solutions d'exposition des données.

\subsection{\label{II-A-1-b}Inventer un format d'échange compatible avec ce modèle de données: RDF}
\titreEntete{Inventer un format d'échange compatible avec ce modèle de données}

\index[ref]{echanges@Échanges!protocoles@Protocoles!http@HTTP}\ac{http} est le protocole utilisé pour le Web de données; le format d'échange est \index[ref]{echanges@Échanges!formats@Formats!rdf@RDF}\ac{rdf}. C'est un standard, développé pour le Web, capable d'assurer l'interopérabilité des données. Seules des URIs peuvent constituer des ressources. Ces ressources sont ensuite reliées par un lien typé dû au formalisme offert par \ac{rdf}. \ac{rdf} ne permet pas, comme cela est le cas avec les encodages XML archivistiques ou codicologiques, un schéma prédéfini, mais un modèle logique de description des ressources.\\

Avec \ac{rdf}, deux ressources ne peuvent être reliées directement, seule leur relation peut être typée afin que la machine puisse interpréter la nature de leur lien, peu importe la localisation des deux ressources. Ainsi, la forme d'un triplet \index[ref]{echanges@Échanges!formats@Formats!rdf@RDF}\ac{rdf} reflète cette distinction: le \og sujet\fg est nécessairement une ressource --- par conséquent une URI ---, il est suivi d'un \og prédicat\fg{} qui défini la nature de la relation avec le troisième élément du triplet, l'\og objet\fg{}, qui peut être une ressource ou un littéral. Le triplet est donc une simple phrase sujet-verbe-complément compréhensible par une machine. Une ressource pouvant être à la fois sujet dans un triplet, prédicat dans un autre, ou objet dans d'autres; un graphe se construit alors. L'information est donc totalement déconstruite pour un humain, mais elle devient compréhensible par une machine, qui permet ensuite la reconstruction de l'information par des \index[ref]{echanges@Échanges!protocoles@Protocoles!sparql@SPARQL}requêtes efficaces sur ces triplets --- cette reconstruction pouvant être personnalisée selon la requête effectuée.

\bigskip
\bigskip
Avec l'apparition du Web de données, un changement d'échelle des référentiels a lieu: ils cessent d'être utilisés par leur seul créateur dès lors qu'ils sont transformés puis envoyés dans le Web de données, ils peuvent désormais être partagés et réutilisés grâce aux URIs. L'utilisation d'un protocole existant, ainsi que d'un nouveau format d'échange, a permis de s'éloigner des modèles d'interopérabilité par conversion et copie, ou par le plus petit dénominateur commun: les référentiels sont des nœuds autour desquels les jeux de données sont rattachés\footnote{C'est l'intéropérabilité de la \og roue et de l'essieu\fg{}, ou\og hub and spoke\fg{}, décrite dans \cite{bermes_convergence_2013}. Voir \reference{annexe_types_interop} (\reference{hub_spoke})}. 
\section{\label{II-A-2}La mise en commun de référentiels au service des institutions}
\titreEntete{La mise en commun de référentiels au service des institutions}

%intro

%conclu
\section{\label{II-A-3}Vers la fin de la notion de référentiels?}
\titreEntete{Vers la fin de la notion de référentiels?}

L'éclatement du document en données sur le Web a permis de grandes avancées pour les institutions patrimoniales qui partagent non plus des notices bibliographiques ou d'autorités, mais des données liées au travers d'URIs. Elles ont trouvé avec le Web de données un protocole ainsi qu'un format d'échange standardisés et utilisés par tous les utilisateurs. Cependant, ce règne de la donnée sur le Web conduit à de nouvelles réflexions quant à la définition des référentiels: personnes, lieux et sujets sont considérés comme des données de référence; pourquoi alors ne pas considérer une œuvre comme une donnée de référence elle aussi? Cette conceptualisation de la réalité conduit ainsi à repenser les modèles de données dans les institutions ainsi que les formats de description des documents, afin de partager, sur le Web, des formats et des standards pour profiter au plus grand nombre.

\subsection{\label{II-A-3-a}Quand tout devient un potentiel référentiel}
\titreEntete{Quand tout devient un potentiel référentiel}

Ce référentiel, qui était une liste de mots contrôlés et hiérarchisés avec les \textit{thesauri}, se tourne, grâce au Web de données, vers une nouvelle définition et de nouveaux usages: un référentiel est désormais un ensemble d'informations susceptibles d'être partagées puis réutilisées dans divers systèmes documentaires pour créer du lien\footcite[§49]{bermes_les_2013}. Ces informations ne sont plus spécifiquement des termes choisis et contrôlés par des documentalistes: tout peut devenir information et par conséquent référentiel si des relations sont établies.\\

Dès la fin du \textsc{XX}\textsuperscript{ème} siècle, deux modèles voient le jour pour repenser la structure de la donnée et la place des référentiels. En 1996, la réflexion autour de la description des collections muséales permet la création du modèle du \index[ref]{modelisation@Modélisation!cidoc@CIDOC-CRM}\ac{cidoccrm}\footcite{noauthor_cidoc-crm_nodate}, modèle orienté document -- objet --- permettant de pouvoir décrire les interactions de chaque objet avec d'autres entités.\\

Parallèlement à ce modèle destiné aux descriptions de collections de musées, les bibliothèques mènent également une réflexion similaire entre 1992 et 1997, permettant ainsi l'élaboration des modèles \index[ref]{modelisation@Modélisation!frbr@FRBR}\ac{frbr}\footnote{Plusieurs modèles \ac{frbr} spécifiques ont vu le jour: les \index[ref]{modelisation@Modélisation!frad@FRAD}\ac{frad} (pour les données d'autorité; voir \cite{federation_internationale_des_associations_de_bibliothecaires_et_de_bibliotheques_fonctionnalites_2010}) et les \index[ref]{modelisation@Modélisation!frsad@FRSAD}\ac{frsad} (pour les sujets; voir \cite{federation_internationale_des_associations_de_bibliothecaires_et_de_bibliotheques_fonctionnalites_2010-1}), destinés à modéliser les relations et les entités des points d'accès.}. 
Il est nécessaire de s'attarder sur ces \index[ref]{modelisation@Modélisation!frbr@FRBR}\ac{frbr} afin de mieux comprendre la structure du \index[ref]{led@Linked Enterprise Data (LED)!ldd@Lac de données (INA)}\index[ref]{modelisation@Modélisation!ldd@Lac de données (INA)}\ldd de l'\ac{ina} expliquée par la suite\footnote{Voir \reference{III-B}.}, bien que celle-ci ne soit pas exactement similaire aux \ac{frbr}.
Trois groupes sont distingués dans les \ac{frbr}\footnote{Le rapport détaillé des \ac{frbr} indique l'ensemble des groupes et des relations possibles, ce que nous ne développerons pas ici. Voir \cite{federation_internationale_des_associations_de_bibliothecaires_et_de_bibliotheques_fonctionnalites_2012}}: l'un correspond à la notice bibliographique elle-même, les deux autres aux points d'accès.\\

La notice bibliographique est, avec les \ac{frbr}, divisée en quatre sections, partant des caractéristiques propres de l'exemplaire décrit --- l'item ---, puis par les caractéristiques de la publication auquel le document appartient --- la manifestation --- et par celles de son contenu --- l'expression ---, pour terminer avec celles de la création abstraite auquel le document appartient --- l'œuvre.\\

Le premier point d'accès est le groupe des personnes et des collectivités qui permettent de décrire les responsabilités de chacun --- auteur, producteur, ayant-droit, \dots ~ ---, de l'œuvre à l'item. Le second point d'accès permet la description du contenu, le sujet de l'activité intellectuelle ou artistique à travers des concepts, des objets, des événements ou des lieux.\\

La création de liens entre les différents groupes et les différentes entités permet une description fine des contenus, des responsabilités et des œuvres. Elle permet également la liaison entre des entités conformément aux principes du Web de données. Enfin, la création infinie de référentiels, avec une œuvre pouvant être sujet d'une autre œuvre par exemple, est possible\footnote{Voir \reference{annexe_nvx_modeles} (\reference{frbr}).}.\\

Avec ces nouveaux modèles --- \index[ref]{modelisation@Modélisation!cidoc@CIDOC-CRM}\ac{cidoccrm} et \index[ref]{modelisation@Modélisation!frbr@FRBR}\ac{frbr} --- , toutes les notions peuvent devenir des référentiels: les œuvres, les expressions, les sujets, les familles, \dots~. Leur avantage est leur partage désormais possible avec d'autres métiers, et plus largement sur le Web de données puisqu'une exposition en \index[ref]{echanges@Échanges!formats@Formats!rdf@RDF}\ac{rdf} est possible et réalisée. Plusieurs institutions peuvent alors utiliser les données modélisées selon les \ac{frbr} afin de créer plus de liens que si elles ne s'étaient appuyées, pour établir la description de leur objet, que sur les référentiels communs, tels que \index[ref]{lod@Linked Open Data (LOD)!lcsh@LCSH}\index[ref]{autorites@Autorités!lcsh@LCSH}\ac{lcsh} ou \index[ref]{lod@Linked Open Data (LOD)!rameau@RAMEAU}\index[ref]{autorites@Autorités!rameau@RAMEAU}\ac{rameau}.

\subsection{\label{II-A-3-b}Vers une uniformisation internationale de la donnée sur le Web et l'adoption de \ac{rdf} comme format de production}
\titreEntete{Uniformisation de la donnée sur le Web et adoption de RDF}

L'apparition des nouveaux modèles centrés sur les entités, dans lesquels toutes les entités sont potentiellement partageables et utilisables par d'autres pour servir de référentiel, favorise de nouvelles réflexions sur le catalogage des documents: la finalité devenant de plus en plus souvent la publication des données sur le \index[ref]{typologie@Typologie!graphe@Graphe de nœuds et de liens}Web sémantique, est-il toujours nécessaire de cataloguer dans un format pour ensuite convertir les données en \ac{rdf}?\\

La nécessité d'améliorer les règles de catalogage dans les pays anglo-saxons dans les années 2000 a conduit à l'utilisation des nouveaux modèles \index[ref]{modelisation@Modélisation!frbr@FRBR}\ac{frbr} dans le code \index[ref]{echanges@Échanges!formats@Formats!rda@RDA}\ac{rda} publié en 2010. Ce changement majeur est testé à partir de 2011 à la Library of Congress, puis adopté en 2013 dans les pays anglo-saxons; la France et ses agences bibliographies ABES et BNF tentent d'intégrer ces nouvelles règles.\\

En effet, \ac{rda} est nativement pensé autour du Web de données et de l'utilisation en ligne des données, de manière à pouvoir ensuite exprimer les entités et leurs relations sous la forme de triplets \index[ref]{echanges@Échanges!formats@Formats!rdf@RDF}\ac{rdf} grâce à l'attribution d'un identifiant à chaque élément ou valeur\footnote{Les vocabulaires \ac{rda} ont fait l'objet d'un groupe de travail après la création du code de catalogage \ac{rda}.}. Ce code de catalogage est destiné à être internationalement utilisé, de manière à uniformiser les données produites et favoriser ainsi les échanges.\\

En adoptant \index[ref]{echanges@Échanges!formats@Formats!rda@RDA}\ac{rda}, le format \ac{marc} est délaissé et semble ne plus pouvoir répondre aux changements impliqués par le Web. Cependant, \ac{rda}, publié et utilisé, subit déjà des évolutions et un nouveau modèle de données, nativement centré sur \ac{rdf}, est créé en 2012: \index[ref]{echanges@Échanges!formats@Formats!bibframe@Bibframe}Bibframe\footcite{library_of_congress_overview_2016}. Le format \index[ref]{echanges@Échanges!formats@Formats!marc@MARC}\ac{marc} est abandonné au profit d'un modèle pensé autour de \ac{rdf}, et, par conséquent, des usages numériques des utilisateurs. Le modèle de données de Bibframe est semblable à celui des \index[ref]{modelisation@Modélisation!frbr@FRBR}\ac{frbr} avec les \textit{work}, les \textit{instance} et les \textit{item}. Seulement, \index[ref]{echanges@Échanges!formats@Formats!rdf@RDF}\ac{rdf} n'apparaît plus comme un format de sortie des données après leur catalogage; il est désormais le format natif de catalogage. Ainsi, chaque donnée, chaque entité, devient un référentiel en ce qu'elle est nativement liée à d'autres ressources.

%conclu
\bigskip
\bigskip
L'impact du Web de données sur la notion de référentiel s'étend au-delà de la réflexion sur la définition et la structure d'un référentiel: toute la modélisation des données des institutions est remise en question. Ces dernières doivent s'adapter et tenter de trouver de nouveaux modèles de données qui puissent répondre d'une part à leurs besoins internes de description et de signalement des collections, d'autre part aux besoins croissants des utilisateurs sur le Web à la recherche des ressources libérées des barrières technologiques et institutionnelles. Ainsi, il n'y a plus de référentiels dans lesquels l'utilisateur peut aller, les catalogues des institutions ouverts en \ac{rdf} sont eux-mêmes ces référentiels.

%conclu

	\chapter{\label{II-B}Partager des structurations similaires de jeux de données par les classes et les propriétés : les ontologies, grammaires communes mais spécifiques}
\titreEntete{Les ontologies, grammaires communes mais spécifiques}

%intro
L'éclatement des référentiels dans le Web de données conduit à la création, ou au renforcement, de liens entre eux, ainsi qu'entre leur données. Ces liens doivent porter une valeur sémantique de manière à ce que l'éclatement ait lieu sans perdre d'informations. De nouveau, il faut alors que le sens des relations soit contrôlé et partagé par le plus grand nombre: les ontologies permettent cela. Ainsi, si le référentiel présenté sous la forme de liste ou de thésaurus disparaît au profit du Web de données, de nouveaux référentiels, adaptés au Web sémantique, prennent le relais.\\

Souvent confondues avec les systèmes organisés de connaissances\footnote{Ils sont fréquemment nommés KOS pour \textit{Knowledge Organization Systems}.}, les ontologies diffèrent par leur formalisme. La \reference{controler} a montré qu'une interopérabilité par les référentiels est possible et que les arbres de classification portent difficilement du sens, l'arborescence permettant essentiellement d'effectuer une classification; avec les ontologies, une interopérabilité sémantique peut avoir lieu.\\

De même qu'avec le Web de données, le milieu bibliothéconomique a été l'un des premiers à adopter massivement les ontologies afin de pouvoir typer les relations entre les entités: l'ontologie est essentielle au Web sémantique.\\

\section{\label{II-B-1}L'ontologie, un vocabulaire structurant}
\titreEntete{L'ontologie, un vocabulaire structurant}

%intro
Dans le chapitre précédent\footnote{Voir \reference{II-A}.}, nous avons évoqué un premier type de référentiel --- les vocabulaires de valeurs --- présent dans le Web de données. De manière à pouvoir décrire ces ressources\footnote{\og L'une des fonctions des ontologies est de permettre de définir la nature des ressources\fg{} in \cite[§49]{bermes_convergence_2013}.}, d'autres référentiels sont nécessaires, les ontologies, en fournissant les classes et les propriétés utiles aux descriptions. Au-delà de l'apport de ces éléments, l'ontologie permet également une description formelle par des axiomes et des règles de raisonnement, visibles dans le Web de données avec \ac{rdfs} et \ac{owl}.\\

L'ontologie informatique est un concept récent, né à la fin du \textsc{XX}\textsuperscript{ème}siècle comme le Web. Plusieurs types d'ontologies existent, reflétant leur caractère universel ou non, leur domaine de description; leur structuration et leur formation doivent cependant répondre à des critères précis de manière à structurer le plus efficacement possible de référentiel.

\subsection{\label{II-B-1-a}Origines de l'ontologie informatique}
\titreEntete{Origines de l'ontologie informatique}

\begin{citationLongue}
	[Les ontologies sont] des vocabulaires de termes --- classes, relations, fonctions, constantes d'objet --- avec des définitions communes, sous la forme d'un texte compréhensible par les humains et applicable à la machine, de contraintes déclaratives dans leur forme la mieux formée.\footnote{\og vocabulaires of representational terms --- classes, relations, functions, object constants ---  with agreed-upon definitions, in the form of human-readable text and machine-enforceable, declarative constraints on their well formed use\fg{} in \cite[p.2]{gruber_role_1991}}
\end{citationLongue}

L'ontologie est d'abord une science philosophique, née avec les \textit{Catégories} d'Aristote, étudiant la réalité des entités, les relations qu'elles entretiennent --- hiérarchie, similarité --- pour trouver les similarités et les différences présentes dans le monde. Au \textsc{XIX}\textsuperscript{ème}siècle, le siècle de la taxonomie, cette science philosophique devient l'étude de l'ensemble des connaissances existantes dans le monde\footcite{welty_supporting_2011}.\\

Ce n'est qu'en 1991\footcite{gruber_role_1991} puis 1993\footcite{gruber_toward_1993} que \nP{Thomas R.}{Gruber}, souhaitant améliorer l'intelligence artificielle et l'indexation structurée, évoque l'ontologie informatique, pensée comme un ensemble d'entités déclaratif destinée au partage des connaissances entre les machines: \og Une ontologie est une spécification explicite d’une conceptualisation\fg{}\footnote{\og An ontology is an explicit specification of a conceptualization. \fg{} in \cite[p.1]{gruber_toward_1993}.}. Cette définition donnée très tôt par \nP{Thomas R.}{Gruber} permet d'observer deux principes de l'ontologie: premièrement, elle est une conceptualisation d'un domaine, par conséquent elle est un choix de description sur un domaine précis; deuxièmement, cette conceptualisation est spécifiée, c'est à dire qu'elle a une description formelle.\\

\nP{Rudi}{Studer} apporte des précisions en 1998\footcite{studer_knowledge_1998} en proposant une nouvelle définition plus spécifique de l'ontologie: \og Une ontologie est
une spécification formelle et explicite d’une conceptualisation partagée\fg{}\footnote{\og An ontologyis a formal, explicit specification of a shared conceptualization\fg{} in \cite{studer_knowledge_1998}}. Une ontologies est formelle de manière à pouvoir être comprise par une machine; elle est une spécification explication par la déclarativité de ses concepts, de ses propriétés, \dots~; elle est partagée car elle prend l'ensemble des connaissances d'une communauté, d'un domaine; enfin, la conceptualisation renvoie au domaine décrit par cette ontologie.\\

L'ontologie est un référentiel de classes et de propriétés, ne s'appliquant qu'à un seul domaine particulier de la connaissance, mais permettant de le structurer. Son fort développement a permis une application dans le Web de données et dans le milieu bibliothéconomique, qui considère les ontologies comme \og des éléments de description de métadonnées\fg{}\footcite{baker_rapport_2012}.

\subsection{\label{II-B-1-b}Des ontologies différentes}
\titreEntete{Des ontologies différentes}

Une grande diversité d'ontologies existe. Certaines sont plus importantes que d'autres du fait du nombre d'utilisations qu'elles entraînent et de leur généralité; d'autres, plus spécifiques, paraissent moins importantes par le faible nombre de liens qu'elles suscitent. Les ontologies peuvent également être classées selon les usages qui en sont faits, selon si leur finalité est une publication sur le Web sémantique ou simplement une utilisation interne à une institution.\\

Au plus haut niveau se trouvent des ontologies \og noyaux\fg{}\footcite[p.4]{isaac_les_2012} qui modélisent les connaissances communs, partageables et réutilisables d'un domaine à un autre: le modèle \ac{cidoccrm} des musées propose ainsi une ontologie réutilisée dans \ac{rdfs} notamment\footnote{Voir \reference{annexe_onto} (\reference{onto_crm}).}. Son vocabulaire correspond aux événements, aux objets, aux moments, \dots ~ ce qui en fait un vocabulaire générique pour d'autres ontologies de plus bas niveau.\\

Au niveau inférieur se trouvent les ontologies de domaine, propres à un domaine en particulier: elles modélisent les connaissances de ce domaine uniquement; elles offrent des concepts et des relations permettant de décrire les activités et les vocabulaires du domaine en question. Les concepts de ces ontologies de domaine sont souvent des spécialisations\footnote{Ainsi que le faisait remarquer \nP{Rudi}{Studer} en 1998 in \cite{studer_knowledge_1998}.} d'ontologies de plus haut niveau. L'ontologie \ac{frbr}\footnote{Décrite dans la \reference{II-B-3}.} peut être considérée comme une de ces ontologies de domaine car elle utilise \ac{rdfs}, les \textit{Dublin Core Terms} (DC Terms), et d'autres ontologies de haut niveau\footnote{Voir \reference{annexe_onto} (\reference{onto_frbr})}.\\

Plus bas encore, il existe des ontologies utilisées dans l'unique cadre d'utilisations applicatives, par un petit nombre d'utilisateurs. Elles ne modélisent par conséquent que les termes et les relations nécessaires à l'application, en utilisant principalement des ontologies de plus haut niveau, notamment celles de domaine\footnote{L'ontologie \textit{Citation Counting and Context Characterization Ontology}(C4O) est l'une de ces ontologies applicatives, ayant peu d'ontologies l'utilisant, et utilisant un grand nombre d'ontologies de plus haut niveau. Voir \reference{annexe_onto} (\reference{onto_c4o}).}.\\

De multiples ontologies existent et diffèrent de cette hiérarchisation\footcite[p.2]{isaac_les_2012} et participent à la diversité des ontologies. Cette diversité est essentielle pour décrire tout type d'objets ou d'événements --- que ce soit dans le milieu culturel ou non ---, ou pour relier à ces ontologies des vocabulaires propres: elle participe à l'interopérabilité des référentiels entre eux. Cette interopérabilité, grâce aux ontologies, peut être entre un système documentaire interne et le Web de données, ou bien entre deux jeux de données du même Web de données.

\subsection{\label{II-B-1-c}Les principes de l'ontologie}
\titreEntete{Les principes de l'ontologie}

Permettre l'interopérabilité entre tout type de données et de jeux de données nécessite une structure et des principes. Dans la publication de 1993\footcite{gruber_toward_1993}, \nP{Thomas R.}{Gruber} édicte déjà cinq critères sans lesquels une ontologie ne peut pas être formée correctement. Ces critères, généraux, contraignent la graphie tout en laissant le champ ouvert aux modifications futures d'une ontologie. Il préconise ainsi:
\begin{itemize}
	\item la clarté des termes décrits: leur description doit être objective, complète, et réalisée dans le langage naturel;
	\item la cohérence des axiomes retenus et l'interdiction de la discordance entre les termes et les axiomes: elle permet la spécialisation de la conceptualisation qui est la définition de l'ontologie;
	\item la possibilité d'étendre l'ontologie même après sa création: cela permet d'accepter les changements d'usages ou de besoins liés à cette ontologie, et par conséquent de la faire évoluer facilement;
	\item le biais d'encodage doit être minimal de manière à permettre la plus grande interopérabilité;
	\item l'engagement dans l'ontologie doit être minimal, les termes utilisés souvent être ceux les plus souvent utilisés: cela permet la réutilisation de l'ontologie
\end{itemize}
\bigskip

Ces cinq critères ontologiques dirigent la structure et la nature des éléments essentiels aux ontologies et les constituant. Les concepts sont le premier de ces éléments: ils permettent de définir des idées, des objets ou des notions. De même que pour les vocabulaires contrôlés\footnote{Voir \reference{I-A-2}.}, plusieurs propriétés peuvent s'appliquer à ces concepts. \nP{Nicola}{Guarino} décrit ces propriétés\footcite{guarino_formal_1998} de généricité --- absence d'extension pour le concept ---, d'identité, de rigidité --- si une instance du concept reste en permanence son instance ---, d'anti-rigidité --- si une instance est principalement définie par son appartenance à un autre concept --- et d'unité des concepts. Ainsi, deux concepts peuvent être disjoints, équivalents ou dépendants.\\

Les relations sont une autre partie essentielle des ontologies, sans lesquelles les concepts n'ont pas de sens entre eux et ne peuvent pas être formalisés. Elles peuvent être inclusives --- hiérarchiques ---, ou ensemblistes avec des unions, des intersections ou des exclusions. À ces relations peuvent d'ajouter des axiomes, des règles, qui viennent régir les contraintes, les relations ou les concepts eux-mêmes.\\

Les différents principes des ontologies les rendent strictes sur leur formation et leur structure, de nombreuses propriétés s'imposant. Complexes, ils permettent néanmoins la création d'un vocabulaire à la fois contrôlé, hiérarchique et utilisable par tous.

%conclu
\bigskip
\bigskip
L'ontologie peut alors trouver son intérêt sur le Web avec une indexation réalisée par les moteurs de recherche; sur le Web et en institution en permettant la description de jeux de données par des ontologies publiques; en structurant la connaissance du monde. L'une des applications principale est bibliothéconomique avec la possibilité de valoriser et de publier les collections sous forme de métadonnées. Enfin, l'ontologie permet de relier deux jeux de données, deux référentiels, pourtant éloignés, selon un vocabulaire commun partagé publiquement.
\section{\label{II-B-2}Des \textit{Knowledge Organization Systems} (KOS) à \ac{skos}: vers l'interopérabilité syntaxique}
\titreEntete{Vers l'interopérabilité syntaxique}

%intro
Les ontologies ressemblent fortement aux \textit{thesauri} et autres vocabulaires contrôlés --- les \ac{kos} --- à cause du contrôle de la graphie, des termes retenus ou rejetés, et de l'établissement de relations entre les termes. Cependant, une ontologie n'est pas nécessairement un thésaurus, alors qu'un thésaurus est une ontologie. Bien que la distinction entre les deux soit mince, elle est essentielle en raison du formalisme qui compose les ontologies.\\

L'interopérabilité sémantique, permise par les ontologies, bâtit de l'interconnexion entre les jeux de données, rend possible l'échange et la publication de données nativement différentes sur le Web de données. La conversion par les institutions patrimoniales d'une partie ou de l'ensemble de leurs vocabulaires en ontologie a été permise par l'ontologie \ac{skos}.

\subsection{\label{II-B-2-a}Distinguer les systèmes d'organisation de la connaissance des ontologies}
\titreEntete{Distinguer les KOS des ontologies}

Les \index[ref]{typologie@Typologie!vocabulaires controles@Vocabulaires contrôlés}\ac{kos} sont des référentiels contrôlés de vedettes et de termes qui ne sont valables que pour un domaine d'activité et de la connaissance. Ils sont souvent organisés par des relations terminologiques et sémantiques qui les font se confondre avec les \index[ref]{typologie@Typologie!ontologie@Ontologie}ontologies en raison de leur formalisation\footcite[p.48]{dalbin_approches_2011}. Plus encore, sur le Web de données, les \ac{kos} ne jouent pas de rôle, ils n'apportent pas de structure, mais complètent des référentiels ou des jeux de données déjà existants. Les \ac{kos} n'offrent qu'une succession de termes destinés à remplir des champs descriptifs, alors que les ontologies dirigent directement les données et leur structure.\\

Les \index[ref]{typologie@Typologie!ontologie@Ontologie}ontologies permettent de dépasser certaines limites des \ac{kos} relevées dans la \reference{controler}\footcite{isaac_les_2012}:
\begin{itemize}
	\item les \index[ref]{typologie@Typologie!thesaurus@Thésaurus}\textit{thesauri} et autres \index[ref]{typologie@Typologie!vocabulaires controles@Vocabulaires contrôlés}vocabulaires sont destinés d'abord à un utilisateur humain qui peut facilement comprendre la structure du vocabulaire\footnote{C'est le cas de la visualisation graphique du thésaurus des noms communs de l'\ac{ina}. Voir \reference{annexe_thesaurus} (\reference{thesaurus_cadreur}).}; les ontologies visent quant à elles à d'abord être comprises par une machine et le \index[ref]{typologie@Typologie!graphe@Graphe de nœuds et de liens}Web sémantique;
	\item les \ac{kos} induisent souvent des relations diverses et variables, notamment dans le cas de relations génériques--spécifiques
\end{itemize}
\medskip
Les limites identifiées sont dues principalement aux relations des termes des \ac{kos}: ces vocabulaires ne sont pas sémantiques, seulement hiérarchiques dans le but de classer et d'aider l'humain dans la compréhension du contexte du terme. L'apport des \index[ref]{typologie@Typologie!ontologie@Ontologie}ontologies est l'ajout de sens aux termes grâce aux relations qu'ils entretiennent entre eux.


\subsection{\label{II-B-2-b}\ac{skos}: exposer les systèmes d'organisation de la connaissance sur le Web de données}
\titreEntete{SKOS: exposer les KOS sur le Web de données}

\index[ref]{relier@Relier!skos@SKOS}\ac{skos} a été créé dans le but de permettre aux institutions d'établir des liens entre leurs référentiels et leurs données. \ac{skos} est \og une ontologie qui se veut simple et compatible avec une majorité d’approches d’organisation des connaissances existantes (thésaurus, classifications\dots)\fg{}\footcite[p.8]{isaac_les_2012}. Elle permet de représenter presque tous les types de vocabulaires en concepts liés. \ac{skos} n'est pas un référentiel, seulement un moyen de faciliter l'interconnexion entre les données, leur échange, et de créer de nouveaux usages jusqu'alors impossibles.\\

Pour cela, \index[ref]{relier@Relier!skos@SKOS}\ac{skos}, vocabulaire \index[ref]{echanges@Échanges!formats@Formats!rdf@RDF}\ac{rdf}, reprend les propriétés des vocabulaires contrôlés, notamment du thésaurus. En effet, il est possible d'indiquer des termes préférentiels, alternatifs, traduits, variants, \dots~; de plus, des notes peuvent être introduites; enfin, des liens --- génériques et hiérarchiques, associatifs, \dots~ sont créés entre les différents concepts\footnote{Voir \reference{skos_modelisation}.}. \ac{skos} a le rôle de l'essieu dans le modèle de l'interopérabilité de la roue et de l'essieu. Cette ontologie offre un petit nombre de concepts servant à la description et à la transcription d'un thésaurus dans un langage compréhensible par une machine. Ainsi, \ac{skos} est une ontologie de domaine: elle hérite d'autres ontologies comme \index[ref]{relier@Relier!dcterms@DC Terms}DC Terms ou \index[ref]{relier@Relier!rdfs@RDFS}\ac{rdfs}, et sert d'ontologie de haut niveau pour des ontologies de plus bas niveau\footnote{La consultation de \url{https://lov.linkeddata.es/dataset/lov/vocabs/frbr} permet de constater ce rôle pivot de \ac{skos} pour les autres ontologies, ainsi que sa dépendance à d'autres ontologies.}.\\

\begin{figure}[!h]
	\centering
	\includegraphics[width=13cm]{images/SKOS_simpleThesaurus.png}
	\caption[Modélisation simplifiée de \ac{skos}]{Modélisation simplifiée de \ac{skos} [Source: \href{https://www.w3.org/Consortium/Offices/Presentations/RDFTutorial/figures/SKOS_simpleThesaurus.png}{www.w3.org}]}
	\label{skos_modelisation}
\end{figure}
\medskip

Enfin, des propriétés de \ac{skos} permettent de créer du lien entre des concepts provenant de différentes sources: ce sont les propriétés d'équivalence ou de similitude <skos:broader>, <skos:exactMatch> ou <skos:closeMatch>. Ces propriétés permettent le rapprochement de plusieurs concepts: le \index[ref]{lod@Linked Open Data (LOD)!lcsh@LCSH}\index[ref]{autorites@Autorités!lcsh@LCSH}\ac{lcsh}\footnote{Voir \reference{lcsh_liens}.} peut ainsi utiliser \ac{skos} pour créer des fiches de liens. De même, la Bibliothèque nationale de France (BnF) contient des références à l'ontologie \index[ref]{autorites@Autorités!dewey@Dewey}\index[ref]{relier@Relier!dewey@Dewey}Dewey qui lui permet ainsi d'obtenir des relations d'équivalence avec ses données. Alors, l'emploi de \ac{skos} nécessite des URIs de manière à créer des triplets \index[ref]{echanges@Échanges!formats@Formats!rdf@RDF}\ac{rdf}.\\

Avec \index[ref]{relier@Relier!skos@SKOS}\ac{skos}, l'interopérabilité sémantique est désormais possible entre les institutions sur le Web de données. Cette ontologie \ac{rdf} permet de rapprocher des jeux de données et des référentiels jusqu'alors séparés et pourtant similaires. Le vocabulaire offert pour décrire les \textit{thesauri} et ensuite les partager dans le Web sémantique permet de nombreuses applications en institutions, et facilite ainsi les opérations d'indexation ou de recherche.

%conclu
\bigskip
\bigskip
Cependant, les vocabulaires des institutions n'étant pas nativement liés entre eux, il reste difficile de les aligner, et par conséquent de passer d'un \index[ref]{typologie@Typologie!vocabulaires controles@Vocabulaires contrôlés}\ac{kos} à une modélisation \index[ref]{relier@Relier!skos@SKOS}\ac{skos}, notamment pour les relations partie--tout. \nP{Sylvie}{Dalbin} prend en exemple\footcite{dalbin_approches_2011} ce type de relation qui peut être exprimé dans le Web sémantique par trois types de relations: hiérarchique, instance--classe, ou bien sous-classe--classe. Ces incertitudes rendent le processus d'ontologisation complexe, qui l'est d'autant plus quand les termes sont dans des langues différentes.\\

Les ontologies apparaissent comme un référentiel essentiel dans le Web de données, plus encore que les autorités qui sont devenues des données; elles ont permis l'apparition d'un Web sémantique. Seulement, la problématique de l'alignement de référentiels entre eux sur le Web de données est toujours présente et ne sera jamais totalement résolue.
\section{\label{II-B-3}Les ontologies dans le Web sémantique}
\titreEntete{Les ontologies dans le Web sémantique}

%intro
La finalité principale des ontologies est leur exposition sur le Web de données. L'utilisation qui s'ensuit permet la création d'un Web sémantique, structuré avec des données partagées. Le référentiel compris dans le sens de \ac{kos} n'est plus aussi présent dans ce Web sémantique; le modèle de description de ces référentiels s'impose sous la forme des ontologies et améliore dans le même temps, par des relations typées, la description des concepts qu'il contient.\\

Cette avancée permet une description plus fine et partagée des documents des institutions patrimoniales: le référentiel est à la fois leurs propres données et celles du Web de données, liées par les ontologies publiques.

\subsection{\label{II-B-3-a}Décrire des ontologies en \ac{rdf}: \ac{rdfs} et \ac{owl}}
\titreEntete{Décrire des ontologies en RDF}

De même que \ac{skos} est une ontologie permettant la description de \ac{kos}, \index[ref]{relier@Relier!rdfs@RDFS}\ac{rdfs} et \index[ref]{relier@Relier!owl@OWL}\ac{owl} sont les représentations des ontologies \index[ref]{echanges@Échanges!formats@Formats!rdf@RDF}\ac{rdf}. Les documents décrits par les institutions le sont par des formats et des logiques différentes. Utiliser une seule ontologie peut s'avérer difficile; en effet, elle peut être trop large ou trop spécifique pour le domaine décrit. C'est pourquoi l'utilisation des URIs, des liens hypertexte du Web, permet d'utiliser autant d'ontologies que nécessaire, et de créer une interopérabilité par parcours de liens, par rebonds sur les URIs.\\

La constitution de ce réseau de liens, utilisé pour la description de documents, n'est possible qu'avec l'utilisation du \index[ref]{typologie@Typologie!graphe@Graphe de nœuds et de liens}Web sémantique et de \index[ref]{echanges@Échanges!formats@Formats!rdf@RDF}\ac{rdf}. C'est pourquoi il a été nécessaire de construire des modèles de représentation des ontologies en \ac{rdf}.\\

\ac{rdfs} est un langage de description simple, destiné à apporter les bases d'une description en \ac{rdf} avec la déclaration de classes --- et de sous-classes --- et de propriétés. Les classes sont des concepts, des types de ressources, identifiés par des URIs. Les propriétés sont les relations qui existent entre les classes. Ainsi, chaque ressource est instanciée à une classe par la propriété --- le prédicat \ac{rdf} --- <http://www.w3.org/1999/02/22-rdf-syntax-ns\#type> (rdf:type). La présence de sous-classes permet de créer des groupes au sein de classes: en déclarant une instance d'une sous-classe, un second triplet est alors implicitement créé entre la sous-classe et la classe avec le prédicat \textit{rdfs:subClassOf}. Tout ce qui s'applique à une classe l'est également pour une sous-classe.\\

\index[ref]{relier@Relier!rdfs@RDFS}\ac{rdfs} admet aussi la déclaration de domaines et de codomaines pour les propriétés. Ces propriétés peuvent être de type ressource quand elles relient deux ressources désignées par des URIs, ou bien de type donnée quand l'objet est un littéral. Ce comportement des propriétés est déclaré avec le domaine et le codomaine: le domaine de la propriété définit la type de la classe sujet, tandis que le codomaine définit la classe de la ressource objet --- ou du type de donnée si c'est un littéral.\\

Malgré les classes et sous-classes, tout comme les domaines et codomaines, \index[ref]{relier@Relier!rdfs@RDFS}\ac{rdfs} reste simple et ne permet pas de description complexe de relations. C'est pourquoi \index[ref]{relier@Relier!owl@OWL}\ac{owl} est une extension de \ac{rdfs}. Des contraintes sur les relations comme la symétrie, l'équivalence, la différence ou la contradiction peuvent être exprimées; de même, la déclaration d'une liste d'instances contrôlées peut être faite. Comme avec \index[ref]{relier@Relier!skos@SKOS}\ac{skos}, il est possible, et souvent indispensable, de déclarer des relations d'équivalences entre les classes, les propriétés ou les instances. Ainsi, une instance peut hériter des propriétés de la classe à laquelle sa classe est équivalente. Pour cela, trois propriétés \ac{owl} existent: 

\noindent<http://www.w3.org/2002/07/owl\#equivalentClass> (owl:equivalentClass), <http://www.w3.org/2002/07/owl\#equivalentProperty> (owl:equivalentProperty) et <http://www.w3.org/2002/07/owl\#sameAs> (owl:sameAs).

\subsection{\label{II-B-3-b}Utilisation des ontologies en institutions}
\titreEntete{Utilisation des ontologies en institutions}

Comme chaque \index[ref]{typologie@Typologie!ontologie@Ontologie}ontologie traite d'un domaine particulier de la connaissance ou du monde, les ontologies sont très nombreuses. Dans sa constante réflexion sur la description, l'indexation et le partage de ses données, le milieu bibliothéconomique s'est emparé du Web de données pour faciliter ses missions et la recherche. Ainsi, plusieurs ontologies sont essentielles dans ce milieu. La première, \index[ref]{relier@Relier!dcterms@DC Terms}DC Terms\footcite{noauthor_dublin_nodate}, et la plus ancienne car créée en 1995, permet la description bibliographique d'un document sur le Web avec quinze propriétés, accompagnées de propriétés affinées --- \textit{abstract} l'est de \textit{description}. L'ontologie Dublin Core est constamment utilisée au sein des bibliothèques\footnote{La \reference{onto_dcterms} montre la quantité d'ontologies l'utilisant.}, et plus généralement dans le Web de données, car elle offre les outils de base servant à la description d'un document.\\

Pour la description des autorités, l'ontologie \index[ref]{relier@Relier!foaf@FOAF}\ac{foaf}\footcite{noauthor_foaf_nodate} est disponible et est également fortement utilisée. Créée au milieu des années 2000, elle permet la description des agents, de groupes et d'organisations, de personnes, mais aussi de réseaux sociaux.\\

\begin{figure}[!h]
	\centering
	\includegraphics[width=13cm]{images/onto_dcterms.png}
	\caption[L'ontologie DC Terms]{DC Terms: représentation des ontologies l'utilisant (à gauche) et de celles qu'elle utilise(à droite) [Source: \url{https://lov.linkeddata.es/dataset/lov/vocabs/dcterms}]}
	\label{onto_dcterms}
\end{figure}
\medskip

Les \index[ref]{typologie@Typologie!ontologie@Ontologie}ontologies propres aux institutions patrimoniales utilisent ces ontologies de haut niveau. Ainsi, \index[ref]{modelisation@Modélisation!bibo@bibo}\ac{bibo} utilise à la fois les \index[ref]{relier@Relier!dcterms@DC Terms}DC Terms et \index[ref]{relier@Relier!foaf@FOAF}\ac{foaf}. \index[ref]{modelisation@Modélisation!cidoc@CIDOC-CRM}\ac{cidoccrm}\footnote{Voir \reference{annexe_onto} (\reference{onto_c4o})}, contrairement aux autres ontologies institutionnelles, n'est pas de bas niveau, mais de haut niveau. En effet, elle souhaite pouvoir décrire n'importe quel type d'objet: elle dispose de quatre-vingt-cinq classes et de plus de deux cent cinquante propriétés.

%conclu
\bigskip
\bigskip
\bigskip
Les \ac{frbr} et le modèle de données de la BnF permettent de conclure sur l'importance des ontologies dans le Web de données pour les institutions. En effet, son modèle de données étant basé sur les \ac{frbr}, tous types de relations sont nécessaires pour relier l'œuvre aux points d'accès --- autorités, sujets, dates, \dots~ Ces relations sont exprimées par des ontologies publiques: l'exposition \ac{rdf} des données permet alors l'utilisation des URIs des ontologies et une description fine du document décrit et de son contexte. Treize ontologies sont ainsi utilisées\footnote{Voir \reference{mdd_bnf}.}, non complètement, mais partiellement, uniquement pour les propriétés nécessaires à la BnF:
\begin{figure}[!h]
	\centering
	\includegraphics[width=15cm]{images/mdd_bnf.PNG}
	\caption[Le modèle de données de la BnF]{Le modèle de données de la BnF [Source: \url{https://data.bnf.fr/images/modele_donnees_2018_02.PNG}]}
	\label{mdd_bnf}
\end{figure}
\begin{itemize}
	\item \ac{foaf} pour les autorités;
	\item \ac{rdf} pour exprimer les instances --- avec rdf:type ---;
	\item \ac{rdfs}
	\item \ac{skos} pour créer le concept préférentiel et définir les sujets proches 
	\item DC Terms pour les descriptions bibliographiques simples
	\item Geo pour les descriptions géographiques
	\item FRBR-RDA, RDAgroup2elements et RDArelationships pour exprimer les entités du modèle \ac{frbr}
	\item des ontologies internes, BNF-onto et BNFroles
	\item l'extension \ac{rdf} \ac{owl}-time
	\item enfin une ontologie du langage de catalogage \ac{marc} MARCrel
\end{itemize}

L'exemple de la BnF montre qu'il n'y a plus de référentiel unique propre à une institution, mais seulement un ensemble de données dispersées dans le Web de données --- utilisation de \ac{rameau} notamment --- avec des ontologies qui permettent la production de sens sur les liens entre les ressources.
	\chapter{\label{II-C}Relier ses données à Wikidata}
\titreEntete{Relier ses données à Wikidata}
	
	%conclu
	%schéma global de la place du référentiel
	\newpage
	Cette seconde partie a mis en évidence une importante évolution dans la place des référentiels et les usages qui en sont faits. Ils ne se trouvent plus en marge des systèmes documentaires: les efforts donnés à les partager pour les réutiliser montre un glissement vers le centre de ces systèmes, sans toutefois l'atteindre. Le lien, plus que la donnée elle-même, est devenu un enjeu pour toutes les institutions: un enrichissement de ses propres données est possible. Ces partages constants ont montré l'importance des formats et des protocoles d'échanges communs et utilisés par tous.\\
	
	Deux cas de figure généraux de la place des référentiels dans les institutions et des liens qu'ils entretiennent entre eux sont apparus: d'une part il peut y avoir une simple réutilisation des données d'une institution dans une autre (\reference{schema_general_relier1}); d'autre part, le Web sémantique a permis la création de nouveaux référentiels avec lesquels il est possible de créer du lien (\reference{schema_general_relier2}).\\
	\begin{figure}[!h]
	\centering
	
	\begin{pspicture}(0,0)(14,7)
		\uput[0](0.8,6.4){Entrepôt de documents A}
		\uput[0](7.8,6.4){Entrepôt de documents B}
		%cercles globaux
		\pscircle(3,3){3}
		\pscircle(10,3){3}
		%cercles du A
		\pscircle(4.8,3.4){0.6}
		\pscircle(3,4.2){0.6}
		\pscircle(1.5,3.2){0.6}
		\pscircle(2.2,1.6){0.6}
		\pscircle(3.8,1.4){0.6}
		%cercles du B
		\pscircle(11.4,3.8){0.6}
		\pscircle(9.2,1.6){0.6}
		%labels du A
		\uput[0](2.6,4.2){D1}
		\uput[0](3.4,1.4){D2}
		\uput[0](1,3.2){D3}
		\uput[0](4.4,3.4){R1}
		\uput[0](1.7,1.6){R2}
		%labels du B
		\uput[0](8.8,1.6){D1}
		\uput[0](10.9,3.8){D2}
		%lignes du A
		\psline(4,2)(4.6,2.8)
		\psline(4.2,3.8)(3.6,4)
		\psline(3,3.6)(2.4,2.1)
		\psline(1.6,2.6)(1.85,2.25)
		%lignes du B
		\psline(8.8,2.1)(5.2,3)
		\psline(10.8,3.8)(5.4,3.4)
	\end{pspicture}
	
	\caption[Réutilisation d'un référentiel entre deux institutions]{Réutilisation d'un référentiel entre deux institutions (R: Référentiel; D: Document)}
	\label{schema_general_relier1}
\end{figure}
	%schémas
	
	Cependant, l'unique utilisation d'un seul référentiel ne permet pas à une institution de combler l'ensemble de ses besoins et de ses usages: plusieurs liens doivent alors être établis avec divers jeux de données, ce qui peut être réalisé par les ontologies et le passage de lien en lien. Bien que le lien soit devenu un élément structurant du Web sémantique, il devient d'autant plus précieux quand il se trouve lui-même objet d'une grande entité\footnote{Comme il a été montré sans autre précision avec \ac{viaf} dans le \reference{II-A}, et ce qui fera la suite de notre propos en \reference{centraliser}.}.
	\begin{figure}[h!]
	\centering
	
	\begin{pspicture}(0,0)(15.2,7)
		\uput[0](0.8,6.4){Entrepôt de documents A}
		\uput[0](9.8,6.4){Entrepôt de documents B}
		%cercles globaux
		\pscircle(3,3){3}
		\pscircle(12,3){3}
		%cercles du A
		\pscircle(3,4.6){0.6}
		\pscircle(1.5,3.6){0.6}
		\pscircle(2.2,1.6){0.6}
		\pscircle(4.2,3.1){0.6}
		%cercles du B
		\pscircle(13.4,3.8){0.6}
		\pscircle(11.2,1.6){0.6}
		%labels du A
		\uput[0](2.6,4.6){D1}
		\uput[0](3.8,2.7){D2}
		\uput[0](1,3.6){R2}
		\uput[0](1.7,1.6){D3}
		%labels du B
		\uput[0](10.8,1.6){D1}
		\uput[0](12.9,4.2){D2}
		
		%lignes du A
		\psline(1.9,4.1)(2.4,4.4)
		\psline(1.5,3)(1.95,2.1)
		\psline(2.8,1.6)(6.95,2.75)
		\psline(6.9,3)(4.8,3.1)
		\psline(3.6,4.7)(6.95,3.3)
		%lignes du B
		\psline(10.7,2)(8.05,2.7)
		\psline(12.8,3.8)(8.05,3.2)
		
		%ref web sem
		\pscircle(7.5,3){0.6}
		\uput[0](7.1,3){R}
	\end{pspicture}
	
	\caption[Utilisation commune d'un jeu de données du Web de données]{Utilisation commune d'un jeu de données du Web de données (R: Référentiel; D: Document)}
	\label{schema_general_relier2}
\end{figure}
	
	\part{\label{centraliser}CENTRALISER. Le référentiel, clé de voûte et pivot (depuis le milieu des années 2010)}	
	
	%intro
	Le Web de données a initié une pratique nouvelle dans la conception des modèles de données. Le lien devient une ressource indispensable qui permet aux institutions et aux systèmes documentaires de déstructurer leurs jeux de données afin de les enrichir, de les partager, de mieux les valoriser. Alors, la création ou l'utilisation d'un référentiel devient quasiment inutile --- sauf en cas de besoins très spécifiques comme c'est le cas à l'\ac{ina}. Le parcours des liens, au travers un graphe, permet bien plus que ce que permet un référentiel interne stocké dans une base de données relationnelles.\\
	
	Le modèle de données comme réseau de liens et de nœuds n'est pas une notion récente, mais elle a trouvé son application et son utilité avec l'apparition de l'informatique et du Web. Ce réseau de nœuds et de liens est alors, depuis la seconde moitié des années 2010, de plus en plus adopté lors des refontes de systèmes d'information, afin de s'adapter aux nouvelles pratiques des utilisateurs sur le Web et aux nouveaux besoins qui en découlent. L'application de ce modèle-réseau en interne créé alors du \ac{led}, l'équivalent du \ac{lod} pour le Web de données.\\
	
	Le \ldd mis en œuvre depuis 2015 à l'\ac{ina} est un \ac{led} qui reprend les principes de déconstruction du document et de l'information au profit de la création de grandes entités et de multiples liens, ce qui permet une grande adaptation aux besoins actuels et futurs auxquels les données doivent ou devront répondre. Les systèmes documentaires ne sont ainsi plus pensés à partir des besoins, mais des données. Le référentiel dans le \ldd prend une place centrale puisqu'il permet la description de l'ensemble des instances et est essentiel aux nouveaux besoins relatifs à l'intelligence artificielle. Cependant, la fusion de multiples référentiels au sein d'un système uniformisé et centralisé est une tâche complexe qui doit éviter les doublons et effacer les différences de structure et de graphie qui existaient auparavant.
	
	\chapter{\label{III-A}Les labyrinthes comme réseaux de données et de liens}
\titreEntete{Les labyrinthes comme réseaux de données et de liens}

%intro
\lettrine{L}a multiplication des liens et de ceux possibles dans le Web de données entraîne, pour un humain, une désorganisation des informations et des référentiels dans ce Web de données. Les chemins à emprunter deviennent multiples et provoquent une ivresse de rebonds et d'informations chez l'utilisateur. Les interfaces de visualisation structurent l'ensemble des informations et des liens du Web de données, qui est devenu un Web où seule une machine peut se repérer rapidement et naviguer aisément. À partir du modèle du graphe d'un jeu de données, le Web de données a permis de créer un graphe à l'échelle du Web, accessible à tous et en tous points, depuis n'importe lequel des jeux de données, des référentiels ou des institutions.\\

La notion de graphe, de réseau de données, découle des nombreuses théories d'arbres de classifications et de descriptions des précédents millénaires. La constatation des limites et de l'échec de ces arbres a conduit à la théorisation, puis l'adoption dans le Web de données et par le milieu bibliothéconomique d'abord, du labyrinthe et du modèle-réseau de données. Le lien devenant l'essence-même de ces réseaux de données, de nouveaux types de référentiels ont vu le jour, notamment les \textit{hubs} de liens qui centralisent les liens et quelques données d'autorités autour d'un même identifiant. Wikidata, d'abord réceptacle structuré des données et des informations des Wikipédias, devient rapidement le hub de liens et d'identifiants le plus utilisé.

\section{\label{III-A-1}Du modèle encyclopédique aux graphes de données}
\titreEntete{Du modèle encyclopédique aux graphes de données}

%intro
Au Moyen-Âge, la dogmatique de l'arbre porphyrien domine. Ce n'est qu'à la Renaissance que le savoir est conçu comme ouvert. L'arbre était pensé selon le monde, pensé lui-même comme un cosmos clos et ordonné; ce même arbre était par ailleurs pensé comme une finitude inaltérable de sphères. Cependant, la pensée de Copernic influe la façon de concevoir le savoir: ce dernier s'efforce de mimer le système planétaire avec ses perspectives variables, des orbites qui deviennent des ellipses, \dots~\\

L'encyclopédie n'est alors plus qu'un amas de connaissances réelles et légendaires; elle devient un index devant décrire le monde et les connaissances, le classifier. La tension pesant sur ce modèle encyclopédique et la quantité infinie de connaissances conduisent à son éclatement au profit d'une forêt où tout est ou peut être relié selon les choix du lecteur.

\subsection{\label{III-A-1-a}Vers les labyrinthes (Renaissance)}
\titreEntete{Vers les labyrinthes}

L'évolution majeure de la Renaissance, faisant suite aux arbres porphyriens puis lulliens, est la nouvelle conception de la structure des éléments du monde: avec Porphyre et ses successeurs, seuls les accidents et les substances sont classifiés; avec la Renaissance, de multiples \index[ref]{typologie@Typologie!index@Index}index d'encyclopédies naissent, accompagnés de réflexions sur les manières d'ordonner le savoir\footnote{\og Nous n'avons plus affaire à une classification de substances et d'accidents, mais à l'index d'une encyclopédie possible et à la tentative de proposer un ordonnancement du savoir\fg{} in \cite{eco_arbre_2010}}. Toute la Renaissance va se concentrer sur cette classification du savoir.\\

La première grande encyclopédie tentant cette classification du savoir est l'\textit{Encyclopaedia septem tomis distincta} de \nP{Johann}{Alsted} en 1620\footcite{alsted_encyclopaedia_1630}: l'index devient la substance-même de cette œuvre. Cette encyclopédie s'inscrit dans la période pansophique de la Renaissance, dans laquelle la réflexion sur une sapience universelle, qui aurait toute l'étendue du savoir, est vive. Si l'arbre de Porphyre voulait simplement être un dictionnaire, un moyen de définir la science, l'index pansophique inspire quant à lui à classifier cette science, et s'éloigne donc du dictionnaire. Cette période pansophique marque bien l'arrêt de la conception hiérarchique du savoir, conçue comme moyen de définir: un autre moyen de décrire ce savoir est possible, en étant plus efficace.\\

Ce nouveau moyen part de la constatation que de multiples chemins peuvent mener à un même savoir. \nP{Francis}{Bacon} le constate, dès 1620, dans l'\textit{Instauratio Magna}\footcite{bacon_instauratio_1620} puis en 1626 dans le \textit{Sylva sylvarum}. Il n'est alors plus question d'arbre unique, mais d'arbres multiples, de labyrinthes avec des chemins ambigus, des ressemblances trompeuses, des spirales et des noeuds complexes\footnote{\og \textit{obliquae et implexae naturarum spirae et nodi}\fg{} in \cite{bacon_instauratio_1620}}. La forêt est un amas de sujets, on n'y trouve plus mais on découvre de nouvelles relations, ce que l'on ne connaissait pas encore et ce que l'on ne cherchait pas. Cependant, la \og tension entre l'arbre et le labyrinthe\fg{}\footcite{eco_arbre_2010} ne faiblit pas: \nP{John}{Wilkins}, à la fin des années 1660, est mis en échec devant ses classifications du savoir qui ne parviennent pas à classer les sujets; une table immense d'index est alors créée pour résoudre cette difficulté.\\

La masse des connaissances à classer étant immense, l'encyclopédie devient un inventaire général des connaissances, incapable de toutes les saisir: \nP{Gottfried Wilhelm}{Leibniz} comprend bien que l'entreprise encyclopédique peut être infinie en raison du nombre de renvois à créer pour s'adapter aux perspectives infinies d'accès à une connaissance. Le labyrinthe prend alors tout son sens: une connaissance est accessible par de multiples points d'accès et ne fait pas partie d'une hiérarchie stricte. Dans le \og Discours préliminaire\fg{} de l'Encyclopédie\footcite{diderot_encyclope_1751}, l'arbre porphyrien et la pensée artistotélicienne sont totalement remis en question. 
\begin{citationLongue}
	Le système général des sciences et des arts est une espèce de labyrinthe, de chemin tortueux, où l'esprit s'engage sans trop connaître la route qu'il doit tenir.\footcite[Discours préliminaire]{diderot_encyclope_1751}
\end{citationLongue}
L'encyclopédie totale et universelle ne pourra jamais voir le jour en raison de son ampleur; elle n'est qu'utopie de la connaissance. Cette utopie, toujours visible aujourd'hui --- les projets \index[ref]{lod@Linked Open Data (LOD)!wikidata@Wikidata}Wikidata ou Wikipédia se veulent le reflet de notre monde ---, ne cessera pas en raison du caractère culturel qu'elle contient: plus que le reflet de notre monde, elle est le reflet de notre culture et de nos cultures spécifiques. L'encyclopédie sert cependant à créer des portions d'encyclopédie, en vue de réaliser des classifications spécifiques à un domaine.

\subsection{\label{III-A-1-b}Des labyrinthes aux graphes}
\titreEntete{Des labyrinthes aux graphes}

L'arrivée de la pensée classificatoire au niveau du labyrinthe a eu lieu à plusieurs reprises, à chaque échec du modèle de l'arbre, notamment avec Porphyre où chaque pas dans l'arbre régénérait sans cesse l'arbre des différences: il n'y a, par conséquent, pas un, mais un nombre infini d'arbres selon leurs contextes. Il existe plusieurs types de labyrinthes dont les trois principaux sont présentés par \nP{Umberto}{Eco}:
\begin{itemize}
	\item Le plus ancien est un labyrinthe classique unicursal, dit de Knossos (\reference{annexe_laby} (\reference{laby_knossos})): la seule possibilité est d'atteindre son centre; en raison de cette caractéristique, il ne peut pas se rapporter à un modèle encyclopédique, ni à un modèle de description des connaissances. En effet, le savoir n'est pas un long couloir dans lequel on accède toujours à la même connaissance.
	\item Le labyrinthe maniériste d'Irrweg permet des choix alternatifs (\reference{annexe_laby} (\reference{laby_irrweg})): toutes les routes mènent à des points morts, sauf un qui est la sortie. Ce labyrinthe est un arbre de décisions, dans lequel les branches sont la représentation des décisions possibles.
	\item Enfin, le labyrinthe qui donne naissance aux réseaux et aux \index[ref]{typologie@Typologie!graphe@Graphe de nœuds et de liens}graphes est le \og labyrinthe réseau\fg{} de \nP{Umberto}{Eco} (\reference{annexe_laby} (\reference{laby_reseau})). Chaque point du labyrinthe peut être connecté à n'importe quel autre point. Cette structure a l'avantage d'être extensible à l'infini, il permet des connexions infinies et des corrections locales qui ne modifient pas le reste du labyrinthe. Évolutif, ce labyrinthe nécessite de l'utilisateur qu'il modifie en permanence l'image qu'il s'en fait: \og Un réseau est un arbre auquel il faut ajouter des couloirs infinis connectant ses noeuds\fg{}\footcite{eco_arbre_2010}.
\end{itemize}
\medskip
Modèle dans lequel les connexions, les liens, sont essentiels, le labyrinthe réseau permet une représentation multidimensionnelle des connaissances et un accès à un point précis par de multiples liens. En 1968, \nP{Ross}{Quillian}\footcite[p.227-270]{minsky_semantic_1968} fait apparaître le réseau sémantique structuré, conçu comme un réseau de nœuds interconnectés\footnote{\og \textit{The memory model consists basically of a mass of nodes interconnected by different kinds of associative links}\fg{} in \cite[p.234]{minsky_semantic_1968}}. Le modèle qu'il décrit part d'un terme souche qui est défini par une série de nœuds, des tokens: ce n'est pour l'instant qu'un arbre. Seulement, les tokens peuvent à leur tour devenir des souches et porter des relations d'association: le réseau est ainsi constamment remodelé et modifié\footnote{\og \textit{Token nodes make it possible for a word's meaning to be built up from other word meanings as ingredients and at the same time to modify and recombine these ingredients into a new configuration}\fg{} in \cite[p.234]{minsky_semantic_1968}}. Ce modèle en réseau permet alors la définition de chaque terme, par ses connexions, avec tous les autres termes; il devient infini et multidimensionnel, non représentable en entier sur un plan bidimensionnel: la complexité du modèle-réseau peut être uniquement traité et compris par une machine. 

%conclu
\bigskip
\bigskip
L'apparition du modèle-réseau, du labyrinthe-réseau, a montré l'importance du lien dès la seconde moitié du \textsc{XX}\textsuperscript{ème}siècle. L'informatique a permis de mettre fin aux structures de connaissances qui étaient concevables par un esprit humain, afin de laisser la machine représenter les données dans toute leur complexité. Ce modèle, beaucoup plus efficace et riche que les arbres, consacre la valeur du lien, qui devient lui-même plus important que la donnée: il définit et légitime la donnée.
\section{\label{III-A-2}Des labyrinthes de relations et d'identifiants: les hubs de liens}
\titreEntete{Les hubs de liens}

%intro
La modélisation des données sous la forme de labyrinthes --- ou de graphes --- a un impact  considérable dans le Web de données: certains jeux de données sont eux-mêmes stockés dans une base de données graphe --- comme Wikidata qui fonctionne sur la base de données Blazegraph---; ou bien les jeux de données peuvent entre eux former un gigantesque graphe, infini. Cette seconde conception du modèle-réseau de \nP{Umberto}{Eco} est au centre du Web de données. Avec la décentralisation des référentiels sur le Web et leur éclatement en de multiples données, il est apparu comme nécessaire de les recentraliser au travers de nouveaux référentiels fournisseurs d'un unique identifiant.\\

Cependant, la recentralisation passe également par l'ajout de données parallèlement à l'ajout des liens. En effet, nous l'avons montré au \reference{II-C}, les données peuvent varier dans leur graphie et leur forme selon le référentiel duquel elles sont issues. Afin d'offrir des données utilisables par tous, certaines plateformes ajoutent, pour chacun de leurs identifiants, des données préférentielles aux côtés des liens pointant vers d'autres référentiels ou jeux de données.

\subsection{\label{III-A-2-a}De la décentralisation des référentiels à leur recentralisation dans le Web de données}
\titreEntete{De la décentralisation des référentiels à leur recentralisation}

La multiplication du nombre de référentiels dans le Web de données conduit à une profusion de données et à leur répétition, sans que soient repérées les données se rapportant à un même concept\footnote{La constellation du Linked Open Data montre cette augmentation croissante du nombre de référentiels dans le Web de données, et l'absence, pour certains, de liens vers d'autres référentiels. Voir \reference{annexe_lod} (\reference{lod_cloud}).}. 

\subsection{\label{III-A-2-b}Apparition des hubs de liens et d'identifiants}
\titreEntete{Apparition des hubs de liens et d'identifiants}

\subsection{\label{III-A-2-c}Les hubs de liens et d'identifiants réceptacles de données}
\titreEntete{Les hubs de liens et d'identifiants réceptacles de données}

%conlu
\section{\label{III-A-3}Wikidata comme hub de liens: aligner les fictions et les séries de l'INA avec Wikidata}
\titreEntete{Wikidata comme hub de liens}

%intro

%conlu

%conclu
\bigskip
\bigskip
\bigskip
Le succès des modèles de données en graphe est incontestable et est repris dans tous les projets d'envergure internationale: nous avons déjà évoqué Wikidata, le projet European Holocaust Research Infrastructure (EHRI)\footnote{Site du projet: \cite{noauthor_european_nodate}\\Utilisation du graphe de données: \cite{blanke_developing_2015}} peut également être cité comme acteur institutionnel se dégageant des bases de données relationnelles et des traditionnels référentiels hiérarchiques et contrôlés. Cependant, ce modèle de graphe nécessite un langage de requête efficace et des rendements élevés. Or, il a été constaté, avec Wikidata et le \ac{sparql}-EndPoint, des lenteurs de retours de résultats, obligeant à utiliser d'autres moyens pour obtenir les entités de Wikidata avec leurs déclarations.\\

La recentralisation des données autour de quelques acteurs du Web de données est une conséquence inattendue de la décentralisation de ces mêmes données qui avait eu lieu quelques années auparavant, avec la publication de jeux de données et de référentiels sur le Web sémantique. Cette recentralisation est née d'un besoin d'obtenir en un même endroit une multiplicité de liens, d'identifiants et d'informations, sans avoir à connaître les institutions ou les jeux de données dans lesquels aller chercher les données nécessaires. Si cette recentralisation concerne ici le Web de données, elle peut également concerner les institutions directement, ainsi que les entreprises: on ne parle alors plus de Linked Open Data, mais de Linked Enterprise Data (LED).
	\chapter{\label{III-B}Le Lac de données de l’INA : le référentiel au centre du modèle}
\titreEntete{Le référentiel au centre du modèle}

%intro

%conclu
	\chapter{\label{III-C}Centraliser les référentiels de l’INA dans le \ldd: l'exemple de l'alignement de deux référentiels de personnes physiques}
\titreEntete{Aligner deux référentiels de personnes physiques}

%intro
La \ac{ddcol} possède un unique référentiel de personnes physiques comme nous avons pu le constater plus tôt\footnote{Voir \reference{I-C-3} et \reference{II-C--3}.}. Le \ldd étant un \ac{led}, il tire son intérêt de la mise en commun de différentes bases de données et jeux de données. Ainsi, c'est l'ensemble des données et des métadonnées de l'\ac{ina} qui entrent dans ce \textit{Lac}: la base de données de la \ac{dj} fait alors l'objet de ce processus de migration vers le \ldd. Cependant, bien que la \ac{dj} représente un silo de données distinct de celui de la \ac{ddcol}, ils partagent tout de même certaines caractéristiques comme l'utilisation massive de personnes physiques, qui sont réunies dans un référentiel dans chaque service.\\

La migration des données de la \ac{ddcol} s'achève en fin d'année 2020 et laisse la place à celle de la \ac{dj}: afin d'éviter toute redondance de concepts dans le \ldd, il convient de rechercher pour une personne physique de la \ac{dj} son équivalent dans le \ldd --- par conséquent dans l'ancien référentiel des personnes physiques et morales de la \ac{ddcol} qui a été déconstruit sous forme de concepts\footnote{Voir \reference{III-B}.}. L'exécution de cet alignement montre à lui seul les problématiques liées au langage naturel, ainsi que la nécessaire présence humaine qui doit superviser les résultats issus d'une automatisation de tâches.

\section{\label{III-C-1}Des jeux de données différents en de multiples points}
\titreEntete{Des jeux de données différents}

%intro
L'alignement des personnes physiques de la \ac{ddcol} avec Wikidata avait déjà démontré l'importance de la donnée structurée afin de créer des points de contacts entre les deux jeux de données et de procéder à leur alignement. Les problématiques liées à la graphie et aux différences de structure ont aussi compliqué cet alignement. Dans le cadre de la mise en relation entre le référentiel des personnes physiques de la \ac{dj} avec celui de la \ac{ddcol}, ces points de contact sont réduits au minimum et peuvent interroger quant à la possibilité de réaliser des alignements sûrs, ou les plus sûrs possibles.

\subsection{\label{III-C-1-a}Enjeux}
\titreEntete{Enjeux}

Le \ldd n'étant pas conçu à partir des\index[ref]{led@Linked Enterprise Data (LED)!ldd@Lac de données (INA)}\index[ref]{modelisation@Modélisation!ldd@Lac de données (INA)} besoins métier mais des données, le référentiel des personnes physiques de la \ac{dj}, se présentant sous la forme d'une table \textit{PERSONNE} de la base de données, ne peut pas être conservé dans sa structure actuelle. En effet, il est uniquement adapté aux besoins de la \ac{dj} et ne correspond pas aux usages que pourrait en faire la \ac{ddcol} ou l'utilisateur final des applications de l'\ac{ina}. Afin d'intégrer ce référentiel dans le \textit{Lac}, il est nécessaire de l'aligner avec les concepts existants, issus du référentiel des personnes physiques de la \ac{ddcol}. Ainsi, un double enrichissement a lieu, celui des concepts par les données de la \ac{dj}, et celui de la \ac{dj} par les données des concepts. Cependant, cet enrichissement devient invisible dans le \ldd puisqu'il n'y a plus de notion de référentiel, et que les distinctions entre \ac{dj} et \ac{ddcol} sont volontairement effacées au profit d'une structure de données plus souple.\\

Cet alignement a pour finalité l'ajout d'un lien \textit{provenance--\ac{dj}} à un concept quand le matricule de la \ac{dj} et le concept sont identiques, ou bien la détection des nombreuses personnes de la \ac{dj} qui ne sont pas des concepts. En effet, la \ac{dj} ayant pour fonction de repérer les ayants-droits et ouvrants-droits de personnes liées à un extrait audiovisuel, ceux-ci ne sont, par conséquent, pas spécifiquement dans la description des documents audiovisuels réalisée par la \ac{ddcol} car ils n'interviennent à aucun moment dans ces documents. Ainsi, nombreuses sont les personnes de la \ac{dj} qui n'ont pas d'équivalent dans la \ac{ddcol} et qu'il est nécessaire de repérer afin de leur créer \textit{in fine} un concept dans le \index[ref]{led@Linked Enterprise Data (LED)!ldd@Lac de données (INA)}\index[ref]{modelisation@Modélisation!ldd@Lac de données (INA)}\ldd.\\

Cette différence entre les jeux de données montre une nouvelle fois comment se sont formées les bases de données --- selon les usages et les besoins --- qui sont à migrer dans le \index[ref]{led@Linked Enterprise Data (LED)!ldd@Lac de données (INA)}\index[ref]{modelisation@Modélisation!ldd@Lac de données (INA)}\ldd: à cause de cette différence, il devient difficile d'estimer l'efficacité et le rendement du processus d'alignement qui va être réalisé. En effet, connaître la raison de l'absence d'alignement de certains matricules des personnes de la \ac{dj} sera uniquement possible par une action humaine. Face à ces enjeux et aux problématiques soulevées par le seul historique des bases de données, l'alignement des deux référentiels comporte plusieurs difficultés supplémentaires déjà évoquées dans les chapitres précédents.

\subsection{\label{III-C-1-b}Points de contact}
\titreEntete{Points de contact}

Trouver des points de contact entre deux jeux de données, plus encore entre deux référentiels, est essentiel lors d'un alignement: plus ces points de contact sont nombreux, plus les comparaisons sont nombreuses et les alignements sûrs. Cependant, les référentiels de la \ac{ddcol} et la \ac{dj} n'en partagent que peu --- sept. De plus, ces points de contact nécessitent la présence de l'information de chaque côté, ce qui est peu la cas entre la \ac{dj} et la \ac{ddcol}.\\

Dans le \index[ref]{led@Linked Enterprise Data (LED)!ldd@Lac de données (INA)}\index[ref]{modelisation@Modélisation!ldd@Lac de données (INA)}\ldd, les concepts disposent notamment d'attributs indiquant le nom, le sexe, les dates de naissance et de décès, ainsi que la note qualité. Cette note qualité n'étant pas scindée dans le \ldd, il est nécessaire, dans cet alignement, d'en extraire la fonction, ou les fonctions, de la personne, en supprimant la mention des pays d'exercice.\\

Si la \ac{dj} dispose de nombreuses données personnelles pour mener à bien ses missions, les données permettant un alignement documentaire sont plus restreintes: hormis le nom et le sexe, seule une date de décès est disponible, ainsi qu'une contribution. En effet, seule la date de décès intéresse le service juridique pour ses applications dans le droit et le reversement des droits aux ayants-droits ou ouvrants-droits: conserver une date de naissance n'a par conséquent aucun usage dans la \ac{dj}.

\noindent Le référentiel des personnes de la \ac{dj} présente une petite normalisation avec les contributions: celles-ci ne sont pas du texte libre, mais du texte contrôlé et choisi parmi une liste d'une vingtaine d'entrées. Ce contrôle du vocabulaire permet, dans l'alignement, des rendements meilleurs après, nous le verrons, un traitement préalable des données des notes qualité de la \ac{ddcol}.\\

Cependant, les points de contact identifiés pour la \ac{dj} et la \ac{ddcol} sont peu nombreux, ce qui complique la détection des homonymes et diminue la fiabilité des alignements futurs.

\subsection{\label{III-C-1-c}Divergences}
\titreEntete{Divergences}

En plus de ces difficultés sur la quantité des points de contact, les deux référentiels diffèrent par leurs structures et leurs graphies. Tout d'abord, les niveaux de description des mêmes attributs sont différents. En effet, alors que le nom du concept du \index[ref]{led@Linked Enterprise Data (LED)!ldd@Lac de données (INA)}\index[ref]{modelisation@Modélisation!ldd@Lac de données (INA)}\ldd est composé de la forme \og \textit{Nom, Prénom}\fg{}, l'état-civil stocké à la \ac{dj} est divisé en deux attributs: un nom et un prénom. Ainsi, avant d'effectuer l'alignement, il est nécessaire de scinder le nom du concept afin de récupérer le nom et le prénom séparément.

\noindent Le jeu de données de la \ac{dj} offre également deux autres attributs, les pseudos de nom et de prénom de chaque personne. Afin d'utiliser ces deux attributs supplémentaires, il est nécessaire de leur trouver un point de contact dans les concepts issus de la \ac{ddcol}: ainsi, il a été considéré qu'un nom de concept ne possédant pas de virgule est un pseudo. Par conséquent, ce pseudo, issu des concepts, peut être comparé avec le pseudo du nom de la \ac{dj}\footnote{Dans les données de la \ac{dj}, c'est le pseudo-nom qui comporte le pseudonyme courant d'une personne; l'attribut pseudo prénom n'est utilisé que pour indiquer une variante du prénom de cette personne.}.\\

En plus de ces différences de niveaux de description, les graphies ne sont pas les mêmes. D'abord, les données de la \ac{dj} sont en majuscules, alors que celles de la \ac{ddcol} sont en minuscules. Si cette difficulté n'est pas majeure, elle nécessite tout de même un traitement dans l'ETL avant de pouvoir procéder à un alignement. De même, afin d'éviter toute variation dans des chaînes de caractères renvoyant à une même personne mais aux graphies différentes, les accents et la ponctuation sont retirés. Les dates, commee lors des alignements décrits dans les chapitres précédents, sont réduites à la seule année.

\noindent La difficulté majeure, posée dans l'alignement de deux référentiels de personnes, est la graphie et l'utilisation des particules des noms. En effet, l'utilisation des particules n'est pas normalisée dans l'Institut, ce qui conduit à la présence de \nP{Louis de }{Funès} dans la \ac{ddcol}, alors que la \ac{dj} conserve la forme \nP{Louis}{de Funès}. Les pratiques d'écriture des noms à particules étant constantes à la \ac{ddcol}, il est possible de transférer cette particule\footnote{Cette particule n'est pas exclusivement \textit{de}, elle peut être de l'une des formes suivantes: \textit{de, des, du, de la}} dans le nom afin d'obtenir  \nP{Louis}{de Funès} dans chaque jeu de données.\\

Enfin, afin de donner aux alignements une plus grande fiabilité, il est essentiel de prendre en compte le texte des notes qualité pour le comparer avec les contributions de la \ac{dj}. Les notes qualité de la \ac{ddcol} comportant plus de vingt mille fonctions différentes, il n'est pas possible de les faire correspondre chacune avec l'une des contributions de la \ac{dj}. Pour cela, seules les contributions et les fonctions des notes qualité les plus courantes ont fait l'objet d'un alignement manuel pour faciliter l'alignement automatique qui va suivre; cinq de ces contributions ont ainsi été pu être traitées:
\begin{itemize}
	\item le terme \textit{Journalisme} de la \ac{dj} est remplacé par \og journaliste\fg{};
	\item \textit{Artiste interprètre} est remplacé par \og chanteu\fg{}\footnote{Les terminaisons sont enlevées dans ces termes de remplacement afin de prendre en compte les variantes de graphie liées au pluriel et au féminin que l'on peut trouver dans les fonctions de la \ac{ddcol}.}
	\item \og realisat\fg remplace \textit{Réalisation}
	\item \og composit\fg remplace \textit{Composition musicale}
	\item enfin, \textit{Réalisation associée} est remplacé par \og realisat\fg{}
\end{itemize}
\medskip
Les chanteurs, les compositeurs, les réalisateurs et les journalistes étant les fonctions les plus courantes dans les deux jeux de données, elles ont été repérées puis traitées. Cependant, une majorité de fonctions ne pourra pas être alignée avec les contributions de la \ac{dj}, et par conséquent limitera la confiance accordée aux alignements.


%conlu
\bigskip
\bigskip
La centralisation de référentiels et de données est nécessaire pour les systèmes documentaires, mais la reprise de ces référentiels et de ces données peut être compliquée par les structures et les normes divergentes selon les jeux de données. Cette absence d'uniformisation annonce déjà des résultats faibles et limités dans la confiance qui peut être accordée. Dans le cas de l'alignement des référentiels de personnes physiques de la \ac{dj} et de la \ac{ddcol} à l'\ac{ina}, les points de contacts sont peu nombreux et peu spécifiques\footnote{Voir \reference{table_contact_dj_ddcol}.}.

\begin{table}[!h]
	\centering
	\begin{tabular}{|c|c|}
		\hline
		\textbf{\ac{dj}} & \textbf{\ac{ddcol}}\\ \hline
		Nom&Nom\\ \hline
		Prénom&Prénom\\ \hline
		Pseudo nom&Pseudo\\ \hline
		Pseudo prénom&\\ \hline
		Sexe&Sexe\\ \hline
		Date de naissance&Date de naissance\\ \hline
		Contribution&Fonction\\ \hline
	\end{tabular}
	\caption[Points de contact entre les référentiels de la \ac{dj} et de la \ac{ddcol}]{Points de contact entre les référentiels de la \ac{dj} et de la \ac{ddcol}}
	\label{table_contact_dj_ddcol}
\end{table}

\section{\label{III-C-2}Établir une méthodologie particulière d'alignement}
\titreEntete{Une méthodologie particulière}

%intro
En raison des difficultés identifiées dans la \reference{III-C-1}, un alignement simple, n'apportant aucune indication de fiabilité, n'est pas possible. De même, les alignements réalisés dans les chapitres précédents utilisent chacun les jeux de données initiaux jusqu'à la fin du traitement, sans en retirer au fur et à mesure les concepts qui viennent d'être alignés. L'alignement des référentiels de la \ac{ddcol} et de la \ac{dj} se distingue des précédents par la nécessité d'une méthodologie particulière, basée sur un indice de confiance attribué à chaque alignement, et sur une succession d'étapes, représentant les niveaux de confiance apportés au type de jointure utilisé.

\subsection{\label{III-C-2-a}Créer un indice de confiance pour chaque alignement}
\titreEntete{Créer un indice de confiance}

Les points de contact entre les jeux de données n'ont pas tous la même valeur dans un alignement. En effet, l'état civil d'une personne, bien qu'essentiel dans un alignement, peut conduire à aligner deux homonymes: c'est pourquoi les points de contact comme les noms, les prénoms ou les pseudos peuvent être considérés comme ayant une faible valeur dans le processus d'alignement. De plus, leur octroyer une valeur forte conduirait à surévaluer les alignements réalisés sur la simple comparaison des noms et prénoms sans autre point de comparaison par rapport aux alignements qui n'auront pas été possibles.\\

En revanche, la valeur des points de comparaison significatifs est considérée comme forte: il s'agit de la date de décès ou de la contribution. En effet, la probabilité que les états civils et les dates de décès de deux homonymes soient identiques est très faible, ce qui peut permettre de donner à cet alignement une valeur plus forte. De même, la correspondance entre une contribution et une fonction est considérée comme très fiable quand les états civils ont déjà été rapprochés: pour cette raison, le point de comparaison sur la contribution est celui qui possède l'indice de confiance le plus élevé, puisqu'il est le point le plus spécifique, et le plus difficile à faire correspondre.\\

Enfin, la comparaison du sexe permet également d'augmenter la fiabilité d'un alignement. Dans la majorité des alignements qui sont réalisés, la comparaison peut sembler évidente à l'humain, mais dans certains cas, comme pour les prénoms \textit{Dominique}, elle est nécessaire et permet la conservation ou non de l'alignement.\\

L'indice de confiance permet une priorisation des points de comparaison, une hiérarchisation de ces derniers. Il se forme à partir de la somme des scores de chaque point de comparaison. Ainsi, dans le cas de l'alignement des référentiels de personnes de la \ac{dj} et de la \ac{ddcol}, l'indice de confiance varie entre 0 --- quand l'alignement n'a pas pu être réalisé --- et 9 --- quand tous les points de comparaison ont été réalisés avec succès --- selon les scores de la \reference{table_scores}.
\begin{table}[!h]
	\centering
	\begin{tabular}{|c|c|}
		\hline
		\textbf{Point de comparaison}&\textbf{Score}\\ \hline
		Nom&1\\ \hline
		Prénom&1\\ \hline
		Pseudo nom&1\\ \hline
		Pseudo prénom&1\\ \hline
		Sexe&1\\ \hline
		Date de décès&2\\ \hline
		Contribution&2\\ \hline
	\end{tabular}
	\caption{Scores attribués à chaque point de comparaison}
	\label{table_scores}
\end{table}

\subsection{\label{III-C-2-b}Des étapes exclusives}
\titreEntete{Des étapes exclusives}

L'attribution d'un indice de confiance ne permet de résoudre que la problématique de l'évaluation finale des alignements. Il subsiste néanmoins une seconde problématique, celle de la présence dans les alignements réalisés de doublons, c'est à dire de personnes de la \ac{dj} alignés avec plusieurs concepts. Une solution pourrait être de supprimer les alignements présents dans ce cas. Or, ce cas survient fréquemment: supprimer les alignements concernés réduirait la quantité de résultats finaux.\\

Ainsi, après une première priorisation des points de comparaison, il est nécessaire d'effectuer ensuite une priorisation des combinaisons de ces points de comparaison. Quatre étapes principales ont été identifiées et constituent cette priorisation.

\noindent D'abord, les alignements réalisés avec le nom, le prénom, les pseudos et la correspondance des fonctions sont considérés comme ceux étant les plus sûrs pour effectuer les rapprochements entre les concepts de la \ac{ddcol} et les matricules de la \ac{dj}. Ces alignements, comme ceux des étapes suivantes, sont des jointures\footnote{Afin de ne récupérer que les alignements qui ont été réalisés, ces jointures sont de type \textit{inner join}.} entre les deux jeux de données réalisées dans l'ETL Talend avec le composant associé, un tMap (\reference{tmap_jointures}).
\begin{figure}[!h]
	\centering
	\includegraphics[width=16cm]{images/tmpa_jointure1_dj.png}
	\caption{L'alignement des personnes de la \ac{dj} et de la \ac{ddcol} pour une jointure dans un tMap de Talend.}
	\label{tmap_jointures}
\end{figure}

\noindent Ensuite, les alignements effectués avec le nom, le prénom et les pseudos, sans avoir pu faire correspondre les fonctions, sont la seconde étape.

\noindent La troisième étape, comme la quatrième, tente de rapprocher deux personnes en prenant en compte les différences de graphie qui peuvent exister. Ainsi, les alignements sont réalisés sur les pseudos, et sur le prénom de la \ac{dj} commençant par le prénom de la \ac{ddcol}\footnote{Une exception a été créé dans cette étape pour les prénoms \textit{Jean} et \textit{Anne}.}. Cette étape permet l'alignement d'une même personne ayant à la \ac{ddcol} le prénom \textit{Louis} et à la \ac{dj} le prénom \textit{Louis Marie}.

\noindent Enfin, l'ensemble des combinaisons possibles étant couvert, il est nécessaire d'effectuer une dernière étape pour effectuer non pas des alignements fiables --- ce qui est le but des trois premières étapes --- mais des alignements permettant d'apporter une aide à un opérateur humain en proposant plusieurs concepts qu'il est possible d'aligner avec un matricule. Ce rapprochement particulier, réalisé sur les seuls nom et prénom, autorise par conséquent la présence de plusieurs concepts alignés avec un même matricule, notamment dans le cas d'homonymes.\\

Distinguer ces quatre étapes permet, à l'issue de chacune d'elles, de récupérer ce qui n'a pas été aligné, tant du côté de la \ac{dj} et de la \ac{ddcol}, afin d'effectuer l'étape suivante avec uniquement ces données non alignées (\reference{orchestration}). Cette récupération évite de créer des alignements doubles avec des concepts différents entre les étapes. C'est également à l'issue de cette récupération que les résultats des jointures précédentes sont comparés afin de supprimer les matricules alignés plusieurs fois, et d'attribuer les scores pour le sexe et la date.\\
\begin{figure}[!h]
	\centering
	\includegraphics[width=16cm]{images/orchestration_partie1_dj.png}
	\caption{L'orchestration de la première étape dans l'ETL Talend.}
	\label{orchestration}
\end{figure}

Cependant, la limite de la récupération des matricules des personnes non alignées de la \ac{dj} et de la \ac{ddcol} est dans la priorisation qui a été faite des étapes. En effet, elle se base uniquement sur des critères définis en amont par un humain selon l'observation des cas généraux d'alignements: bien que ce soit une méthode apportant de la fiabilité aux alignements, cette fiabilité n'est pas nécessairement celle qui est la plus optimale. De même, cette récupération pourrait avoir lieu après chaque alignement de matricule: cela ajouterait néanmoins une nouvelle priorisation, basée sur l'ordre de passage, qui est plus arbitraire encore, puisque cet ordre de passage des matricules de la \ac{dj} dans le processus d'alignement n'est régi par aucun ordre significatif\footnote{Une base de données relationnelle n'étant pas ordonnée.}.

%conlu
\bigskip
\bigskip
Face aux nombreuses difficultés, l'indice de confiance et la priorisation des étapes permettent de réduire les mauvais alignements, et d'apporter une précision sur la fiabilité d'un alignement. Cependant, l'automatisation de ce processus présente des limites comme la définition de règles dirigeant le processus d'alignement. L'alignement des soixante-dix mille matricules de la \ac{dj} et des plus de trois cent mille concepts de la \ac{ddcol} ne peut pas être réalisé sur la base de quelques règles puis être considéré comme fiable.
\section{\label{III-C-3}Des résultats à la hauteur des données initiales}
\titreEntete{Des résultats à la hauteur des données initiales}

%intro

%conlu

%conclu
\bigskip
\bigskip
Dans le projet de centralisation des données de l'\ac{ina}, la centralisation des référentiels est indispensable, ces derniers étant devenus les pivots du système d'information. La conservation d'un référentiel utilisé par un seul jeu de données, celui de la \ac{dj}, ne correspond pas aux principes du \ac{led} et du \ldd. Ainsi, sa migration dans ce \textit{Lac} a nécessité un traitement important pour aboutir à l'alignement de ses matricules avec les concepts du \ldd. Face aux résultats de cet alignement, l'indice de confiance attribué permet une meilleure visibilité du travail effectué automatiquement, et permet aux superviseurs, qui sont nécessaires, d'approuver ou de refuser chaque alignement, afin d'éviter l'introduction d'erreurs dans le \ldd.
	
	\newpage
	
	%conclu
	Avec la théorisation du labyrinthe à la fin de l'Ancien Régime et du modèle-réseau par \nP{Umberto}{Eco}, les référentiels ont trouvé une nouvelle place dans les systèmes d'information. En effet, bien que déconstruits en nœuds et en liens, ils occupent désormais une place centrale dans ces systèmes. Ils sont la cible de toutes les attentions afin de pouvoir s'adapter aux besoins de l'intelligence artificielle, aux usages nouveaux de l'utilisateur sur le Web, aux besoins des professionnels de l'information et de la documentation. Les jeux de données peuvent devenir des jeux de données de référence dès lors qu'il sont réutilisés ou partagés.\\
	
	Seulement, la centralisation des données décentralisées par le modèle-réseau est apparue comme nécessaire et a créé des hubs de liens qui permettent d'accéder en un lieu unique à un corpus de liens relatif à un seul concept. Le parcours de graphe et de liens ne requiert alors qu'un simple passage par ce hub pour trouver la ressource recherchée. Au sein d'un \ac{led}, il est également possible de créé un hub de liens: c'est alors le concept lui-même qui devient un hub grâce aux identifiants qu'il peut porter.\\
	
	L'ouverture d'une institution vers le Web de données a été facilitée par le Web de données et le modèle-réseau. Cependant, le but premier de la création d'un \ac{led} est la mise en cohérence des données propres à une institution, ce qui a conduit à la création du \ldd à l'\ac{ina}. Or, cette mise en cohérence, des référentiels notamment, est compliquée par les pratiques de rédaction des termes et de choix des attributs stockés selon les usages qui les régissent.

	\chaptertoc{Conclusion}
\titreEntete{Conclusion}

%historique de struct ref: arbre au laby et modele reseau

%changement des usages et des besoins...

%... qui produit changement place ref
	
	\appendix
	\part*{Annexes}	
	\addcontentsline{toc}{part}{Annexes}
	\setcounter{chapter}{0}

\chapter{\label{annexe_index_schoepflin}Les index de la Renaissance, termes contrôlés et classification alphabétique (les index de l'\textit{Alsatia Illustrata} de \nP{Jean-Daniel}{Schoepflin})}
\titreEntete{Annexe \thechapter}

\begin{figure}[!h]
	\centering
	\begin{minipage}[c]{.46\linewidth}
		\includegraphics[width=6cm]{images/index_auctorum_alsatia.jpg}
		\caption{Index auctorum}
	\end{minipage} \hfill
	\begin{minipage}[c]{.46\linewidth}
		\includegraphics[width=6cm]{images/index_rerum_alsatia.jpg}
		\caption{Index rerum}
	\end{minipage} 
	\medskip
	Extraits des deux index de l'œuvre de \nP{Jean-Daniel}{Schoepflin} [Source: \url{http://bibliotheque-numerique.inha.fr/idurl/1/12532}, p.804 et 813]
\end{figure}

\chapter{\label{annexe_types_interop}Les différents types d'interopérabilité}
\titreEntete{Annexe \thechapter}

\begin{figure}[!h]
	\centering
	\includegraphics[width=12cm]{images/interop_conversion_copie.jpeg}
	\medskip
	\caption[L'interopérabilité par conversion et copie]{L'interopérabilité par conversion et copie [Source: \cite{bermes_2_2013}]}
\end{figure}

\begin{figure}[!h]
	\centering
	\includegraphics[width=12cm]{images/interop_denom_commun.jpeg}
	\medskip
	\caption[L'interopérabilité par le plus petit dénominateur commun]{L'interopérabilité par le plus petit dénominateur commun [Source: \cite{bermes_2_2013}]}
\end{figure}

\chapter{\label{annexe_thesaurus}Le thésaurus de noms communs de l'\ac{ina}}
\titreEntete{Annexe \thechapter}

\begin{figure}[!h]
	\centering
	\includegraphics[width=7cm]{images/cadreur_hierarchie.png}
	\medskip
	\caption[Extrait du thésaurus de noms communs de l'\ac{ina}]{Extrait du thésaurus de noms communs de l'\ac{ina} autour du terme \og Cadreur\fg{}}
\end{figure}
	
	\backmatter
	
	
%bibliographie ici dans les normes de l'école
%\printbibliography[title= Bibliographie sélective, prenote=intro]%postnote est aussi possible
\part*{Bibliographies sélectives}	
\addcontentsline{toc}{part}{Bibliographies sélectives}
\thispagestyle{empty}

\printbibliography[heading=subbibliography, keyword={ina}, title={L'INA: institution, utilité et applications des référentiels}]
\newpage
\thispagestyle{empty}
\printbibliography[heading=subbibliography, keyword={blod}, title={Les référentiels avec le Linked Open Data}]
\newpage
\thispagestyle{empty}
\printbibliography[heading=subbibliography, keyword={lod}, title={Le Web de données et les référentiels}]
\titreEntete{Le Web de données et les référentiels}
\newpage
\thispagestyle{empty}
\printbibliography[heading=subbibliography, keyword={graphe}, title={Utiliser les noeuds et les liens pour créer un référentiel global}]
\newpage
\thispagestyle{empty}
\printbibliography[heading=subbibliography, keyword={aligner}, title={Aligner des référentiels dans d'autres domaines}]
\thispagestyle{empty}


\printindex[ref]

	\listoffigures
	
	\listoftables

	\tableofcontents
	
\end{document}