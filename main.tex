%%%%%%%%%%%%%%%%%%%%%
%%%%% PREAMBULE %%%%%
%%%%%%%%%%%%%%%%%%%%%
\documentclass[a4paper,12pt,twoside]{book}
\usepackage{fontspec}

%%%%% Index %%%%%
\usepackage{index}
\usepackage{imakeidx}%pour les index, à charger avant hyperref
\makeindex
\makeindex[name=referentiels, title=Index des noms de référentiels]
\makeindex[name=ina, title=Index des termes concernant l'Institut national de l'Audiovisuel]

\usepackage[pdfusetitle, pdfsubject ={Mémoire TNAH}, pdfkeywords={institut national de l'audiovisuel; référentiel; thésaurus; vocabulaire contrôlé; vocabulaire hiérarchique; ontologie; web de données; Wikidata; liens; alignement}]{hyperref}
\usepackage[english, french]{babel}
\usepackage{morewrites}
\usepackage{tocbibind} %paquet pour mettre index et bib dans la toc

% configurer le document selon les normes de l'école
\usepackage[margin=2.5cm]{geometry}
\usepackage{setspace}
\onehalfspacing
\setlength\parindent{1cm}

\usepackage{lettrine}

%%%%% Dessin %%%%%
\usepackage{qtree}
\usepackage{pstricks}
\usepackage{tikz} %paquet pour dessiner ; à placer avant graphicx
\usepackage{graphicx} %paquet image


\usepackage{caption} %pour que la mention de figure n'apparaisse pas dans les légendes de l'image


%%%%% Bibliographie %%%%%
\usepackage[backend=biber, sorting=nyt, style=enc, maxbibnames=3]{biblatex}
\addbibresource{bibliographie/bib.bib}
\nocite{*}
%\defbibnote{intro}{Cette bibliographie contient toutes les références utilisées pour ce cours} %pour des notes introductives dans le début de la biblio

%%%%% Abréviations %%%%%
\usepackage{acro}
\DeclareAcronym{ddcol}{short = DDCOL, long = Direction déléguée aux collections}
\DeclareAcronym{dj}{short = DJ, long = Direction juridique}
\DeclareAcronym{dsi}{short = DSI, long = Direction des systèmes d'information}
\DeclareAcronym{ead}{short = EAD, long = Encoded Archival Description}
\DeclareAcronym{epic}{short = ÉPIC, long = Établissement public à caractère industriel et commercial}
\DeclareAcronym{ina}{short = INA, long = Institut national de l'Audiovisuel}
\DeclareAcronym{isan}{short = ISAN, long = \textit{International Standard Audiovisual Number}}
\DeclareAcronym{lcsh}{short = LCSH, long = Library of Congress Subject Headings}
\DeclareAcronym{marc}{short = MARC, long = MAchine-Readable Cataloging}
\DeclareAcronym{oaipmh}{short = OAI-PMH, long = Open Archive Intiative Protocol for Metadata Harvesting}
\DeclareAcronym{oclc}{short = OCLC, long = Online Computer Library Center}
\DeclareAcronym{ortf}{short = ORTF, long = Office de la radio-télévision française}
\DeclareAcronym{rameau}{short = RAMEAU, long =Répertoire d'autorité-matière encyclopédique et alphabétique unifié}
\DeclareAcronym{unimarc}{short = UNIMARC, long =UNIversal MAchine-Readable Cataloging }


%%%%% Nouvelles commandes %%%%%
\newcommand{\reference}[1]{\autoref{#1}: \nameref{#1}}

\newcommand{\chaptertoc}[1]{\chapter*{#1}
	\addcontentsline{toc}{chapter}{#1}
	\markboth{\slshape\MakeUppercase{#1}}{\slshape\MakeUppercase{#1}}}

\newcommand{\titreEntete}[1]{\markboth{\slshape\MakeUppercase{#1}}{}}

\newcommand{\nP}[2]{#1 \textsc{#2}}

%%%%% Nouveaux environnements %%%%%
\newenvironment{citationLongue}{\begin{quotation}\og}{\fg{}\end{quotation}}

\author{Maxime Challon - M2 TNAH}
\title{Les référentiels en institutions patrimoniales : évolution des pratiques et repositionnement. L’exemple des référentiels de l’Institut national de l’Audiovisuel.}

%%%%%%%%%%%%%%%%%%%%
%%%%% DOCUMENT %%%%%
%%%%%%%%%%%%%%%%%%%%
\begin{document}
	\renewcommand*\appendixautorefname{Annexe}
		
	\frontmatter
	\begin{titlepage}
		\begin{center}
			
			\bigskip
			
			\begin{large}
				\'ECOLE NATIONALE DES CHARTES
			\end{large}
			\begin{center}\rule{2cm}{0.02cm}\end{center}
			
			\bigskip
			\bigskip
			\bigskip
			\begin{Large}
				\textbf{Maxime Challon}\\
			\end{Large}
			\begin{normalsize} \textit{licencié ès histoire}
			\end{normalsize}
			
			\bigskip
			\bigskip
			\bigskip
			
			\begin{Huge}
				\textbf{Les référentiels en institutions patrimoniales : évolution des pratiques et repositionnement}\\
			\end{Huge}
			\bigskip
			\bigskip
			\begin{LARGE}
				\textbf{L’exemple des référentiels de l’Institut national de l’Audiovisuel}\\
			\end{LARGE}
			
			\bigskip
			\bigskip
			\bigskip
			\begin{large}
			\end{large}
			\vfill
			
			\begin{large}
				Mémoire 
				pour le diplôme de master \\
				\og{} Technologies numériques appliquées à l'histoire \fg{} \\
				\bigskip
				2020
			\end{large}
			
		\end{center}
	\end{titlepage}

\thispagestyle{empty}	
\cleardoublepage
	
		\chapter*{Résumé}
	\titreEntete{Résumé}
\addcontentsline{toc}{chapter}{Résumé}
	\medskip
	Ce mémoire, réalisé pour l'obtention du diplôme de Master 2 \og Technologies numériques appliquées à l'histoire\fg{} de l'École nationale des Chartes, retrace l'évolution des pratiques documentaires sur les référentiels en institution patrimoniale à travers l'étude des référentiels de l'\ac{ina} et leurs alignements. Cette étude de l'évolution des formes et des structures des référentiels est liée à l'évolution de la place de ces référentiels au sein des systèmes documentaires, ainsi qu'aux besoins qui leur sont liés.\\
	
	\textbf{Mots-clés~:} institut national de l'audiovisuel; référentiel; thésaurus; vocabulaire contrôlé; vocabulaire hiérarchique; ontologie; web de données; Wikidata; liens; alignement.
	
	\textbf{Informations bibliographiques~:} Maxime Challon, \textit{Les référentiels en institutions patrimoniales : évolution des pratiques et repositionnement. L’exemple des référentiels de l’Institut National de l’Audiovisuel.}, mémoire de master \og{}Technologies numériques appliquées à l'histoire\fg{}, dir. Gautier Poupeau, École nationale des chartes, 2020.
	
		\chapter*{Remerciements}
	\titreEntete{Remerciements}
	\addcontentsline{toc}{chapter}{Remerciements}
	
	\lettrine{M}es remerciements vont tout d'abord à Gautier \textsc{Poupeau}, mon maître de stage, qui m'a accueilli, guidé, conseillé et intégré à son équipe malgré le travail à distance imposé par le contexte actuel. Je souhaite également remercier Axel \textsc{Roche-Dioré} pour ses explications et son soutien dans la réalisation technique de mon stage.\\
	
	J'adresse aussi mes remerciements aux membres du pôle \og Ingénierie de la Donnée\fg{} pour le temps qu'ils m'ont accordé, Lauryne \textsc{Lemosquet}, Otmane \textsc{Elabboubi} et Akli \textsc{Abdi}, ainsi qu'à Florence \textsc{Bréant}, cheffe de projet pour le \textit{Lac de données}. \\
	
	Que soit également remercié l'ensemble du département \og Architecture et Innovation\fg{} de l'\ac{ina} pour l'accompagnement fourni tout au long de mon stage, notamment Stanislas \textsc{de Maigret} et Matthieu \textsc{Boricaud} pour le déploiement de l'application, et Olivio \textsc{Ségura} pour la présentation des archives de l'\ac{ina}.
	
	\chaptertoc{Liste des abréviations}
	\printacronyms[heading=none]
	
		\chapter*{Introduction}
\addcontentsline{toc}{chapter}{Introduction}
% texte de l'intro ici



\thispagestyle{empty}
\cleardoublepage
	
	\mainmatter
	
	\part{\label{controler}CONTRÔLER. A la recherche de clés (années 1960 – fin des années 1990)}
	
	\chapter{\label{I-A}Le référentiel comme clé}
\titreEntete{Le référentiel comme clé}

Considéré comme une simple aide ou outil au service du documentaliste ou de l'utilisateur, le référentiel trouve d'abord sa place comme fournisseur de clés. Son utilisation principale est d'offrir au document décrit des vedettes qui puissent permettre une classification ou une recherche aisée de ce document. Cependant, pour être efficaces, ces vedettes doivent partager un langage contrôlé, des règles de graphie, de syntaxe, \dots D'abord conservées sur des fichiers papier en institutions patrimoniales, ces vedettes ont été parmi les premiers éléments rétroconvertis, donnant naissance aux fichiers d'autorité numériques, et permettant une interopérabilité entre les référentiels par le biais des portails numériques.

\section{\label{I-A-1}Du langage libre au langage contrôlé: vers l'indexation}
\titreEntete{Vers l'indexation}

\begin{quote}
	\og La nature n'a pas juré de ne nous offrir que des objets exprimables par des formes simples de langage \footcite[p.18]{valery_variete_1936} \fg{}
\end{quote}

Le langage permet aux hommes de communiquer entre eux. Ce langage libre, naturel, comprend l'ensemble des langues, et donne aux hommes la possibilité de décrire le plus précisément possible le monde qui les entoure, sans jamais atteindre le description idéale. Seulement, ce langage conduit à des variations graphiques ou syntaxiques, selon la déclinaison des noms ou la conjugaison des verbes. La polysémie est également l'une des conséquences de ce langage naturel selon le contexte de chaque mot. Enfin, le langage libre conduit à la synonymie. Toutes ces caractéristiques du langage humain perturbent et complexifient la tâche de description documentaire, bien qu'elles soient essentielles à la communication entre les hommes.\\

Afin de régler ces confusions possibles entre les mots et de régir leur formation, des langages contrôlés ont très vite fait leur apparition. Ils permettent de décrire des concepts, des thèmes, des ouvrages, tout en permettant un classement potentiel. Ce recours aux langages contrôlés est une pratique très ancienne, née avant l'apparition des \textit{codices} lorsque déjà la recherche d'informations était nécessaire. Pratique millénaire, l'attribution de termes contrôlés à une information se perpétue encore actuellement, par exemple sous la forme de \og hashtags\fg{} sur les réseaux sociaux, qui permettent de décrire un texte et de le retrouver ensuite aux côtés d'autres similaires.\\

Dans l'Antiquité, les index n'existent pas encore. Cependant, des \index[ref]{typologie@Typologie!vocabulaires controles@Vocabulaires contrôlés}vocabulaires contrôlés sont utilisés pour le classement et pour la mémorisation des textes. Ces termes contrôlés se retrouvent dans des notes marginales, des tables de concordance ou bien dans les catalogues. Au \textsc{III}\textsuperscript{ème}siècle av. J.C., \nP{Callimaque}{de Silène} réalise le catalogue de la bibliothèque d'Alexandrie en utilisant le genre du texte pour lui déterminer une classe, puis les \textit{volumina} sont rangés dans des rayons selon un ordre alphabétique, ces rayons reflétant les classes attribuées selon le genre.\\

Au Moyen-Âge, les premiers \index[ref]{typologie@Typologie!index@Index}index apparaissent, s'ajoutant aux tables de concordance. \nP{Isidore}{de Séville} ne créé qu'un classement alphabétique dans son Livre X des \textit{Étymologies}, sans indexer son ouvrage. Cinq siècles plus tard, les vedettes commencent à être normalisées dans certaines œuvres, le nominatif ou l'ablatif étant considérés comme la forme retenue, et rassemblées dans un index alphabétique\footnote{\nP{Jean}{Berger}, dans son analyse du \textit{Liber de honoribus}, le plus vieil index alphabétique compilé au \textsc{XII}\textsuperscript{ème}siècle, étudie avec précision l'indexation des chartes du Cartulaire de Saint-Julien de Brioude: les lieux et les personnes sont ainsi indexés. Voir \cite[pp.97 et suivantes]{berger_indexation_2006}}.\\

Avec la Renaissance puis l'Ancien Régime, l'indexation devient plus fine et les index de fin de volumes de plus en plus imposants. Ils permettent au lecteur un accès direct aux passages du texte contenant l'entrée d'index. Plus encore, ces index lient une classification générale suivie d'alphabétique, tout en normalisant leurs entrées\footnote{\nP{Jean-Daniel}{Schoepflin} dans son \textit{Alsatia illustrata} de 1751 créé ainsi deux index distincts: l'un pour les personnes (\textit{Index auctorum}), l'autre pour les termes évoqués dans son œuvre(\textit{Index rerum}). L'ensemble des noms est indexé au nominatif puis ils sont parfois subdivisés en thèmes ou événements. L'index devient ainsi indépendant de la graphie et de la grammaire de la langue utilisée. Voir \cite{schoepflin_alsatia_1751}. Voir \reference{annexe_index_schoepflin}.}\footnote{\nP{Robert}{Estienne} pousse plus loin encore l'indexation, un siècle et demi avant \nP{Jean-Daniel}{Schoepflin}, en créant de multiples \index[ref]{typologie@Typologie!index@Index}index: celui des populations, des villes, \dots. Ces index sont eux-mêmes subdivisés, normalisés et classés alphabétiquement, les rendant œuvre à part entière. Voir \cite{estienne_thesaurus_1573}.}.

\section{\label{I-A-2}Une clé entre les données: une terminologie maîtrisée, objectif des vocabulaires contrôlés}

\section{\label{I-A-3}Une clé entre les jeux de données: l'interopérabilité par les fichiers d'autorité et les portails}
\titreEntete{Une clé entre les jeux de données}

Comme nous l'avons évoqué précédemment (voir \reference{I-A-2}), les vocabulaires contrôlés sont de nouveaux langages, spécifiques et uniformisés, se substituant au langage naturel humain pour un domaine précis. Le vocabulaire est par conséquent un référentiel propre à l'institution qui l'a créé et a pour seul utilisateur cette institution. Seulement, deux institutions aux activités proches créent deux vocabulaires similaires, se distinguant par la complétude de certaines vedettes ou par des variantes de graphies.\\

Le domaine bibliothéconomique a été le premier à informatiser en masse ses vocabulaires et ses \index[ref]{autorites@Autorités!fichiers autorite@Fichiers d'autorité}fichiers d'autorité, permettant ainsi une amélioration de l'expérience utilisateur et du catalogage, tout comme un partage possible avec des institutions proches.

\subsection{\label{I-A-3-a}La naissance des autorités par rétroconversion}
\titreEntete{Les fichiers d'autorité}

\begin{citationLongue}
	Les fichiers d'autorité appartiennent bien à un	ensemble : fonctionnant comme un tout, avec des règles d’interdépendance et d’interopérabilité de ses constituants, ils permettent le contrôle de la cohérence des métadonnées bibliographiques.\footcite[p.6]{aymonin_arabesques_2017}
\end{citationLongue}

Avant la naissance du Web, chaque ouvrage était décrit dans un catalogue et classé par ordre alphabétique des noms d'auteurs. Des \index[ref]{autorites@Autorités!fichiers autorite@Fichiers d'autorité}catalogues thématiques ont été créés, de même que des fichiers physiques en bibliothèque, permettant la recherche de documents selon un sujet précis. Cependant, l'indexation des documents est réduite au titre, à l'auteur, et à quelques sujets. En effet, la structure même d'un fichier papier en bibliothèque nécessite de dupliquer la notice d'un exemplaire en plusieurs notices qui vont être placées par la suite dans le fichier correspondant au sujet.\\

Ces fichiers physiques des bibliothèques, bien qu'utiles aux lecteurs par leur classement thématique, présentent plusieurs difficultés: d'abord, l'indexation se trouve limitée à quelques mots; ensuite, la création d'un fichier thématique est complexe à réaliser par le choix des vedettes et produit alors un immense silence; enfin, la consultation d'une fiche par un lecteur empêche un second de la consulter dans le même temps.\\

Dès les années 1970, les bibliothèques se sont engagées dans une vaste opération de rétroconversion de leurs notices documentaires. Les fichiers physiques et les notices cartonnées sont alors informatisés et \og reproduits presque à l’identique [\dots] sous forme de bases de données\fg{}\footcite{bermes_du_2013}. L'informatisation des notices et des fichiers permet, par conséquent, d'améliorer l'indexation des documents. L'utilisateur va donc pouvoir trouver plus rapidement plus de documents correspondant à sa recherche. Ainsi, les autorités \index[ref]{lod@Linked Open Data (LOD)!lcsh@LCSH}\index[ref]{autorites@Autorités!lcsh@LCSH}\ac{lcsh}, créées en 1914 sous format papier, ont été informatisées; les autorités \index[ref]{lod@Linked Open Data (LOD)!rameau@RAMEAU}\index[ref]{autorites@Autorités!rameau@RAMEAU}\ac{rameau} créées dans les années 1980 reprennent celles de \ac{lcsh} en les complétant.\\

Cependant, ces \index[ref]{autorites@Autorités!fichiers autorite@Fichiers d'autorité}fichiers d'autorité comportent, comme nous l'avons évoqué plus haut (\reference{I-A-2-c}), des formes retenues et des formes rejetées des termes, ce qui crée de multiples renvois à l'intérieur des fichiers physique ou informatique. L'arrivée des moteurs de recherche, dans les années 2000, permet de supprimer ces différences de termes, en indexant à la fois les formes retenues et les formes rejetées et en permettant de trouver directement la vedette recherchée.

\subsection{\label{I-A-3-b}Partager des vocabulaires: à la recherche de la meilleure interopérabilité}
\titreEntete{Partager des vocabulaires}

La problématique du partage des référentiels entre institutions se pose avant l'informatisation des catalogues et des fichiers des bibliothèques. En effet, le format \index[ref]{echanges@Échanges!formats@Formats!marc@MARC}\ac{marc}, né en 1968 à la Bibliothèque du Congrès, permet l'échange de données entre les institutions et la \og duplication les notices d’un catalogue à un autre\fg{}\footcite{bermes_convergence_2013}. Malgré de multiples variantes nationales, l'\ac{unimarc} reste aujourd'hui le format d'échange privilégié entre les bibliothèques.\\

Pour partager les \index[ref]{autorites@Autorités!fichiers autorite@Fichiers d'autorité}fichiers d'autorité et aboutir à une interopérabilité totale des données entre deux institutions par le biais des machines, différents protocoles d'échange ont été utilisés --- ou délaissés en fonction des difficultés imposées par chacun. Dès les années 1980 est développé le \index[ref]{echanges@Échanges!protocoles@Protocoles!z3950@Z39-50}protocole Z39-50. Ce protocole permet d'interroger une base de données de manière synchrone, selon la requête du client, et de récupérer des données en format \ac{marc}\footcite{bibliotheque_nationale_de_france_protocole_nodate}.\\

Ce protocole Z39-50 est destiné aux catalogueurs qui peuvent ainsi \og repérer puis télécharger une notice dans un catalogue distant plutôt que d’avoir à la saisir \textit{ex nihilo}\fg{}\footcite{bermes_convergence_2013}. Le partage, \og par conversion et copie\fg{}\footnote{\cite{bermes_convergence_2013}. Voir \reference{annexe_types_interop} (\reference{conversion}).}, n'est alors qu'une simple copie de données, dont la mise à jour est difficile. L'existence de ce protocole, bien que destiné aux professionnels de la documentation, a suscité la création de \index[ref]{relier@Relier!portails@Portails}portails de consultation de notices documentaires ou de \index[ref]{autorites@Autorités!fichiers autorite@Fichiers d'autorité}fichiers d'autorité, interrogeant de manière synchrone les bases de données: cette utilisation orientée utilisateur du \index[ref]{echanges@Échanges!protocoles@Protocoles!z3950@Z39-50}protocole Z39-50 permet à la Bibliothèque nationale de France d'offrir différents services(intégration des notices dans \index[ref]{lod@Linked Open Data (LOD)!oclc@OCLC}\ac{oclc}, recherche dans le Catalogue Collectif de France (CCFR), \dots\footcite{bibliotheque_nationale_de_france_protocole_nodate}). Cependant, face aux temps de réponses importants et aux résultats appauvris retournés par la requête, les portails se sont révélés décevants et peu efficaces. De plus, l'utilisation d'un portail nécessite, de la part de l'utilisateur, qu'il connaisse précisément ce qu'il cherche, de manière à se connecter au \index[ref]{relier@Relier!portails@Portails}portail correspondant (qui lui-même doit être connu de cet utilisateur)\footcite{dalbin_approches_2011}.\\

La multiplication des formats d'échanges --- \index[ref]{echanges@Échanges!formats@Formats!marc@MARC}\ac{marc} et \ac{unimarc} pour les bibliothèques, \index[ref]{echanges@Échanges!formats@Formats!ead@EAD}\ac{ead} pour les archives ---, ainsi que la volonté d'offrir au public lien entre les différentes bases de données patrimoniales, ont conduit à la création d'un nouveau protocole, \index[ref]{echanges@Échanges!protocoles@Protocoles!oaipmh@OAI-PMH}\ac{oaipmh}. Ce protocole asynchrone repose sur deux acteurs: le fournisseur qui met à disposition ses données dans un \og entrepôt\fg{}, et le moissonneur qui collecte ces données pour les intégrer à son système\footcite{bibliotheque_nationale_de_france_protocole_nodate-1}.\\

Cependant, si les performances du protocole sont améliorées avec \ac{oaipmh}, les documents et les \index[ref]{autorites@Autorités!fichiers autorite@Fichiers d'autorité}fichiers d'autorité ne peuvent pas être sélectionnés et filtrés: un format d'échange simple, minimal, est nécessaire. Ce format est le \index[ref]{echanges@Échanges!formats@Formats!dublin core@Dublin Core}Dublin Core\footcite{noauthor_dublin_nodate} comprenant quinze champs d'informations.
Ce partage de données et de métadonnées entre les institutions permet une \og interopérabilité par le plus petit dénominateur commun\fg{}\footnote{\cite{bermes_convergence_2013}. Voir \reference{annexe_types_interop} (\reference{denom})}, qui est le Dublin Core. Celui-ci peut cependant présenter un appauvrissement des données puisque les champs sont très réduits, ou au contraire permettre de grandes différences au sein d'un même champ.

\bigskip
\bigskip
\bigskip
Auteurs, catalogueurs et bibliothécaires ont très vite ressenti le besoin de se dégager du langage naturel de manière à renvoyer rapidement vers des passages de leur texte ou à décrire le plus précisément possible les documents, pour faciliter la lecture ou la recherche de l'utilisateur final. D'abord effectuées sur des supports papier, ces opérations de descriptions ont été informatisées et ont permis le partage de données et de métadonnées entre les institutions: les notices et les fichiers d'autorité disponibles sont la source constante d'interrogations quant au meilleur moyen de les mettre à disposition, tant pour le professionnel que pour le public. L'ouverture de ces vocabulaires a permis une amélioration des descriptions et une uniformisation des pratiques d'indexation.\\

Cependant, ces vocabulaires contrôlés restent peu précis et sont limités à leur terminologie pour en tirer le sens: il ne comprennent pas de terminologie sémantique, qui permettrait d'améliorer plus encore la description effectuée, en pouvant se référer aux termes parents, frères ou fils. L'anneau de synonymie évoqué (\reference{I-A-2-b}) permet la prise en compte des synonymes, mais ne donne pas de sens supplémentaire à la vedette.
	\chapter{\label{I-B}Les référentiels à l’INA}
\titreEntete{Les référentiels à l’INA}

%intro: historique INA

\section{\label{I-B-1}De multiples fonds à décrire}
\titreEntete{De multiples fonds à décrire}


\subsection{\label{I-B-1-a}Les archives professionnelles}
\titreEntete{Les archives professionnelles}

Après l'éclatement de l'\ac{ortf}, des fonds divers deviennent la propriété de l'\ac{ina} et participent à la diversité des fonds audiovisuels conservés à l'\ac{ina}. Ainsi, les \og Actualités françaises\fg{} --- la presse filmée diffusée dans les salles de cinéma --- ont été transférées à l'\ac{ina} en 1974. Les grands moments des débuts de la télévision --- le premier journal télévisé, les premières grandes émissions ou les magazines de reportage --- sont conservés et participent, avant l'instauration du dépôt légal, à la mémoire de l'audiovisuel français; de même pour les fonds radio qui retracent les grands moments historiques depuis les années 1940, et qui deviennent de plus en plus complets avec la généralisation de la bande magnétique à partir du milieu des années 1950.\\

Les archives professionnelles de l'\ac{ina} ne sont que des archives: elles ne couvrent pas l'ensemble de la production audiovisuelle depuis les années 1940 ou la création de l'Institut. Des tranches horaires de diffusion télé- ou radiodiffusées ne sont ainsi pas conservées; peu de traces demeurent alors de la production audiovisuelle, notamment avant l'arrivée du kinéscope. Pour combler ces manques, l'\ac{ina} dispose d'un fonds photographique créé à partir des services de l'\ac{ortf} ou de l'\ac{ina} et portant sur la réalisation des émissions et des tournages.\\

Enfin, des délégations régionales se chargent de la conservation et de la communication des archives télévisées et radiodiffusées des stations régionales --- apparues dans les années 1950: la vie et l'histoire des régions sont ainsi couvertes par l'\ac{ina}.\\

Les fonds d'archives professionnelles de l'\ac{ina} sont conséquents et divers, témoins de la vie et de la société française depuis l'Après-Guerre. Irremplaçables, leur description n'en est pas moins difficile par la diversité des sujets évoqués ou présents dans les documents.

\subsection{\label{I-B-1-b}Les fonds issus du dépôt légal}
\titreEntete{Les fonds issus du dépôt légal}

Depuis la loi sur le dépôt légal de l'audiovisuel de 1992, l'\ac{ina} en est le dépositaire. À partir de 1995, l'\ac{ina} enregistre la globalité de la programmation des stations de Radio-France --- France-Inter, France-Musique, France-Culture, France-Info et France-Bleue ---, enregistrement étendu en 2001 aux stations privées généralistes comme RTL ou NRJ\footnote{En 2020, l'ensemble des stations captées au titre du dépôt légal par l'INA est décrit dans l'\reference{annexe_dl_captation} (\reference{dl_radio})}.\\

Pour les programmes télévisés, le dépôt légal ne concerne d'abord --- entre 1995 et 2001 --- que les sept chaînes principales --- TF1, France 2, France 3, Canal +, M6, Arte, France 5 --- et leurs programmes en première diffusion. La captation directe et intégrale des chaînes n'apparaît qu'en 2002 et est élargie à douze autres chaînes. Enfin, depuis 2005, les chaînes de la Télévision numérique terrestre (TNT) sont toutes captées\footnote{Les chaînes de télévision captées en 2002 pour le dépôt légal sont décrites dans l'\reference{annexe_dl_captation} (\reference{dl_tv})}.

\bigskip
\bigskip

La diversité des fonds d'archives, la captation directe en intégralité des chaînes de télévision et de radio, ainsi que la captation de sites web, plateformes ou comptes de réseaux sociaux au titre du dépôt légal audiovisuel, représentent une masse très importante de données et de documents à conserver. En 2019\footcite[p.5]{institut_national_de_laudiovisuel_rapport_2019}, l'\ac{ina} conserve 20 873 143 heures de programmes de télévision et de radio, dont plus de 18 millions captés par le dépôt légal. 1,2 million de photos s'ajoutent à ces documents. La majorité de ces documents, issus du dépôt légal, sont destinés à une gestion patrimoniale et à une valorisation dans l'INAthèque, alors que les documents des archives professionnelles sont destinées à la valorisation commerciale au travers notamment du site \href{https://www.inamediapro.com}{INAMediaPro} destiné aux professionnels.
\section{\label{I-B-2}Un système documentaire pluriel répondant aux besoins}
\titreEntete{Un système documentaire pluriel}

La masse des documents, l'évolution de leur récupération auprès des producteurs et leurs usages divers conduisent à la création d'un système documentaire pluriel, créé à partir des besoins et non des données. Les deux usages commerciaux et patrimoniaux des documents\footnote{É. Alquier décrit ces deux usages dans un article de 2017 évoque ces usages et la plateforme qui les met en œuvre. En 2020, le service de vidéo à la demande Madelen vient s'ajouter à l'offre \og grand public\fg{} de l'\ac{ina}. Voir \cite{alquier_production_2017}.}, évoqués précédemment, dirigent le nombre des bases de données, leur structure et le partage de référentiels. Avant le projet du \ldd lancé en 2015, le système documentaire de l'\ac{ina} est pluriel, constitué de plusieurs bases de données distinctes ainsi que de plusieurs référentiels non communs.\\

Deux types de données sont présents dans les bases de l'\ac{ina}. D'abord, il y a du texte libre, décrivant les titres propres des documents, les titres de collections ou indiquant un identifiant, ou bien des notes diverses ou des chiffres. Ensuite, il y a les données contrôlées, issues de référentiels et de lexiques, permettant de décrire les contenus, les particularités de ces contenus et les événements associés à ces contenus (diffusion, archivage, exploitation)\footnote{Voir \reference{annexe_type_donnees_axel} (\reference{type_donnees_axel}).}.

\subsection{\label{I-B-2-a}Les bases de données du dépôt légal (DL)}
\titreEntete{Les bases de données du dépôt légal}

L'\ac{ina} capte en permanence et en direct plus de 170 chaînes de télévision et stations de radio\footcite[p.5]{institut_national_de_laudiovisuel_rapport_2019}. Ce flux ininterrompu est décrit lors du catalogage par des techniciens spécialisés dans la gestion de collections multimédia: pour chaque document y est notamment indiqué le titre, le générique, les dates et heures de diffusion, ainsi que des descripteurs pour indexer la chaîne de diffusion, les thématiques présentes, \dots\\

Quand le document est décrit, les données complétées par le technicien de gestion des collections multimédia dans son interface graphique sont dirigées vers les bases de données du dépôt légal, scindées en quatre pour correspondre à la provenance du document. Ainsi, bien que les documents proviennent de la même source --- la captation pour le dépôt légal ---, ils sont éclatés dans quatre bases de données différentes pour correspondre à leur provenance:
\begin{itemize}
	\item la base DLRADIO (Dépôt Légal de la Radio) comprend les documents diffusés en radio, sans autre distinction de provenance
	\item la base DLTV (Dépôt Légal de la Télévision (Nationale)) ne comprend pas l'ensemble des documents diffusés à la télévision, mais seulement les chaînes nationales
	\item la base DLREG (Dépôt Légal de la Télévision Régionale) comprend les documents télévisuels diffusés sur une chaîne de télévision régionale comme France 3
	\item enfin, la base DLSAT (Dépôt Légal de la Télévision Satellite) comprend les documents diffusés sur les chaînes de télévision satellite
\end{itemize}
\bigskip

Cependant, malgré cette scission des données dans plusieurs bases de données, ces quatre bases partagent un même schéma pour les \index[ref]{typologie@Typologie!vocabulaires controles@Vocabulaires contrôlés}référentiels. Ce schéma permet de trouver des tables comprenant la signification d'identifiants de provenance de chaînes (le lien entre le code \og FR5\fg{} présent dans les données peut ainsi être établi avec son terme développé), de provenance de données, \dots Ce schéma est un fournisseur de mots-clés destinés à permettre la description, l'indexation et la recherche des documents.

\subsection{\label{I-B-2-b}Les bases de données des archives professionnelles (DA)}
\titreEntete{Les bases de données des archives professionnelles}

Le dépôt légal se concentre sur la diffusion des documents et conserve alors à chaque instant l'ensemble de ce qui est diffusé à la télévision ou à la radio --- les émissions, les films, les publicités, les journaux télévisés, \dots~ Cette conservation des premières diffusions et des rediffusions permet, ainsi que l'exige le dépôt légal, d'avoir un panorama complet du paysage audiovisuel français, comme c'est le cas à la Bibliothèque nationale de France pour les imprimés ou les périodiques.\\

Les archives professionnelles ne sont pas soumises à cette exhaustivité: lorsqu'un producteur de contenu audiovisuel mandate l'\ac{ina} pour la conservation et/ou la commercialisation de ses contenus, les données de ces contenus sont récupérées dans les bases du dépôt légal puis copiées dans celles des archives professionnelles. Ainsi, la même donnée est dupliquée au \ac{dl} et au \ac{da}. Cependant, le \ac{da} s'intéressant non pas à la diffusion elle-même du document mais au document lui-même, ces données vont être transformées et complétées de manière à être plus précises et à avoir une meilleure description. Cette description plus fine permet la vente des extraits.\\

Comme pour le \ac{dl}, le \ac{da} possède plusieurs bases de données partageant les mêmes référentiels:
\begin{itemize}
	\item la base DAV (Archives Professionnelles de la Télévision Nationale)
	\item la base DAVREG (Archives Professionnelles de la Télévision Régionale)
	\item la base DAVRAD (Archives Professionnelles de la Radio)
\end{itemize}
\bigskip

Ces trois bases de données sont appuyées par plusieurs lexiques et \index[ref]{typologie@Typologie!thesaurus@Thésaurus}\textit{thesauri}, notamment celui des noms communs\footnote{L'importance --- et la complexité --- de ce thésaurus au \ac{da} nécessite une interface graphique, \og Totem\fg{}, pour le visualiser et cataloguer les documents. Un exemple de visualisation de ce thésaurus est possible en \reference{annexe_thesaurus} (\reference{thesaurus_cadreur}).} et des personnes physiques et morales.

\subsection{\label{I-B-2-c}La base de donnés juridique (DJ)}
\titreEntete{La base de donnés juridique}

La base de données \og Adaje\fg{} de la \ac{dj} comprend l'ensemble des données permettant d'identifier et de rémunérer les ouvrants-droit\footnote{Personnes auxquelles les droits ont été ouverts, le producteur lui-même ou ses ayants-droit.} et les ayants-droit\footnote{\og Un ayant droit est une personne ayant acquis un droit d'une autre personne\fg{} in \url{https://droit-finances.commentcamarche.com/faq/4010-ayant-droit-definition}.} des documents et extraits vendus. Cette base juridique contient par conséquent des tables de Personnes, de Contributions, d'Informations personnelles, \dots\\

Les bases \ac{dl} et \ac{da}, et celle de la \ac{dj} n'ont aucun lien entre elles, mais leurs données semblent redondantes notamment pour les personnes physiques et morales. Le projet du \index[ref]{led@Linked Enterprise Data (LED)!ldd@Lac de données (INA)}\index[ref]{modelisation@Modélisation!ldd@Lac de données (INA)}\ldd\footnote{Ce projet est évoqué au \reference{III-B}} devrait permettre l'alignement de ces bases entre elles en évitant les doublons: la base de la \ac{dj} enrichira notamment les concepts de personnes physiques et morales déjà créés à partir des données de la \ac{ddcol}.

\bigskip
\bigskip

Plusieurs référentiels, parfois similaires, sont présents dans les bases de la \ac{ddcol}\footnote{Voir \reference{annexe_bdd_ina} (\reference{bdd_ddcol_ina}).} et la \ac{dj} présentées ici. Leur structure\footnote{Ces structures sont détaillées dans les chapitres consacrés aux alignements des données de l'\ac{ina}.} est différente selon les usages qui ont conduit à leur création, et aux besoins qui en résultent: des notes qualités décrivant la fonction précise des personnes sont présentes dans le lexique des personnes de la \ac{ddcol} alors que seul un domaine d'activité général est conservé à la \ac{dj}. Les systèmes documentaire et juridique de l'\ac{ina} ne sont pas interopérables et n'ont pas été conçus pour l'être: d'un côté, soit l'événement de diffusion est prioritaire, soit l'extrait documentaire l'est; de l'autre, l'information juridique joue ce rôle. Les usages sont tous différents et dirigent le stockage des données dans l'Institut.
\section{\label{I-B-3}Multiplication des sources de données et des référentiels}
\titreEntete{Multiplication des sources de données et des référentiels}

De manière à améliorer et enrichir ses données, à faciliter le travail de catalogage, de description et d'indexation, l'\ac{ina} récupère des métadonnées et des données à l'extérieur, auprès de plusieurs fournisseurs. Certains fournisseurs deviennent alors eux-mêmes des référentiels, dont l'identifiant qu'ils fournissent est présent dans les bases de données de l'\ac{ina} aux côtés des données fournies.\\

Ainsi, l'\ac{ina} reçoit des informations concernant les chaînes de provenance, les noms du générique avec les titres, les audiences et le public cible du document\footnote{Voir \reference{annexe_fournisseurs_exterieurs} (\reference{sheldon_mediametrie}).}, ou encore les grilles de diffusion prévisionnelles et réelles. L'ensemble de ces informations permet d'accompagner la tâche de catalogage en fournissant des champs préremplis. Les fournisseurs de ces données sont multiples\footnote{Voir \reference{annexe_fournisseurs_exterieurs} (\reference{enrichissement_dl}).} et fournissent des données tant sur les programmes que sur les producteurs eux-mêmes: 
\begin{itemize}
	\item Les données prévisionnelles de diffusion de la télévision sont achetées auprès de la société Plurimédia\footnote{Voir \url{http://www.plurimedia.fr/}.}. Les fictions, les documentaires, les dessins animés, les émissions de toutes natures, les magazines, \dots sont ainsi décrits au préalable par cette société.
	\item Les données réelles de la diffusion télévisuelle et radio --- date, horaires, parts d'audience, public --- sont fournies par Médiamétrie\footnote{Voir \url{https://www.mediametrie.fr/}.}, en complément des données ---programmation, diffusion, description des contenus --- reçues de la part des diffuseurs eux-mêmes.
	\item Des informations complémentaires sur les programmes sont acquises auprès d'agences de presse comme Kantarmédia\footnote{Voir \url{https://www.kantarmedia.com/fr}.}; pour les producteurs, les informations sont obtenues depuis la société Karl More Productions France.
\end{itemize}

%concli
	\chapter{\label{I-C}L'arbre, un vocabulaire contrôlé hiérarchique}
\titreEntete{L'arbre, un vocabulaire contrôlé hiérarchique}

Nous l'avons évoqué (\reference{I-C-2}), le contexte d'un terme de vocabulaire peut lui donner un sens complémentaire ou différent. La hiérarchisation des vocabulaires permet un ajout de contexte à chaque terme, mais également un accroissement de la précision de la définition donnée à ce terme. Le vocabulaire hiérarchique contrôlé le plus fréquent est le thésaurus: la diversité de ses relations et de ses caractéristiques lui permet une adaptation à chaque vocabulaire. Cependant, la hiérarchie n'offre plus assez d'autorités pour décrire précisément les données de l'\ac{ina}.

\section{\label{I-C-1}L'arbre de Porphyre: origines et influences}
\titreEntete{L'arbre de Porphyre: origines et influences}

%intro
La définition d'un terme est une réflexion millénaire, et la recherche d'un référentiel, d'un dictionnaire pur, n'est toujours pas aboutie. En effet, l'intelligence artificielle nécessitant des référentiels solides, la réflexion sur la pureté du dictionnaire utilisé est constante. \nP{Umberto}{Eco} considère que le dictionnaire \og ne devrait comporter, pour la définition d'un  terme, que les propriétés nécessaires et suffisantes pour distinguer ce concept d'un autre\fg\footcite[chap.1]{eco_arbre_2010}. Ces propriétés nécessaires à la définition du terme ne doivent pas être une connaissance du monde, mais bien des propriétés analytiques: \og Animal\fg{} est une propriété analytique de \og Chien\fg{} alors que l'aboiement est une connaissance.\\

La théorisation du dictionnaire remonte à l'Antiquité et a eu de nombreuses influences dans les systèmes classificatoires jusqu'à nos jours: les vocabulaires utilisés en institutions patrimoniales sont pour la plupart des hiérarchies de termes.

\subsection{\label{I-C-1-a}L'arbre de Porphyre}
\titreEntete{L'arbre de Porphyre}

La pensée aristotélicienne considère la définition d'un terme comme la forme substantielle, c'est à dire les attributs essentiels: l'\og homme\fg{} est un \og Animal rationnel mortel\fg{}\footcite[chap.1]{eco_arbre_2010}. L'assemblage de ces propriétés essentielles crée une définition, mais chacune de ces propriétés peut s'appliquer à d'autres entités. \\

Le commentateur des \textit{Catégories} d'Aristote au \textsc{III}\textsuperscript{ème}siècle, Porphyre, établit des arbres pour décrire le monde: celui des \og Substances\fg{} a le plus de postérité en étant \og un ensemble hiérarchisé et fini de genres et de substances\fg{}\footcite[chap.1]{eco_arbre_2010}. Il part du \textit{Summus genus}, la Substance, pour atteindre une espèce indivisible, définie uniquement par ses attributs analytiques appelés genres\footnote{Voir \reference{arbre_porphyre_analytique}}. Un arbre de Porphyre est, par conséquent, une succession de genres divisés en espèces qui deviennent elles-mêmes des genres.\\

\begin{figure}[!h]
	\centering
	\Tree[.\textsc{Substance} 
		Incorporelle
		[.Corporelle 
			{Non vivante}
			[.Vivante 
				{Non animale}
				[.Animale 
					[.\textbf{Homme/Cheval} ]]]]]
	
	\caption[Arbre porphyrien de l'homme avec les seuls attributs analytiques]{Arbre porphyrien de l'homme avec les seuls attributs analytiques [d'après \cite{eco_arbre_2010}]}
	\label{arbre_porphyre_analytique}
\end{figure}

\begin{figure}[!h]
	\centering
	\includegraphics[width=12cm]{images/arbre_porprhyre_differences.png}
	\caption[Arbre porphyrien prenant en compte les différences]{Arbre porphyrien prenant en compte les différences [Source: \cite[chap.1]{eco_arbre_2010}]}
	\label{arbre_porphyre_differences}
\end{figure}

L'impossibilité de la distinction entre l'homme et le cheval impose de tenir compte des différences qui ne sont pas des attributs analytiques: \og La rationalité est la différence de l'homme\fg{}\footcite[chap.1]{eco_arbre_2010}. Ainsi, ces différences vont s'ajouter aux genres des espèces. Ces différences deviennent elles-mêmes divisibles et constitutives: elles deviennent genres. Ces différences sont essentielles pour distinguer une espèce d'une autre (voir \reference{arbre_porphyre_differences}).\\

Cependant, si la prise en compte des différences permet de différencier l'homme du cheval, elles ne permettent pas de distinguer le cheval de l'âne par exemple. Un même genre doit donc être utilisé plusieurs fois dans l'arbre, ce qui le rend infini, et l'établissement d'un dictionnaire impossible à réaliser (voir \reference{arbre_porphyre_boucle}).\\

\begin{figure}[!h]
	\centering
	\includegraphics[width=12cm]{images/arbre_porphyre_boucle.png}
	\caption[Infinitude de l'arbre de Porphyre]{Infinitude de l'arbre de Porphyre [Source: \cite[chap.1]{eco_arbre_2010}]}
	\label{arbre_porphyre_boucle}
\end{figure}

Face à cette impossibilité de décrire le monde avec des divisions uniques dans un seul arbre, c'est à dire d'établir un dictionnaire universel, absolu et global, la seule solution paraît être la création d'un nombre infini d'arbres, composés de propriétés s'articulant selon le contexte et le domaine d'utilisation de l'arbre: d'un seul arbre insaisissable, une forêt réorganisable à l'envi et à l'infini est apparue, laissant le choix à l'utilisateur de l'arbre utilisé selon le sujet.

\subsection{\label{I-C-1-b}L'encyclopédisme (Antiquité - Moyen-Âge): la recherche d'un arbre global mimant le monde réel}
\titreEntete{L'encyclopédisme}

L'utopie de saisie totale du monde se retrouve dans l'encyclopédisme, dès l'\textit{Historia naturalis} de \nP{Pline}{l'Ancien}. Sur le même principe que l'arbre porphyrien, la hiérarchie de \index[ref]{typologie@Typologie!index@Index}l'index de cette encyclopédie de 37 volumes part de l'original vers le dérivé, du naturel à l'artifice: \og Une encyclopédie, pour s’organiser, tente de suivre le modèle de l’arbre --- qui est toujours plus ou moins consciemment celui de la subdivision binaire d’un arbre porphyrien\fg{}\footnote{\cite[chap.1]{eco_arbre_2010}. Voir \reference{index_pline}}. Cependant, l'index d'une encyclopédie se distingue des termes d'un arbre porphyrien en ce qu'il est défini dans un autre développement --- un article d'encyclopédie ---, alors que les termes de l'arbre de Porphyre ne peuvent pas être définis par la suite.

\begin{figure}[!h]
	\centering
	\includegraphics[width= 13cm]{images/index_pline.png}
	\caption{Extrait de l'arborescence de l'index de \nP{Pline}{l'Ancien}}
	\label{index_pline}
\end{figure}

Avec le passage au christianisme, l'encyclopédisme doit décrire les textes sacrés et non plus le monde. Ainsi, des éléments moralisateurs et allégoriques se retrouvent dans les \index[ref]{typologie@Typologie!index@Index}index, devant les éléments matériels du monde\footnote{La tradition moralisatrice encyclopédique naît avec le \textit{Physiologos} d'un auteur grec et s'inspire de l'œuvre de \nP{Pline}{l'Ancien}, puis se poursuit tout au long du Moyen-Âge avec les \textit{Étymologies} d'\nP{Isidore}{de Séville} notamment.}. À partir du \textsc{XIII}\textsuperscript{ème}siècle, les encyclopédies montrent l'ordre qui les dirige: cela conduit à \textit{L'arbre de science} de \nP{Raymond}{Lulle} qui crée seize arbres représentant l'Être, chacun représentant un savoir différent en se divisant en sept parties (racines, tronc, branches, rameaux, feuilles, fleurs, fruits)\footcite[chap.10]{eco_arbre_2010}. Contrairement à l'arbre de Porphyre qui est un arbre vide que l'on peut remplir selon le contexte, les arbres que propose \nP{Raymond}{Lulle} sont pleins et ont pour vocation de décrire et de classer le monde, la Grande Chaîne de l'Être.

\subsection{\label{I-C-1-c}Influences: une diversité de référentiels hiérarchiques}
\titreEntete{Influences: une diversité de référentiels hiérarchiques}

La pensée aristotélicienne puis le commentaire porphyrien ont produit une tradition de hiérarchisation du monde qui s'est poursuivie pendant plus d'un millénaire, sans cesse confrontée à l'impossibilité d'une description totale de ce monde. La multiplicité des arbres est, chez \nP{Umberto}{Eco} puis dans celle de \nP{Raymond}{Lulle}, la conclusion de leur réflexion. L'influence de cette tradition de description est sensible jusqu'à aujourd'hui, notamment dans le domaine de l'indexation et de la bibliothéconomie.\\

En effet, une diversité de référentiels est apparue, chacun étant dérivé d'un arbre. Des schémas de classification sont définissables à l'infini, emboîtant les genres, les espèces et les différences\footnote{\og Un simple artifice classificatoire consiste à emboîter des genres, des espèces et des différences sans en expliquer le \textit{definiendum}\fg{} in \cite[chap.1]{eco_arbre_2010}}. La \index[ref]{typologie@Typologie!taxonomie@Taxonomie}taxonomie naît de ce modèle d'arbre. Elle n'a pas pour but de dire comment repérer le concept décrit, elle permet seulement de classer en renvoyant, pour chaque nœud, vers un autre chapitre où l'on décrit ces propriétés. La taxonomie, bien qu'historiquement appliquée aux sciences de la terre, a été reprise par \nP{Melvil}{Dewey} dans sa classification décimale Dewey\index[ref]{autorites@Autorités!dewey@Dewey} en 1876.\\

Définie comme un \og classement hiérarchique de termes préférentiels\fg{} par \nP{Louis}{Rosenfeld} et \nP{Peter}{Morville}\footcite{rosenfeld_information_2015}, la taxonomie ne veut pas définir, mais simplement permettre l'utilisation correcte et logique du terme, par l'attribution de catégories et l'utilisation exclusive de relations hiérarchiques.\\

Les \index[ref]{typologie@Typologie!thesaurus@Thésaurus}\textit{thesauri}\footnote{Ils sont décrits comme une \og liste organisée de termes contrôlées et normalisés (descripteurs et non-descripteurs) servant à l’indexation des documents et des questions dans un système documentaire\fg{} dans \cite{degez_thesauroglossaire_2001}. Peu formels, ils sont néanmoins le vocabulaire le plus utilisé pour l'indexation. L'un des \textit{thesauri} les plus utilisés est le \ac{gemet}(\cite{noauthor_general_nodate}). Le \ac{gemet}\index[ref]{autorites@Autorités!gemet@GEMET} est disponible en plus de trente langues et diffusé par l'Agence européenne de l'Énergie. Voir \reference{I-C-2}.} utilisent plus de relations et de types de termes, de manière à indexer des contenus avec des mots-clés et faciliter ainsi la recherche. Ce vocabulaire contrôlé et hiérarchique reste proche du langage naturel en y intégrant les variantes, les synonymes, les descriptions, les traductions et les équivalences.\\

Pour avoir une plus grande formalisation du thésaurus, il faut utiliser une \index[ref]{typologie@Typologie!ontologie@Ontologie}ontologie. Cette ontologie est la spécification formelle d'un espace de noms, d'un domaine particulier de la connaissance\footnote{L'une des ontologies les plus utilisées, notamment dans le web sémantique, est \ac{foaf}. \ac{foaf}\index[ref]{relier@Relier!foaf@FOAF} permet la description précise des personnes. Voir \cite{noauthor_foaf_nodate}.}. Elle identifie alors les objets à décrire, leurs relations au sein de ce domaine ainsi que leurs propriétés. L'ontologie n'est pas utilisée directement dans l'indexation ou la recherche, elle est d'abord utilisée pour instancier et raisonner, en s'éloignant du langage naturel avec l'utilisation d'identifiants techniques.\\

Les \index[ref]{typologie@Typologie!taxonomie@Taxonomie}taxonomies, les \textit{thesauri} ainsi que les \index[ref]{typologie@Typologie!ontologie@Ontologie}ontologies héritent tous du modèle de l'arbre, la description ou la classification par la hiérarchie étant la plus efficace. Ces \index[ref]{typologie@Typologie!vocabulaires controles@Vocabulaires contrôlés}vocabulaires sont les plus complexes par les relations qui les composent. \nP{Louis}{Rosenfeld} et \nP{Peter}{Morville}\footnote{\cite{rosenfeld_information_2015}. Voir \reference{frise_voca}} considèrent l'anneau de synonymie comme le plus simple des vocabulaires, avec des relations d'équivalence, alors que les \index[ref]{autorites@Autorités!fichiers autorite@Fichiers d'autorité}fichiers d'autorité et les taxonomies, fonctionnant sur la hiérarchie, sont plus complexes. Les \index[ref]{typologie@Typologie!thesaurus@Thésaurus}\textit{thesauri} et les ontologies sont encore plus complexes puisqu'ils sont constitués de relations hiérarchiques et associatives.

\begin{figure}[!h]
	\centering
	
	\begin{pspicture}(0,2)(10,8)
		\psline[linewidth=1.5pt]{->}(0,5)(10,5)
		\uput[0](-2.5,5){\textsc{Simple}}
		\uput[0](10.5,5){\textsc{Complexe}}
		\uput[0](3,3){\textbf{\textsc{Type de relations}}}
		\uput[0](2.5,7){\textbf{\textsc{Type de vocabulaire}}}
		\uput[0](0,4){Équivalence}
		\uput[0](3.8,4){Hiérarchique}
		\uput[0](7.6,4){Associative}
		\uput[0](-1,6.3){Anneau de synonymie}
		\uput[0](2.5,5.5){Fichiers d'autorité}
		\uput[0](5,6.1){Taxonomies}
		\uput[0](7,5.5){Thésaurus}
		\uput[0](9,6.3){Ontologies}
	\end{pspicture}
	
	\caption[Classification des vocabulaires selon leur complexité]{Classification des vocabulaires selon leur complexité [d'après \cite{rosenfeld_information_2015}]}
	\label{frise_voca}
\end{figure}

\section{\label{I-C-2}Les \textit{thesauri}, vocabulaire contrôlé hiérarchique le plus fréquent}
\titreEntete{Les thesauri}

\section{\label{I-C-3}Passer du texte libre à un vocabulaire contrôlé: aligner des notes qualité et un thésaurus de noms communs}
\titreEntete{Passer du texte libre à un vocabulaire contrôlé}



%conclu
\bigskip
\bigskip
\bigskip
L'utilisation d'un thésaurus permet d'aligner des termes ensemble et de relier du texte libre avec un vocabulaire contrôlé de manière à disposer d'un vocabulaire commun de description. Plus encore, la hiérarchie d'un thésaurus permet la classification d'un ensemble de concepts --- ici les fonctions --- selon quelques catégories globales. Le thésaurus a par conséquent la double fonction d'offrir un enrichissement du terme préférentiel par ses relations d'association, et de proposer une classification par ses relations hiérarchiques.\\

Aligner du texte libre avec un thésaurus nécessite plusieurs étapes et la prise en compte des différences de langage --- l'un étant un contrôle minimal du langage humain naturel, l'autre un vocabulaire contrôlé natif --- :
\begin{itemize}
	\item une normalisation est d'abord nécessaire de chaque côté de l'alignement selon les mêmes règles;
	\item puis un alignement selon l'exactitude avec le terme préférentiel peut être réalisé
	\item suivi par un autre alignement selon l'exactitude avec une variante du terme préférentiel;
	\item ensuite, aligner selon l'exactitude du commencement de la fonction avec le terme préférentiel est possible, 
	\item ainsi qu'aligner selon l'exactitude du commencement de la fonction avec une variante du terme préférentiel
	\item pour enfin aligner selon l'exactitude du deuxième terme non polysémique de la fonction avec le terme préférentiel
\end{itemize}


	
	\part{\label{relier}RELIER. Vers le partage de référentiels communs (début des années 2000 – milieu des années 2010)}
	
	\chapter{\label{II-A}Le web de données: une exposition commune des référentiels}
	\titreEntete{Le web de données: une exposition commune des référentiels}
	\chapter{\label{II-B}Partager des structurations similaires de jeux de données par les classes et les propriétés : les ontologies, grammaires communes mais spécifiques}
	\titreEntete{Les ontologies, grammaires communes mais spécifiques}
	\chapter{\label{II-C}Relier ses données à Wikidata}
	\titreEntete{Relier ses données à Wikidata}
	
	\part{\label{centraliser}CENTRALISER. Le référentiel, clé de voûte et pivot (depuis le milieu des années 2010)}	
	
	\chapter{\label{III-A}Les labyrinthes comme réseaux de données et de liens}
	\titreEntete{Les labyrinthes comme réseaux de données et de liens}
	\chapter{\label{III-B}Le Lac de données de l’INA : le référentiel au centre du modèle}
	\titreEntete{Le référentiel au centre du modèle}
	\chapter{\label{III-C}Centraliser les référentiels de l’INA dans le Lac de données: l'exemple de l'alignement de deux référentiels de personnes physiques}
	\titreEntete{Aligner deux référentiels de personnes physiques}

	\chaptertoc{Conclusion}
\titreEntete{Conclusion}

%historique de struct ref: arbre au laby et modele reseau

%changement des usages et des besoins...

%... qui produit changement place ref
	
	\appendix
	\part*{Annexes}	
	\addcontentsline{toc}{part}{Annexes}
	\setcounter{chapter}{0}

\chapter{\label{annexe_index_schoepflin}Les index de la Renaissance, termes contrôlés et classification alphabétique (les index de l'\textit{Alsatia Illustrata} de \nP{Jean-Daniel}{Schoepflin})}
\titreEntete{Annexe \thechapter}

\begin{figure}[!h]
	\centering
	\begin{minipage}[c]{.46\linewidth}
		\includegraphics[width=6cm]{images/index_auctorum_alsatia.jpg}
		\caption{Index auctorum}
	\end{minipage} \hfill
	\begin{minipage}[c]{.46\linewidth}
		\includegraphics[width=6cm]{images/index_rerum_alsatia.jpg}
		\caption{Index rerum}
	\end{minipage} 
	\medskip
	Extraits des deux index de l'œuvre de \nP{Jean-Daniel}{Schoepflin} [Source: \url{http://bibliotheque-numerique.inha.fr/idurl/1/12532}, p.804 et 813]
\end{figure}

\chapter{\label{annexe_types_interop}Les différents types d'interopérabilité}
\titreEntete{Annexe \thechapter}

\begin{figure}[!h]
	\centering
	\includegraphics[width=12cm]{images/interop_conversion_copie.jpeg}
	\medskip
	\caption[L'interopérabilité par conversion et copie]{L'interopérabilité par conversion et copie [Source: \cite{bermes_2_2013}]}
\end{figure}

\begin{figure}[!h]
	\centering
	\includegraphics[width=12cm]{images/interop_denom_commun.jpeg}
	\medskip
	\caption[L'interopérabilité par le plus petit dénominateur commun]{L'interopérabilité par le plus petit dénominateur commun [Source: \cite{bermes_2_2013}]}
\end{figure}

\chapter{\label{annexe_thesaurus}Le thésaurus de noms communs de l'\ac{ina}}
\titreEntete{Annexe \thechapter}

\begin{figure}[!h]
	\centering
	\includegraphics[width=7cm]{images/cadreur_hierarchie.png}
	\medskip
	\caption[Extrait du thésaurus de noms communs de l'\ac{ina}]{Extrait du thésaurus de noms communs de l'\ac{ina} autour du terme \og Cadreur\fg{}}
\end{figure}
	
	\backmatter
	
	
%bibliographie ici dans les normes de l'école
%\printbibliography[title= Bibliographie sélective, prenote=intro]%postnote est aussi possible
%\printbibliography[heading=subbibliography, keyword={semantique}, title={Sémantique}]%biblio sélective pour un mot clé donné

\printindex[referentiels]
\printindex[ina]

	\listoffigures

	\tableofcontents
	
\end{document}